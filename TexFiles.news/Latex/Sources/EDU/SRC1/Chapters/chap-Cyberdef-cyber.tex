%-------------------------------------------------------------
%               FR CYBERDEF SECOPS COURSE
%                           SECOPS
%.                    Vulnerability/Threat/Incident
%
%                   Introduction Cyberdefense
%                    chap-Cyberdef-intro.tex
%
%                     2020 eduf@ction
%-------------------------------------------------------------
\uchap{Chapitre de définition sur la cyber}

% ***** CHAPTER CYBERDEF


\section{Le prefix Cyber}

Cyber est un préfixe qui désigne une relation avec les technologies de l'information (TI). Tout ce qui a trait à l'informatique, comme l'internet, relève de os jour de la catégorie cyber au sens cybernétique Il convient de noter que le terme "cyber" a plutôt aujourd'hui une connotation sécurité dans l'informatique et dans technologies de l'information. Par exemple, dans informatique des années 1980 à 2000 n'utilisait pas le terme cyber pour désigner la sécurité informatique ou la sécurité des systèmes d'information.

\subsection{Une origine}

À la fin des années 1940, le terme cybernétique a été inventé par le mathématicien Norbert Wiener. Il est défini comme l'étude des systèmes de contrôle et de la communication entre les personnes et les machines. Weiner a utilisé le mot cybernétique du grec ancien, qui est lié à l'idée de gouverner. Dans son livre Cybernetics, Weiner décrit un système informatique qui fonctionne par rétroaction, essentiellement un système autonome. Cette idée était révolutionnaire dans les années 1940.

\subsection{Quelques mots courants}

Cyber est généralement utilisé comme préfixe d'un mot composé. Comme de nombreux noms composés, ceux qui comportent le mot "cyber" peuvent être écrits en un seul mot (cyberspace), en deux mots (cyber espace) ou en un mot composé (cyber espace). Par souci de cohérence, tous les exemples seront écrits en un seul mot. Les termes cyber couramment utilisés sont les suivants :


\begin{itemize}

	\item  Cyberespace : Une métaphore pour décrire le terrain non-physique créé par les systèmes informatiques.
	\item Cybersécurité : Les technologies et processus conçus pour protéger les ordinateurs, les réseaux et les données contre les accès non autorisés, les vulnérabilités et les attaques menées par des cybercriminels via l'internet.
	\item Cybercriminalité : Tout crime commis à l'aide de l'informatique ou qui vise l'informatique.
	\item Cyberattaque : L'accès non autorisé à des informations privées ou confidentielles contenues dans un système ou un réseau informatique.
	\item Cybermenaces :
	\item Cyberguerre : 
	\item Cyberintimidation : Toute forme de harcèlement en ligne.
	\item Cyberforensics : L'application de méthodes scientifiquement prouvées pour rassembler, traiter, interpréter et utiliser des preuves numériques afin de fournir une description concluante des activités cybercriminelles.
	\item Cybernétique : La science des communications et des systèmes de contrôle automatique des machines et des êtres vivants.
	\item Cybersécurité : Techniques et organisations pour faire face aux menaces
	\item cyberdéfense : 

\end{itemize}




