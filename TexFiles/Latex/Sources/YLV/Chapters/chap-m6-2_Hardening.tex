\section{Règles techniques de sécurisation : durcissement}

Les actions de durcissement, en anglais \textbf{hardening}, consistent à améliorer le niveau de sécurité des systèmes via des actions de configuration, choix techniques et process.

Le durcissement s'inscrit naturellement dans le process de conception securisée par défaut décrit dans le chapitre précédent. Il est tout aussi important que le maintien en condition opérationnelle et sécurisée et peut réduire significativement les risques cyber (ce qui est bien évidemment le but final).

Ce chapitre décrit les différentes techniques de sécurisation appliquées à la gestion des droits et accès, les systèmes d'exploitation et applications, le matériel (batiments équipements) et et enfin les réseaux.
La liste des exemples présentées n'est pas exhaustives et bien évidemment évolutive en fonction des avancées technologiques et des besoins de sécurisation qui peuvent évoluer dans le temps.

\subsection{Sécurisation des droits et accès} 
%(PKI, MFA, journalisation, contrôles)

Pourquoi déployer une infrastructure de gestion de clés?

La mise en place d'une infrastructure de gestion de clés (IGC), ou Public Key Infrastructure (PKI) en anglais, est essentielle pour garantir la sécurité des communications et des transactions numériques. En intégrant des mécanismes tels que l'authentification à facteurs multiples (MFA), les organisations renforcent la protection des accès en exigeant plusieurs preuves d'identité. Cette méthode utilise le principe de coupler un élément que l'on possède (carte ou donc ici une clé PKI par exemple) et que l'on connait (mot de passe, ou réponse à une question).
L'ANSSI définit dans son cyberdico la PKI\footnote{PKI: https://cyber.gouv.fr/le-cyberdico#P} comme un "Ensemble organisé de composantes fournissant des services de gestion des clés cryptographiques et des certificats de clés publiques au profit d’une communauté d’utilisateurs". 

De plus, la journalisation des activités permet de suivre et d'analyser les accès et les modifications, assurant ainsi un suivi et une traçabilité indispensable en cas d'incident. 

Enfin, l'implémentation de contrôles permet de gérer efficacement les clés cryptographiques, de la création jusqu'à la destruction, réduisant ainsi les risques de compromission et assurant la confidentialité et l'intégrité des données.

Voir au chapitre 3 les principes cryptographiques et fonctionnalités des IGC (PKI).

\subsection{Systèmes d'exploitation et applications}
% Hardening UNIX/WINDOWS kesako?
% ACL, admin dédié, supervision et administration via réseau chiffré, tout doit être loggé ! Exclusion services inutiles
% 80 443 principe qui se décline sur toutes les applis

Les systèmes d'exploitation Windows, Linux et même MacOS sont des cibles privilégiées par les attaquants. Il est donc important qu'ils soient configurés afin d'être le moins possible vulnérable à des attaques. Le Center for Internet Security (CIS) propose des configurations types pour chaque système et il est recommandé de les appliquer. Ces recommandations et configurations sont appelés "durcissement des systèmes d'exploitation" car le but est bien de renforcer la sécurité.

Par exemple, pour les environnements UNIX/Linux, l'activation de SeLinux est recommandée, tandis que pour Windows, l'utilisation de stratégies de groupe (GPO) et d'AppLocker est conseillée. En ce qui concerne les applications, il est important de ne pas afficher publiquement la version utilisée et d'appliquer des règles de conception visant à protéger les données confidentielles.

L'administration des réseaux et des systèmes doit également suivre des pratiques de mises en oeuvre sécurisées, telles que l'utilisation de listes de contrôle d'accès (ACL) afin de limiter les accès aux réseaux et utilisateurs spécifiques. Il est conseillé de séparer le réseau d'administration des autres réseaux de l'entreprise et de réaliser la supervision via un réseau chiffré, en utilisant des protocoles sécurisés comme SNMPv3 et SSH. De plus, il est essentiel de remplacer les mots de passe par défaut par des mots de passe forts et de les stocker dans une base de données sécurisée (coffre fort numérique).

Sur tous les systèmes, OS, application ou même réseaux, il convient d'exclure les services inutiles, en faisant attention aux serveurs web qui peuvent être activés par défaut. La conservation des journaux d'événements est primordiale pour permettre une investigation efficace en cas de problème. Enfin, il est recommandé de privilégier les protocoles sécurisés, tels que HTTPS, SMTPS et IMAPS, et d'utiliser des certificats valides pour garantir la sécurité des communications.


\subsection{Hardware (HSM), Datacenter}
% Chiffrement, zone hardware dédiée (mémoire, voire carte dédiée), 
% DC : salles, controles d’accès, caméras, vigiles, etc.

Le durcissement et la sécurisation du matériel sont des éléments essentiels pour protéger les infrastructures informatiques. L'une des premières mesures à envisager est le chiffrement des données, qui permet de garantir la confidentialité et l'intégrité des informations sensibles, même en cas de compromission physique.

Les modules de sécurité matériels (HSM - Hardware Security Module) représentent une solution robuste pour la protection des secrets cryptographiques et des données sensibles. Ces dispositifs physiques, conçus avec des mécanismes anti-effraction, offrent un environnement hautement sécurisé pour le stockage et la gestion des clés cryptographiques, des certificats numériques et autres secrets. Les HSM assurent non seulement la protection physique des données, mais effectuent également les opérations cryptographiques critiques directement dans leur environnement sécurisé, évitant ainsi l'exposition des clés dans la mémoire du système hôte. Ils sont particulièrement utilisés dans les infrastructures à clés publiques (PKI) décrits dans le premier paragraphe de ce chapitre, les systèmes de paiement et les environnements nécessitant une conformité réglementaire stricte. Les HSM intègrent des mécanismes de contrôle d'accès, de journalisation des opérations et de sauvegarde sécurisée, garantissant ainsi l'intégrité et la disponibilité des secrets tout au long de leur cycle de vie.

Il est également recommandé de créer des zones matérielles dédiées, telles que des mémoires sécurisées ou des cartes dédiées, pour stocker des données critiques et des clés de chiffrement. Cela limite l'accès aux informations sensibles et réduit les risques d'exposition.

Les centres de données ou Datacenter en anglais, sont des batiments dédiés à l'hébergement de serveurs informatiques. Leur fonction impose naturellement un contrôle d'accès afin de restreindre l'entrée aux personnes autorisées uniquement. Leur sécurisation peut être assurée par l'utilisation de caméras de surveillance et la présence de vigiles qui assureront la surveillance en temps réel des activités afin de réagir rapidement en cas d'incident.

\subsection{Sécurisation des réseaux}
%VPN, chiffrement

La sécurisation et le durcissement des réseaux informatiques nécessitent une approche multicouche intégrant plusieurs niveaux de protection. Comme décrit dans le 3ème chapitre, la protection périmétrique constitue la première ligne de défense, s'appuyant sur des pare-feu nouvelle génération (NGFW) couplés à des systèmes de détection et de prévention d'intrusion (IDS/IPS).

Les réseaux de transit, particulièrement vulnérables aux attaques, requièrent une attention spécifique avec le déploiement de protocoles de chiffrement robustes comme TLS 1.3 et AES-256 (versions de en 2023-2024, ces protocoles évoluent et leur choix doit être en adéquation du besoin de la version disponible). La segmentation réseau, via l'utilisation de VLAN et de zones DMZ, ainsi que le filtrage du trafic par des listes de contrôle d'accès (ACL), renforcent cette protection.

Pour sécuriser les connexions distantes, les VPN intègrent une authentification forte reposant sur des certificats et l'authentification multifacteur (MFA), des tunnels IPsec ou SSL/TLS (cf. chapitre 3 également), ainsi qu'un chiffrement de bout en bout des communications.

Comme explicité et décrit plus précisement dans le chapitre X, les réseaux doivent être supervisés via un dispositif de monitoring continu assuré par des SIEM et des SOC, permettant la détection des anomalies et incidents.

%Begin FRAME----------------------------
\mode<presentation>{\texframe
% contenu affiché sur Article et Beamer
%- - - - - - - - - - - - - - - - - - - - - - - - 
{Règles techniques de sécurisation} % titre de la diapo
{Hardening} % sous titre de la diapo
{
%begin slide- - - - - - - - - - - - - - - - - 
 \begin{enumerate}
     \item IAM : PKI, MFA, journalisation, contrôles
     \item Systèmes d'exploitation et application
     \item Matériel (HSM) et locaux (Data Center)
     \item Réseaux : VPN, chiffrement, supervision
 \end{enumerate}
%end slide- - - - - - - - - - - - - - - - - - - 
}}
%End FRAME------------------------------

%Begin FRAME----------------------------
\mode<presentation>{\texframe
% contenu affiché sur Article et Beamer
%- - - - - - - - - - - - - - - - - - - - - - - - 
{Règles techniques de sécurisation : IAM} % titre de la diapo
{IAM} % sous titre de la diapo
{
%begin slide- - - - - - - - - - - - - - - - - 
 \begin{itemize}
     \item Public Key Infrastructure (PKI) : pourquoi déployer une infrastructure de gestion de clés?
     \item MFA : Multiple Factor Access
     \item journalisation
     \item contrôles
 \end{itemize}
%end slide- - - - - - - - - - - - - - - - - - - 
}}
%End FRAME------------------------------

%Begin FRAME----------------------------
\mode<presentation>{\texframe
% contenu affiché sur Article et Beamer
%- - - - - - - - - - - - - - - - - - - - - - - - 
{Règles techniques de sécurisation : OS & applications} % titre de la diapo
{Systèmes d'exploitation et applications} % sous titre de la diapo
{
%begin slide- - - - - - - - - - - - - - - - - 
Ref : documents édités par le CIS (Center for Internet Security)
\begin{itemize}
    \item Hardening OS
    \begin{itemize}
        \item UNIX/LINUX : SeLinux
        \item WINDOWS : GPO, Applocker
    \end{itemize}
    \item Applications : 
        \begin{itemize}
            \item ne pas afficher en accès publique la version utilisée
            \item règles de design pour protéger les données confidentielles
        \end{itemize}
    \item Administration
        \begin{itemize}
            \item Access Control List (ACL) : limiter les accès aux réseaux/utilisateurs dédiés
            \item réseau admin dédié : séparer les réseaux administration des autres réseaux de l'entreprise
            \item supervision et administration via réseau chiffré : utilisation de protocoles sécurisés tels que : SNMPv3, SSH
            \item remplacement des mots de passe par défaut par des mots de passe forts
            \item stockage des mots de passe dans une base de données sécurisée (coffre fort)
        \end{itemize}
    \item Exclusion services inutiles : attention aux serveurs web lancés par défaut, etc.
    \item Journaux d'événements : garder toutes les traces nécessaires à l'investigation en cas de problème
    \item 80 --> 443 : en règle général, préférer les protocoles sécurisés tels que HTTPs, SMTPs, IMAPs, etc.
    \item Utilisation de certificats valides
\end{itemize}
%end slide- - - - - - - - - - - - - - - - - - - 
}}
%End FRAME------------------------------

%Begin FRAME----------------------------
\mode<presentation>{\texframe
% contenu affiché sur Article et Beamer
%- - - - - - - - - - - - - - - - - - - - - - - - 
{Règles techniques de sécurisation : Matériel et DC} % titre de la diapo
{Matériel et Locaux (Data Center)} % sous titre de la diapo
{
%begin slide- - - - - - - - - - - - - - - - - 
 \begin{itemize}
    \item Chiffrement, zone hardware dédiée (mémoire, voire carte dédiée)
    \item HSM
    \item Datacenter : salles, contrôles d’accès, caméras, vigiles
\end{itemize}
%end slide- - - - - - - - - - - - - - - - - - - 
}}
%End FRAME------------------------------


%Begin FRAME----------------------------
\mode<all>{\texframe
% contenu affiché sur Article et Beamer
%- - - - - - - - - - - - - - - - - - - - - - - - 
{Points à retenir} % titre de la diapo
{} % sous titre de la diapo
{
%begin slide- - - - - - - - - - - - - - - - - 
\begin{itemize}
    \item \#PKI, \#CIS, \#ACL
    \item \#hardening, \#HSM
\end{itemize}
%end slide- - - - - - - - - - - - - - - - - - - 
}}
%End FRAME------------------------------