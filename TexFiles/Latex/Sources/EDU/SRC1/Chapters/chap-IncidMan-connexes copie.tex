%-------------------------------------------------------------
%               FR CYBERDEF SECOPS COURSE
%                      INCIDENT MANAGEMENT
%
%                    Technics & Methods Annexes
%
%                              2020 eduf@ction
%-------------------------------------------------------



%==================================================
\section{Méthodes et techniques connexes}
%--------------------------------------------------------------------------------

\subsection{Sécurité Offensive vs Défensive}

\begin{techworkbox}{Offensive Security}
Un sujet intéressant pour \fichetech. 
\end{techworkbox}

La sécurité offensive est une approche de la sécurité qui vise à identifier et à exploiter les vulnérabilités et les faiblesses d'un système de sécurité dans le but de le compromettre. Elle est souvent utilisée par les entreprises et les organisations pour évaluer la robustesse de leurs systèmes de sécurité et pour identifier les faiblesses qui pourraient être exploitées par des attaquants malveillants.

\begin{itemize}
  \item Tests d'intrusion : Les tests d'intrusion consistent à élaborer un scénario d'attaque contre un système  dans le but de réaliser un objectif malveillant. Ces tests sont  réalisés par des équipes de sécurité (audit, redteam) qui utilisent des outils et des techniques similaires à celles des attaquants malveillants.
  \item Recherche de vulnérabilités : La recherche de vulnérabilités consiste à identifier les faiblesses et les points faibles d'un système et en particulier des systèmes de sécurité dans le but de d'inclure ces vulnérabilités dans une chaine d'attaque.
\end{itemize}

La sécurité offensive est un peu controversée, car elle implique de simuler en profondeur des attaques malveillantes contre les systèmes et les systèmes de sécurité eux-mêmes. Cependant, elle est considérée comme un moyen efficace d'identifier et de corriger les vulnérabilités avant qu'elles ne soient exploitées par des attaquants malveillants.

\subsection{Threats Hunting}


\begin{techworkbox}{Chasse et ATP}
Un sujet intéressant pour \fichetech. 
\end{techworkbox}

Le Threat Hunting est une technique qui consiste à rechercher activement des menaces cachées au sein d'un système. Il s'agit d'une approche proactive de la sécurité qui vise à identifier les menaces qui ont réussi à passer inaperçues des outils de sécurité traditionnels et à les éliminer avant qu'elles ne causent des dommages.

Il implique généralement l'utilisation de différentes techniques et outils pour analyser les données de sécurité et rechercher des anomalies ou des comportements suspects qui pourraient indiquer la présence d'une menace. Les équipes de Threat Hunting travaillent souvent en collaboration avec les équipes de sécurité de l'information et de l'analyse de données pour identifier et éliminer les menaces cachées.

Le Threat Hunting fait partie des expertises de plus en plus demandées ces dernières années en raison de l'augmentation des attaques de malware avancées et des menaces ciblées, qui peuvent passer inaperçues des outils de sécurité traditionnels. Il est considéré comme un moyen efficace de renforcer la sécurité d'un système et de protéger contre les menaces qui pourraient autrement échapper aux défenses.

Ces métiers vont au delà de la simple "chasse" aux menaces discrètes et s'élargie vers les techniques mises en place pour gérer les interactions entre l'attaquant et les équipes de défense comme par exemple provoquer une continuité de l'attaque avec des objectifs qui peuvent aller du maintien de l'attaque pour découvrir les scénarios pensés par l'attaquant.
%
%Exemple se mettre en proxy et modifier les fichiers ex-filtrées pour les corrompre et faire en sort que l'attaquant reste plus longtemps.
%%introduction sur la surveillance opérationnelle du quotidien. les outils de « visibilités » permettant de voir, percevoir ... et anticiper.
%Réagir, gestion de crise, à quel moment géré t’on la crise.

\subsection{HoneyPots}

\begin{techworkbox}{HoneyPots}
Un sujet intéressant pour \fichetech. 
\end{techworkbox}

Les honeypots sont des systèmes de sécurité particulier. Ils visent à attirer et à identifier les attaquants  en simulant des cibles vulnérables ou attractives au sein d'un réseau ou d'un système d'information d'entreprise. Ils sont conçus pour ressembler aux systèmes existants ou à des données sensibles plausibles. Toutefois, ils sont isolés du système sensibles et ne contiennent pas d'informations ou de données sensibles pour l'entreprise. Ils sont équipées pour permettre de facilement maintenir le contact avec l'attaquant pour le pousser à dévoiler ses techniques et objectifs.

Lorsqu'un attaquant pénètre un honeypot, il lui est laissé la latitude de conduire son attaque, ses actions sont enregistrées, ce qui permet aux équipes de sécurité de comprendre comment des attaques sont menées et de développer des stratégies pour se protéger contre elles. Les honeypots peuvent également être utilisés pour détourner l'attention des attaquants et les empêcher de cibler des cibles réelles.

Il existe différents types de honeypots, tels que les honeypots de réseau, les honeypots de système et les honeypots de données. Ils sont utilisés en combinaison avec d'autres technique de sécurité, tels que les pare-feux et les systèmes de détection...
Ce sont des techniques utilisées soit par les équipes de détection et de réponse soit par les équipes de protection. En fonction de l'usage les méthodes et les concepts d'usage peuvent être différents.


\subsection{Hackback}

\begin{techworkbox}{Hackback}
Un sujet intéressant pour \fichetech. 
\end{techworkbox}

Le hackback est une réponse à une attaque qui consiste à attaquer un système informatique qui a été utilisé pour mener une attaque contre un autre système. Il s'agit d'une approche controversée de la sécurité qui implique de prendre l'initiative et de mener une attaque contre un attaquant malveillant plutôt que de se contenter de se défendre contre lui.

Le hackback est souvent utilisé pour collecter des informations sur l'attaque initiale et sur l'attaquant, qui peuvent être utilisées pour mieux se protéger contre les futures attaques. Cependant, il est également controversé en raison des risques qu'il pose, notamment en termes de dommages potentiels causés aux systèmes ciblés et de la possibilité de violer les lois et les réglementations en matière de sécurité de l'information.

Il est important de noter que le hackback est  considéré comme étant à la limite de la légalité dans la majorité pays et qu'il peut entraîner des poursuites judiciaires si utilisé de manière inappropriée. La réponse externe d'une entreprise est généralement judiciaire. 

En France, en particulier, l'article 21 du chapitre IV "Dispositions relatives à la protection des infrastructures vitales contre la cybermenace" stipule bien que l'État, et seul l'État, peut "répondre à une attaque informatique qui vise les systèmes d’information affectant le potentiel de guerre ou économique, la sécurité ou la capacité de survie de la Nation.
Par ailleurs ce droit à la réponse active ne s'improvise pas. Car avant d'estimer avoir une réponse active à réaliser, encore faut-il être certain de l'identité de son assaillant numérique, ce qui n'est pas encore spécialement facile, si ce n'est même extrêmement complexe.


