\section{Système de management de la sécurité de l'information : SMSI}

Selon la norme ISO 9000, un système de management est défini comme un ensemble de processus interconnectés ou interactifs, qui sont mis en œuvre par une organisation pour établir, mettre en œuvre, maintenir et améliorer la gestion de la qualité. Ce système vise à atteindre les objectifs de qualité de l'organisation en assurant la cohérence et l'efficacité de ses processus.

Toujours sela la norme ISO 9000, les éléments clés d'un système de management sont : 
\begin{itemize}
   \item processus interconnectés : le système repose sur une série de processus qui interagissent pour atteindre les objectifs de qualité. Chaque processus a ses propres activités, ressources, et résultats attendus;
   \item approche processus : la norme encourage une approche basée sur la gestion des processus, permettant une meilleure compréhension, gestion et amélioration continue;
   \item orientation client : le système vise à satisfaire les exigences des clients et à améliorer leur satisfaction;
   \item amélioration continue : la norme insiste sur la nécessité d'améliorer constamment l'efficacité du système de management;
   \item leadership : la direction doit s'engager activement pour établir une vision claire et soutenir la mise en œuvre du système;
   \item implication du personnel : la participation et la compétence du personnel sont essentielles pour le succès du système;
   \item gestion des ressources : la disponibilité et la gestion efficace des ressources (humaines, matérielles, informationnelles) sont fondamentales;
   \item évaluation et amélioration : la surveillance, la mesure, l'analyse et l'audit des processus permettent d'identifier les opportunités d'amélioration.
\end{itemize}

Pourquoi alors définir un système de management et se lancer dans un processus de certification?
Car les apports sont nombreux et contribuent à la performance globale de l'organisation.
Voici quelques éléments principaux : 
\begin{itemize}
   \item Amélioration de la qualité : Permet de mieux répondre aux attentes des clients en assurant une cohérence et une conformité des produits et services.
   \item Satisfaction client accrue : En se concentrant sur la satisfaction des besoins et attentes des clients, l'organisation peut renforcer leur fidélité.
   \item Efficacité opérationnelle : La gestion structurée des processus permet d'optimiser les ressources, réduire les coûts et limiter les erreurs.
   \item Meilleure gestion des risques : La mise en place d'un système de management facilite l'identification, l'évaluation et la maîtrise des risques liés aux activités.
   \item Engagement et motivation du personnel : La participation active des employés dans un cadre clair et structuré favorise leur implication et leur développement professionnel.
   \item Amélioration continue : Le système encourage une démarche d'amélioration constante, permettant à l'organisation de s'adapter aux évolutions du marché et de ses environnements.
   \item Reconnaissance et crédibilité : La certification ISO ou d'autres labels issus du système de management renforcent la crédibilité de l'organisation auprès de ses partenaires, clients et parties prenantes.
   \item Conformité réglementaire : Facilite le respect des exigences légales et réglementaires applicables à l'activité.
\end{itemize}

Quelle méthode utiliser pour l'amélioration continue des processus?
%shéma PDCA
La référence des méthodes est le cycle de Deming ou PDCA pour Plan (Planifier), Do (Réaliser), Check (Vérifier) et Act (Agir).
le cycle PDCA sert à structurer la démarche d'amélioration continue. Il permet à l'organisation de planifier ses objectifs et ses processus, de mettre en œuvre ces processus, de vérifier leur efficacité, puis d'agir pour corriger ou améliorer les pratiques. Ce cycle favorise une gestion dynamique et proactive, assurant que le système évolue en permanence pour mieux répondre aux exigences et attentes (donc instaurer un climat de confiance), tout en améliorant la performance globale.

Il existe plusieurs systèmes de management, comme celui sur la qualité (ISO 9001), l'environnement (ISO 14001) mais nous allons nous attarder plus précisement dans ce chapitre sur celui de la sécurité de l'information (SMSI) l'ISO 27001. Le système de management d la continuité d'activité (SMCA) ISO 22301 sera lui abordé un peu plus loin dans ce chapitre.

Définition du SMSI
Un SMSI, ou Système de Management de la Sécurité de l'Information, est un cadre organisé qui permet à une organisation de gérer de manière systématique la sécurité de ses informations. Selon la norme ISO/IEC 27001, il s'agit d'un ensemble de politiques, de processus, de procédures, de structures organisationnelles et de ressources qui sont mis en place pour protéger la confidentialité, l'intégrité et la disponibilité des informations.

Un SMSI repose sur plusieurs éléments clés. Tout d’abord, il inclut une politique de sécurité qui définit les orientations et les engagements de l’organisation en matière de sécurité de l’information (cf. le chapitre précédent pour plus de détails sur les politiques). Ensuite, il implique une analyse des risques, permettant d’identifier, d’évaluer et de traiter les menaces potentielles pesant sur les actifs informationnels.

L’organisation de la sécurité est également essentielle, avec la mise en place de structures, de responsabilités et de ressources dédiées à la gestion de la sécurité. Des contrôles de sécurité, tant techniques qu’organisationnels et physiques, sont déployés pour protéger l’information. Il est également crucial de disposer d’un processus de gestion des incidents, afin de détecter, répondre et se remettre rapidement en cas de problème. La sensibilisation et la formation du personnel jouent un rôle important pour garantir que tous comprennent les enjeux de sécurité. Enfin, le système doit faire l’objet d’une amélioration continue, grâce à la surveillance, aux audits et aux revues régulières pour assurer son efficacité et son évolution.

Pour mettre en œuvre un SMSI, il est recommandé que la direction s’engage activement dans le projet. Il faut adopter une approche basée sur les risques, en priorisant les actions en fonction des menaces identifiées. La communication interne doit être claire et régulière pour sensibiliser l’ensemble du personnel. La documentation précise des politiques, procédures et mesures est également essentielle. Enfin, il est important de réaliser des audits réguliers pour vérifier la conformité et l’efficacité du système, et ainsi l’améliorer en permanence.

Que trouve-t-on alors dans la norme ISO 27001 ? 

Dans la norme ISO 27001, on trouve tout d’abord une description claire des exigences que doit respecter un SMSI. Cela inclut la nécessité d’établir une politique de sécurité, d’évaluer et de traiter les risques liés à la sécurité de l’information, ainsi que de définir des objectifs et des plans d’action pour atteindre ces objectifs.

La norme détaille également les exigences relatives à la gouvernance, à la gestion des ressources, à la sensibilisation du personnel, ainsi qu’à la documentation et à la gestion des enregistrements. Elle insiste sur l’importance de la surveillance, de l’audit interne, des revues de direction et de l’amélioration continue du système.

En complément, la norme ISO 27001 est accompagnée d’un ensemble de contrôles de sécurité (définis dans l’annexe A) que l’organisation peut choisir d’appliquer en fonction de ses risques spécifiques. Ces contrôles couvrent un large spectre, allant de la gestion des accès à la sécurité physique, en passant par la gestion des incidents et la continuité des activités.

Quel est le but d'être certifié ISO 27001 ? 

Le but principal d'obtenir la certification ISO 27001 est de démontrer à ses partenaires, clients et parties prenantes que l'organisation a mis en place un système de gestion de la sécurité de l'information efficace et conforme aux meilleures pratiques internationales. Cette certification vise à renforcer la confiance en assurant que les actifs informationnels sont protégés contre les risques, les menaces et les vulnérabilités.

En étant certifiée, une organisation peut également bénéficier d'une meilleure gestion des risques liés à la sécurité, réduire la probabilité d'incidents de sécurité, et assurer la continuité de ses activités. De plus, la certification ISO 27001 peut constituer un avantage concurrentiel en attestant de la sérieux et du professionnalisme de l'organisation en matière de sécurité de l'information.

%normatif ISO27001; périmètre de certification \& mesures ISO27002 à appliquer

%Définition SMSI (voir cours Fadoua sur l'intro)

%Begin FRAME----------------------------
\mode<presentation>{\texframe
% contenu affiché sur Article et Beamer
%- - - - - - - - - - - - - - - - - - - - - - - - 
{Système de management de la sécurité de l'information : SMSI} % titre de la diapo
{Introduction à ISO/EIC 27001} % sous titre de la diapo
{
%begin slide- - - - - - - - - - - - - - - - - 
	
 \begin{enumerate}
    \item Présentation de la norme ISO/EIC 27001
    \item Périmètre de certification
    \item Mesures de sécurité : ISO/EIC 27002
 \end{enumerate}
    
%end slide- - - - - - - - - - - - - - - - - - - 
}}
%End FRAME------------------------------

\subsection{Very short intro to ISO/EIC 27001}

%Begin FRAME----------------------------
\mode<presentation>{\texframe
% contenu affiché sur Article et Beamer
%- - - - - - - - - - - - - - - - - - - - - - - - 
{Very short intro to ISO/EIC 27001} % titre de la diapo
{Introduction à ISO/EIC 27001} % sous titre de la diapo
{
%begin slide- - - - - - - - - - - - - - - - - 
	Quel est le but d'ISO 27k1 ? Pourquoi l'utiliser ?

    
%end slide- - - - - - - - - - - - - - - - - - - 
}}
%End FRAME------------------------------


% \subsection{Historique puis Basiques}
% que trouve t on dans la norme ?
\subsection{Qu’est ce que le périmètre de certification ?}

Le périmètre de certification ISO 27001 est le point de départ de tout projet de certification. Il est indispensable de bien définir quel élément du système d'information devra respecter les exigences de la norme. Le périmètre peut inclure une application, une ferme de serveurs, une équipe de collaborateurs, un data-center, etc. Il doit être clairement délimité et documenté pour assurer une compréhension commune au sein de l'organisation et faciliter la mise en œuvre des contrôles et des processus nécessaires. Une définition précise du périmètre permet également d’évaluer plus efficacement les risques, de cibler les ressources appropriées et de garantir la conformité aux exigences de la norme tout au long du processus de certification.

%Begin FRAME----------------------------
\mode<presentation>{\texframe
% contenu affiché sur Article et Beamer
%- - - - - - - - - - - - - - - - - - - - - - - - 
{Périmètre de certification} % titre de la diapo
{Introduction à ISO/EIC 27001} % sous titre de la diapo
{
%begin slide- - - - - - - - - - - - - - - - - 
% Plutot que de parler SOA, parler de domaine et de périmètre d'application , éventuellement pointeur vers un exemple (Annexe A de la 27001)
% Slide avec les différents chapitres, thèmes de l'annexe A
%Zoom sur les 
% Panorama des normes
% dérouler la norme ? pour suivre sa structure 
% comment on l'améliore (PDCA)? on contrôle, on se structure pour mettre en place, engagement de la direction
% ssms apporte de la confiance, la confiance apporte du business

	\begin{itemize}
    \item Quel est le périmètre de certification ? 
    \item SOA : définition, rôle --> BOF 
 \end{itemize}

    
%end slide- - - - - - - - - - - - - - - - - - - 
}}
%End FRAME------------------------------

\subsection{ISO/EIC 27002}
%Exemples à donner


La norme ISO/IEC 27002 est un guide de bonnes pratiques et de recommandations pour la gestion de la sécurité de l'information. Elle fournit un ensemble de contrôles de sécurité détaillés que les organisations peuvent mettre en œuvre pour gérer efficacement les risques liés à la sécurité de l'information.

Définition de la norme ISO 27002


La norme ISO 27002 offre un catalogue de contrôles de sécurité répartis en plusieurs domaines, tels que la gestion des accès, la sécurité physique, la gestion des incidents, la continuité des activités, etc. Elle sert de référence pour sélectionner, mettre en œuvre et gérer ces contrôles en fonction des risques spécifiques à chaque organisation. Contrairement à la norme ISO 27001, elle ne définit pas d'exigences obligatoires, mais fournit des recommandations pour renforcer la sécurité.

Quelles différences entre les deux normes ISO 27001 et ISO 27002?

Comme vu aux paragraphes précédents, la norme ISO 27001 définit les exigences pour établir, mettre en œuvre, maintenir et améliorer un Système de Management de la Sécurité de l'Information (SMSI). Elle est donc une norme certifiable, qui permet à une organisation d'obtenir une certification officielle.
La norme ISO 27002 est elle un guide de bonnes pratiques, destiné à aider à la sélection et à la mise en œuvre de contrôles de sécurité. Elle n’est pas certifiable en soi.

En conclusion, on peut retenir les différences suivantes : 

\begin{itemize}
   \item ISO 27001 est une norme normative avec des exigences obligatoires pour la certification;
   \item ISO 27002 est une norme de recommandations, fournissant des lignes directrices pour la mise en œuvre;
   \item ISO 27001 sert à établir un cadre de gestion de la sécurité, avec des processus, des politiques et des contrôles;
   \item ISO 27002 accompagne cette démarche en proposant des contrôles concrets et des bonnes pratiques pour leur mise en œuvre.
\end{itemize}

Quelles sont les mesures ?

La norme ISO/IEC 27002 propose un ensemble de 114 contrôles de sécurité répartis en 14 domaines ou sections principales. Ces contrôles couvrent un large éventail de mesures visant à gérer efficacement les risques liés à la sécurité de l'information.

Ces contrôles sont regroupés en 14 domaines, tels que :

\begin{enumerate}
    \item Politique de sécurité de l'information
\item Organisation de la sécurité de l'information
\item Sécurité des ressources humaines
\item Gestion des actifs
\item Contrôle d’accès
\item Cryptographie
\item Sécurité physique et environnementale
\item Sécurité des opérations
\item Sécurité des communications
\item Acquisition, développement et maintenance des systèmes d’information
\item Relations avec les fournisseurs
\item Gestion des incidents de sécurité de l'information
\item Aspects de la sécurité de l'information dans la gestion de la continuité d’activité
\item Conformité
 \end{enumerate}

Voir ci-dessous quelques exemples concrets de mesures de sécurité:

Contrôles d’accès

\begin{itemize}
   \item Mise en place de mots de passe robustes et de politiques de gestion des mots de passe.
\item Utilisation de l’authentification à deux facteurs pour accéder aux systèmes sensibles.
\item Attribution de droits d’accès en fonction du principe du moindre privilège, pour limiter l’accès aux seules informations nécessaires à chaque utilisateur.
 \end{itemize}

Sécurité physique :

\begin{itemize}
   \item Installation de caméras de surveillance et de systèmes d’alarme dans les data-centers et autres zones sensibles.
\item Contrôle d’accès par badge ou biométrie pour entrer dans les locaux sensibles.
\item Protection contre les incendies et les inondations dans les centres de données (Datacenters).
 \end{itemize}

Gestion des incidents :

\begin{itemize}
   \item Mise en place d’un plan de réponse aux incidents pour détecter, analyser et répondre rapidement aux cyberattaques ou  incidents de sécurité.
\item Formation du personnel à la reconnaissance et à la signalisation des incidents de sécurité.
 \end{itemize}

Sécurité des réseaux :

\begin{itemize}
   \item Utilisation de pare-feu et de systèmes/sondes de détection/prévention d’intrusions (IDS/IPS).
\item Chiffrement des données transmises via des protocoles sécurisés comme TLS ou IPSEC.
\item Segmentation des réseaux pour limiter la propagation d’éventuelles attaques.
 \end{itemize}

Sauvegarde et continuité :

\item Réalisation régulière de sauvegardes des données critiques, stockées hors site ou dans le cloud.
\item Mise en place de plans de reprise d’activité (PRA) pour assurer la continuité en cas d’incident majeur.
 \end{itemize}

Sensibilisation et formation :

\item Organisation de sessions de formation pour sensibiliser les employés aux risques liés à la sécurité (phishing, ingénierie sociale, etc.).
\item Envoi de campagnes de sensibilisation régulières pour maintenir une vigilance constante.
 \end{itemize}


%Begin FRAME----------------------------
\mode<presentation>{\texframe
% contenu affiché sur Article et Beamer
%- - - - - - - - - - - - - - - - - - - - - - - - 
{ISO/EIC27002} % titre de la diapo
{quelles sont les mesures de sécurité?} % sous titre de la diapo
{
%begin slide- - - - - - - - - - - - - - - - - 
	\begin{itemize}
    \item Définition mesures de sécurité 
    \item Exemples
 \end{itemize}
%end slide- - - - - - - - - - - - - - - - - - - 
}}
%End FRAME------------------------------