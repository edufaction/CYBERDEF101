Les fonctionnalités de détection de point de terminaison et de réponse de Microsoft Defender - PACM permettent de détecter des attaques avancées qui sont en temps réel et exploitables. Les analystes de la sécurité peuvent hiérarchiser efficacement les alertes, obtenir une visibilité sur l’ensemble des violations et prendre des mesures pour remédier aux menaces.
En cas de détection d’une menace, des alertes sont créées dans le système pour qu’un analyste examine. Les alertes associées aux mêmes techniques d’attaque ou affectées au même agresseur sont agrégées dans une entité appelée incident. L’agrégation des alertes de cette manière permet aux analystes de rechercher et de répondre à des menaces collectivement.
Microsoft Defender ATP collecte en continu le comportement du terminal via de la télémétrie cyber. Cela inclut les informations de processus, les activités réseau, les composants optiques intégrés au noyau et au gestionnaire de mémoire, les activités de connexion utilisateur, les modifications de la base de Registre et du système de fichiers, etc.  Les fonctionnalités de réponse vous permettent d’apporter une correction rapide aux menaces en agissant sur les entités affectées.