%===========================================
% QUIZZ CYBEREDU ANSSI
%===========================================


% Q.........................................
\begin{multi}[multiple=true]{CyberEdu-X2.1}
	Donner 2 exemples de données électroniques sensibles pour un étudiant :
\item Adresse postale
\item* Nom et numéro de sécurité sociale
\item* Numéro de carte bancaire
\item Nom de famille
\end{multi}

% Q.........................................
\begin{multi}[multiple=true]{CyberEdu-X2.2}
	Donner 2 exemples de données électroniques sensibles pour une université/école :
\item Le nom et l'origine de l'université
\item Les noms des professeurs
\item* Les brevets déposés
\item* Les épreuves d'examens à venir (non encore passés)
\end{multi}

% Q.........................................
\begin{multi}[multiple=true]{CyberEdu-X2.3}
	Dans un réseau, qu'est-ce qu'on entend par une zone de confiance?
\item Le hotspot wifi offert aux visiteurs, exemple à la gare SNCF
\item* Le réseau interne (où sont hébergés les postes des utilisateurs et les serveurs)
\item Le réseau Internet
\item Une zone démilitarisée (DMZ)
\end{multi}

% Q.........................................
\begin{multi}[multiple=true]{CyberEdu-X2.4}
	Quand parle-t-on d'une authentification mutuelle entre deux entités?
\item Lorsque des deux entités sont administrées par la même personne
\item* Lorsque chacune des entités doit s'authentifier vis-à-vis de l'autre
\item Lorsque la communication entre les deux entités est chiffrée
\item Lorsque les deux entités sont situées sur le même réseau
\end{multi}

% Q.........................................
\begin{multi}[multiple=true]{CyberEdu-X2.5}
	Dans un réseau, l'usage du BYOD peut entrainer (choisir la (ou les) proposition(s) vraie(s)) :
\item Une restriction du périmètre à sécuriser
\item* La propagation de codes malveillants
\item* La fuite de données de l'entreprise
\item Une meilleure sécurité du SI
\end{multi}

% Q.........................................
\begin{multi}[multiple=true]{CyberEdu-X2.6}
	Quel est le principe célèbre en matière de gestion de flux sur un réseau?
\item* Tout ce qui n'est pas autorisé est interdit
\item Tout ce qui est autorisé n'est pas interdit
\item Tout ce qui est interdit est interdit
\end{multi}

% Q.........................................
\begin{multi}[multiple=true]{CyberEdu-X2.7}
	Un  pare-feu  peut être aussi bien matériel (appliance dédiée) que logiciel?
\item* Vrai
\item Faux
\end{multi}

% Q.........................................
\begin{multi}[multiple=true]{CyberEdu-X2.8}
	Entourer la (ou les) proposition(s) vraie(s) qui peut (ou peuvent) servir de mesure de sécurisation des accès distants à un réseau :
\item* Utiliser un serveur d'authentification centralisé comme TACACS+
\item Utiliser Internet
\item Utiliser un protocole sécurisé tel que telnet ou ftp
\item* Utiliser un VPN
\end{multi}

% Q.........................................
\begin{multi}[multiple=true]{CyberEdu-X2.9}
	Entourer la (ou les) bonne(s) mesure(s) de sécurisation de l'administration
\item Rendre les interfaces d'administration disponibles à tous depuis Internet
\item Tous les administrateurs doivent utiliser le même compte pour se connecter
\item* Utiliser un réseau dédié pour l'administration
\item* Authentifier mutuellement les postes des administrateurs et les serveurs à administrer.
\end{multi}

% Q.........................................
\begin{multi}[multiple=true]{CyberEdu-X2.10}
	Quelle est la technologie la plus appropriée pour sécuriser son accès Wifi:
\item WEP
\item WPA
\item WPS
\item* WPA2
\end{multi}

% Q.........................................
\begin{multi}[multiple=true]{CyberEdu-X2.11}
	Entourer la (ou les) proposition(s) vraie(s) lors de l'usage d'un hotspot Wifi?
\item* Il peut s'agir d'un faux point d'accès ;
\item* Les autres personnes connectées peuvent voir mes communications
\item Je suis protégé des personnes malveillantes
\item Je suis sur un réseau de confiance, je peux désactiver mon pare-feu.
\end{multi}

% Q.........................................
\begin{multi}[multiple=true]{CyberEdu-X2.12}
	Pourquoi vérifier l'intégrité d'un logiciel?
\item Pour m'assurer qu'il ne contient pas de virus
\item* Pour m'assurer que le logiciel que je télécharge n'a pas été corrompu
\item Pour m'assurer que le logiciel fonctionne bien comme promis
\item Pour m'assurer qu'il est gratuit
\end{multi}

% Q.........................................
\begin{multi}[multiple=true]{CyberEdu-X2.13}
	Laquelle (ou lesquelles) des expressions suivantes est (sont) vraie(s) pour un logiciel téléchargeable?
\item toujours gratuit
\item* Peut être  open source
\item* Peut contenir des logiciels espions
\item* Peut être un programme malveillant
\end{multi}

% Q.........................................
\begin{multi}[multiple=true]{CyberEdu-X2.14}
	Citer une bonne pratique de configuration de son antivirus
\item Avoir un antivirus d'un éditeur connu
\item Avoir un jour installé un antivirus
\item* Tenir son antivirus à jour (mise à jour des signatures et du moteur)
\item Interdire l'analyse antivirale à certains répertoires ou périphériques.
\end{multi}

% Q.........................................
\begin{multi}[multiple=true]{CyberEdu-X2.15}
	Sélectionner la (ou les) proposition(s) vraie(s) parmi les suivantes. Un antivirus:
\item peut détecter tous les virus et programmes malveillants, y compris ceux non découverts
\item protège de toutes les menaces
\item* ne peut détecter que les virus qui sont connus dans sa base de signatures
\item* doit être actif, et à jour pour être utile
\end{multi}

% Q.........................................
\begin{multi}[multiple=true]{CyberEdu-X2.16}
	Choisir un (ou des) symptôme(s) potentiel(s) d'infection par un code malveillant
\item* Mon antivirus est désactivé
\item* Mon ordinateur fonctionne plus lentement
\item  J'ai plusieurs pages Web  qui s'ouvrent toutes seules
\item Des fichiers ou des répertoires sont créés automatiquement sur mon poste
\end{multi}

% Q.........................................
\begin{multi}[multiple=true]{CyberEdu-X2.17}
	Les mises à jour logicielles servent à améliorer les logiciels et à corriger les failles de sécurité
\item* Vrai
\item Faux
\end{multi}

% Q.........................................
\begin{multi}[multiple=true]{CyberEdu-X2.18}
	Vous pouvez protéger la confidentialité vos données en :
\item* Les chiffrant
\item En calculant leur empreinte de manière à vérifier leur intégrité
\item En les envoyant vers des supports externes ou vers le Cloud
\item En  les publiant sur Internet
\end{multi}

% Q.........................................
\begin{multi}[multiple=true]{CyberEdu-X2.19}
	Sélectionner le (ou les) moyen(s) de durcissement d'une configuration
\item* Modifier les mots de passe par défaut
\item* Désinstaller les logiciels inutiles
\item Activer le mode  débogage USB  sur les téléphones
\item* Sécuriser le BIOS à l'aide d'un mot de passe
\end{multi}

% Q.........................................
\begin{multi}[multiple=true]{CyberEdu-X2.20}
	Sélectionner le (ou les) principes(s) à prendre en compte lors de l'attribution de privilèges utilisateurs
\item  Tout ce qui n'est pas interdit, est autorisé
\item*  Moindre privilège
\item*  Besoin d'en connaitre
\item  Droit administrateur pour tous
\end{multi}

% Q.........................................
\begin{multi}[multiple=true]{CyberEdu-X2.21}
	Entourer la (ou les) mauvaise(s) pratique(s) pour les mots de passe
\item* Je crée un mot de passe très long et très complexe, dont je ne me souviens pas
\item* Ma date de naissance me sert de mot de passe
\item* Je stocke mes mots de passe en clair dans un fichier texte
\item* Mon mot de passe doit avoir au plus 7 caractères
\end{multi}

% Q.........................................
\begin{multi}[multiple=true]{CyberEdu-X2.22}
	Entourer la (ou les) bonne(s) pratique(s) pour les mots de passe
\item J'enregistre mes mots de passe sur chaque navigateur Internet
\item* Je crée un mot de passe long et complexe dont je peux me souvenir * facilement
\item J'écris mon mot de passe sur un post-it que je cache sous mon clavier/PC
\item* J'utilise un porte-clés de mots de passe
\end{multi}

% Q.........................................
\begin{multi}[multiple=true]{CyberEdu-X2.23}
	Entourer la (ou les) bonne(s) pratique(s) de navigation sur Internet
\item Je suis victime de ransomware, je paye la rançon
\item* J'évite de communiquer avec des inconnus
\item J'accepte toutes les demandes sur les médias sociaux
\item Je donne mon mot de passe de messagerie à  l'administrateur   lorsqu'il me le demande.
\end{multi}

% Q.........................................
\begin{multi}[multiple=true]{CyberEdu-X2.24}
	Citer deux moyens de sécurisation physique des biens/équipements
\item Mettre les équipements sensibles dans une salle sans contrôle d'accès
\item* Attacher les équipements sensibles avec des càbles de sécurité
\item Nommer tous les équipements de la même façon
\item* Utiliser des filtres de confidentialité pour les écrans
\end{multi}

% Q.........................................
\begin{multi}[multiple=true]{CyberEdu-X2.25}
	Choisir l' (ou les) exemple(s) d'incidents de sécurité
\item* Le vol d'un équipement/terminal
\item La création d'un compte utilisateur pour un nouvel étudiant
\item* La présence d'un code malveillant sur un poste
\item* La divulgation sur un forum des noms, prénoms, et numéros de sécurité sociale des étudiants
\end{multi}

% Q.........................................
\begin{multi}[multiple=true]{CyberEdu-X2.26}
	Choisir la (ou les) bonne(s) réaction(s) face à un incident de sécurité :
\item Désactiver/désinstaller son antivirus
\item* Appliquer les règles/consignes reçues par exemple dans la charte informatique
\item* Chercher à identifier la cause de l'incident
\item Désactiver son pare-feu (personnel par exemple)
\end{multi}

% Q.........................................
\begin{multi}[multiple=true]{CyberEdu-X2.27}
	Sélectionner la (ou les) raison(s) pour laquelle (ou lesquelles) les audits de sécurité peuvent être effectués :
\item* Pour obtenir une certification ou un agrément
\item* Pour trouver des faiblesses et les corriger
\item* Pour évaluer le niveau de sécurité
\item Provoquer des incidents de sécurité
\end{multi}


