\subsection{L'analyse} 
\subsubsection{L'analyse statique}
Réalisée en sandbox, sur une machine virtuelle ou une machine dédiée avec des outils pré-installés comme InetSim,FakeNet ou Mozzle, elle débute par une analyse statique pour identifier les éléments et la composition du malware.
L'examen du code et des fonctions appelées permettent d'évaluer les capacités du botnet.
\subsubsection{L'analyse dynamique} 
Cette étape est relativement utile pour la compréhension de la menace car elle présente la machine infectée sous plusieurs états à l'aide de snapshots\footnote{copie des données/modifications apportées à un système}.
Ces captures instantanées situent l'avancement de l'infection lors de l'attaque.
Il est souvent nécessaire en présence d'algorithme chiffré d'utiliser cette méthode pour désobfusquer le code et comprendre la structure du malware.


