

\section{Construction de la Politique de sécurité du système d'information}
\subsection{Politique générale de sécurité}

%Begin FRAME----------------------------
\mode<presentation>{\texframe
% contenu affiché sur Article et Beamer
%- - - - - - - - - - - - - - - - - - - - - - - - 
{Construction de la Politique de sécurité du système d'information} % titre de la diapo
{Soustitre} % sous titre de la diapo
{
%begin slide- - - - - - - - - - - - - - - - - 
	Contenu du texte ce que l'on veut comme par exemple
 \begin{itemize}
    \item Définition d'une politique de sécurité
    \item Intégration de la politique de sécurité dans la gouvernance du SI
 \end{itemize}
%end slide- - - - - - - - - - - - - - - - - - - 
}}
%End FRAME------------------------------


%Begin FRAME----------------------------
\mode<presentation>{\texframe
% contenu affiché sur Article et Beamer
%- - - - - - - - - - - - - - - - - - - - - - - - 
{Concept de PSG} % titre de la diapo
{La politique générale de sécurité} % sous titre de la diapo
{
%begin slide- - - - - - - - - - - - - - - - - 
La politique de sécurité générale (PSG) est généralement basée sur l'architecture de l'ISO/EIC 27001, et donne le cadre général de conformité pour les projets, les organisations sous-jacentes (divisions, filiales), les produits ainsi que l'organisation des responsabilités.

Cette PSG est déclinée en différentes politiques de sécurité par secteurs (souvent structurées par les chapitres de l'ISO/EIC 27001)

\begin{itemize}
	\item IAM
	\item Filtrage et sécurité périmétrique
	\item Détection et rémédiation
	\item continuité d'activité
\end{itemize}

%end slide- - - - - - - - - - - - - - - - - - - 
}}
%End FRAME------------------------------

\subsection{Comment passe t-on de l’analyse de risques à une PSSI ?}

Inputs AR --> identification des biens à protéger \& à intégrer dans la politique de sécurité

%Begin FRAME----------------------------
\mode<presentation>{\texframe
% contenu affiché sur Article et Beamer
%- - - - - - - - - - - - - - - - - - - - - - - - 
{Approche par les risques} % titre de la diapo
{Contenu de la politique de sécurité} % sous titre de la diapo
{
%begin slide- - - - - - - - - - - - - - - - - 

\begin{itemize}
	\item Identification des biens essentiels à protéger
	\item Intégration dans la politique de sécurité
\end{itemize}

%end slide- - - - - - - - - - - - - - - - - - - 
}}
%End FRAME------------------------------


\subsection{Exemples de chapitres de PSSI}
\begin{itemize}
    \item Gestion de l’administration des équipements
    \begin{itemize}
        \item protocoles d’accès autorisés
        Seuls les protocoles d'accès sécurisés type SSHv3 ou via une solution HTTPs avec certificats générés par l'IGC (Infrastructure de gestion de clés) de l'entreprise
        L'accès aux machines via le protocole RDP (Remote Desktop Protocol de Microsoft) est soumis à conditions (accès Interne uniquement et via un réseau d'administration dédié)
        Sont bannis les protocoles de type TELNET
        \item protocole et méthode de supervision autorisés
        Seuls sont autorisés les protocoles de supervision de type SNMPv3 respectant les configurations décrites dans la documentation dédiée (utilisation de certificats par exemple).
    \end{itemize}
    \item Politique de mot de passe
    \item Gestion des fournisseurs
\end{itemize}


%Begin FRAME----------------------------
\mode<presentation>{\texframe
% contenu affiché sur Article et Beamer
%- - - - - - - - - - - - - - - - - - - - - - - - 
{Exemples de chapitres de PSSI} % titre de la diapo
{Contenu de la politique de sécurité} % sous titre de la diapo
{
%begin slide- - - - - - - - - - - - - - - - - 

\begin{itemize}
	 \item Gestion de l’administration des équipements
    \begin{itemize}
        \item protocoles d’accès autorisés : SSHv3, HTTPs, interdiction de TELNET
        \item protocoles de supervision autorisés : SNMPv3
    \end{itemize}
    \item Politique de mot de passe
    \item Gestion des fournisseurs
\end{itemize}

%end slide- - - - - - - - - - - - - - - - - - - 
}}
%End FRAME------------------------------