%===========================================
% QUIZZ CYBEREDU ANSSI
%===========================================

  
% Q..................................................................................
\begin{multi}[multiple=true]{CyberEdu-X4.1}
De quelle famille de normes internationales, une organisation peut-elle s'inspirer pour intégrer la sécurité en son sein ?
\item*27000
\item 9000
\item 14000
\end{multi}
% Q..................................................................................
\begin{multi}[multiple=true]{CyberEdu-X4.2}
	Citer un exemple représentatif d'une organisation devant avoir recours à une certification de sécurité. 
\item* pour respecter une réglementation
\item* pour s'améliorer
\item pour se protéger contre des menaces spécifiques
\end{multi}

% Q..................................................................................
\begin{multi}[multiple=true]{CyberEdu-X4.3}
Très souvent dans les entreprises, les informations ont toutes le même niveau de confidentialité  toutes non confidentielles  
\item  Vrai 
\item* Faux.
\end{multi}

% Q..................................................................................
\begin{multi}[multiple=true]{CyberEdu-X4.4}
Pour une bonne intégration de la sécurité dans l'organisation, le personnel doit être sensibilisé à la sécurité conformément à leurs fonctions 
\item* Vrai 
\item Faux.
\end{multi}


% Q..................................................................................
\begin{multi}[multiple=true]{CyberEdu-X4.5}
Citer les procédures de gestion des départs du personnel indispensables impactant la sécurité 
\item* Retrait des accès 
\item* Restitution du matériel fourni (badge, ordinateur, …)
\item Solde de tout compte
\end{multi}


% Q..................................................................................
\begin{multi}[multiple=true]{CyberEdu-X4.6}
La sécurité c'est comme la cerise sur le g\^ateau  elle doit être prise en compte à la fin d'un projet
\item	Vrai 
\item*	Faux.
\end{multi}
% Q..................................................................................
\begin{multi}[multiple=true]{CyberEdu-X4.7}
Le but d'une analyse de risques est de déterminer pour un périmètre donné (projet par exemple), les risques qui peuvent porter sur les biens non sensibles 
\item	Vrai 
\item*	Faux
\end{multi}

% Q..................................................................................
\begin{multi}[multiple=true]{CyberEdu-X4.8}
Sélectionner la phase qui résume le plus la démarche d'analyse de risques
\item Identifier les agents menacants et les neutraliser 
\item Identifier les acteurs importants du projet 
\item Inventorier les biens
\item* Déterminer les risques et les traiter
\end{multi}

\item Identifier les agents menaçants et les neutraliser 
\item Identifier les acteurs importants du projet 
\item Inventorier les biens
\item* Déterminer les risques et les traiter

% Q..................................................................................
\begin{multi}[multiple=true]{CyberEdu-X4.9}
Est-ce que tous les risques issus d'une analyse de risques doivent-ils être traités par une mesure de réduction des risques ?
\item* Non seuls ceux dont le niveau de criticité est supérieur au seuil de tolérance
\item Oui tous les risques doivent être couverts
\end{multi}
% Q..................................................................................
\begin{multi}[multiple=true]{CyberEdu-X4.10}
Choisir la (les) proposition(s) correcte(s). Au cours de l'analyse de risques, les contre-mesures sont des mesures de réduction de risque qui peuvent être 
\item*	techniques et organisationnelles 
\item	techniques uniquement 
\item	organisationnelles uniquement 
\item* déclinées des objectifs de sécurité définis.
\end{multi}
% Q..................................................................................
\begin{multi}[multiple=true]{CyberEdu-X4.11}
Choisir la (les) proposition(s) correcte(s) 
\item*	Il est plus facile d'attaquer un système que de le rendre invulnérable 
\item	Il est facile de créer un système sans aucune vulnérabilité 
\item	Pour défendre un système, il suffit de le protéger de manière périmétrique 
\item*	La défense en profondeur peut être appliquée pour protéger un système.
\end{multi}


% Q..................................................................................
\begin{multi}[multiple=true]{CyberEdu-X4.12}
Choisir la (les) proposition(s) correcte(s). La défense en profondeur est un principe d'origine militaire qui consiste à avoir plusieurs lignes de défense constituant des barrières autonomes pour défense un système 
\item*	Vrai
\item	Faux.
\end{multi}
% Q..................................................................................
\begin{multi}[multiple=true]{CyberEdu-X4.13}
Choisir la (les) proposition(s) correcte(s). Pour une organisation, l'usage des services du Cloud doit prendre en compte 
\item*	les exigences légales relatives aux données hébergées
\item*	les mécanismes de sécurité tels que le chiffrement des données stockées proposés par le fournisseur du service
\item* le devenir des données hébergées à la fin du contrat
\item*	les certifications dont dispose le fournisseur du service Cloud.
\end{multi}
% Q..................................................................................
\begin{multi}[multiple=true]{CyberEdu-X4.14}
L'une des difficultés de l'intégration de la sécurité dans une organisation, est celle des choix éclairés en matière de produits de confiance 
\item*	Vrai 
\item	Faux.
\end{multi}
% Q..................................................................................
\begin{multi}[multiple=true]{CyberEdu-X4.15}
Dans une organisation, la sécurité est critique. Elle doit être imposée à tous sans consultation
\item	Vrai 
\item*	Faux.
\end{multi}
% Q..................................................................................
\begin{multi}[multiple=true]{CyberEdu-X4.16}
Le  Shadow IT  ou  Shadow Cloud  est une pratique qui consiste pour les utilisateurs à souscrire directement aux services Cloud sans la consultation et aval de leur DSI et en souvent en dépit de la politique de sécurité 
\item*	Vrai 
\item	Faux
\end{multi}
% Q..................................................................................
\begin{multi}[multiple=true]{CyberEdu-X4.17}
Choisir la (les) proposition(s) correcte(s). Le Big Data peut constituer une opportunité en sécurité car il peut permettre de 
\item	d'envoyer les données sensibles de l'organisation en clair vers le Cloud 
\item*	d'utiliser une capacité de traitement de manière à effectuer l'analyse d'évènements de sécurité en temps réel 
\item*	de corréler les traces provenant de différents équipements réseau pour détecter des menaces persistantes avancées (APT) 
\item*	de surveiller le trafic réseau en temps réel pour détecter les botnets.
\end{multi}
% Q..................................................................................
\begin{multi}[multiple=true]{CyberEdu-X4.18}
Citer un métier, avec une principale activité associée, sollicité dans chaque phase d'un cycle d'un projet 
\item*	Expression de besoin  Chef de projet/consultant MOA pour définir les exigences de sécurité issues d'une analyse de risque
\item*	Développement  Chef de projet/consultant MOE, architecte, concepteur/développeur  pour spécifier, concevoir/développer les mesures de sécurité
\item*	Validation  auditeur technique ou organisationnel pour contrôler la conformité et l'efficacité des mesures de sécurité
\item*	Exploitation  technicien ou administrateur pour maintenir en condition de sécurité (mise à jour des patchs et des bases de signature), analyste pour faire la veille sur les vulnérabilités ou détecter des incidents de sécurité
\end{multi}

% Q..................................................................................
\begin{multi}[multiple=true]{CyberEdu-X4.19}	
Les compétences recherchées en cybersécurité sont uniquement techniques 
\item Vrai 
\item* Faux.
\end{multi}

% Q..................................................................................
\begin{multi}[multiple=true]{CyberEdu-X4.20}
La cybersécurité est un secteur ayant peu de perspective d'embauche 
\item	Vrai 
\item*	Faux.
\end{multi}

