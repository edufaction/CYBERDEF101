\section{Introduction Architecture}
Nous allons dans ce chapitre aborder les sujets d'architectures sécurisées afin d'appréhender la complexité de l'environnement technique du client.
L'analyse des menaces et des risques sur un périmètre défini permet d'identifier les fonctions de sécurité qui réduiront les impacts et qui doivent être inscrites dans la politique de sécurité et mises en oeuvre au sein de l'architecture conçue.
L'ensemble de ces fonctions ansi que les composants de l'architecture doivent être sécurisés à l'aide de contrôle, le filtrage des accès, cloisonnement et l'utilisation d'algorithmes cryptographiques.
Nous identifierons ensuite les composants services, terminaux et réseaux du Système d'information afin de les intégrer dans la politique de sécurité protective.
Puis nous découvrirons l'évolution des architectures de sécurité périmétrique depuis le modèle du château fort jusqu'au ZeroTrust et les services assurant les fonctions de sécurité comme le pare feu, les proxy, les bastions et les solutions utilisées sur les architectures CLOUD.
Enfin, nous étudierons dans la dernière partie de ce chapitre la sécurité du poste de travail avec les solutions déployées pour assurer la sécurité des données de l'entreprise et de l'utilisateur.

%CONSTRUIRE vs Politique Architecture de sécurité vs Sécurité des architectures (Crypto, Sécurité Perimétrique). - PERIMETRE, CLOISENEMENT, ZERO TRUST, FILTRAGE, %DETECTION, BASTION  … CHÂTEAU FORT (Mur) / AEROPORT (Portique) - Contrôle d'accès à tout objets

%Appréhender la complexité de l'environnement technique du client (Protection, Défense, Résilience) et identifier les composants services, terminaux et réseaux du %SI afin de les intégrer dans la politique de sécurité protective.
%Découvrir les modèles d'architecture de sécurité périmétrique afin d'assurer le filtrage nécessaire et la détection des menaces sur le SI.


%\picframe{../Latex/Sources/YLV/Pictures/img-crypto1}{Label sous titre de l'image}{0.8}{labelRef}

%Begin FRAME----------------------------
\mode<presentation>{\texframe
% contenu affiché sur Article et Beamer
%- - - - - - - - - - - - - - - - - - - - - - - - 
{Architectures, Composants et Sécurité} % titre de la diapo
{Introduction} % sous titre de la diapo
{
%begin slide- - - - - - - - - - - - - - - - - 
Construire une architecture sécurisée : politique, sécurité des solutions, idendification des composants
 \begin{enumerate}
     \item De l’analyse de risques aux fonctions de sécurité
     \item Architecture et composants de système d’information
     \item Modèles de sécurité et technologies de sécurité protectrices
     \item Sécurité Endpoints
 \end{enumerate}
    
%end slide- - - - - - - - - - - - - - - - - - - 
}}
%End FRAME------------------------------
