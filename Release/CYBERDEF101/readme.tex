\begin{quote}
Cyberdef101 (c) 2022-2025 - Production Orange Security School
\end{quote}

\section{Element de Cybersécurité
d'entreprise}\label{element-de-cybersuxe9curituxe9-dentreprise}

\section{Module 1 - Contexte et
Environnement}\label{module-1---contexte-et-environnement}

\begin{quote}
\textbf{S'exprimer dans le même langage que les acteurs du domaine} -
Histoire de la cybersécurité: cryptologie, sécurité réseau, premiers
virus - Cybermenaces et leurs auteurs: cybercriminalité, cyberguerre -
Types et vecteurs d'attaques - Darkweb et Darkbusiness - Organisation de
la sécurité en entreprise: rôles du directeur sécurité et du RSSI
\textbf{\emph{Compétences acquises}}: compréhension de l'écosystème
cyber, connaissance des menaces actuelles, familiarisation avec le
jargon du domaine
\end{quote}

\section{Module 2 - Risques
numériques}\label{module-2---risques-numuxe9riques}

\begin{quote}
\textbf{Déterminer les évènements redoutés de l'entreprise}
\end{quote}

\begin{itemize}
\tightlist
\item
  Définitions et référentiels de gestion des risques
\item
  Les risques liés au numérique (IOT, Cloud, IA\ldots)
\item
  Concepts clés: impacts, biens primordiaux, menaces, scénarios
\item
  Cadres normatifs: ISO 27005, intégration dans SMSI
\item
  Des risques aux objectifs de sécurité: risques résiduels, PDCA
\item
  Veille sécurité: CERT (Vulnérabilités, menaces), DarkWeb, IES, OSINT
  \textgreater{}\textbf{\emph{Compétences acquises}}: capacité à
  déterminer les événements redoutés, compréhension des méthodologies
  d'analyse de risques \# Module 3 - Architectures et technologies
  \textgreater{}\textbf{Appréhender la réalité technologique~de
  l'environnement cyber d'une entreprise}
\end{itemize}

\subsection{Architectures (Part A)}\label{architectures-part-a}

\subsection{Architectures (Part B)}\label{architectures-part-b}

\subsection{Architectures (Part C)}\label{architectures-part-c}

\begin{longtable}[]{@{}
  >{\raggedright\arraybackslash}p{(\linewidth - 2\tabcolsep) * \real{0.5000}}
  >{\raggedright\arraybackslash}p{(\linewidth - 2\tabcolsep) * \real{0.5000}}@{}}
\toprule\noalign{}
\begin{minipage}[b]{\linewidth}\raggedright
PRZ.PDF
\end{minipage} & \begin{minipage}[b]{\linewidth}\raggedright
DOC.PDF
\end{minipage} \\
\midrule\noalign{}
\endhead
\bottomrule\noalign{}
\endlastfoot
\pandocbounded{\includegraphics[keepaspectratio]{Teacher/Lessons/Slides/L-Orange-Cyberdef101-M3c-Architectures.prz.pdf.dir/PNG/Pic-1.png}}
& \href{file.pdf}{Notes de cours} \\
\end{longtable}

\section{Module 4 - Management de la
sécurité}\label{module-4---management-de-la-suxe9curituxe9}

\begin{quote}
\textbf{Appréhender les enjeux de conformité et de pilotage d'une
entreprise}
\end{quote}

\section{Module 5 - La sécurité
opérationnelle}\label{module-5---la-suxe9curituxe9-opuxe9rationnelle}

\begin{quote}
\textbf{S'avoir s'insérer dans l'organisation de la Cyberdéfense de
l'entreprise}
\end{quote}

\section{Module 6 - La sécurité dans les
projets}\label{module-6---la-suxe9curituxe9-dans-les-projets}

\begin{quote}
\textbf{Différentier la sécurité dans les projets et la sécurité de
l'entreprise}
\end{quote}

\section{Module 7 - Le marché de la
sécurité}\label{module-7---le-marchuxe9-de-la-suxe9curituxe9}

\begin{quote}
\textbf{Se positionner dans l'écosystème cyber du marché}
\end{quote}
