%-------------------------------------------------------------
%               FR CYBERDEF SECOPS COURSE
%              $Chapitre : Threat Management
%                         $theme : Caracterization
%              $File : chap-ThreatMan-caract.tex
%                             2020 eduf@ction
%-------------------------------------------------------------
\uchap{chap-ThreatMan-caract.tex}
%-------------------------------------------------------------

\subsection{Caractérisation de la menace}

Nous avons évoqué les mécanismes de surveillance et de détection constitués en particulier d'outils comme le SIEM, ou l'EDR.
Il est toutefois important de disposer de compétences et de processus structurés pour caractériser la menace.
Cette caractérisation permet de lever les doutes, de caractériser les éléments d'une attaque dont les impacts mais surtout les raisons et les intentions des attaquant. Ces éléments sont évidement primordiaux pour définir une réponse adaptée.

\subsection{De l'usage d'un CSIRT}

Nous verrons l'usage d'un CSIRT dans le chapitre sur la réponse à incident toutefois nous pouvons donner ici les fonctions ou compétences clefs pour effectuer cette caractérisation :

\begin{itemize}
  \item Rétro-conception : afin d'analyser un code malveillant;
  \item Liaisons police/gendarmerie : afin de lancer au plus tôt les investigations via les services de l'état;
  \item Base \g{enrichie} d'IP \& Url malveillantes : afin de caractériser non seulement la provenance, mais aussi l'intentionalité de l'attaque;
  \item Etudes de cyber-intelligence sur le secteur d'activité de l'entreprise : afin de corréler une agression avec des risques dans un environnement politique ou économique particulier.
\end{itemize}
