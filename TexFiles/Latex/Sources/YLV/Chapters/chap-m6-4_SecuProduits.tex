\section{Sécurité des produits}

%La conformité technique aux référentiels/normes est un élément fondamental pour s'assurer que les produits logiciels et hardware sont conformes aux normes et réglementations du secteur.

La conformité technique aux référentiels et normes constitue un pilier essentiel pour garantir que les produits logiciels et matériels répondent aux exigences réglementaires et aux standards de sécurité de l'industrie. Cette démarche de conformité permet non seulement de valider la robustesse des solutions déployées mais aussi d'assurer leur interopérabilité et leur niveau de sécurité.

Ce chapitre présentera la conformité aux implementations normatives les plus connues ainsi que les dispositifs de certifications existants.

\subsection{Conformité aux implémentations normatives}
Pour aider à protéger les informations sensibles et les infrastructures critiques, divers référentiels ont été développés pour définir l'état de l'art en matière de sécurité informatique. Ces référentiels fournissent des cadres, des normes et des meilleures pratiques pour guider les organisations dans la mise en place de mesures de sécurité efficaces. Ce chapitre présente une liste ci-après de ces référentiels. Cette liste n'est évidemment pas exhaustive et est évolutive.

Il est à noter que la certification d'un périmètre ou système est aussi un atout concurrentiel majeur pour une entreprise. Une certification comme l'ISO/EIC 27001 témoigne de l'engagement de l'entreprise envers la protection des données et la gestion des risques, renforçant ainsi la confiance des clients et des partenaires. Par exemple, une entreprise de services financiers certifiée ISO 27001 peut se démarquer sur le marché en rassurant ses clients sur la sécurité de leurs informations sensibles, ce qui peut conduire à une fidélisation accrue et à l'acquisition de nouveaux clients, tout en minimisant les risques de violations de données.

\subsubsection{ISO/IEC 27001}
L'ISO/IEC 27001 est une norme internationale pour les systèmes de management de la sécurité de l'information (SMSI). Elle spécifie les exigences pour établir, mettre en œuvre, maintenir et améliorer continuellement un SMSI. C'est une norme qui sert de "livre de chevet" pour tous les dirigeants de la sécurité, RSSI ou directeur de la sécurité. Elle donne les bonnes pratiques de gouvernance ainsi que les mesures à appliquer pour contrôler la mise en oeuvre. Cette norme est certifiante pour les SMSI mais aussi pour les utilisateurs implémenteurs ou auditeurs.
	
\subsubsection{NIST Cybersecurity Framework (CSF)}
Le NIST Cybersecurity Framework (CSF) est un cadre développé par le National Institute of Standards and Technology (NIST) pour améliorer la gestion des risques de cybersécurité. Il se compose de cinq fonctions principales : Identifier, Protéger, Détecter, Répondre et Récupérer. De très nombreux documents ont été édités par le NIST et servent aux organisations et responsables.
	
\subsubsection{CIS Controls}
Les CIS Controls sont un ensemble de 20 contrôles de sécurité critiques publiés par le Center for Internet Security (CIS). Ils sont conçus pour aider les organisations à se défendre contre les cybermenaces les plus courantes. Cf. chapitre 6.2 pour plus d'information sur l'application et l'utilisation de ces controles.
	
\subsubsection{COBIT}
Ensuite, le COBIT développé par ISACA est un Framework de gouvernance et de management des technologies de l'information (IT). L'ISACA (Information Systems Audit and Control Association) est une organisation mondiale à but non lucratif qui se concentre sur la gouvernance, la sécurité, la gestion des risques et l'audit des systèmes d'information. Elle fournit des certifications, des ressources et des normes pour les professionnels en technologie de l'information. Le COBIT fournit des outils pour aligner les objectifs de l'entreprise avec les objectifs de l'IT.
	
\subsubsection{PCI DSS}
La norme PCI DSS est une norme de sécurité des données pour l'industrie des cartes de paiement. Elle vise à protéger les informations des titulaires de carte et à réduire la fraude liée aux cartes de paiement.
	
\subsubsection{GDPR}
Le Règlement général sur la protection des données de l'Union européenne (ou GDPR en anglais pour General Data Protection Regulation) impose des obligations strictes aux organisations qui collectent ou traitent des données personnelles de résidents de l'UE.
	
\subsubsection{HIPAA}
l'HIPAA (Health Insurance Portability and Accountability Act) est une loi américaine sur la portabilité et la responsabilité en matière d'assurance maladie. Elle établit des normes pour la protection des informations de santé des patients.
	
\subsubsection{SOC 2}
La norme Service Organization Controls 2 (SOC 2) est une norme de conformité pour les organisations de services. Elle se concentre sur les contrôles relatifs à la sécurité, la disponibilité, l'intégrité du traitement, la confidentialité et la vie privée.
	
\subsubsection{OWASP Top Ten}
Le top 10 OWASP est liste des dix principales vulnérabilités de sécurité des applications web, publiée par l'Open Web Application Security Project (OWASP). Elle sert de guide pour les développeurs et les professionnels de la sécurité. Ce guide est très précieux pour les développeurs et les professionnels de la sécurité. D'ailleurs, il existe d'autres types de Top10 pour guider la sécurisation des mobiles ou des modèles d'IA par exemple.
	
\subsubsection{ITIL}
L'Information Technology Infrastructure Library (ITIL) est une bibliothèque pour l'infrastructure des technologies de l'information. Elle fournit un ensemble de processus des meilleures pratiques pour la gestion des services informatiques, y compris la gestion de la sécurité de l'information. Pour information, la version actuelle en 2025 est la version 4.
	
\subsubsection{SANS Top 20}
Le top 20 publié par SANS Institute liste les 20 contrôles de sécurité critiques conçus pour aider les organisations à se protéger contre les cybermenaces les plus courantes.

\subsubsection{CSA CCM}
Le  Cloud Controls Matrix (CCM) est un cadre de contrôle de la cybersécurité pour le cloud, publiée par la Cloud Security Alliance (CSA). Elle fournit des recommandations pour sécuriser les environnements de cloud computing.
	
\subsubsection{ISO/IEC 27017}
La norme ISO/IEC 27017 donne un code de pratique pour les contrôles de sécurité de l'information basés sur ISO/IEC 27002 pour les services cloud. Elle fournit des lignes directrices spécifiques pour les fournisseurs de services cloud.
	
\subsubsection{ISO/IEC 27018}
La norme ISO/IEC 27018 donne un code de pratique pour la protection des informations personnelles (PII) dans les services cloud publics. Elle se concentre sur la protection des données personnelles dans les environnements de cloud computing.
	
\subsubsection{ENISA}
L'Agence de l'Union européenne pour la cybersécurité publie des lignes directrices et des recommandations pour améliorer la cybersécurité en Europe. Elle fournit des ressources pour aider les organisations à se protéger contre les cybermenaces. Par exemple, on peut citer le texte sur la sécurité de la 5G.
	
\subsection{La confiance certifiée}
Dans le domaine de la cybersécurité on peut faire certifier des produits logiciels ou matériels avec la Certification de Sécurité de Premier Niveau (CSPN) ou les critères communs. En France cela passe par l'ANSSI.

Quel est l'intérêt de ce type de certification ?

Pour le vendeur :
\begin{itemize}
	\item confiance accrue : la certification CSPN renforce la crédibilité du produit auprès des clients potentiels;
	\item avantage concurrentiel : se démarquer sur le marché en offrant des produits reconnus pour leur sécurité;
	\item accès à de nouveaux marchés : facilite l'entrée sur des marchés sensibles où la sécurité est primordiale.
\end{itemize}

Pour l'acheteur :
\begin{itemize}
	\item garantie de sécurité : assurance que le produit a été évalué et répond à des normes de sécurité élevées;
	\item réduction des risques : diminution des vulnérabilités potentielles dans l'infrastructure de l'entreprise;
	\item conformité réglementaire : aide à respecter les exigences légales et réglementaires en matière de sécurité des systèmes d'information.
\end{itemize}

\subsubsection{Certification de Sécurité de Premier Niveau (CSPN)}
La CSPN mise en place par l’ANSSI en 2008 consiste en des tests en « boîte noire » effectués en temps et délais contraints. La CSPN est une alternative aux évaluations Critères Communs, dont le coût et la durée peuvent être un obstacle, et lorsque le niveau de confiance visé est moins élevé. Cette certification s’appuie sur des critères, une méthodologie et un processus élaborés par l’ANSSI publiés sur leur site.
(source ANSSI)

Plusieurs produits dont les fonctionnalités ont été décrites dans le chapitre 3 sont certifiés CSPN, les EDR Harfanglab ou la suite logicielle de sécurisation du poste de travail de Stormshield\footnote{liste des produits certifiés ANSSI CSPN: https://cyber.gouv.fr/produits-certifies?sort_bef_combine=field_date_de_certification_value_DESC&type_1\%5Bproduit_certifie_cspn\%5D=produit_certifie_cspn}.


\subsubsection{critères communs}
La certification dite tierce partie est la certification de plus haut niveau, qui permet à un client de s’assurer par l’intervention d’un professionnel indépendant, compétent et contrôlé, appelé organisme certificateur, de la conformité d’un produit à un cahier des charges ou à une spécification technique. 

La certification tierce partie apporte au client la confirmation indépendante et impartiale qu’un produit répond à un cahier des charges ou à des spécifications techniques publiées. Ces spécifications techniques peuvent être élaborées dans un cadre normatif ou non.
(source ANSSI)
Les produits utilisés pour chiffrer des documents, des emails voire des postes de travail comme Zed! et ZoneCentral de PRIM'X sont par exemples certifiés critères communs\footnote{liste des produits certifiés ANSSI CC: https://cyber.gouv.fr/produits-certifies?sort_bef_combine=field_date_de_certification_value_DESC&type_1\%5Bproduit_certifie_cc\%5D=produit_certifie_cc&field_categorie_target_id\%5B536\%5D=536&field_categorie_target_id\%5B534\%5D=534}

%AJOUTER EXEMPLES

%Begin FRAME----------------------------
\mode<presentation>{\texframe
% contenu affiché sur Article et Beamer
%- - - - - - - - - - - - - - - - - - - - - - - - 
{Sécurité des produits} % titre de la diapo
{Conformité aux implémentations normatives} % sous titre de la diapo
{
%begin slide- - - - - - - - - - - - - - - - - 
La conformité technique aux référentiels et normes permet de garantir que les produits logiciels et matériels répondent aux exigences réglementaires et aux standards de sécurité de l'industrie

But?
\begin{itemize}
	\item valider la robustesse des solutions déployées 
	\item assurer leur interopérabilité et leur niveau de sécurité
	\item mais aussi \dots
	\item ... un atout concurrentiel!
\end{itemize}
%end slide- - - - - - - - - - - - - - - - - - - 
}}
%End FRAME------------------------------

%Begin FRAME----------------------------
\mode<presentation>{\texframe
% contenu affiché sur Article et Beamer
%- - - - - - - - - - - - - - - - - - - - - - - - 
{Sécurité des produits : Liste référentiels et normes} % titre de la diapo
{Liste 1} % sous titre de la diapo
{
%begin slide- - - - - - - - - - - - - - - - - 
\begin{itemize}
	\item ISO/IEC 27001
	\item NIST Cybersecurity Framework (CSF)
	\item CIS Controls
	\item COBIT
	\item PCI DSS
	\item GDPR
	\item HIPAA
	\item SOC 2
	\item OWASP Top Ten
	\item ITIL
	\item SANS Top 20
	\item CSA CCM
	\item ISO/IEC 27017
	\item ISO/IEC 27018
	\item ENISA
\end{itemize}
%end slide- - - - - - - - - - - - - - - - - - - 
}}
%End FRAME------------------------------

%Begin FRAME----------------------------
\mode<presentation>{\texframe
% contenu affiché sur Article et Beamer
%- - - - - - - - - - - - - - - - - - - - - - - - 
{Sécurité des produits : La confiance certifiée} % titre de la diapo
{La confiance certifiée} % sous titre de la diapo
{
%begin slide- - - - - - - - - - - - - - - - - 

Quel est l'intérêt des certifications CSPN et Critères Communs ?

Pour le vendeur :
\begin{itemize}
	\item confiance accrue : la certification CSPN renforce la crédibilité du produit auprès des clients potentiels
	\item avantage concurrentiel : se démarquer sur le marché en offrant des produits reconnus pour leur sécurité
	\item accès à de nouveaux marchés : facilite l'entrée sur des marchés sensibles où la sécurité est primordiale
\end{itemize}

Pour l'acheteur :
\begin{itemize}
	\item garantie de sécurité : assurance que le produit a été évalué et répond à des normes de sécurité élevées
	\item réduction des risques : diminution des vulnérabilités potentielles dans l'infrastructure de l'entreprise
	\item conformité réglementaire : aide à respecter les exigences légales et réglementaires en matière de sécurité des systèmes d'information
\end{itemize}
%end slide- - - - - - - - - - - - - - - - - - - 
}}
%End FRAME------------------------------

%Begin FRAME----------------------------
\mode<presentation>{\texframe
% contenu affiché sur Article et Beamer
%- - - - - - - - - - - - - - - - - - - - - - - - 
{Sécurité des produits : CSPN et critères communs} % titre de la diapo
{La confiance certifiée} % sous titre de la diapo
{
%begin slide- - - - - - - - - - - - - - - - - 

\begin{itemize}
	\item Certification de Sécurité de Premier Niveau (CSPN) : tests en « boîte noire »
	
	--> Exemples : Harfanglab EDR et Stormshield security (poste de travail)
	\item critères communs : certification qui permet à un client de s’assurer par l’intervention organisme certificateur, de la conformité d’un produit à un cahier des charges ou à une spécification technique
	
	--> Exemples : Les produits Zed! et ZoneCentral de PRIM'X
\end{itemize}
%end slide- - - - - - - - - - - - - - - - - - - 
}}
%End FRAME------------------------------

%Begin FRAME----------------------------
\mode<all>{\texframe
% contenu affiché sur Article et Beamer
%- - - - - - - - - - - - - - - - - - - - - - - - 
{Points à retenir} % titre de la diapo
{} % sous titre de la diapo
{
%begin slide- - - - - - - - - - - - - - - - - 
\begin{itemize}
    \item Conformité == répondre aux exigences réglementaires et aux standards de sécurité de l'industrie
    
    --> \#ISO/IEC 27001 \#NIST \#CIS Controls \#COBIT \#PCI DSS \#GDPR
	--> \#HIPAA \#SOC 2 \#OWASP Top Ten \#ITIL \#SANS Top20 \#CSA CCM
	--> \#ISO/IEC 27017 \#ISO/IEC 27018 \#ENISA
    
    \item La confiance certifié
    
     --> \#CSPN \#Critères Communs
\end{itemize}
%end slide- - - - - - - - - - - - - - - - - - - 
}}
%End FRAME------------------------------