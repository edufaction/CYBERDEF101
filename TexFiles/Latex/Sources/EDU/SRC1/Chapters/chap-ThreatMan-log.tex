%-------------------------------------------------------------
%               FR CYBERDEF SECOPS COURSE
%              $Chapitre : Threat Management
%                   $theme : Threat Detection
%               $File :  chap-ThreatMan-log.tex
%                             2020 eduf@ction
%-------------------------------------------------------------
\uchap{chap-ThreatMan-log.tex}
%-------------------------------------------------------------

\subsection{Log Management }

\subsubsection{Traces, journaux, logs}

Dans le domaine informatique et télécom, le terme log (ou ses synonymes traces, journaux) est généralement un fichier, une base de données ou tout autre moyen de stocker des informations.  On parle  donc de stockage d'un historique d'événements qu'un logiciel ou un système souhaite \g{tracer}. 
Ce mot qui est le diminutif  de \textit{logging}, est traduit en français par \g{journal}. Le log est donc un journal horodaté, qui stocke temporellement les différents événements qui se sont produits sur un  logiciel, un ordinateur, un serveur, etc. Il permet ainsi d'analyser avec une fréquence programmée (heure par heure,  minute par minute, etc) l'activité  d'un processus technique.  

\mode<all>{\picframe{../Latex/Sources/EDU/SRC1/Pictures/img-log}{Sources de log}{0.8}{lbllog}}

Vous trouverez des éléments très intéressants  dans le 
\ulink {https://www.microfocus.com/media/white-paper/the_complete_guide_to_log_and_event_management_wp_fr.pdf}{Guide LOG MANAGEMENT} édité par NetIQ.

Dans notre cas d'usage, les log sont les éléments techniques  bruts d'un capteur d'événements. L'objectif est d'assurer que l'ensemble des journaux d'évènements contient suffisamment d'informations pour assurer des corrélations permettant de  constituer des signatures d'attaques. 

\begin{notebox}{Sources choisies des logs }
Pour détecter des attaques en temps réel, il faut disposer des informations caractéristiques de ces attaques captées dans le système d'information. Cela nécessite donc de sélectionner les bonnes sources (équipements réseaux, ou informatiques), les bons journaux, les bonnes traces dans ces journaux.  Ces choix sont donc primordiaux. Pour faire ces choix, il est nécessaire de connaître la capacité des équipements et des systèmes logiciels de générer ces traces. Au coeur de ces évènements il sera alors possible par corrélation de détecter des scénarios complexes d'attaques
\end{notebox} 

%SOURCE  : https://www.microfocus.com/media/white-paper/the_complete_guide_to_log_and_event_management_wp_fr.pdf

Une grande majorité des équipements (réseau, serveurs, terminaux (endpoint), des bases de données ou des applications d’un systèmes d'information peuvent aujourd’hui générer des logs ou traces.  Ces fichiers contiennent, pour chaque équipe, la liste de tous les évènements  \g{traçables} qui se sont déroulés pendant l'execution : réussite ou échec d’une connexion, redémarrage, utilisation des ressources (mémoire, ...).


\mode<all>{\picframe{../Latex/Sources/EDU/SRC1/Pictures/img-logman}{Les logs au coeur de la détection}{0.8}{lbllogman2}}


L'exploitation de ces traces est souvent complexe car chaque équipement dispose de ses propres fonctions de gestion des traces, avec encore dans de nombreux cas un format  d'enregistrement et de stockage propriétaire. Il faut consulter ces logs équipement par équipement. Heureusement, il existe des outils qui permettent de centraliser et de \g{normaliser} ces traces.

On peut citer par exemple, SYSLOG, qui par ailleurs n'est pas le seul type d'outil pour assurer la collecte et la normalisation des traces. SYSLOG reste pourtant un outil de référence dans les architectures de collecte.

\subsubsection{Services et protocole SYSLOG}

Le protocole Syslog est un protocole réseau qui permet de transporter les messages de journalisation générés par les applications vers une machine hébergeant un serveur Syslog. 

Quand un système veut conserver les traces d'un événement, il est possible, d'utiliser syslog pour communiquer les détails de l'événement à un daemon syslog (serveur syslog) qui va le conserver dans une base de données.

Le protocole Syslog est structuré autour de la notion de périphérique, de relais et de collecteur dans une architecture Syslog.

\begin{itemize}
  \item Un \textbf{périphérique} est une machine ou une application qui génère des messages Syslog;
  \item Un \textbf{relais} est une machine ou une application qui reçoit des messages Syslog et les retransmet à une autre machine;
  \item Un \textbf{collecteur} est une machine ou une application qui reçoit des messages Syslog mais qui ne les retransmet pas.
\end{itemize}

Tout périphérique ou relais sera vu comme un émetteur lorsqu'il envoie un message Syslog et tout relais ou collecteur sera vu comme un récepteur lorsqu'il reçoit un message Syslog.

L'intérêt d'un serveur/collecteur Syslog est donc de permettre une centralisation de ces journaux d'événements, permettant de repérer plus rapidement et efficacement les défaillances de machines présentes sur un réseau. On trouvera par exemple, sur le site 
\ulink{https://homputersecurity.com/2018/03/01/comment-mettre-en-place-un-serveur-syslog/}{homputersecurity.com} des éléments pour déployer un serveur SYSLOG, et une description détaillée sur le site \ulink{https://ram-0000.developpez.com/tutoriels/reseau/Syslog/}{Developpez.com}.

 \subsubsection{L'usage des LOGs}
 
 % todo modifcation
 
 Dans un usage de cybersécurité, les traces, journaux et logs informatiques et réseaux structurent les sources d'évènements et de stockages des informations. Ils sont donc :
 
 \begin{itemize}

   \item Un outil indispensable au processus de \textbf{détection de menace}. La gestion des logs (ou d’événements) s’avère en effet un outil primordial pour les analyses à posteriori, mais peut aussi servir dans la détection en temps réel pour peu que les outils d'analyse puissent le faire; Nous verrons cela dans la partie sur la détection de la menace.

   \item  \textbf{Une couverture légale }. Confrontée à une plainte, une entreprise peut utiliser ces traces pour gérer un litige avec un tiers  en attestant de la non-implication de son système d’information ou, a contrario, assumer le litige tout en remontant jusqu’à l’utilisateur concerné. La société peut également utiliser ces traces pour fournir des éléments aux services de polices. La fourniture d'élément probant à valeur légale nécessite quelques précautions. 

   \item  \textbf{Le dépistage des malversations internes ou de comportements déviants}. Les flux illégaux, les flux de données déviants (copies de fichiers en masse avant qu'un salarié quitte l'entreprise par exemple)

\end{itemize}

 Évidement, un outil de gestion de LOG ne serait à lui seul et sans fonction de corrélation avoir la capacité de détecter des évènements liés entre eux. 
 
\begin{techworkbox}{Syslog et cybersécurité}
	L'environnement SYSLOG possède une richesse fonctionnelle qui nécessiterait une présentation détaillée pour en appréhender les capacités et la puissance d'usage. C'est un sujet pour \fichetech de référence. Il existe de nombreuses documentations sur internet, toutefois une présentation détaillée d'une architecture SYSLOG pour un usage de cybersécurité est un sujet à explorer. 
\end{techworkbox}

\subsubsection{Puits de logs}

La construction d’un « puits de log » est une première brique de réponse : il s’agit de collecter, à l’aide d’un outil automatisé du marché, l’ensemble des journaux d’équipements dans un espace de stockage unique. L’un des critères de sélection de cet outil est justement sa capacité à reconnaître différents formats de logs (syslog, traps SNMP, formats propriétaires…).

Le volume d’informations centralisées peut vite exploser : il est important d’éviter la collecte de données inutiles. Par ailleurs, le système peut également être gourmand en puissance de calcul en fonction des périmètres de recherches effectuées.

On parle de log management à partir du moment où les données contenues dans ces  puits sont traitées et exploitées, par exemple pour retrouver un élément dangereux (virus, problème de sécurité…), ou un comportement malveillant (fuite d’information, suppression de données…). Il est nécessaire de cadrer en amont les finalités du projet,  qui peuvent être multiples.

\subsubsection{Outils d'analyses}

Après avoir \textbf{collectés, stockés} des évènements dans un format compréhensible (structuré ou non), il est nécessaire de disposer d'outils de recherche et d'analyses de ces logs. Il existe de nombreux outils  dont beaucoup de codes open source pour se faire.


\mode<all>{\picframe{../Latex/Sources/EDU/SRC1/Pictures/img-log-level}{Les niveaux d'outillage}{0.8}{lblloglevel}}


Nous pouvons citer par exemple \ulink{https://www.graylog.org}{Graylog} basé sur MongoDB pour la gestion des métadonnées et Elasticsearch pour le stockage des logs et la recherche textuelle. Graylog  permet de mieux comprendre l’utilisation d'un système d'information tant dans l'amélioration la sécurité (comportements, évènements à risques, indicateurs de compromission (IOC)) que dans l'usage des applications et services.

%TODO Menaces LOG : Cas des end point
%\subsubsection{le cas des End Point}
%Syslog - osquery

\begin{techworkbox}{Analyse de logs}
	Les outils d'indexation et d'analyse de logs sont nombreux et chacun possède des avantages et des inconvénients, des facilités d'usage de déploiement et des fonctionnalités différentes. C'est un sujet de réalisation pour \fichetech
\end{techworkbox}






