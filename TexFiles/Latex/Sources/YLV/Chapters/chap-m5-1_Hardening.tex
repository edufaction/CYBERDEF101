
\section{Règles techniques de sécurisation / durcissement}
\subsection{IAM} 
(PKI, MFA, journalisation, contrôles), ?
\subsection{Systèmes d'exploitation}
Hardening UNIX/WINDOWS kesako?
ACL, admin dédié, supervision et administration via réseau chiffré, tout doit être loggé ! Exclusion services inutiles
80 443 principe qui se décline sur toutes les applis
\subsection{Hardware (HSM), DC}
Chiffrement, zone hardware dédiée (mémoire, voire carte dédiée), 
DC : salles, controles d’accès, caméras, vigiles, etc.
\subsection{Réseaux}
VPN, chiffrement


%Begin FRAME----------------------------
\mode<presentation>{\texframe
% contenu affiché sur Article et Beamer
%- - - - - - - - - - - - - - - - - - - - - - - - 
{Règles techniques de sécurisation} % titre de la diapo
{Hardening} % sous titre de la diapo
{
%begin slide- - - - - - - - - - - - - - - - - 
 \begin{itemize}
     \item IAM : PKI, MFA, journalisation, contrôles
     \item Systèmes d'exploitation
         \begin{itemize}
             \item Hardening UNIX/WINDOWS
             \item ACL, réseau admin dédié, supervision et administration via réseau chiffré
             \item Exclusion services inutiles
             \item Journaux d'événements
             \item 80 --> 443
        \end{itemize}
     \item Matériel et Locaux (Data Center)
        \begin{itemize}
            \item Chiffrement, zone hardware dédiée (mémoire, voire carte dédiée), 
            \item DC : salles, controles d’accès, caméras, vigiles, etc.
        \end{itemize}
     \item Réseaux : VPN, chiffrement
 \end{itemize}
%end slide- - - - - - - - - - - - - - - - - - - 
}}
%End FRAME------------------------------