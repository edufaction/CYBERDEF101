\section{De l’analyse de risques aux fonctions de sécurité}
Via exemples de traitement des risques 

Nous explorerons dans ce chapitre l'utilisation des plans de traitement des risques comme fondement pour définir et mettre en œuvre des fonctions de sécurité essentielles, telles que le filtrage, le contrôle d'accès basé sur les rôles (RBAC) et la cryptographie. 
En intégrant une approche systématique de la gestion des risques, les organisations peuvent non seulement identifier et évaluer les menaces potentielles, mais aussi prioriser les mesures de sécurité en fonction de leur impact et de leur probabilité. 
Ainsi, ce chapitre mettra en lumière les synergies entre la gestion des risques et les fonctions de sécurité, offrant un cadre pour une protection robuste et proactive des actifs du système d'information.

%Begin FRAME----------------------------
\mode<presentation>{\texframe
% contenu affiché sur Article et Beamer
%- - - - - - - - - - - - - - - - - - - - - - - - 
{De l’analyse de risques aux fonctions de sécurité} % titre de la diapo
{exemples de traitement des risques} % sous titre de la diapo
{
%begin slide- - - - - - - - - - - - - - - - - 
	
 \begin{enumerate}
     \item Filtrage : cloisonnement, Multitenant
     \item Accès : RBAC (Role Based Access Control), droit d’en connaître
     \item Cryptographie : intégrité des données, protection des flux (IPSec, VPN SSL)
     %\mode<all>{\input{../Latex/Sources/YLV/Chapters/chap-m3-1-3-Crypto.tex}}
 \end{enumerate}
    
%end slide- - - - - - - - - - - - - - - - - - - 
}}
%End FRAME------------------------------
\subsection{Filtrage}

Afin de réduire les risques liés à la perte, l'intégrité ou la disponibilité des données de production, il est conseillé de cloisonner le système d'information et le réseau et donc de séparer les environnements de production, tests, pré-production et développement.

La notion de cloisonnement est aussi indispensable dans les cas d'hébergement et fourniture de services auprès de plusieurs clients. En effet, il est primordial d'assurer l'étanchéité et la confidentialité des données de chaque client. Le terme utilisé est celui de Multitenant et peut décrire le cas d'une solution ou application proposée et vendue à plusieurs clients ou entités.

Les services, utilisateurs ou zones d'exploitations spécifiques doivent être isolées entre elles. De plus, tout flux entre les différentes zones d'utilisateurs ou services doivent être contrôlés afin d'éviter les compromissions de données de production par celles de tests par exemple.

Nous verrons dans le chapitre 3.3 les solutions techniques pour assurer ce cloisonnement et ce filtrage.


%Begin FRAME----------------------------
\mode<presentation>{\texframe
% contenu affiché sur Article et Beamer
%- - - - - - - - - - - - - - - - - - - - - - - - 
{Cloisonnement} % titre de la diapo
{} % sous titre de la diapo
{
%begin slide- - - - - - - - - - - - - - - - - 
--> isoler les zones entre elles en fonction : 
\begin{itemize}
    \item des groupes, utilisateurs et clients (Multitenant)
    %ajouter schéma multitenant (exmple AWS)
    \item des services et exposition (Internet, Intranet/internes, etc.)
    \item des zones d'exploitation : production, tests ou entraînements
\end{itemize}
! Attention de contrôler les flux entre les zones...
%end slide- - - - - - - - - - - - - - - - - - - 
}}
%End FRAME------------------------------

\subsection{Accès}

%Gérer les utilisateurs, leurs accès et les droits qui leur sont associés constitue le cœur de l'Identity and Access Management (IAM), incluant des concepts tels que le contrôle d'accès basé sur les rôles (RBAC) et le principe du besoin d'en connaître.

L'Identity and Access Management (IAM), ou Gestion des Identités et des Accès, est un cadre essentiel qui permet de gérer les identités numériques des utilisateurs et de contrôler leur accès aux ressources d'une organisation. 
Dans ce contexte, le Privileged Access Management (PAM), ou Gestion des Comptes à Privilèges, joue un rôle crucial en centralisant la gestion des profils d'administrateur. 
Il garantit que le principe du moindre privilège est appliqué, permettant ainsi aux utilisateurs d'accéder uniquement aux ressources nécessaires à l'exercice de leurs fonctions.

Le processus de gestion des identités englobe quand à lui l'ensemble du cycle de vie des utilisateurs, depuis leur création jusqu'à leur suppression. 
Cela inclut des procédures de contrôle des habilitations, assurant que les droits d'accès sont régulièrement examinés et ajustés en fonction des besoins changeants de l'organisation.
Cette régularité est cruciale pour maintenir le niveau de sécurité optimal.
Plus précisément, l'Identity Access Governance (IAG) automatise la création, la gestion et la certification des comptes d'utilisateurs, des rôles et des droits d'accès. 
Elle supervise l'approvisionnement des comptes utilisateurs, la gestion des mots de passe, l'application des politiques, ainsi que la gouvernance et la révision des accès.

%Todo Déf IAM, process ET procédures

%Begin FRAME----------------------------
\mode<presentation>{\texframe
% contenu affiché sur Article et Beamer
%- - - - - - - - - - - - - - - - - - - - - - - - 
{Gestion des identités : IAM} % titre de la diapo
{Identity Access Management} % sous titre de la diapo
{
%begin slide- - - - - - - - - - - - - - - - - 
\begin{itemize}
    \item Définition
    \begin{itemize}
        \item IAM (identity and access management) Gestion des Identités et des Accès
        \item PAM (privileged access management) Gestion des comptes à privilèges , centralise la gestion des profils d'administrateur et assure que l'accès au moindre privilège est appliqué pour donner aux utilisateurs uniquement l'accès dont ils ont besoin.
    \end{itemize}
    \item Process
    Gestion des identités - cycle de vie
    \item Procédures
    Contrôle des habilitations
\end{itemize}
%end slide- - - - - - - - - - - - - - - - - - - 
}}
%End FRAME------------------------------


%Begin FRAME----------------------------
\mode<presentation>{\texframe
% contenu affiché sur Article et Beamer
%- - - - - - - - - - - - - - - - - - - - - - - - 
{Gestion des identités : IAM} % titre de la diapo
{Identity Access Management} % sous titre de la diapo
{
%begin slide- - - - - - - - - - - - - - - - - 
\begin{itemize}
    \item Processus de gestion des identifiants, arrivées, départs, demandes d'habilitations, etc.
    \item Politique de contrôle des accès : certificats, authentification multifacteurs (MFA)
    \item IAG (identity access governance)
        automatise la création, la gestion et la certification des comptes d'utilisateurs, des rôles et des droits d'accès pour les utilisateurs
        provisionnement des utilisateurs, gestion des mots de passe, gestion des politiques, gouvernance des accès et revue des accès
\end{itemize}
%end slide- - - - - - - - - - - - - - - - - - - 
}}
%End FRAME------------------------------

%Begin FRAME----------------------------
\mode<presentation>{\texframe
% contenu affiché sur Article et Beamer
%- - - - - - - - - - - - - - - - - - - - - - - - 
{Gestion des identités : PAM} % titre de la diapo
{privileged access management} % sous titre de la diapo
{
%begin slide- - - - - - - - - - - - - - - - - 
\begin{itemize}
    \item PAM (privileged access management) Gestion des comptes à privilèges, centralise la gestion des profils d'administrateur et assure que l'accès au moindre privilège est appliqué pour donner aux utilisateurs uniquement l'accès dont ils ont besoin.
\end{itemize}
%end slide- - - - - - - - - - - - - - - - - - - 
}}
%End FRAME------------------------------

% Crypto
% gestion intégrité des données, protection des flux (IPSec, VPN SSL)
\mode<all>{\input{../Latex/Sources/YLV/Chapters/chap-m3-1-3-Crypto.tex}}









