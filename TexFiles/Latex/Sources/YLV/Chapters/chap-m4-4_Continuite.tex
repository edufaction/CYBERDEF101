
\section{Continuité d'activité}
\subsection{Qu’est-ce que la continuité d’activité ?}
Bases de la norme ISO22301

La continuité d’activité consiste à mettre en place des mesures pour que les opérations essentielles d’une organisation puissent continuer, même en cas d’incident ou de crise. Par exemple, une entreprise de e-commerce peut prévoir un plan pour continuer à traiter les commandes si son centre de données tombe en panne ou si une cyberattaque bloque ses systèmes. De même, une usine peut avoir des plans pour continuer à produire en utilisant des fournisseurs alternatifs ou en déplaçant temporairement ses activités.

La norme ISO 22301 fournit un cadre pour structurer cette démarche. Elle insiste sur l’importance d’évaluer les risques, comme une inondation ou une panne électrique, et de planifier en conséquence. Par exemple, une banque peut élaborer un plan pour assurer la continuité de ses services en cas de coupure de courant ou de cyberattaque, en utilisant des centres de secours ou des sauvegardes de données.

La norme demande aussi à la direction de s’engager activement dans cette démarche. Elle doit définir des objectifs clairs, comme réduire le temps de reprise après un incident, et allouer les ressources nécessaires. Ensuite, il faut tester régulièrement ces plans, par exemple en organisant des exercices pour vérifier si les employés savent comment réagir en cas d’urgence.

Enfin, la norme encourage une amélioration continue. Si un test révèle une faiblesse, l’organisation doit ajuster ses plans pour mieux faire face à de futurs incidents. Par exemple, si une entreprise découvre que ses sauvegardes de données ne sont pas suffisantes, elle doit renforcer ses mesures de sauvegarde pour garantir la disponibilité des informations critiques.

\subsection{Définitions}
Sources : NIST, ANSSI
\begin{itemize}
    \item Business Impact Analysis (BIA)
    Process of analyzing operational functions and the effect that a disruption might have on them.
    \item Business Continuity Plan (BCP)
    The documentation of a predetermined set of instructions or procedures that describe how an organization’s mission/business processes will be sustained during and after a significant disruption.
    \item Plan de Continuité d'Activité (PCA) / Disaster Recovery Plan (DRP)
     A written plan for recovering one or more information systems at an alternate facility in response to a major hardware or software failure or destruction of facilities.
    \item Plan de Reprise d'Activité (PRA)
\end{itemize}


%Begin FRAME----------------------------
\mode<presentation>{\texframe
% contenu affiché sur Article et Beamer
%- - - - - - - - - - - - - - - - - - - - - - - - 
{Continuité d'activité : ISO22301} % titre de la diapo
{Définitions} % sous titre de la diapo
{
%begin slide- - - - - - - - - - - - - - - - - 

\begin{itemize}
    \item Business Impact Analysis (BIA)
    \item Business Continuity Plan (BCP)
    \item Plan de Continuité d'Activité (PCA) / Disaster Recovery Plan (DRP)
    \item Plan de Reprise d'Activité (PRA)
\end{itemize}
%end slide- - - - - - - - - - - - - - - - - - - 
}}
%End FRAME------------------------------



