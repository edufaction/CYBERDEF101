

\section{Système de management de la sécurité de l'information : SMSI}
normatif ISO27001; périmètre de certification \& mesures ISO27002 à appliquer

Définition SMSI (voir cours Fadoua sur l'intro)

%Begin FRAME----------------------------
\mode<presentation>{\texframe
% contenu affiché sur Article et Beamer
%- - - - - - - - - - - - - - - - - - - - - - - - 
{Système de management de la sécurité de l'information : SMSI} % titre de la diapo
{Introduction à ISO/EIC 27001} % sous titre de la diapo
{
%begin slide- - - - - - - - - - - - - - - - - 
	
 \begin{enumerate}
    \item Présentation de la norme ISO/EIC 27001
    \item Périmètre de certification
    \item Mesures de sécurité : ISO/EIC 27002
 \end{enumerate}
    
%end slide- - - - - - - - - - - - - - - - - - - 
}}
%End FRAME------------------------------

\subsection{Very short intro to ISO/EIC 27001}

%Begin FRAME----------------------------
\mode<presentation>{\texframe
% contenu affiché sur Article et Beamer
%- - - - - - - - - - - - - - - - - - - - - - - - 
{Very short intro to ISO/EIC 27001} % titre de la diapo
{Introduction à ISO/EIC 27001} % sous titre de la diapo
{
%begin slide- - - - - - - - - - - - - - - - - 
	Quel est le but d'ISO 27k1 ? Pourquoi l'utiliser ?

    
%end slide- - - - - - - - - - - - - - - - - - - 
}}
%End FRAME------------------------------


\subsection{Historique puis Basiques}
que trouve t on dans la norme ?
\subsection{Qu’est ce que le périmètre de certification ?}

Le périmètre de certification est le point de départ de tout projet de certification.
Il est indispensable de bien définir quel élément du système d'information devra respecter les exigences de la norme.
Le périmètre peut inclure une application, une ferme de serveurs, une équipe de collaborateurs, un data-center, etc.

%Begin FRAME----------------------------
\mode<presentation>{\texframe
% contenu affiché sur Article et Beamer
%- - - - - - - - - - - - - - - - - - - - - - - - 
{Périmètre de certification} % titre de la diapo
{Introduction à ISO/EIC 27001} % sous titre de la diapo
{
%begin slide- - - - - - - - - - - - - - - - - 
% Plutot que de parler SOA, parler de domaine et de périmètre d'application , éventuellement pointeur vers un exemple (Annexe A de la 27001)
% Slide avec les différents chapitres, thèmes de l'annexe A
%Zoom sur les 
% Panorama des normes
% dérouler la norme ? pour suivre sa structure 
% comment on l'améliore (PDCA)? on contrôle, on se structure pour mettre en place, engagement de la direction
% ssms apporte de la confiance, la confiance apporte du business

	\begin{itemize}
    \item Quel est le périmètre de certification ? 
    \item SOA : définition, rôle --> BOF 
 \end{itemize}

    
%end slide- - - - - - - - - - - - - - - - - - - 
}}
%End FRAME------------------------------

\subsection{ISO/EIC27002, quelles sont les mesures de sécurité ?}
Exemples à donner

%Begin FRAME----------------------------
\mode<presentation>{\texframe
% contenu affiché sur Article et Beamer
%- - - - - - - - - - - - - - - - - - - - - - - - 
{ISO/EIC27002} % titre de la diapo
{quelles sont les mesures de sécurité?} % sous titre de la diapo
{
%begin slide- - - - - - - - - - - - - - - - - 
	\begin{itemize}
    \item Définition mesures de sécurité 
    \item Exemples
 \end{itemize}
%end slide- - - - - - - - - - - - - - - - - - - 
}}
%End FRAME------------------------------