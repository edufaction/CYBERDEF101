\section{Sécurité des produits}

La conformité technique aux référentiels/normes est un élément fondamental pour s'assurer que les produits logiciels et hardware sont conformes aux normes et réglementations du secteur.

\subsection{Conformité aux implémentations normatives}
\begin{itemize}
	\item Protocoles réseaux
	\item Normes environnementales
\end{itemize}

\subsection{La confiance certifiée}

Dans le domaine de la cybersécurité on peut faire certifier des produits logiciels ou matériels avec la Certification de Sécurité de Premier Niveau (CSPN) ou les critères communs. En France cela passe par l'ANSSI.

\subsubsection{Certification de Sécurité de Premier Niveau (CSPN)}
La CSPN mise en place par l’ANSSI en 2008 consiste en des tests en « boîte noire » effectués en temps et délais contraints. La CSPN est une alternative aux évaluations Critères Communs, dont le coût et la durée peuvent être un obstacle, et lorsque le niveau de confiance visé est moins élevé. Cette certification s’appuie sur des critères, une méthodologie et un processus élaborés par l’ANSSI publiés sur le présent site.
(source ANSSI)

\subsubsection{critères communs}
La certification dite tierce partie est la certification de plus haut niveau, qui permet à un client de s’assurer par l’intervention d’un professionnel indépendant, compétent et contrôlé, appelé organisme certificateur, de la conformité d’un produit à un cahier des charges ou à une spécification technique. 

La certification tierce partie apporte au client la confirmation indépendante et impartiale qu’un produit répond à un cahier des charges ou à des spécifications techniques publiées. Ces spécifications techniques peuvent être élaborées dans un cadre normatif ou non.
(source ANSSI)
AJOUTER EXEMPLES
%Begin FRAME----------------------------
\mode<presentation>{\texframe
% contenu affiché sur Article et Beamer
%- - - - - - - - - - - - - - - - - - - - - - - - 
{Sécurité des produits} % titre de la diapo
{Conformité aux implémentations normatives} % sous titre de la diapo
{
%begin slide- - - - - - - - - - - - - - - - - 

\begin{itemize}
	\item Protocoles réseaux
	\item Normes environnementales
\end{itemize}
%end slide- - - - - - - - - - - - - - - - - - - 
}}
%End FRAME------------------------------
%Begin FRAME----------------------------
\mode<presentation>{\texframe
% contenu affiché sur Article et Beamer
%- - - - - - - - - - - - - - - - - - - - - - - - 
{Sécurité des produits} % titre de la diapo
{La confiance certifiée} % sous titre de la diapo
{
%begin slide- - - - - - - - - - - - - - - - - 

\begin{itemize}
	\item Certification de Sécurité de Premier Niveau (CSPN) : tests en « boîte noire »
	\item critères communs : certification qui permet à un client de s’assurer par l’intervention organisme certificateur, de la conformité d’un produit à un cahier des charges ou à une spécification technique
\end{itemize}
%end slide- - - - - - - - - - - - - - - - - - - 
}}

%End FRAME------------------------------

%Begin FRAME----------------------------
\mode<all>{\texframe
% contenu affiché sur Article et Beamer
%- - - - - - - - - - - - - - - - - - - - - - - - 
{Points à retenir} % titre de la diapo
{} % sous titre de la diapo
{
%begin slide- - - - - - - - - - - - - - - - - 
\begin{itemize}
    \item
    \item
\end{itemize}
%end slide- - - - - - - - - - - - - - - - - - - 
}}
%End FRAME------------------------------