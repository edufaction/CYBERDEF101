%-------------------------------------------------------------
%               FR CYBERDEF SECOPS COURSE
%                                        SECOPS
%                                           Intro
%
%                       Introduction Cyberdefense
%                        Chap-Intro-attaquants.tex
%
%                              2020 eduf@ction
%-------------------------------------------------------------


Avant de nous lancer dans la mise en place de mécanismes de sécurité, de défense et de résilience, il est intéressant de revenir aux éléments essentiels d'une cyberdéfense d'entreprise. En effet, se protéger, se défendre, résister fait écho au concept d'attaque. Je vous propose important de définir et de caractériser une attaque et un attaquant. 

\utocomplete

Un cyberattaquant ou hacker est une personne ou un groupe de personnes qui utilisent des techniques de piratage informatique pour accéder à des systèmes d'information ou des réseaux sans autorisation ou en détournant des autorisations existantes ou pour causer des dommages.
 
Le terme "hacker" est souvent utilisé de manière négative pour décrire quelqu'un qui accède à des systèmes de manière illégale ou qui utilise des connaissances en informatique de manière malveillante. Cependant, le terme peut également être utilisé de manière plus positive pour décrire quelqu'un qui a une expertise solide en informatique et qui utilise ses connaissances de manière constructive pour résoudre des problèmes complexes ou créer de nouvelles technologies.

Il existe plusieurs types de hackers, qui varient en fonction de leurs motivations et de leurs méthodes. Certains hackers sont motivés par des raisons financières et cherchent à voler des informations sensibles ou à demander une rançon en échange de la restauration de fichiers verrouillés par un logiciel de rançon. D'autres hackers sont motivés par des raisons politiques ou militaires et cherchent à récupérer des informations sensibles ou à causer des dommages à des systèmes ou réseaux pour perturber leur fonctionnement. Enfin, il existe également des hackers qui agissent simplement pour le plaisir de découvrir comment fonctionnent les choses et de trouver des failles de sécurité dans les systèmes d'information.