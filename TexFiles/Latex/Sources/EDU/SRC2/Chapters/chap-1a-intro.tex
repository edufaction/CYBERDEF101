%-----------------------------------------------------%
% Section File											
%-----------------------------------------------------%

\section{Contexte}

Pour démarrer la leçon plongeons nous dans quelques attaques qui ont marqué une époque, ou du moins ont été révélatrice de la monté rapide des cybermenaces sur l'environnement.

Avant d'être un affaire de criminalité et de cybercriminalité, les attaques informatiques ont été des affaires d'état, ou impliquant des états. De nos jours il y a un retour aux sources dans ce cyberespace qui est devenu un enjeu géostratégique.

On peut parler de Stuxnet, il y a plus de dix ans.

Stuxnet est le premier malware ayant causé des destructions physiques dans le monde réel. Ce virus fut développé par les services secrets d'Israël et des États-Unis, dans le but de faire échouer le programme nucléaire de l'Iran.

\ulink{https://fr.wikipedia.org/wiki/Stuxnet}{Histoire de Stuxnet}.

Mais on peut plonger plus loin, dans la guerre de l'information, l'espionage avec le réseau ECHELON. avec leur station d'interception ROEM située à Menwith Hill, au Royaume-Uni.
ECHELON est le nom de code utilisé pendant de nombreuses années par les services de renseignement des États-Unis pour désigner un réseau utilisé pour la surveillance et l'interception des télécommunications.

Mais revenons au 21ième siécle.




\subsection{Quelques fuites de données célèbre}

Sony

AsthonWhesley


TV5monde

WannaCry 

\subsection{Les chaines logiciels}

SolarWind

Log4J

\subsubsection{Libwebp}
Libwebp est une bibliothèque open-source que les programmes intègrent pour obtenir la capacité d'encoder et de décoder des images au format WebP, un format moderne de compression sans perte/avec perte largement utilisé dans les applications web publiées par Google. Des centaines d'applications utilisent libwebp pour prendre en charge le format d'image WebP, de sorte qu'une vulnérabilité dans cette bibliothèque peut avoir des conséquences en cascade.

Après quelques jours de confusion dans la communauté de la sécurité suite à la divulgation initiale de la faille, répertoriée sous le nom de CVE-2023-4863, les analystes se sont rendu compte que l'impact du problème était vaste, allant au-delà des navigateurs web. Comme il a été rapporté que des pirates informatiques ont exploité activement CVE-2023-4863 dans des attaques, le risque pour des millions de personnes utilisant des logiciels encore affectés est élevé.


Pole-emploi, différents CHU


\section{les attaquants}

\subsection{Les script kiddies}

\section{Un monde numérique étendu}


\subsection{le cloud}

\subsection{l'AI}


\section{Quelques définitions}