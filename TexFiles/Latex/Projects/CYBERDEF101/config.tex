%-----------------------------------------------
%               FR CYBERDEF SECOPS COURSE
%                         GLOBAL CONFIG  File
%                             2020 eduf@ction
%-----------------------------------------------

% just used for quick access to samples on project editing tool
%%-------------------------------------------------------------
%               FR CYBERDEF SECOPS COURSE
%              SAMPLES for edx & u  commands
%                             2020 eduf@ction
%-------------------------------------------------------------


% 												I M A G E S   
%-------------------------------------------------------------------------
%\upicture{../Latex/Sources/EDU/Pictures/img-liovar-incidents}{Incidents}{01}{lb:mtbf}
%-------------------------------------------------------------------------


%\mode<all>{\picframe{../Latex/Sources/EDU/Pictures/img-cycle}{Cycle de vie de gouvernance Cyberdef}{0.3}{lbl:cycle}}


% Planche PRZ 
% begin frame \mode<all>{\texframe{Déroulement}{présentation Gestion de la menace}
%{

%}} % end frame



%\begin{table}[!hbtp]
%\resizebox{\textwidth}{!}{%
%\begin{tabular}{ l|l l}
%%\begin{tabular}{l*{3}{S[table-format=2.1]}}
%\toprule
%\textbf{Package\_name }                 & \textbf{C\&C }                                        & \textbf{sha256}                                                            \\ \hline
%caracal.raceinspace.astronaut & http://api.lulquid.xyz                       & f1d32c17a169574369088...\\ \hline
%
%\bottomrule
%\end{tabular}
%}
%\caption{Exemple de  données de TI (Threat Intelligence)}
%\label{tab:ExempleTI}
%\end{table}

%----------------------------
% Path
\def\upath{../Latex}

 %----------------------------
 % TEMPLATE, THEME, & Beamer Model

%-------------------------------------------------------------
%			 FR CYBERDEF SECOPS COURSE
% 					INIT Commands File
% 		Select BEAMER or Article/book Mode
% 			with sub extension .doc or .prz
%                           2020 eduf@ction
%-------------------------------------------------------------
\RequirePackage{xstring}
\RequirePackage[realmainfile]{currfile}

\newcommand{\udocname}{\jobname}

\IfSubStr*{\currfilebase}{.prz}{\newcommand{\PRZMODE}{YES}}{\newcommand{\DOCMODE}{YES}}
   
\IfSubStr*{\currfilebase}{.edt}{\newcommand{\EDTMODE}{YES}}{\newcommand{\NOMODE}{YES}}

%===========================
% EDX LIBRARY
%===========================


\def\utemplatepath{*}
\def\utemplatetheme{*} % orange ou cnam
\def\uproject{*} % nom du projet


% Chargement des éléments de l'auteur
\newcommand{\setauthor}[1]{\def\uUniversity{Executive Education Polytechnique}
\newcommand{\uauthor}{Yann-Arzel LE VILLIO}
\def\uauthortrigram{YLV}
\newcommand{\uauthorwriter}{\uauthor} %Beamer
\newcommand{\uproa}{Enseignant Sécurité ESIR}
\newcommand{\uprob}{Directeur Technique et Scientifique Orange Security School}
\newcommand{\umaila}{yannarzel.levillio@orange.com}
\newcommand{\usitea}{http://campus.orange.com}
\newcommand{\usiteb}{http://www.esir.fr}
\newcommand{\umailb}{}

\def\uauthorphoto{../Latex/Sources/\uauthortrigram/Author/photo.png}


\def\ubio{\small \color{white} Yann-Arzel LE VILLIO (Enseignant à l'ESIR, Titulaire d'un Master 2 de l'IFSIC) est actuellement Directeur Technique et Scientifique d'Orange Security School, le centre de formation et d'entrainement Cybersécurité et Cyberdéfense du groupe Orange après plusieurs années comme expert sécurité et architecte sécurité. Il intervient dans l'Executive Master in Cybersecurity de l'Ecole Polytechnique dont il a participé à sa création. Spécialiste de la pédagogie en matière de Sécurité, de cybersécurité, il s'est spécialisé dans l'apport de l'IA dans la pédagogie en formation Cybersécurité.
\\}%}


% pour les LESSONS et les QUIZZ

\newcommand{\lessonID}[1]{\renewcommand{\lessonID}{#1}}




\def\utemplatepath{../Latex/Templates/Magic}

\def\edxlibspath{\utemplatepath/edxlibs}
%%-----------------------------------------
% Load Edxlibs in MAGIC path
%-----------------------------------------
% Need : \edxlibspath defined

%---------------------------------------------------------------
% Placeholder Module - eduf@ction - 2023
%---------------------------------------------------------------

%\usepackage[usenames,svgnames,table]{xcolor}

\usepackage{xcolor}

\newcommand{\placeholder}[1]{\noindent%
   \fcolorbox{LightBlue}{LightCyan}{%
      \begin{minipage}{0.9\columnwidth}%
         \color{DodgerBlue}\textbf{\tiny{TODO~~}}%
         \color{MidnightBlue}#1%
      \end{minipage}%
   }%
   \par%
}
\newcommand{\placeholderinl}[1]{\noindent%
   \fcolorbox{LightBlue}{LightCyan}{\color{DodgerBlue}!}%
   ~{\color{MidnightBlue}#1}~%
   \fcolorbox{LightBlue}{LightCyan}{\color{DodgerBlue}<}%
}

%---------------------------------------------------------------
% Epigraphe Module - eduf@ction - 2023
%---------------------------------------------------------------


\usepackage{times}
\usepackage{xcolor}
\usepackage{epigraph}
\usepackage{tabto}
\usepackage{ifthen}
\setlength{\epigraphwidth}{0.75\linewidth}
\setlength{\afterepigraphskip}{\baselineskip}
\setlength{\beforeepigraphskip}{0.4\baselineskip}


\definecolor{epigraphecolor}{RGB}{125,125,125}


\makeatletter
\newenvironment{@leftepigraph}{
   \setlength\topsep{0pt}
   \setlength\parskip{0pt}
   \renewcommand{\epigraphflush}{flushleft}
}{}
\newenvironment{@rightepigraph}{
   \setlength\topsep{0pt}
   \setlength\parskip{0pt}
   \renewcommand{\epigraphflush}{flushright}
   \renewcommand{\sourceflush}{flushright}
   \renewcommand{\textflush}{flushright}
}{}
\newcommand{\@bgquote}[2]{\tabto*{#1}{\fontsize{100pt}{0pt}\selectfont{}\color{epigraphecolor!50}\smash{\raisebox{-60pt}{#2}}}\tabto*{0pt}}
\newcommand{\@generalquote}[6]{\begin{#6}\epigraph{\@bgquote{#5}{#4}#3}{\textsc{#2}\ifthenelse{\equal{#1}{}}{}{\\\textit{#1}}}\end{#6}}
\newcommand{\lquote}[3][]{\@generalquote{#1}{#2}{#3}{''}{\dimexpr(\epigraphwidth-50pt)}{@leftepigraph}}
\newcommand{\rquote}[3][]{\@generalquote{#1}{#2}{#3}{``}{0pt}{@rightepigraph}}
\makeatother
%-----------------------------------------
% Module QuizzMoodle in MAGIC path
%-----------------------------------------
% Need : \edxlibspath defined


%-----------------------------------------
% Define counter for question

\newcounter{uQuizznum}
\renewcommand\theuQuizznum{\arabic{uQuizznum}}
\newcommand{\uQuizznum}{\stepcounter{uQuizznum}ID\theuQuizznum}

\setcounter{uQuizznum}{0}
\newcommand\QuizzIDprefix{no-prefix}
\newcommand\QuizzName{no-name}

\newcommand{\QuizzID}{\QuizzIDprefix-\uQuizznum}
 à mettre dans les class article, book ...

\def\uauthorphoto{\utemplatepath/commons.inc/CommonsPictures/missionauthorphoto.png}


\def\utemplatetheme{orange}
\def\uproject{CYBERDEF101}
\def\ubeamermodel{Olannion}


%-------------------------------
% Parametres locaux des noms pour les fichiers de configs Courses, Book et Lessons
\newcommand*{\localcoursename}{CYBERDEF101}
\newcommand*{\localcourseinstitute}{Orange CyberSchool}
\newcommand*{\localcoursetitle}{Eléments de cybersécurité d'entreprise}

%\setbreakpage{}

\newcommand{\upictureext}{pdf}

 %----------------------------
 % Basic Var
 
 %\setcounter{cntx}{\year+1}
\newcommand{\INFODistrib}{Eléments de cours}
\newcommand{\uJournalInfo}{\localcourseinstitute, Direction de la formation}

\newcommand{\udescription}{}



 %\newcommand{\uJournalInfo}{CNAM Bretagne, Cybersécurité SEC101, eduf@ction}

\newcommand{\uinstituteshort}{\localcourseinstitute}
\newcommand{\uinstitute}{\localcourseinstitute}
\newcommand{\uchaire}{Direction technique \& scientifique}
\newcommand{\uversion}{COURS INTERNE}

%----------------------------
% All other variables are defined in
 % file : *.lesson.tex
\newcommand{\ushorttitle}{\localcoursename}
\newcommand{\uCoursetittle}{\localcoursename}
\newcommand{\uMinilogo}{\localcoursename Orange}
\newcommand{\uCourseLongName} {\localcoursetitle}
%\newcommand{\utitle}{Cyberdéfense d'entreprise}

%----------------------------
% Book versus Course Notes
%----------------------------
% Use "e document' "e cours"
\newcommand{\edoc}{e document\xspace}
\newcommand{\ecours}{e cours\xspace} 
\newcommand{\etitle}{\localcoursetitle} 
\newcommand{\fichetech}{une fiche TECHNO}


%%----------------------------
%% Spécifique BOOK
% %----------------------------
\newcommand{\printer}{Publication limitée pour  \uinstitute \xspace}

%%----------------------------
% Attention : Use hard coded Pictures Path
% (PicframeS)
\newcommand{\picpath}{../Latex/Sources/EDU/SRC1/Pictures}
%-------------------------------------------------------------

%%----------------------------
%% Spécifique COURSE
% %----------------------------

\newcommand {\parttitlecourse}{TITRE PARTIE (config)}
\newcommand{\coursenumber}{1}

\newcommand{\ucourseresume}{\normalsize Aborder la sécurité des systèmes d’information sous l’angle d'une sécurité dynamique est un axe qui depuis quelques années apporte de nouvelle manière d’aborder la protection, la défense, et la résilience des systèmes d’information. La transformation digitale de l'entreprise modifie et rend plus flous les périmètres des systèmes d’informations. Cela nécessite une approche élargie du risque numérique et des nouvelles architectures de systèmes d'information intégrant des technologies de cloud et de cyberdéfense. Malgré la mise en place de mesures et de technologies de protection de plus en plus élaborées, l'impact d'une attaque ayant franchi ces frontières poreuses à considérablement augmenté.  Cette compilation des notes de cours élaborée dans le cadre d'un cours d'introduction à la cybersécurité et la cyberdéfense d'entreprise aborde les grands éléments fondamentaux permettant d'appréhender le domaine. Il permet d'en comprendre les enjeux, les codes, et acteurs. Protéger l'ensemble de l'entreprise alors qu'il est complexe de définir ses frontières est illusoire. Identifier les actifs essentiels ou vitaux et mettre en place les moyens adaptés à leur protection et leur défense est une démarche tactique qui permet de graduellement réduire ses cyber-risques tout en assurant un pilotage globale de la gouvernance de la sécurité sur ses volets de conformité.}


\newcommand{\uGlossaryAcronyms}{
\loadglsentries{../Latex/Sources/commons/acronymes.tex}
\loadglsentries{../Latex/Sources/commons/glossaire.tex}
}
