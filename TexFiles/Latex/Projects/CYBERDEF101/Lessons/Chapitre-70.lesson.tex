%========================================
%. CONFIGURATION ARTICLE  COURS  SEC 101
%========================================
% 				CONTRIBUTIONS
%========================================

%************************************************************
% Chargement des variables du modèle
%************************************************************




%-------------------------------------------------------------
%               FR CYBERDEF SECOPS COURSE
%                          CONFIG NOTES File
%          SET BEAMER or Article/book Mode
%                    PRZMODE OR DOCMODE
%                             2020 eduf@ction
%-------------------------------------------------------------

% insert InterTitlle for each file compose main "Notes"
%\newcommand{\INTERTITLE}{OK}

 %------------------------------------------
 % Use global project CYBERDEF 
 %            Configuration file
 %------------------------------------------
%-----------------------------------------------
%               FR CYBERDEF SECOPS COURSE
%                         GLOBAL CONFIG  File
%                             2020 eduf@ction
%-----------------------------------------------

% just used for quick access to samples on project editing tool
%\input{Latex/Projects/CNAMSEC101/Lessons/Commons/edxsamples.tex}

%----------------------------
% Path
\newcommand{\upath}{Latex}

\newcommand{\utemplatepath}{Latex/Templates/Magic}
\newcommand{\utemplatetheme}{cnam} % MAGIC TEMPLATE Var
\newcommand{\uproject}{CNAMSEC101}


\newcommand{\upictureext}{pdf}

 %----------------------------
 % Basic Var
 
 %\setcounter{cntx}{\year+1}
\newcommand{\INFODistrib}{Notes de cours SECOPS 2022-2023}
\newcommand{\uJournalInfo}{Orange Cyber et Cloud avec CNAM Bretagne, Cybersécurité SEC101, edufaction}

\newcommand{\udescription}{La sécurité opérationnelle au coeur de la cyberdéfense d'entreprise}

\newcommand{\uCnam}{Conservatoire National des Arts et Métiers}
\newcommand{\uOCD}{Orange Campus Cyber}
\newcommand{\uauthor}{Eric DUPUIS}
\newcommand{\uauthorwriter}{edufaction}
\newcommand{\uproa}{Enseignement sous la direction du Professeur Véronique Legrand, \uCnam, Paris, France}
\newcommand{\uprob}{Directeur \uOCD}
\newcommand{\umaila}{eric.dupuis@lecnam.net}
\newcommand{\usitea}{http://www.cnam.fr}
\newcommand{\usiteb}{http://www.orange.com}
\newcommand{\umailb}{eric.dupuis@orange.com}
 %\newcommand{\uJournalInfo}{CNAM Bretagne, Cybersécurité SEC101, eduf@ction}
\newcommand{\uinstituteshort}{CNAM}
\newcommand{\uinstitute}{\uCnam}
\newcommand{\uchaire}{Chaire de Cybersécurité}
\newcommand{\uversion}{Cnam Bretagne}





%----------------------------
% All other variables are defined in
 % file : *.lesson.tex
\newcommand{\ushorttitle}{CYBERDEF SEC101}
\newcommand{\uCoursetittle}{Eléments de Sécurité Opérationnelle}
\newcommand{\uMinilogo}{Cyberdéfense d'entreprise}
\newcommand{\uCourseLongName} {Eléments de sécurité opérationnelle en cyberdéfense d'entreprise}
%\newcommand{\utitle}{Cyberdéfense d'entreprise}

%----------------------------
% Book versus Course Notes
%----------------------------
% Use "e document' "e cours"
\newcommand{\edoc}{e document\xspace}
\newcommand{\ecours}{e cours\xspace} 
\newcommand{\etitle}{éléments de sécurité opérationnelle en cyberdéfense d'entreprise} 
\newcommand{\fichetech}{une fiche TECHNO}


%%----------------------------
%% Spécifique BOOK
% %----------------------------
\newcommand{\printer}{Publication limitée pour le \uCnam \xspace et Orange Campus Cyber}

%%----------------------------
% Attention : Use hard coded Pictures Path
% (PicframeS)
\newcommand{\picpath}{Latex/Sources/EDU/Pictures}
%-------------------------------------------------------------

%%----------------------------
%% Spécifique COURSE
% %----------------------------

\newcommand {\parttitlecourse}{TITRE PARTIE (config)}
\newcommand{\coursenumber}{1}

\newcommand{\uGlossaryAcronyms}{
\loadglsentries{Latex/Sources/commons/acronymes.tex}
\loadglsentries{Latex/Sources/commons/glossaire.tex}
}

 %------------------------------------------
 
\renewcommand{\uMinilogo}{\includegraphics[width=0.04\paperwidth]{\utemplatepath/commons.inc/CommonsPictures/\utemplatetheme.theme/shield-20.pdf}}
 
\newcommand {\Ucontribute}
{
%%-------------------------------------------------------------
%               FR CYBERDEF SECOPS COURSE
%                            Contributions Rules
%                              2020 eduf@ction
%-------------------------------------------------------------

% \section{Contributions}

% \subsection{Comment contribuer}

Les notes et les présentations sont réalisées sous \hologo{LaTeX}. 

Vous pouvez contribuer au projet du cours CYBERDEF101. Les contributions peuvent se faire sous deux formes :

\begin{itemize}
  \item Corriger, amender, améliorer les notes publiées. A chaque session des modifications et évolutions sont apportées pour tenir compte des corrections de fond et de formes. 
  \item Ajouter, compléter, modifier des parties de notes sur la base de votre lecture du cours et de vos expertises dans chacun des domaines évoqués.
\end{itemize}

 Les fichiers sources sont publiés sur GITHUB dans l'espace : \ulink{https://github.com/edufaction/CYBERDEF101}{(edufaction/CYBERDEF101)}.
 
%%-------------------------------------------------------------
%               FR CYBERDEF SECOPS COURSE
%                Contributions DATAS & NAMES
%                              2020 eduf@ction
%-------------------------------------------------------------

\subsection{Les contributeurs/auteurs du cours}

\subsubsection{co-auteurs}

%-------------------------------------------------------------
%               FR CYBERDEF SECOPS COURSE
%                Contributions DATAS & NAMES
%                              2020 eduf@ction
%-------------------------------------------------------------
%           CO AUTEURS - Rédacteur de chapitre
%-------------------------------------------------------------

\coauthor{Eric DUPUIS}{2019-2023}{Chapitre 1 à 7}{CYBERDEF101}
\coauthor{Yann Arzel Le Villio}{2022-2023}{Chapitre 1 à 7}{CYBERDEF101}

\subsubsection{contributeurs}

%-------------------------------------------------------------
%               FR CYBERDEF SECOPS COURSE
%                Contributions DATAS & NAMES
%                              2020 eduf@ction
%-------------------------------------------------------------
%                 CONTRIBUTIONS AU COURS
%-------------------------------------------------------------


\coauthor{Céline JUBY}{2023}{Contributions d'amélioration et relectures}{Orange Security School}


} 

%------------------------------------------
 %     BEAMER OU ARTICLE MODEL
 %------------------------------------------
  
\ifdefined\PRZMODE 
	\providecommand{\umainload}{\documentclass[ignorenonframetext,allowframebreaks,aspectratio=169,t,8pt, xcolor=table]{beamer}
\usepackage[utf8]{inputenc}

\usepackage{booktabs} 
\usepackage{subcaption}

\usepackage[T1]{fontenc}
\usepackage{csquotes}
\usepackage[french]{babel}
\usepackage{lmodern}
\usepackage{multicol}

\input{\utemplatepath/commons.inc/CommonsPictures/\utemplatetheme.theme/colors.inc} % set uMainColor

%-------------------------------------------------------------
%               FR CYBERDEF SECOPS COURSE
%                                  EDX PACKAGES
%.    Commons packages for BOOK, NOTES (.doc) 
%                            and BEAMER (.prz)
%                              2020 eduf@ction
%-------------------------------------------------------------

%\usepackage[hyphens,spaces,obeyspaces]{url}
%\usepackage[hyphens,spaces,obeyspaces]{url}



\PassOptionsToPackage{hyphens,spaces,obeyspaces}{url}

% \usepackage[utf8]{inputenc}

% \usepackage[T1]{fontenc}
%\usepackage{markdown}

\usepackage{fontspec}

\setsansfont{Arial}%{Arial}
\setmainfont{Arial}%{Times New Roman}
\setmonofont{Arial}%{Consolas}

\usepackage[french]{babel}

%\usepackage{cleveref} suppression ...

\usepackage{ccicons}

\usepackage{tikz} % Required for drawing custom shapes
\usetikzlibrary{tikzmark} 
\usetikzlibrary{mindmap,trees, backgrounds}


\usepackage{verbatim}
\usepackage{wrapfig}
\usepackage{xpatch}

\usepackage{hologo}

\usepackage[figure]{totalcount}

\usepackage{fontawesome}

\usepackage[skins]{tcolorbox} % don't forget SKINS option

\usepackage{listings}
\usepackage{upquote}
%\usepackage{xcolor} loaded by edxlibs

\definecolor{grey}{RGB}{50,80,80}

\lstset{
upquote=true,columns=flexible, basicstyle=\ttfamily,
language=HTML, 
frameround=tttt,
commentstyle=\color{gray},
identifierstyle=\color{blue},
keywordstyle=\color{uMainColor}\bfseries,
xleftmargin=2em,
xrightmargin=2em,
aboveskip=\topsep,
belowskip=\topsep, 
frame=single,
rulecolor=\color{uMainColor},
backgroundcolor=\color{uMainColor!5},
breaklines,
breakindent=1.5em,
showspaces=false,
showstringspaces=false,
showtabs=false,
xleftmargin=2em,
xrightmargin=2em,
aboveskip=1em,
belowskip=1em,
}



\usepackage{graphicx} % Required for including pictures
%\graphicspath{{Pictures/}} % Specifies the directory where pictures are stored

\usepackage{booktabs} % Required for nicer horizontal rules in tables

\usepackage{setspace}
\setstretch{1,2}

\usepackage{csquotes}

\usepackage{fancyhdr}

\usepackage{makeidx} 

\usepackage{xspace} 

\usepackage{floatflt} 

\usepackage{hyperref}


%pdfdisplaydoctitle={\utitle}, Attention le titre ne doit pas contenir de saut de ligne 

\hypersetup{
pdftitle={\uCoursetittle},
pdfauthor={\uauthor},
hidelinks,
backref=true,
pagebackref=true,
hyperindex=true,
pdfcenterwindow = true,
colorlinks=true,
breaklinks=true,
bookmarks=true,
allcolors=uMainColor,
pdfsubject = {\udescription},%
pdfcreator = {MACTEX on TEXPAD},%
bookmarksopen=true}
%


% urlcolor=cnam,
% runcolor=uMainColor,
%\pdfstringdefDisableCommands{%
%  \def\\{}%
%  \def\texttt#1{<#1>}%
%}

\usepackage[style=numeric,citestyle=numeric,sorting=nyt,sortcites=true,autopunct=true,hyperref=true,abbreviate=false,backref=true,backend=biber]{biblatex}

%\usepackage{cleveref} old position

\usepackage{multicol}

%--------------------------------------------------------
% pour la génération du QRCODE en page de garde d'article
\usepackage{qrcode} 

%--------------------------------------------------------
% POur la date d'impression en premiere page d'article
\usepackage[useregional]{datetime2}

%--------------------------------------------------------
% Gestion des glossaires et Sources/EDU/acronymes
% Un fichier acroymes.tex centralisé
\usepackage[acronym]{glossaries}

%\makeglossaries
\makenoidxglossaries 
\loadglsentries{../Latex/Sources/commons/acronymes.tex}
\loadglsentries{../Latex/Sources/commons/glossaire.tex}




%--------------------------------------------------------
% Pour exclure des "environnement", ici techworkbox par exemple, travaux pour étudiants, non affiché dans le BOOK.
\usepackage{comment}


%--------------------------------------------------------
% pour filigramme sur les pages
\usepackage{eso-pic}



\usepackage{etoolbox}

\usepackage{breakurl}



%-------------------------------------------------------------
%               FR CYBERDEF SECOPS COURSE
%                                  EDX PACKAGES
%.    Commons COMMANDS for BOOK, NOTES (.doc) 
%                            and BEAMER (.prz)
%                              2020 eduf@ction
%-------------------------------------------------------------

\newcommand{\ucite}[1]{\cite{#1}}
%\newcommand{\ucite}[1]{REF BIB #1}
\newcommand{\ugls}[1]{\gls{#1}}
%\newcommand{\ugls}[1]{GLOSSAIRE (#1)}
\newcommand{\uac}[1]{\acrshort{#1}}



% \newcommand{\ExtractBeforeLastSlash}[1]{\directlua{extract_before_last_slash("#1")}}
% \newcommand{\ExtractAfterLastSlash}[1]{\directlua{extract_after_last_slash("#1")}}
% \newcommand{\RemoveExtension}[1]{\directlua{remove_extension("#1")}}
% \newcommand{\ExtractExtension}[1]{\directlua{extract_extension("#1")}}


\newcommand{\includepicturesubpath}[1]
{#1}

%{\ExtractBeforeLastSlash{#1}/\upicturethemepath\RemoveExtension{\ExtractAfterLastSlash{#1}}.\ExtractExtension{#1}}
    

%---------------------------------------------------
%										 	AUDIO NOTE 
%---------------------------------------------------

\newcommand\audionote[2]{%
  \note{#2}
  \immediate\write\audioexport{==AudioSlide=============}
  \immediate\write\audioexport{#1}
  \immediate\write\audioexport{\insertframenumber}
  \immediate\write\audioexport{---------------------------------}
  \immediate\write\audioexport{#2}
  \immediate\write\audioexport{---------------------------------}
}%

%---------------------------------------------------
%										 	TEXT STYLE SHORTCUT 
%---------------------------------------------------

\newcommand{\tcb}[1]{\textbf{\textcolor{uMainColor}{#1}}}

\newcommand{\voir}[1]{(Schéma~\ref{#1},~page~\pageref{#1}}

\newcommand{\graphicxextension}{pdf}

\newcommand\myemptypage{
   
    \thispagestyle{empty}
    \addtocounter{page}{-1}
    \newpage
  \null
    %CECI EST UNE PAGE VIDE
       \vfill
    }


\newcommand{\uNA}[2]
{
\newacronym{a#1}{#1}{#2}
}

\newcommand{\coauthor}[4]
{
\noindent (#2) \textbf{#1} - #4 : \textit{#3}
}


\newcommand\UKword[1]{\emph{#1}\xspace}
\newcommand\textUK[1]{\textit{\textbf{#1}}\xspace}
\newcommand\Ucenter[1]{\begin{center}{#1}\end{center}}

\newcommand\head[1]{\textbf{#1}}
\newcommand\tb[1]{\textbf{#1}}
\newcommand{\g}[1]{\og #1 \fg{}}
\newcommand{\uload}[1] {\input{../Latex/#1}}
\newcommand{\ucurcolor}{black}

\newcommand{\mak}[1]{\faicon{\aArrowCircleORight} #1}


%-----------------------------------------------------
%										 	\INFOGitHub 
%---------------------------------------------------


\newcommand{\GITfilename}{https://github.com/edufaction/CYBERDEF101/raw/main/Release/Builder/\jobname.pdf}

%\newcommand{\safeqrcode}[2][]{%
%\qrcode[#1]{\detokenize{#2}} }

\newcommand{\safeqrcode}[2][]{\qrcode[#1]{#2}} 

\newcommand{\INFOGitHub}
{
\begin{center} 
\setstretch{1.5}
 {Vérifiez la disponibilité d'une version plus récente de} \\
 {\ulink{\GITfilename}{\textbf{\jobname.pdf }sur GITHUB CYBERDEF} }   \\ 
{{\huge\ccbyncndeu}}  \\  
{Publication en \href{https://creativecommons.org/licenses/by-nc-nd/2.0/fr/}{Creative Common BY-NC-ND} by eduf@ction}    \\  % Copyright notice
{\safeqrcode[height=4 cm,padding]{\GITfilename}}  
\end{center} 
}

\newcommand{\INFOGitHubS}
{
\noindent{\safeqrcode[height=3 cm]{\GITfilename}} \\
\noindent {\tiny Téléchargez une version à jour} 
}



%-----------------------------------------------------
%										 	\upspicture 
%---------------------------------------------------
\newcommand{\upspicture}[3]
{
\renewcommand{\ucurcolor}{#3}
\resizebox{#2\textwidth}{!}{\input{../#1}}
}


%-----------------------------------------------------
%										 	\uindex 
%---------------------------------------------------


\newcommand{\uindex}[1]
{
#1\index{#1}\xspace
}

\newcommand{\ulindex}[1]{\index{#1}}

%-----------------------------------------------------
%										 	\includer 
%---------------------------------------------------

\newcommand{\includer}[1]
{
\input{../Latex/Sources/EDU/SRC1/Chapters/#1}
}

%-----------------------------------------------------
%										 	\edxdico 
%---------------------------------------------------

\newcommand{\edxdico}[2] {\textbf{#1}:\textit{#2} \index{#1}}

%-----------------------------------------------------
%  command DEF : InTextx
%-----------------------------------------------------
\newcommand\InTexta{no text}
\newcommand\InTextb{no text}
\newcommand\InTextc{no text}
\newcommand\rnc{\renewcommand}


%-----------------------------------------------------
%										 	ENV warningbox
%---------------------------------------------------

\newtcolorbox{warningbox}[2][]
{
  colframe = uMainColor!25,
  colback  = black!05,
  coltitle = uMainColor!20!black,
  fonttitle = \bfseries, 
  title    = #2,
  #1,
}


\newtcolorbox{techworkbox}[2][]{sharp corners,colframe = uMainColor!25,colback  = black!05,coltitle = uMainColor!20!black,fonttitle = \bfseries, title={\textcolor{uMainColor}{\faFile~\small {Sujet de mémoire SECOPS}}} : #2,#1,}


\newtcolorbox{notebox2}[2][]
{
sharp corners, 
colback = uMainColor!5!white, 
colframe = uMainColor!75!black,
fonttitle = \bfseries, 
colbacktitle= uMainColor!85!black,
title=#2,#1
}

\newtcolorbox{notebox}[2][]{colback=uMainColor!5!white,
colframe=uMainColor!75!black,fonttitle=\bfseries,
colbacktitle=uMainColor!85!black,enhanced,
attach boxed title to top right={yshift=+1mm},
title=#2,#1}


\newtcolorbox{toolsboxin}[2][]{sharp corners, colback=uMainColor!0!white,
colframe= black!25,fonttitle=\bfseries,
colbacktitle=black!60,enhanced,
attach boxed title to top left={yshift=+1mm},
title=#2,#1}


\newtcolorbox{techworkin}[2][]{sharp corners, colback=uMainColor!5!white,
colframe= uMainColor!100, fonttitle=\bfseries,
colbacktitle=black!5, enhanced,
 coltitle=black!50,
attach boxed title to top right ={yshift=+1mm},
title=#2,#1}



%-----------------------------------------------------
%										 	ENV ToolsBox
%---------------------------------------------------


\newcommand\tooleditor{No EDITOR in Datas.tex}
\newcommand\toolurl{No URL INSERTED in Datas.tex}
\newcommand\toolclass{No EDITOR in Datas.tex}
\newcommand\toolname{No NAME in Datas.tex}
\newcommand\toolanalyst{No ANALYST in Datas.tex}

\newcommand{\toolsbox}[2]
{
\input{../Latex/Sources/EDU/SRC1/SecComponents/#2.#1/datas.tex} 
\begin{toolsboxin}[sidebyside,righthand width=.7\textwidth]{#1}
\includegraphics[width=0.9\textwidth]{../Latex/Sources/EDU/SRC1/SecComponents/#2.#1/icon.png} \\
\tcblower
\input{../Latex/Sources/EDU/SRC1/SecComponents/#2.#1/description.tex}
\end{toolsboxin}
\noindent \small {\faCogs~Classe : \uppercase{\textbf{#2}}, Site de référence : \textbf{\ulink{\toolurl}{#1}}}
\newline
\small {\faGlobe~Editeur : \textbf{\tooleditor}}
\small {\faEye~Analyste : \textbf{\toolanalyst}}
}

%\renewcommand\toolsclass{SIEM}
%\renewcommand\toolsname{RSA Netwitness}
%\renewcommand\toolseditor{RSA}
%\renewcommand\toolsurl{https://www.rsa.com/fr-fr/products/threat-detection-response}

%-----------------------------------------------------
%										 	ENV  TechWork
%---------------------------------------------------


\newcommand{\techwork}[2]
{
\begin{techworkin} [righthand width=.8\textwidth]{\faCogs~\small {Fiche TECHNO} : #1}#2
\end{techworkin}
}

%-----------------------------------------------------
%										 	\uchap
%---------------------------------------------------
\ifdefined\INTERTITLE
\providecommand{\uchap}[1]
	{
	\begin{LARGE}
	\textbf{\Ucenter{#1 \\ -oOo-}}	
	\end{LARGE}
	}
\else
	\providecommand{\uchap}[1]{}
\fi

%-----------------------------------------------------
%											\uexpand
%---------------------------------------------------
\newcommand {\uexpand}[1]{
\mode<all>\input{../Latex/Sources/EDU/SRC1/Chapters/#1}
}


%-----------------------------------------------------
%											\spicture 
% direct include graphic with Scrivener
%-----------------------------------------------------

\newcommand{\spicture}[3]{
\begin{figure}[!hbtp] %h !hbtp
  \begin{center}
	 \includegraphics[width=0.8\textwidth]{#1}
  \end{center}
\caption{#3 - (slbl-#2)\label{slbl-#2}}
\end{figure}
}

%\StrBefore{#2}{.pdf}

%-----------------------------------------------------
%											\upicture 
%---------------------------------------------------

\ifdefined\PRZMODE

\providecommand{\upicture}[4]{
\framesubtitle{#2}
%\begin{figure}[!h] %h
  \begin{center}
	 \includegraphics[height=#3\textheight, width=#3\textwidth, keepaspectratio ]{\includepicturesubpath{#1.pdf}}
  \end{center}
%  \end{figure}
}

\else

\providecommand{\upicture}[4]{
\begin{figure}[!hbtp] %h !hbtp
  \begin{center}
	 \includegraphics[width=#3\textwidth]{\includepicturesubpath{#1.pdf}}
  \end{center}
\caption{#2 \label{#4}}
\end{figure}
}
	
\fi


\ifdefined\PRZMODE

\providecommand{\uwpicture}[4]{
\framesubtitle{#2}
%\begin{figure}[!h] %h
  \begin{center}
	 \includegraphics[height=0.8\textheight, width=0.8\textwidth, keepaspectratio ]{\includepicturesubpath{#1.pdf}}
  \end{center}
%  \end{figure}
}
\else
\providecommand{\uwpicture}[4]
{
\begin{wrapfigure}{O}{#3\textwidth} % inside / Outside
  \begin{center}
	 \includegraphics[width=#3\textwidth]{\includepicturesubpath{#1.pdf}}
\end{center}
\caption{#2 \label{#4}}
\end{wrapfigure}
}
\fi

%-----------------------------------------------------
%	\picframe for BEAMER (Image included in Beamer Slide)
%---------------------------------------------------

\ifdefined\PRZMODE

\providecommand{\picframe}[4]
{
\begin{frame}
\frametitle<presentation>{#2}
  \begin{center}

   \begin{tikzpicture}[remember picture, overlay]
    \node at (current page.center) {\includegraphics[width=#3\paperwidth, height=#3\paperheight,keepaspectratio]{\includepicturesubpath{#1.pdf}}};
    \end{tikzpicture}

  \end{center}
\end{frame}

}
\else
\providecommand{\picframe}[4]
{
\begin{figure}[!hbtp] % !hbtp ou !h 
\centering 
%{\includegraphics[width=#3\paperwidth, height=#3\paperheight,keepaspectratio]{\includepicturesubpath{#1.pdf}}}
\includegraphics[width=#3\textwidth]{\includepicturesubpath{#1.pdf}}
\caption{#2 \label{#4}}
\end{figure}
}
\fi

%-----------------------------------------------------
%.  for BEAMER (Image included in Beamer Slide)
%	\picframeS Simplified with Name and Label from One Param
%---------------------------------------------------


\ifdefined\PRZMODE

\providecommand{\picframeS}[2]
{
\begin{frame}
\frametitle<presentation>{#2}
  \begin{center}
	 \includegraphics[height=0.95\textheight, width=0.95\textwidth, keepaspectratio ]{\picpath/img-#1.\graphicxextension}
  \end{center}
\end{frame}

}
\else
\providecommand{\picframeS}[2]
{
\begin{figure}[!hbtp] % !hbtp ou !h 
\centering \includegraphics[width=0.9\textwidth]{\picpath/img-#1.\graphicxextension}
\caption{(s) #2 \label{lbl:#1}}
\end{figure}
}

\fi


%{figure}[!h]
%{wrapfigure}{O}{#3\textwidth}
%-----------------------------------------------------
%	FRAME			\wpicframe 
%---------------------------------------------------

\ifdefined\PRZMODE

\providecommand{\wpicframe}[4]
{
\begin{frame}
\frametitle<presentation>{#2}
  \begin{center}
	 \includegraphics[height=0.95\textheight, width=0.95\textwidth, keepaspectratio ]{#1.\graphicxextension}
  \end{center}
\end{frame}

}
\else
\providecommand{\wpicframe}[4]
{
\begin{wrapfigure}{O}{#3\textwidth}
\centering \includegraphics[width=#3\textwidth]{#1.\graphicxextension}
\caption{#2 \label{#4}}
\end{wrapfigure}
}
\fi

%-----------------------------------------------------
%	FRAME										\texframe 
%---------------------------------------------------
\ifdefined\PRZMODE

\providecommand{\texframe}[3]
{
\begin{frame}
\frametitle<presentation>{#1}
\framesubtitle<presentation>{#2}
#3
\end{frame}
}
\else
\providecommand{\texframe}[3]
{
#3
}

\fi

%-----------------------------------------------------
%											\uref 
%---------------------------------------------------
%\newcommand{\uref}[2]{(Voir~#1~\cref{#2} page~\pageref{#2})}

\newcommand{\uref}[2]{(Voir~#1~\ref{#2} page~\pageref{#2})}
%-----------------------------------------------------
%											\ubg
%---------------------------------------------------
\newcommand{\ubg}[1]{
\textbf{\g{\uppercase{#1}}}
}
%-----------------------------------------------------
%											\updfimage 
%---------------------------------------------------

\newcommand{\updfimage}[3]
{
\begin{wrapfigure}{R}{#3\textwidth}
  \begin{center}
	 \resizebox{#3\textwidth}{!}{\includegraphics{\includepicturesubpath{#1.pdf}}}
  \end{center}
\caption{#2}
\end{wrapfigure}
}


%************************PACKAGE***

%\usepackage{tikz}% Required for drawing custom shapes
\usetikzlibrary{shadows}


%-----------------------------------------------------
%											 		\rem 
% insertion d'une remarque avec R
%---------------------------------------------------
\newcommand\rem[1]{%
   \marginpar{%
   \tikzpicture[baseline={(title.base)}]
      \node[inner sep=5pt,text width=4cm,drop shadow={shadow yshift=-5pt,shadow xshift=5pt,uMainColor},fill=white] (box) {\vskip5pt \nointerlineskip #1};
      \node[right=10pt,inner sep=0pt,outer sep=10pt] (title) at (box.north west) {\bfseries\color{uMainColor}Remarque};
      \draw[draw=uMainColor,very thick](title.west)--(box.north west)--(box.south west)--(box.south east)--(box.north east)--(title.east);
      \fill[uMainColor]([yshift=-10pt]box.north west)--+(-5pt,-5pt)--+(0pt,-10pt);
   \endtikzpicture}%
}
%-----------------------------------------------------
%											 \utikzimage 
% insertion d'une image sur fichier tkz.tex
%---------------------------------------------------

\newcommand{\utikzimage}[3]
{
\begin{wrapfigure}{R}{#3\textwidth}
  \begin{center}
	 \resizebox{#3\textwidth}{!}{\input{../#1.tkz.tex}}
  \end{center}
\caption{#2}
\end{wrapfigure}
}

%-----------------------------------------------------
%														\link 
% Url et footnote du lien
%-----------------------------------------------------

% TT every character and hyphenate after it
\def\hyphenateAndTtWholeString #1{\xHyphenate#1$\wholeString\unskip}

\def\xHyphenate#1#2\wholeString {\if#1$%
    \else\transform{#1}%
    \takeTheRest#2\ofTheString\fi}

\def\takeTheRest#1\ofTheString\fi
{\fi \xHyphenate#1\wholeString}

\def\transform#1{\url{#1}\hskip 0pt plus 1pt}

% Define the \urlx command which works like \url, but with line brakes
\def\urlx #1{\href{#1}{\hyphenateAndTtWholeString{#1}}}
  

 %-----------------------------------------------------
%														\link 
%-----------------------------------------------------  
  
  \newcommand{\ulink}[2]
  {\href{#1}
  {#2~\raisebox{-0.2ex}{\faExternalLink}\footnote{\url{#1}}}\xspace}
  
  \newcommand{\wikipedia}[1]
  { \href{https://fr.wikipedia.org/wiki/#1}{\textcolor{grey}{#1}}\raisebox{0.5ex}{\tiny{\textcolor{uMainColor}{\faWikipediaW}}}}
  


 %-----------------------------------------------------
%														\utodo 
%-----------------------------------------------------

  \newcommand{\utodo}{\begin{center}{\noindent \textcolor{cyan}{\faGears~{En cours de rédaction, DRAFT non publiable} \faGears}}\end{center}}
  
    \newcommand{\utocomplete}{\begin{center}{\noindent \textcolor{blue}{\faClipboard~{à compléter, les éléments ne donnent qu'une vue trop réduite ou trop parcellaire du sujet}}}\end{center}}
  

 





%-------------------------------------------------------
\usepackage{\utemplatepath/LessonModel/Beamer/\ubeamermodel.model/beamerthemeUfg}
%-------------------------------------------------------

% a charger après hyperref, pour avoir les guillemets sous beamer
\usepackage{ae}
\usepackage{aecompl}

\renewcommand{\familydefault}{pag}

\usetheme{Ufg}


%-------------------------------------------------------
% Optional: a subtitle to be displayed on the title slide
\title[\ushorttitle]{\utitle}
\subtitle{\tiny\uCourseLongName}

%-------------------------------------------------------
% The author(s) of the presentation:
%  - again first a short version to be displayed at the bottom;
%  - next the full list of authors, which may include contact information;

\author[\uauthorwriter]{ \LARGE{\uauthor} \\ \medskip {\small \url{\umaila} \and \url{\umailb}\\ {\small \url{\usitea}}}}

%-------------------------------------------------------
% The institute:
%  - to start the name of the university as displayed on the top of each slide
%    this can be adjusted such that you can also create a Dutch version
%  - next the institute information as displayed on the title slide

\institute[\uinstituteshort]{{\uinstitute \\ \uchaire}}


%-------------------------------------------------------
% Add a date and possibly the name of the event to the slides
%  - again first a short version to be shown at the bottom of each slide
%  - second the full date and event name for the title slide
\date[\uversion]{\tiny{Publication \INFODistrib ~du \\ \DTMnow}}


%-------------------------------------theorems--------------
\newtheorem{conj}{Conjetura}
\newtheorem{defi}{Definição}
\newtheorem{teo}{Teorema}
\newtheorem{lema}{Lema}
\newtheorem{prop}{Proposição}
\newtheorem{cor}{Corolário}
\newtheorem{ex}{Exemplo}
\newtheorem{exer}{Exercício}

\setbeamertemplate{theorems}[numbered]
\setbeamertemplate{caption}[numbered]

%-------------------------------------------------------------%
%----------------------- Primary Definitions -----------------%

% This command set the default Color, is also possible to choose a custom color
\setPrimaryColor{uMainColor} 

% First one is logo in title slide (we recommend use a horizontal image), and second one is the logo used in the remaining slides (we recommend use a square image)
\setLogos {\utemplatepath/LessonModel/Beamer/BeamerPictures/\utemplatetheme.theme/ufg/logos/infw.png} {\utemplatepath/LessonModel/Beamer/BeamerPictures/\utemplatetheme.theme/ufg/logos/infw2.png} 


% \AtBeginSection[]
% {
% \begin{frame}
% \frametitle{Sommaire}
% \tableofcontents[currentsection, hideothersubsections]
% \end{frame}
% }

\begin{document}

\mode<all>{\setLayout{titlepage}}
%================================
\begin{frame}[plain]
                 \titlepage
\end{frame}
%================================

\mode<all>{\setLayout{vertical}}
%================================
\begin{frame}
	\frametitle{Abstract}
	\begin{techworkin} [righthand width=.6\textwidth]{\faCogs~\small {Hashtags} : \ukeywords}\uabstract
	\end{techworkin}
	\end{frame}
%================================
   

\mode<all>{\setLayout{horizontal}}
%================================
\begin{frame}
\frametitle{Sommaire}
\begin{multicols}{2}
\tableofcontents[hideallsubsections]
\end{multicols}
\end{frame}
%================================

%-------------------------------------------------------
%  BEAMER MAIN TEX configured with ubody
%-------------------------------------------------------
\mode<all>{\setLayout{vertical}}

\mode<all>{\ubody}


\mode<all>{\setLayout{mainpoint}}
%-------------------------------------------------------
% Last Frame
%-------------------------------------------------------
\begin{frame}
\frametitle{des questions ?}
\framesubtitle{contacter \umaila}

\end{frame}

\mode<all>{\setLayout{vertical}}
%-------------------------------------------------------
% Contributions
%-------------------------------------------------------
\begin{frame}
\frametitle{Contributions}
\begin{columns}
\begin{column}{0.1\textwidth}

    \begin{figure}
           \centering
           \includegraphics[width=0.9\textwidth]{Latex/Sources/EDU/Contribs/contribute/ContribPictures/github.pdf}\\
                \includegraphics[width=0.9\textwidth]{Latex/Sources/EDU/Contribs/contribute/ContribPictures/overleaf.png}
 \end{figure}
\end{column}
\begin{column}{0.8\textwidth}
           %-------------------------------------------------------------
%               FR CYBERDEF SECOPS COURSE
%                            Contributions Rules
%                              2020 eduf@ction
%-------------------------------------------------------------

% \section{Contributions}

% \subsection{Comment contribuer}

Les notes et les présentations sont réalisées sous \hologo{LaTeX}. 

Vous pouvez contribuer au projet du cours CYBERDEF101. Les contributions peuvent se faire sous deux formes :

\begin{itemize}
  \item Corriger, amender, améliorer les notes publiées. A chaque session des modifications et évolutions sont apportées pour tenir compte des corrections de fond et de formes. 
  \item Ajouter, compléter, modifier des parties de notes sur la base de votre lecture du cours et de vos expertises dans chacun des domaines évoqués.
\end{itemize}

 Les fichiers sources sont publiés sur GITHUB dans l'espace : \ulink{https://github.com/edufaction/CYBERDEF101}{(edufaction/CYBERDEF101)}.
 
\end{column}
\end{columns}
\end{frame}


\mode<all>{\setLayout{titlepage}}
%-------------------------------------------------------
% Link to download file and copyright
%-------------------------------------------------------
\begin{frame}
\frametitle{Mises à jour régulières}
\framesubtitle{Eduf@ction~\umaila}
%\INFOGitHub
{\begin{center} 
\setstretch{2}
{Vérifiez la disponibilité d'une version plus récente de} \\
{\ulink{\GITfilename}{\textbf{\udocname.pdf }sur GITHUB CYBERDEF} }   \\ 
{{\huge\ccbyncndeu}}  \\  
{\the\year~eduf@ction - Publication en Creative Common BY-NC-ND }    \\  % Copyright notice
{\safeqrcode[padding]{\GITfilename}}  
\end{center} }
\end{frame}

\end{document}



}
\else

\ifdefined\EDTMODE 
	\providecommand{\umainload}{\input{\utemplatepath/LessonModel/Article/edt-article.tex}}
	\else
	\providecommand{\umainload}{%-------------------------------------------------------------
%               FR CYBERDEF SECOPS COURSE
%                          ARTICLE MAIN FILE
%                             2020 eduf@ction
%-------------------------------------------------------------

\documentclass[10pt,fleqn,twoside]{\utemplatepath/LessonModel/Article/edxstyle} %edxstyle

\usepackage{beamerarticle}

\usepackage{ae}
\usepackage[french]{babel}
%\usepackage[utf8]{inputenc}
%\usepackage[T1]{fontenc}
%\usepackage{lmodern}




\usepackage{titletoc} % Required for manipulating the table of contents

\input{\utemplatepath/commons.inc/CommonsPictures/\utemplatetheme.theme/colors.inc}

%-------------------------------------------------------------
%               FR CYBERDEF SECOPS COURSE
%                                  EDX PACKAGES
%.    Commons packages for BOOK, NOTES (.doc) 
%                            and BEAMER (.prz)
%                              2020 eduf@ction
%-------------------------------------------------------------

%\usepackage[hyphens,spaces,obeyspaces]{url}
%\usepackage[hyphens,spaces,obeyspaces]{url}



\PassOptionsToPackage{hyphens,spaces,obeyspaces}{url}

% \usepackage[utf8]{inputenc}

% \usepackage[T1]{fontenc}
%\usepackage{markdown}

\usepackage{fontspec}

\setsansfont{Arial}%{Arial}
\setmainfont{Arial}%{Times New Roman}
\setmonofont{Arial}%{Consolas}

\usepackage[french]{babel}

%\usepackage{cleveref} suppression ...

\usepackage{ccicons}

\usepackage{tikz} % Required for drawing custom shapes
\usetikzlibrary{tikzmark} 
\usetikzlibrary{mindmap,trees, backgrounds}


\usepackage{verbatim}
\usepackage{wrapfig}
\usepackage{xpatch}

\usepackage{hologo}

\usepackage[figure]{totalcount}

\usepackage{fontawesome}

\usepackage[skins]{tcolorbox} % don't forget SKINS option

\usepackage{listings}
\usepackage{upquote}
%\usepackage{xcolor} loaded by edxlibs

\definecolor{grey}{RGB}{50,80,80}

\lstset{
upquote=true,columns=flexible, basicstyle=\ttfamily,
language=HTML, 
frameround=tttt,
commentstyle=\color{gray},
identifierstyle=\color{blue},
keywordstyle=\color{uMainColor}\bfseries,
xleftmargin=2em,
xrightmargin=2em,
aboveskip=\topsep,
belowskip=\topsep, 
frame=single,
rulecolor=\color{uMainColor},
backgroundcolor=\color{uMainColor!5},
breaklines,
breakindent=1.5em,
showspaces=false,
showstringspaces=false,
showtabs=false,
xleftmargin=2em,
xrightmargin=2em,
aboveskip=1em,
belowskip=1em,
}



\usepackage{graphicx} % Required for including pictures
%\graphicspath{{Pictures/}} % Specifies the directory where pictures are stored

\usepackage{booktabs} % Required for nicer horizontal rules in tables

\usepackage{setspace}
\setstretch{1,2}

\usepackage{csquotes}

\usepackage{fancyhdr}

\usepackage{makeidx} 

\usepackage{xspace} 

\usepackage{floatflt} 

\usepackage{hyperref}


%pdfdisplaydoctitle={\utitle}, Attention le titre ne doit pas contenir de saut de ligne 

\hypersetup{
pdftitle={\uCoursetittle},
pdfauthor={\uauthor},
hidelinks,
backref=true,
pagebackref=true,
hyperindex=true,
pdfcenterwindow = true,
colorlinks=true,
breaklinks=true,
bookmarks=true,
allcolors=uMainColor,
pdfsubject = {\udescription},%
pdfcreator = {MACTEX on TEXPAD},%
bookmarksopen=true}
%


% urlcolor=cnam,
% runcolor=uMainColor,
%\pdfstringdefDisableCommands{%
%  \def\\{}%
%  \def\texttt#1{<#1>}%
%}

\usepackage[style=numeric,citestyle=numeric,sorting=nyt,sortcites=true,autopunct=true,hyperref=true,abbreviate=false,backref=true,backend=biber]{biblatex}

%\usepackage{cleveref} old position

\usepackage{multicol}

%--------------------------------------------------------
% pour la génération du QRCODE en page de garde d'article
\usepackage{qrcode} 

%--------------------------------------------------------
% POur la date d'impression en premiere page d'article
\usepackage[useregional]{datetime2}

%--------------------------------------------------------
% Gestion des glossaires et Sources/EDU/acronymes
% Un fichier acroymes.tex centralisé
\usepackage[acronym]{glossaries}

%\makeglossaries
\makenoidxglossaries 
\loadglsentries{../Latex/Sources/commons/acronymes.tex}
\loadglsentries{../Latex/Sources/commons/glossaire.tex}




%--------------------------------------------------------
% Pour exclure des "environnement", ici techworkbox par exemple, travaux pour étudiants, non affiché dans le BOOK.
\usepackage{comment}


%--------------------------------------------------------
% pour filigramme sur les pages
\usepackage{eso-pic}



\usepackage{etoolbox}

\usepackage{breakurl}



%-------------------------------------------------------------
%               FR CYBERDEF SECOPS COURSE
%                                  EDX PACKAGES
%.    Commons COMMANDS for BOOK, NOTES (.doc) 
%                            and BEAMER (.prz)
%                              2020 eduf@ction
%-------------------------------------------------------------

\newcommand{\ucite}[1]{\cite{#1}}
%\newcommand{\ucite}[1]{REF BIB #1}
\newcommand{\ugls}[1]{\gls{#1}}
%\newcommand{\ugls}[1]{GLOSSAIRE (#1)}
\newcommand{\uac}[1]{\acrshort{#1}}



% \newcommand{\ExtractBeforeLastSlash}[1]{\directlua{extract_before_last_slash("#1")}}
% \newcommand{\ExtractAfterLastSlash}[1]{\directlua{extract_after_last_slash("#1")}}
% \newcommand{\RemoveExtension}[1]{\directlua{remove_extension("#1")}}
% \newcommand{\ExtractExtension}[1]{\directlua{extract_extension("#1")}}


\newcommand{\includepicturesubpath}[1]
{#1}

%{\ExtractBeforeLastSlash{#1}/\upicturethemepath\RemoveExtension{\ExtractAfterLastSlash{#1}}.\ExtractExtension{#1}}
    

%---------------------------------------------------
%										 	AUDIO NOTE 
%---------------------------------------------------

\newcommand\audionote[2]{%
  \note{#2}
  \immediate\write\audioexport{==AudioSlide=============}
  \immediate\write\audioexport{#1}
  \immediate\write\audioexport{\insertframenumber}
  \immediate\write\audioexport{---------------------------------}
  \immediate\write\audioexport{#2}
  \immediate\write\audioexport{---------------------------------}
}%

%---------------------------------------------------
%										 	TEXT STYLE SHORTCUT 
%---------------------------------------------------

\newcommand{\tcb}[1]{\textbf{\textcolor{uMainColor}{#1}}}

\newcommand{\voir}[1]{(Schéma~\ref{#1},~page~\pageref{#1}}

\newcommand{\graphicxextension}{pdf}

\newcommand\myemptypage{
   
    \thispagestyle{empty}
    \addtocounter{page}{-1}
    \newpage
  \null
    %CECI EST UNE PAGE VIDE
       \vfill
    }


\newcommand{\uNA}[2]
{
\newacronym{a#1}{#1}{#2}
}

\newcommand{\coauthor}[4]
{
\noindent (#2) \textbf{#1} - #4 : \textit{#3}
}


\newcommand\UKword[1]{\emph{#1}\xspace}
\newcommand\textUK[1]{\textit{\textbf{#1}}\xspace}
\newcommand\Ucenter[1]{\begin{center}{#1}\end{center}}

\newcommand\head[1]{\textbf{#1}}
\newcommand\tb[1]{\textbf{#1}}
\newcommand{\g}[1]{\og #1 \fg{}}
\newcommand{\uload}[1] {\input{../Latex/#1}}
\newcommand{\ucurcolor}{black}

\newcommand{\mak}[1]{\faicon{\aArrowCircleORight} #1}


%-----------------------------------------------------
%										 	\INFOGitHub 
%---------------------------------------------------


\newcommand{\GITfilename}{https://github.com/edufaction/CYBERDEF101/raw/main/Release/Builder/\jobname.pdf}

%\newcommand{\safeqrcode}[2][]{%
%\qrcode[#1]{\detokenize{#2}} }

\newcommand{\safeqrcode}[2][]{\qrcode[#1]{#2}} 

\newcommand{\INFOGitHub}
{
\begin{center} 
\setstretch{1.5}
 {Vérifiez la disponibilité d'une version plus récente de} \\
 {\ulink{\GITfilename}{\textbf{\jobname.pdf }sur GITHUB CYBERDEF} }   \\ 
{{\huge\ccbyncndeu}}  \\  
{Publication en \href{https://creativecommons.org/licenses/by-nc-nd/2.0/fr/}{Creative Common BY-NC-ND} by eduf@ction}    \\  % Copyright notice
{\safeqrcode[height=4 cm,padding]{\GITfilename}}  
\end{center} 
}

\newcommand{\INFOGitHubS}
{
\noindent{\safeqrcode[height=3 cm]{\GITfilename}} \\
\noindent {\tiny Téléchargez une version à jour} 
}



%-----------------------------------------------------
%										 	\upspicture 
%---------------------------------------------------
\newcommand{\upspicture}[3]
{
\renewcommand{\ucurcolor}{#3}
\resizebox{#2\textwidth}{!}{\input{../#1}}
}


%-----------------------------------------------------
%										 	\uindex 
%---------------------------------------------------


\newcommand{\uindex}[1]
{
#1\index{#1}\xspace
}

\newcommand{\ulindex}[1]{\index{#1}}

%-----------------------------------------------------
%										 	\includer 
%---------------------------------------------------

\newcommand{\includer}[1]
{
\input{../Latex/Sources/EDU/SRC1/Chapters/#1}
}

%-----------------------------------------------------
%										 	\edxdico 
%---------------------------------------------------

\newcommand{\edxdico}[2] {\textbf{#1}:\textit{#2} \index{#1}}

%-----------------------------------------------------
%  command DEF : InTextx
%-----------------------------------------------------
\newcommand\InTexta{no text}
\newcommand\InTextb{no text}
\newcommand\InTextc{no text}
\newcommand\rnc{\renewcommand}


%-----------------------------------------------------
%										 	ENV warningbox
%---------------------------------------------------

\newtcolorbox{warningbox}[2][]
{
  colframe = uMainColor!25,
  colback  = black!05,
  coltitle = uMainColor!20!black,
  fonttitle = \bfseries, 
  title    = #2,
  #1,
}


\newtcolorbox{techworkbox}[2][]{sharp corners,colframe = uMainColor!25,colback  = black!05,coltitle = uMainColor!20!black,fonttitle = \bfseries, title={\textcolor{uMainColor}{\faFile~\small {Sujet de mémoire SECOPS}}} : #2,#1,}


\newtcolorbox{notebox2}[2][]
{
sharp corners, 
colback = uMainColor!5!white, 
colframe = uMainColor!75!black,
fonttitle = \bfseries, 
colbacktitle= uMainColor!85!black,
title=#2,#1
}

\newtcolorbox{notebox}[2][]{colback=uMainColor!5!white,
colframe=uMainColor!75!black,fonttitle=\bfseries,
colbacktitle=uMainColor!85!black,enhanced,
attach boxed title to top right={yshift=+1mm},
title=#2,#1}


\newtcolorbox{toolsboxin}[2][]{sharp corners, colback=uMainColor!0!white,
colframe= black!25,fonttitle=\bfseries,
colbacktitle=black!60,enhanced,
attach boxed title to top left={yshift=+1mm},
title=#2,#1}


\newtcolorbox{techworkin}[2][]{sharp corners, colback=uMainColor!5!white,
colframe= uMainColor!100, fonttitle=\bfseries,
colbacktitle=black!5, enhanced,
 coltitle=black!50,
attach boxed title to top right ={yshift=+1mm},
title=#2,#1}



%-----------------------------------------------------
%										 	ENV ToolsBox
%---------------------------------------------------


\newcommand\tooleditor{No EDITOR in Datas.tex}
\newcommand\toolurl{No URL INSERTED in Datas.tex}
\newcommand\toolclass{No EDITOR in Datas.tex}
\newcommand\toolname{No NAME in Datas.tex}
\newcommand\toolanalyst{No ANALYST in Datas.tex}

\newcommand{\toolsbox}[2]
{
\input{../Latex/Sources/EDU/SRC1/SecComponents/#2.#1/datas.tex} 
\begin{toolsboxin}[sidebyside,righthand width=.7\textwidth]{#1}
\includegraphics[width=0.9\textwidth]{../Latex/Sources/EDU/SRC1/SecComponents/#2.#1/icon.png} \\
\tcblower
\input{../Latex/Sources/EDU/SRC1/SecComponents/#2.#1/description.tex}
\end{toolsboxin}
\noindent \small {\faCogs~Classe : \uppercase{\textbf{#2}}, Site de référence : \textbf{\ulink{\toolurl}{#1}}}
\newline
\small {\faGlobe~Editeur : \textbf{\tooleditor}}
\small {\faEye~Analyste : \textbf{\toolanalyst}}
}

%\renewcommand\toolsclass{SIEM}
%\renewcommand\toolsname{RSA Netwitness}
%\renewcommand\toolseditor{RSA}
%\renewcommand\toolsurl{https://www.rsa.com/fr-fr/products/threat-detection-response}

%-----------------------------------------------------
%										 	ENV  TechWork
%---------------------------------------------------


\newcommand{\techwork}[2]
{
\begin{techworkin} [righthand width=.8\textwidth]{\faCogs~\small {Fiche TECHNO} : #1}#2
\end{techworkin}
}

%-----------------------------------------------------
%										 	\uchap
%---------------------------------------------------
\ifdefined\INTERTITLE
\providecommand{\uchap}[1]
	{
	\begin{LARGE}
	\textbf{\Ucenter{#1 \\ -oOo-}}	
	\end{LARGE}
	}
\else
	\providecommand{\uchap}[1]{}
\fi

%-----------------------------------------------------
%											\uexpand
%---------------------------------------------------
\newcommand {\uexpand}[1]{
\mode<all>\input{../Latex/Sources/EDU/SRC1/Chapters/#1}
}


%-----------------------------------------------------
%											\spicture 
% direct include graphic with Scrivener
%-----------------------------------------------------

\newcommand{\spicture}[3]{
\begin{figure}[!hbtp] %h !hbtp
  \begin{center}
	 \includegraphics[width=0.8\textwidth]{#1}
  \end{center}
\caption{#3 - (slbl-#2)\label{slbl-#2}}
\end{figure}
}

%\StrBefore{#2}{.pdf}

%-----------------------------------------------------
%											\upicture 
%---------------------------------------------------

\ifdefined\PRZMODE

\providecommand{\upicture}[4]{
\framesubtitle{#2}
%\begin{figure}[!h] %h
  \begin{center}
	 \includegraphics[height=#3\textheight, width=#3\textwidth, keepaspectratio ]{\includepicturesubpath{#1.pdf}}
  \end{center}
%  \end{figure}
}

\else

\providecommand{\upicture}[4]{
\begin{figure}[!hbtp] %h !hbtp
  \begin{center}
	 \includegraphics[width=#3\textwidth]{\includepicturesubpath{#1.pdf}}
  \end{center}
\caption{#2 \label{#4}}
\end{figure}
}
	
\fi


\ifdefined\PRZMODE

\providecommand{\uwpicture}[4]{
\framesubtitle{#2}
%\begin{figure}[!h] %h
  \begin{center}
	 \includegraphics[height=0.8\textheight, width=0.8\textwidth, keepaspectratio ]{\includepicturesubpath{#1.pdf}}
  \end{center}
%  \end{figure}
}
\else
\providecommand{\uwpicture}[4]
{
\begin{wrapfigure}{O}{#3\textwidth} % inside / Outside
  \begin{center}
	 \includegraphics[width=#3\textwidth]{\includepicturesubpath{#1.pdf}}
\end{center}
\caption{#2 \label{#4}}
\end{wrapfigure}
}
\fi

%-----------------------------------------------------
%	\picframe for BEAMER (Image included in Beamer Slide)
%---------------------------------------------------

\ifdefined\PRZMODE

\providecommand{\picframe}[4]
{
\begin{frame}
\frametitle<presentation>{#2}
  \begin{center}

   \begin{tikzpicture}[remember picture, overlay]
    \node at (current page.center) {\includegraphics[width=#3\paperwidth, height=#3\paperheight,keepaspectratio]{\includepicturesubpath{#1.pdf}}};
    \end{tikzpicture}

  \end{center}
\end{frame}

}
\else
\providecommand{\picframe}[4]
{
\begin{figure}[!hbtp] % !hbtp ou !h 
\centering 
%{\includegraphics[width=#3\paperwidth, height=#3\paperheight,keepaspectratio]{\includepicturesubpath{#1.pdf}}}
\includegraphics[width=#3\textwidth]{\includepicturesubpath{#1.pdf}}
\caption{#2 \label{#4}}
\end{figure}
}
\fi

%-----------------------------------------------------
%.  for BEAMER (Image included in Beamer Slide)
%	\picframeS Simplified with Name and Label from One Param
%---------------------------------------------------


\ifdefined\PRZMODE

\providecommand{\picframeS}[2]
{
\begin{frame}
\frametitle<presentation>{#2}
  \begin{center}
	 \includegraphics[height=0.95\textheight, width=0.95\textwidth, keepaspectratio ]{\picpath/img-#1.\graphicxextension}
  \end{center}
\end{frame}

}
\else
\providecommand{\picframeS}[2]
{
\begin{figure}[!hbtp] % !hbtp ou !h 
\centering \includegraphics[width=0.9\textwidth]{\picpath/img-#1.\graphicxextension}
\caption{(s) #2 \label{lbl:#1}}
\end{figure}
}

\fi


%{figure}[!h]
%{wrapfigure}{O}{#3\textwidth}
%-----------------------------------------------------
%	FRAME			\wpicframe 
%---------------------------------------------------

\ifdefined\PRZMODE

\providecommand{\wpicframe}[4]
{
\begin{frame}
\frametitle<presentation>{#2}
  \begin{center}
	 \includegraphics[height=0.95\textheight, width=0.95\textwidth, keepaspectratio ]{#1.\graphicxextension}
  \end{center}
\end{frame}

}
\else
\providecommand{\wpicframe}[4]
{
\begin{wrapfigure}{O}{#3\textwidth}
\centering \includegraphics[width=#3\textwidth]{#1.\graphicxextension}
\caption{#2 \label{#4}}
\end{wrapfigure}
}
\fi

%-----------------------------------------------------
%	FRAME										\texframe 
%---------------------------------------------------
\ifdefined\PRZMODE

\providecommand{\texframe}[3]
{
\begin{frame}
\frametitle<presentation>{#1}
\framesubtitle<presentation>{#2}
#3
\end{frame}
}
\else
\providecommand{\texframe}[3]
{
#3
}

\fi

%-----------------------------------------------------
%											\uref 
%---------------------------------------------------
%\newcommand{\uref}[2]{(Voir~#1~\cref{#2} page~\pageref{#2})}

\newcommand{\uref}[2]{(Voir~#1~\ref{#2} page~\pageref{#2})}
%-----------------------------------------------------
%											\ubg
%---------------------------------------------------
\newcommand{\ubg}[1]{
\textbf{\g{\uppercase{#1}}}
}
%-----------------------------------------------------
%											\updfimage 
%---------------------------------------------------

\newcommand{\updfimage}[3]
{
\begin{wrapfigure}{R}{#3\textwidth}
  \begin{center}
	 \resizebox{#3\textwidth}{!}{\includegraphics{\includepicturesubpath{#1.pdf}}}
  \end{center}
\caption{#2}
\end{wrapfigure}
}


%************************PACKAGE***

%\usepackage{tikz}% Required for drawing custom shapes
\usetikzlibrary{shadows}


%-----------------------------------------------------
%											 		\rem 
% insertion d'une remarque avec R
%---------------------------------------------------
\newcommand\rem[1]{%
   \marginpar{%
   \tikzpicture[baseline={(title.base)}]
      \node[inner sep=5pt,text width=4cm,drop shadow={shadow yshift=-5pt,shadow xshift=5pt,uMainColor},fill=white] (box) {\vskip5pt \nointerlineskip #1};
      \node[right=10pt,inner sep=0pt,outer sep=10pt] (title) at (box.north west) {\bfseries\color{uMainColor}Remarque};
      \draw[draw=uMainColor,very thick](title.west)--(box.north west)--(box.south west)--(box.south east)--(box.north east)--(title.east);
      \fill[uMainColor]([yshift=-10pt]box.north west)--+(-5pt,-5pt)--+(0pt,-10pt);
   \endtikzpicture}%
}
%-----------------------------------------------------
%											 \utikzimage 
% insertion d'une image sur fichier tkz.tex
%---------------------------------------------------

\newcommand{\utikzimage}[3]
{
\begin{wrapfigure}{R}{#3\textwidth}
  \begin{center}
	 \resizebox{#3\textwidth}{!}{\input{../#1.tkz.tex}}
  \end{center}
\caption{#2}
\end{wrapfigure}
}

%-----------------------------------------------------
%														\link 
% Url et footnote du lien
%-----------------------------------------------------

% TT every character and hyphenate after it
\def\hyphenateAndTtWholeString #1{\xHyphenate#1$\wholeString\unskip}

\def\xHyphenate#1#2\wholeString {\if#1$%
    \else\transform{#1}%
    \takeTheRest#2\ofTheString\fi}

\def\takeTheRest#1\ofTheString\fi
{\fi \xHyphenate#1\wholeString}

\def\transform#1{\url{#1}\hskip 0pt plus 1pt}

% Define the \urlx command which works like \url, but with line brakes
\def\urlx #1{\href{#1}{\hyphenateAndTtWholeString{#1}}}
  

 %-----------------------------------------------------
%														\link 
%-----------------------------------------------------  
  
  \newcommand{\ulink}[2]
  {\href{#1}
  {#2~\raisebox{-0.2ex}{\faExternalLink}\footnote{\url{#1}}}\xspace}
  
  \newcommand{\wikipedia}[1]
  { \href{https://fr.wikipedia.org/wiki/#1}{\textcolor{grey}{#1}}\raisebox{0.5ex}{\tiny{\textcolor{uMainColor}{\faWikipediaW}}}}
  


 %-----------------------------------------------------
%														\utodo 
%-----------------------------------------------------

  \newcommand{\utodo}{\begin{center}{\noindent \textcolor{cyan}{\faGears~{En cours de rédaction, DRAFT non publiable} \faGears}}\end{center}}
  
    \newcommand{\utocomplete}{\begin{center}{\noindent \textcolor{blue}{\faClipboard~{à compléter, les éléments ne donnent qu'une vue trop réduite ou trop parcellaire du sujet}}}\end{center}}
  

 




\input{\utemplatepath/commons.inc/edx.styles.tex}

%\definecolor{uMainColor}{RGB}{255,128,0}   



%-------------------------------------------------
%	BIBLIOGRAPHY
%-------------------------------------------------

\addbibresource{../Latex/Sources/commons/bibliography.bib} % BibTeX bibliography file
\defbibheading{bibempty}{}

%-------------------------------------------------
%	COLORS & BORDERS
%-------------------------------------------------
\setlength{\fboxrule}{0.75pt} % Width of the border around the abstract
%\definecolor{uMainColor}{RGB}{255,128,0} % Color of the article title and section
\definecolor{color2}{RGB}{220,220,220} % Color of the boxes behind the abstract and headings

%-------------------------------------------------
% ITEMS DEFINITION
%------------------------------------------------
%\usepackage{enumitem}

\setlist [itemize,1]{label=\color{uMainColor}\faCaretRight }


\usepackage{environ}
\newwrite\audioexport



\begin{document}





  %-------------------------------------------------------
  % AudioNote Open File
  %-------------------------------------------------------
\mode<all>{\immediate\openout\audioexport=\jobname.txt}


\addtocontents{toc}{\protect\hypertarget{toc}{}}

%-------------------------------------------------
%	ARTICLE INFORMATION
%-------------------------------------------------

\Abstract {\uabstract%===========================
% COURS "INTRO CYBERDEF101"
% Abstract général des articles
%===========================


Il fait partie du cours introductif aux fondamentaux de la sécurité des systèmes d'information, de la cybersécurité, et de la cyberdéfense.
Le cours est constitué d'un ensemble de notes de synthèse indépendantes compilées en un document unique, mais édité par chapitre dans le cadre de ce cours.\\
C\edoc ne constitue pas à lui seul le référentiel du cours CYBERDEF101. Il compile des notes de cours mises à disposition de l'auditeur comme support pédagogique partiel à ce cours introductif à la cyberdéfense d'entreprise. }
\JournalInfo{\uJournalInfo} % Journal information
\Archive{Notes de cours éditées le  \DTMnow} % Additional notes (e.g. copyright, DOI, review/research article)
\PaperTitle{\utitle} % Article title
\Authors{\uauthor\textsuperscript{1,}\textsuperscript{2}*} % Authors
\affiliation{\textsuperscript{1}\textit{\uproa}} % Author affiliation
\affiliation{\textsuperscript{2}\textit{\uprob}} % Author affiliation
\affiliation{*\textbf{email}: \umaila\ -- \umailb} % Corresponding author
\Keywords{\ukeywords} % Keywords 
\newcommand{\keywordname}{Mots clefs} % Defines the keywords heading name

%-------------------------------------------------
%	MAKE TITLE
%-------------------------------------------------

\maketitle

\begin{tikzpicture}[remember picture,overlay]\node at (current page.north west)[xshift=3cm, yshift=-3cm, text opacity=1.0]{\includegraphics[width=0.1\paperwidth]{\utemplatepath/commons.inc/CommonsPictures/\utemplatetheme.theme/shield-20.pdf}}; 
\end{tikzpicture} 


\AddToShipoutPicture{%
\AtPageCenter{%
\begin{tikzpicture}[remember picture,overlay]
\node [rotate=60,scale=5,text opacity=0.1] at (current page.center) { \textit{\color{uAlertColor}\INFODistrib} };
\end{tikzpicture} 
%\begin{tikzpicture}[remember picture,overlay]
%\node [rotate=60,scale=10,text opacity=0.3]
%at (current page.center) {moi};
%\end{tikzpicture}
}%
}


%-------------------------------------------------	
\begin{center}
	\textcolor{uMainColor}{\textbf{\INFODistrib}} % Distribution information
\end{center}

\INFOGitHub % Link to download file and copyright
%-------------------------------------------------

%-------------------------------------------------
	\newpage
	\tableofcontents  % TOC
	\iftotalfigures
  			\listoffigures
	\fi
%-------------------------------------------------

%-------------------------------------------------
\newpage
\ubody % MAIN BODY defined external
%-------------------------------------------------

	\printbibliography
 \printglossary[type=\acronymtype]
%	\printglossary
%-------------------------------------------------	
\newpage
\Ucontribute % Contribution TEXT
%-------------------------------------------------

  %-------------------------------------------------------
  % AudioNote Open File
  %-------------------------------------------------------
  \mode<all>{\immediate\closeout\audioexport}

\end{document}
}
	\fi
\fi


 
\lessonID{7.0}
%************************************************************
% Chargement des variables dédiées à l'article
%************************************************************

\setauthor{EDU}

%========================================
\newcommand {\ukeywords}
%========================================
{%---------------------------------------------------------------------
Anticipation, veille, alerte, réponse, CERT, SOC
}%

%========================================
\newcommand {\utitle}
{%---------------------------------------------------------------------
Introduction à la SECOPS
}%---------------------------------------------------------------------

%========================================
\newcommand {\uabstract}
{%---------------------------------------------------------------------
C\edoc introduit le triptype de la partie cyberdéfense de la sécurité opérationnelle : Anticiper, Détecter, Réagir  et ceci sur les trois grands invariants des risques numériques : les vulnérabilités, les menaces et l'impact.
}
%---------------------------------------------------------------------

%************************************************************
%  variable définissant  le corps de l'article
%************************************************************

%========================================
\newcommand {\ubody}
{%---------------------------------------------------------------------
%$BEGIN_UBODY


\mode<all>{\input{../Latex/Sources/EDU/SRC1/Chapters/chap-VTI-intro.tex}}



%$END_UBODY
}%---------------------------------------------------------------------

%************************************************************
% Chargement  du MODELE
%************************************************************

\umainload



