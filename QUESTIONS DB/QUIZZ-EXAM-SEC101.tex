\documentclass[12pt]{article}

% \usepackage[draft]{moodle}
 \usepackage{moodle}
\usepackage{ae}
%\usepackage[french]{babel}
\usepackage[utf8]{inputenc}
\usepackage[T1]{fontenc}
\usepackage{lmodern}




 \begin{document}

E. Dupuis - SEC101 - Quiz generated \LaTeX's \textsf{moodle} (\moodleversion, \moodledate). Import the derived file \texttt{\jobname-moodle.xml} on Moodle.
  
\begin{quiz}{SEC101 PACK 2024 Questions}

  \begin{multi}[points=1, tags={M1Sec101, Risques}]{Sécurité des Systèmes d'Information}
    Quelle est la première étape du processus de gestion des risques en sécurité de l'information ?
    \item Analyse des menaces
    \item Identification des actifs
    \item Évaluation des vulnérabilités
    \item* Identification des risques
  \end{multi}

  \begin{multi}[points=1,tags={M2Sec101, PSSI}]{Politiques de Sécurité}
    Quelle politique de sécurité définit les règles de base pour l'utilisation des ressources informatiques d'une organisation ?
    \item* Politique d'utilisation acceptable
    \item Politique de cryptographie
    \item Politique de pare-feu
    \item Politique de sauvegarde
  \end{multi}

  \begin{multi}[points=1]{Cybersécurité}
    Quelle est la principale menace à la sécurité liée à l'ingénierie sociale ?
    \item Malware
    \item* Manipulation psychologique
    \item Attaque par déni de service (DDoS)
    \item Vol de données
  \end{multi}

  \begin{multi}[points=1]{Architectures de Sécurité Informatique}
    Quel composant d'un pare-feu est responsable de la vérification des paquets de données entrants et sortants ?
    \item* Filtre de paquets
    \item Processeur principal
    \item Antivirus intégré
    \item Répartiteur de charge
  \end{multi}

  \begin{multi}[points=1]{Sécurité des Systèmes d'Information}
    Quelle norme de sécurité définit les exigences pour la gestion des informations sensibles par les organisations ?
    \item ISO 9001
    \item ISO 14001
    \item* ISO 27001
    \item ISO 22000
  \end{multi}

  \begin{multi}[points=1]{Politiques de Sécurité}
    Quelle est la principale fonction d'une politique de pare-feu ?
    \item Gérer les identités des utilisateurs
    \item Gérer les mises à jour logicielles
    \item* Contrôler le trafic réseau
    \item Gérer les politiques de sauvegarde
  \end{multi}

  \begin{multi}[points=1]{Cybersécurité}
    Quelle méthode de cryptographie utilise une clé publique et une clé privée pour chiffrer et déchiffrer les données ?
    \item* Cryptographie asymétrique
    \item Cryptographie symétrique
    \item Cryptographie par transposition
    \item Cryptographie quantique
  \end{multi}

  \begin{multi}[points=1]{Architectures de Sécurité Informatique}
    Quelle technologie est utilisée pour authentifier un utilisateur en utilisant des empreintes digitales, des iris ou des caractéristiques biométriques similaires ?
    \item* Biométrie
    \item Token d'authentification
    \item Mot de passe
    \item Carte à puce
  \end{multi}

  \begin{multi}[points=1]{Sécurité des Systèmes d'Information}
    Quelle est la meilleure pratique pour se prémunir contre les attaques par phishing ?
    \item Utiliser un pare-feu
    \item* Sensibiliser les employés à la sécurité
    \item Activer la détection d'intrusion
    \item Mettre en place un VPN
  \end{multi}

  \begin{multi}[points=1]{Politiques de Sécurité}
    Quelle politique de sécurité définit les règles de stockage, de gestion et de protection des mots de passe ?
    \item* Politique de gestion des mots de passe
    \item Politique de pare-feu
    \item Politique de cryptographie
    \item Politique d'utilisation acceptable
  \end{multi}
  
  % Question 11
  \begin{multi}[points=1]{Cybersécurité}
    Quel type d'attaque vise à saturer un réseau ou un service en submergeant de trafic malveillant ?
    \item Attaque de phishing
    \item Attaque par injection SQL
    \item Attaque par déni de service distribué (DDoS)
    \item* Attaque par déni de service (DoS)
  \end{multi}

  % Question 12
  \begin{multi}[points=1]{Architectures de Sécurité Informatique}
    Quel dispositif de sécurité est conçu pour empêcher les intrusions non autorisées en surveillant le trafic réseau et en bloquant les menaces potentielles ?
    \item* IDS (Système de Détection d'Intrusion)
    \item Antivirus
    \item Firewall
    \item VPN
  \end{multi}

  % Question 13
  \begin{multi}[points=1]{Sécurité des Systèmes d'Information}
    Quelle couche du modèle OSI est principalement responsable de la sécurité des données en transit ?
    \item* Couche application
    \item Couche transport
    \item Couche liaison de données
    \item* Couche réseau
  \end{multi}

  % Question 14
  \begin{multi}[points=1]{Politiques de Sécurité}
    Quelle politique de sécurité permet de spécifier les niveaux d'accès aux ressources informatiques en fonction des rôles des utilisateurs ?
    \item Politique de sauvegarde
    \item* Politique de contrôle d'accès
    \item Politique de cryptographie
    \item Politique d'utilisation acceptable
  \end{multi}

  % Question 15
  \begin{multi}[points=1]{Cybersécurité}
    Quelle est la principale menace associée aux logiciels malveillants qui chiffrent les fichiers d'un utilisateur et demandent une rançon pour les déverrouiller ?
    \item* Ransomware
    \item Spyware
    \item Adware
    \item Worm
  \end{multi}

  % Question 16
  \begin{multi}[points=1]{Architectures de Sécurité Informatique}
    Quel composant réseau est utilisé pour segmenter un réseau en zones isolées afin de limiter la propagation des menaces ?
    \item Routeur
    \item Commutateur
    \item* Pare-feu de zone
    \item Répéteur
  \end{multi}

  % Question 17
  \begin{multi}[points=1]{Sécurité des Systèmes d'Information}
    Quelle méthode de chiffrement convertit chaque caractère individuellement dans un message en un caractère chiffré ?
    \item Chiffrement de flux
    \item* Chiffrement par substitution
    \item Chiffrement de bloc
    \item Chiffrement asymétrique
  \end{multi}

  % Question 18
  \begin{multi}[points=1]{Politiques de Sécurité}
    Quelle action est recommandée pour une politique de sécurité en cas de perte ou de vol d'un dispositif mobile contenant des données sensibles ?
    \item Ignorer la situation
    \item Révoquer les privilèges d'accès aux utilisateurs
    \item* Signaler immédiatement la perte ou le vol
    \item Effectuer une sauvegarde des données
  \end{multi}

  % Question 19
  \begin{multi}[points=1]{Cybersécurité}
    Quel type d'attaque tente d'exploiter les faiblesses connues dans les logiciels ou les systèmes pour prendre le contrôle de ces derniers ?
    \item Attaque de phishing
    \item Attaque par déni de service (DDoS)
    \item Attaque par ingénierie sociale
    \item* Exploitation de vulnérabilités
  \end{multi}

  % Question 20
  \begin{multi}[points=1]{Architectures de Sécurité Informatique}
    Quelle technologie est utilisée pour surveiller en temps réel le trafic réseau à la recherche de comportements suspects ou malveillants ?
    \item* SIEM (Security Information and Event Management)
    \item Antivirus
    \item Pare-feu
    \item VPN
  \end{multi}

% Question 21
\begin{multi}[points=1]{Sécurité des Systèmes d'Information}
  Quel est l'objectif principal d'un système de gestion des identités (IdM) ?
  \item* Gérer les autorisations et l'authentification des utilisateurs
  \item Gérer les mises à jour logicielles
  \item Surveiller le trafic réseau
  \item Effectuer des analyses de vulnérabilité
\end{multi}

% Question 22
\begin{multi}[points=1]{Politiques de Sécurité}
  Quel type de politique de sécurité spécifie les règles concernant le stockage, l'accès et la distribution des informations classifiées ?
  \item Politique de contrôle d'accès
  \item Politique de pare-feu
  \item* Politique de classification de l'information
  \item Politique de chiffrement
\end{multi}

% Question 23
\begin{multi}[points=1]{Cybersécurité}
  Quelle est la meilleure pratique pour se protéger contre les attaques par force brute visant les mots de passe ?
  \item Utiliser un pare-feu robuste
  \item Ne pas utiliser de mots de passe
  \item* Utiliser des mots de passe forts et un mécanisme de verrouillage après plusieurs tentatives infructueuses
  \item Activer la détection d'intrusion
\end{multi}

% Question 24
\begin{multi}[points=1]{Architectures de Sécurité Informatique}
  Quelle technologie permet de masquer l'adresse IP réelle d'un dispositif en la remplaçant par une adresse IP publique ?
  \item VPN
  \item IDS
  \item* NAT (Network Address Translation)
  \item Pare-feu
\end{multi}

% Question 25
\begin{multi}[points=1]{Sécurité des Systèmes d'Information}
  Quelle norme de sécurité est spécifiquement conçue pour protéger les informations de santé des patients dans le secteur de la santé ?
  \item ISO 9001
  \item* HIPAA (USA)(Health Insurance Portability and Accountability Act) ou ISO 27799(Europe)
  \item ISO 27001
  \item GDPR (General Data Protection Regulation)
\end{multi}

% Question 26
\begin{multi}[points=1]{Politiques de Sécurité}
  Quelle est la première étape dans la gestion d'une violation de données ?
  \item Notifier les autorités gouvernementales
  \item* Identifier la source de la violation et la contenir
  \item Communiquer immédiatement à tous les employés
  \item Engager une équipe de sécurité externe
\end{multi}

% Question 27
\begin{multi}[points=1]{Cybersécurité}
  Quelle est la principale menace pour la sécurité des objets connectés (IoT) ?
  \item Attaques par déni de service (DDoS)
  \item* Manque de mise à jour des logiciels et des micrologiciels
  \item Phishing ciblé
  \item Utilisation de mots de passe forts
\end{multi}

% Question 28
\begin{multi}[points=1]{Architectures de Sécurité Informatique}
  Quelle méthode de chiffrement utilise une seule clé pour chiffrer et déchiffrer les données ?
  \item* Cryptographie symétrique
  \item Cryptographie asymétrique
  \item Cryptographie par substitution
  \item Cryptographie par transposition
\end{multi}

% Question 29
\begin{multi}[points=1]{Sécurité des Systèmes d'Information}
  Quel type d'attaque vise à tromper les utilisateurs en leur faisant croire qu'ils interagissent avec un site web légitime alors qu'il s'agit d'une fausse copie ?
  \item Attaque de force brute
  \item Attaque par déni de service (DDoS)
  \item* Attaque de phishing
  \item Attaque par injection SQL
\end{multi}

% Question 30
\begin{multi}[points=1]{Politiques de Sécurité}
  Quelle politique de sécurité spécifie les règles pour la gestion des certificats numériques utilisés dans les communications sécurisées ?
  \item* Politique de gestion des certificats
  \item Politique de contrôle d'accès
  \item Politique de chiffrement
  \item Politique de classification de l'information
\end{multi}


% Question 31
\begin{multi}[points=1]{Sécurité des Systèmes d'Information}
    Quel terme désigne une technique visant à vérifier l'authenticité d'un utilisateur en demandant quelque chose qu'il sait (comme un mot de passe) et quelque chose qu'il possède (comme un smartphone) ?
    \item* Authentification à deux facteurs
    \item Authentification biométrique
    \item Authentification unique (SSO)
  \end{multi}

  % Question 32
  \begin{multi}[points=1]{Politiques de Sécurité}
    Quelle politique de sécurité concerne la surveillance et la gestion des journaux d'activité pour détecter les incidents de sécurité ?
    \item* Politique de gestion des journaux (log)
    \item Politique de cryptographie
    \item Politique de contrôle d'accès
    \item Politique de pare-feu
  \end{multi}

  % Question 33
  \begin{multi}[points=1]{Cybersécurité}
    Quelle technique de sécurité permet de cacher les données sensibles dans un fichier ou un message sans altérer son apparence externe ?
    \item Chiffrement asymétrique
    \item Cryptographie par substitution
    \item* Stéganographie
    \item Vigenère cipher
  \end{multi}

  % Question 34
  \begin{multi}[points=1]{Architectures de Sécurité Informatique}
    Quel mécanisme de sécurité permet de garantir que les données n'ont pas été modifiées en transit entre l'expéditeur et le destinataire ?
    \item* Intégrité des données
    \item Confidentialité des données
    \item Disponibilité des données
    \item Authentification des données
  \end{multi}

  % Question 35
  \begin{multi}[points=1]{Sécurité des Systèmes d'Information}
    Quel protocole de sécurité réseau est couramment utilisé pour établir des connexions VPN sécurisées ?
    \item* IPsec
    \item SSH
    \item HTTP
    \item RDP
  \end{multi}

  % Question 36
  \begin{multi}[points=1]{Politiques de Sécurité}
    Quelle est la principale raison de l'application de mises à jour de sécurité régulières sur un système ?
    \item Améliorer les performances
    \item Ajouter de nouvelles fonctionnalités
    \item Réduire les coûts d'exploitation
    \item* Corriger les vulnérabilités connues
  \end{multi}

  % Question 37
  \begin{multi}[points=1]{Cybersécurité}
    Quelle est la méthode la plus courante pour protéger un réseau sans fil (Wi-Fi) contre les accès non autorisés ?
    \item Chiffrement par substitution
    \item Pare-feu de zone
    \item* Chiffrement WPA/WPA2
    \item Biométrie
  \end{multi}

  % Question 38
  \begin{multi}[points=1]{Architectures de Sécurité Informatique}
    Quel type de pare-feu permet de surveiller le trafic entrant et sortant en analysant son contenu pour détecter les menaces ?
    \item Pare-feu de zone
    \item Pare-feu d'application
    \item* Pare-feu de nouvelle génération (NGFW)
    \item Pare-feu d'état
  \end{multi}

  % Question 39
  \begin{multi}[points=1]{Sécurité des Systèmes d'Information}
    Quelle méthode de sécurité consiste à rendre illisible une information en la transformant en un format inintelligible, réversible uniquement avec une clé de déchiffrement ?
    \item Stéganographie
    \item* Chiffrement
    \item Biométrie
    \item Token d'authentification
  \end{multi}

  % Question 40
  \begin{multi}[points=1]{Politiques de Sécurité}
    Quelle politique de sécurité se concentre sur la définition des procédures et des responsabilités en cas d'incident de sécurité majeur ?
    \item Politique de contrôle d'accès
    \item Politique de sauvegarde
    \item* Politique de gestion des incidents de sécurité
    \item Politique de pare-feu
  \end{multi}


  % Question 41
  \begin{multi}[points=1]{Cybersécurité}
    Quelle mesure de sécurité consiste à supprimer complètement l'accès aux ressources d'un utilisateur lorsqu'il quitte une organisation ou un projet ?
    \item* Décommissionnement du compte
    \item Réinitialisation du mot de passe
    \item Changement du nom d'utilisateur
    \item Mise en quarantaine du compte
  \end{multi}

  % Question 42
  \begin{multi}[points=1]{Architectures de Sécurité Informatique}
    Quel type de pare-feu examine le trafic réseau à la recherche de comportements anormaux et peut prendre des mesures en temps réel pour bloquer les menaces ?
    \item Pare-feu d'état
    \item Pare-feu de zone
    \item Pare-feu d'application
    \item* Pare-feu comportemental (BFW)
  \end{multi}

  % Question 43
  \begin{multi}[points=1]{Sécurité des Systèmes d'Information}
    Quelle mesure de sécurité consiste à conserver une copie identique des données à un instant donné pour une éventuelle restauration en cas de perte de données ?
    \item* Sauvegarde
    \item Cryptographie
    \item Archivage
    \item Désactivation du compte
  \end{multi}

  % Question 44
  \begin{multi}[points=1]{Politiques de Sécurité}
    Quel concept de sécurité informatique vise à minimiser les risques en distribuant les ressources et les données sur plusieurs serveurs ou emplacements géographiques ?
    \item* Redondance
    \item Pare-feu
    \item Authentification unique (SSO)
    \item Chiffrement
  \end{multi}

  % Question 45
  \begin{multi}[points=1]{Cybersécurité}
    Quelle technique de protection de réseau identifie et isole automatiquement les appareils non conformes ou malveillants ?
    \item Cryptographie quantique
    \item* Network Access Control (NAC)
    \item SIEM (Security Information and Event Management)
    \item Pare-feu d'application
  \end{multi}

  % Question 46
  \begin{multi}[points=1]{Architectures de Sécurité Informatique}
    Quel type d'attaque consiste à intercepter et à détourner les communications entre deux parties pour les espionner ?
    \item Attaque par déni de service (DoS)
    \item* Attaque de l'homme du milieu (Man-in-the-Middle)
    \item Attaque par injection SQL
    \item Attaque par force brute
  \end{multi}

  % Question 47
  \begin{multi}[points=1]{Sécurité des Systèmes d'Information}
    Quelle est la principale raison de la mise en œuvre de la gestion des identités et des accès (IAM) ?
    \item Améliorer les performances du réseau
    \item Réduire les coûts d'exploitation
    \item* Gérer les droits d'accès des utilisateurs
    \item Protéger les données sensibles
  \end{multi}

  % Question 48
  \begin{multi}[points=1]{Politiques de Sécurité}
    Quelle est la principale raison de l'application de la politique de chiffrement des données ?
    \item Contrôler l'accès aux ressources informatiques
    \item Surveiller les journaux d'activité
    \item* Protéger la confidentialité des données
    \item Appliquer des mises à jour de sécurité
  \end{multi}

  % Question 49
  \begin{multi}[points=1]{Cybersécurité}
    Quelle technologie de sécurité vise à identifier les modèles de comportement suspects ou malveillants dans le trafic réseau ?
    \item Antivirus
    \item* Analyse comportementale
    \item Pare-feu de nouvelle génération (NGFW)
    \item Chiffrement WPA/WPA2
  \end{multi}

  % Question 50
  \begin{multi}[points=1]{Architectures de Sécurité Informatique}
    Quelle est la principale fonction d'un proxy en matière de sécurité informatique ?
    \item* Filtrer et contrôler le trafic réseau
    \item Établir des connexions VPN
    \item Analyser les journaux d'activité
    \item Surveiller les vulnérabilités du système
  \end{multi}
  

% Question 51
\begin{multi}[points=1]{Sécurité des Systèmes d'Information}
    Quelle mesure de sécurité permet de s'assurer qu'un utilisateur a l'autorisation d'accéder à une ressource spécifique ?
    \item Contrôle d'intégrité
    \item* Contrôle d'accès
    \item Contrôle de flux
    \item Contrôle de routage
  \end{multi}

  % Question 52
  \begin{multi}[points=1]{Politiques de Sécurité}
    Quelle politique de sécurité définit les règles et procédures pour la protection des informations sensibles lors de leur transmission par voie électronique ?
    \item Politique de cryptographie
    \item Politique de sauvegarde
    \item* Politique de sécurité des communications
    \item Politique d'utilisation acceptable
  \end{multi}

  % Question 53
  \begin{multi}[points=1]{Cybersécurité}
    Quelle est la principale mesure de sécurité pour empêcher l'accès non autorisé à un réseau sans fil (Wi-Fi) ?
    \item VPN (Virtual Private Network)
    \item Pare-feu de nouvelle génération (NGFW)
    \item* Chiffrement WPA3
    \item Filtrage MAC
  \end{multi}

  % Question 54
  \begin{multi}[points=1]{Architectures de Sécurité Informatique}
    Quel est l'objectif principal de la sécurité périmétrique d'un réseau ?
    \item Protéger les données en transit
    \item Contrôler l'accès aux données
    \item Assurer la redondance des serveurs
    \item* Empêcher les menaces d'atteindre le réseau interne
  \end{multi}

  % Question 55
  \begin{multi}[points=1]{Sécurité des Systèmes d'Information}
    Quelle technologie de sécurité permet de suivre les activités des utilisateurs et des systèmes afin de détecter les comportements anormaux ?
    \item Pare-feu d'application
    \item Chiffrement par substitution
    \item* SIEM (Security Information and Event Management)
    \item VPN (Virtual Private Network)
  \end{multi}

  % Question 56
  \begin{multi}[points=1]{Politiques de Sécurité}
    Quelle politique de sécurité traite des procédures à suivre pour déclasser, détruire ou archiver des informations sensibles en fin de vie ?
    \item Politique de gestion des mots de passe
    \item Politique de sauvegarde
    \item* Politique de gestion de la fin de vie des données
    \item Politique de cryptographie
  \end{multi}

  % Question 57
  \begin{multi}[points=1]{Cybersécurité}
    Quelle est la principale menace liée aux attaques de type "zero-day" ?
    \item* L'absence de correctif de sécurité disponible
    \item L'absence de surveillance de sécurité
    \item La lenteur des réponses aux incidents
    \item L'absence de pare-feu
  \end{multi}

  % Question 58
  \begin{multi}[points=1]{Architectures de Sécurité Informatique}
    Quel mécanisme de sécurité permet de confirmer l'identité d'un utilisateur avant de lui accorder l'accès à un système ou à des données ?
    \item Chiffrement des données
    \item* Authentification
    \item Contrôle d'intégrité
    \item Chiffrement par substitution
  \end{multi}

  % Question 59
  \begin{multi}[points=1]{Sécurité des Systèmes d'Information}
    Quelle est la première étape du processus de gestion des incidents de sécurité ?
    \item* Détection de l'incident
    \item Analyse des causes profondes
    \item Réponse immédiate à l'incident
    \item Rapport de l'incident
  \end{multi}

  % Question 60
  \begin{multi}[points=1]{Politiques de Sécurité}
    Quelle politique de sécurité concerne l'utilisation des clés de chiffrement, leur stockage et leur gestion ?
    \item Politique de gestion des mots de passe
    \item Politique de sauvegarde
    \item Politique de gestion de la fin de vie des données
    \item* Politique de gestion des clés de chiffrement
  \end{multi}


  % Question 61
  \begin{multi}[points=1]{Sécurité des Systèmes d'Information}
    Quel terme désigne le processus de vérification de l'identité d'un utilisateur, généralement à l'aide d'un nom d'utilisateur et d'un mot de passe ?
    \item* Authentification
    \item Autorisation
    \item Audit
    \item Attribution
  \end{multi}

  % Question 62
  \begin{multi}[points=1]{Architectures de Sécurité Informatique}
    Quelle technologie permet de créer un réseau privé virtuel (VPN) en utilisant une connexion sécurisée sur un réseau public, comme Internet ?
    \item Commutateur Ethernet
    \item Pare-feu de zone
    \item* Tunnel VPN (Virtual Private Network)
    \item Routeur sans fil
  \end{multi}

  % Question 63
  \begin{multi}[points=1]{Cybersécurité}
    Quelle est la principale menace associée à l'ingénierie inversée dans le domaine de la sécurité informatique ?
    \item Divulgation de données
    \item Perte de disponibilité
    \item* Perte de la confidentialité
    \item Perte d'intégrité
  \end{multi}

  % Question 64
  \begin{multi}[points=1]{Politiques de Sécurité}
    Quelle politique de sécurité traite de la façon dont les données sensibles doivent être stockées, protégées et détruites à la fin de leur durée de vie utile ?
    \item Politique de gestion des incidents de sécurité
    \item Politique de contrôle d'accès
    \item* Politique de gestion de la confidentialité des données
    \item Politique de sauvegarde
  \end{multi}

  % Question 65
  \begin{multi}[points=1]{Sécurité des Systèmes d'Information}
    Quel est l'objectif principal d'un pare-feu d'application Web (WAF) ?
    \item Bloquer les attaques de phishing
    \item Contrôler l'accès aux données
    \item* Protéger les applications Web contre les vulnérabilités et les attaques
    \item Filtrer le trafic réseau
  \end{multi}

  % Question 66
  \begin{multi}[points=1]{Cybersécurité}
    Quelle est la principale mesure de sécurité pour protéger les systèmes contre les virus informatiques et les logiciels malveillants ?
    \item Authentification à deux facteurs
    \item Chiffrement des données
    \item* Logiciels antivirus
    \item Firewall d'application
  \end{multi}

  % Question 67
  \begin{multi}[points=1]{Architectures de Sécurité Informatique}
    Quel est le rôle principal d'un proxy dans une architecture réseau sécurisée ?
    \item Chiffrement des données
    \item* Faire office de mandataire (Intermédiaire) entre les utilisateurs et les ressources sur Internet
    \item Authentification des utilisateurs
    \item Détection d'intrusion
  \end{multi}

  % Question 68
  \begin{multi}[points=1]{Politiques de Sécurité}
    Quelle politique de sécurité concerne la classification et l'étiquetage des informations en fonction de leur sensibilité ?
    \item Politique de gestion des incidents de sécurité
    \item* Politique de classification des données
    \item Politique de pare-feu
    \item Politique de contrôle d'accès
  \end{multi}

  % Question 69
  \begin{multi}[points=1]{Sécurité des Systèmes d'Information}
    Quelle technologie de sécurité est conçue pour protéger un réseau local (LAN) contre les accès non autorisés de l'extérieur ?
    \item VPN (Virtual Private Network)
    \item IDS (Système de Détection d'Intrusion)
    \item* Pare-feu
    \item SIEM (Security Information and Event Management)
  \end{multi}

  % Question 70
    %70 : ça se discute ? divulgation == intégrité
  \begin{multi}[points=1]{Cybersécurité}
    Quelle est la principale menace associée à une attaque de type "man-in-the-middle" ?
    \item Perte de disponibilité
    \item Perte de la confidentialité
    \item* Altération des données en transit
  \end{multi}


% Question 71
\begin{multi}[points=1]{Sécurité des Systèmes d'Information}
    Quelle est la principale menace pour la sécurité des informations stockées sur des supports physiques tels que des disques durs ou des clés USB perdus ou volés ?
    \item Attaque de phishing
    \item Attaque par déni de service (DDoS)
    \item Attaque par ingénierie sociale
    \item* Perte ou vol de données
  \end{multi}
  


  % Question 72
  \begin{multi}[points=1]{Architectures de Sécurité Informatique}
    Quel type de dispositif de sécurité inspecte le trafic réseau à la recherche de signatures de logiciels malveillants connus ?
    \item* IDS (Système de Détection d'Intrusion)
    \item Pare-feu d'application
    \item VPN (Virtual Private Network)
    \item Pare-feu de zone
  \end{multi}

  % Question 73
  \begin{multi}[points=1]{Politiques de Sécurité}
    Quelle politique de sécurité définit les règles pour garantir la disponibilité et l'accès continu aux données en cas de catastrophe ?
    \item Politique de contrôle d'accès
    \item Politique de sauvegarde
    \item* Politique de continuité des activités (PCA)
    \item Politique de cryptographie
  \end{multi}

  % Question 74
  \begin{multi}[points=1]{Cybersécurité}
    Quel type d'attaque informatique vise à perturber ou à détruire délibérément des systèmes informatiques ou des réseaux ?
    \item Attaque par injection SQL
    \item Attaque de phishing
    \item* Attaque par déni de service distribué (DDoS)
  \end{multi}

  % Question 75
  \begin{multi}[points=1]{Sécurité des Systèmes d'Information}
    Quel processus permet de rendre les données lisibles uniquement pour des utilisateurs autorisés, tout en empêchant les utilisateurs non autorisés d'y accéder ?
    \item Authentification
    \item Chiffrement
    \item Biométrie
    \item* Contrôle d'accès
  \end{multi}

  % Question 76
  \begin{multi}[points=1]{Architectures de Sécurité Informatique}
    Quelle technologie permet de créer un réseau privé virtuel sécurisé en utilisant une connexion Internet publique ?
    \item SIEM (Security Information and Event Management)
    \item IDS (Système de Détection d'Intrusion)
    \item* VPN (Virtual Private Network)
    \item Pare-feu de nouvelle génération (NGFW)
  \end{multi}

  % Question 77
  \begin{multi}[points=1]{Politiques de Sécurité}
    Quelle politique de sécurité définit les règles pour l'utilisation des médias sociaux au sein de l'entreprise ?
    \item Politique de contrôle d'accès
    \item Politique de gestion de la fin de vie des données
    \item* Politique des médias sociaux
    \item Politique de cryptographie
  \end{multi}

  % Question 78
  \begin{multi}[points=1]{Cybersécurité}
    Quelle mesure de sécurité consiste à diviser un réseau en segments logiques pour réduire la surface d'attaque ?
    \item Chiffrement WPA/WPA2
    \item Chiffrement par substitution
    \item* Segmentation réseau
    \item Authentification à deux facteurs
  \end{multi}

  % Question 79
  \begin{multi}[points=1]{Sécurité des Systèmes d'Information}
    Quel est le principal objectif de la sauvegarde régulière des données ?
    \item Empêcher les attaques par injection SQL
    \item Réduire la latence du réseau
    \item* Assurer la récupération des données en cas de perte ou de panne
    \item Renforcer le pare-feu
  \end{multi}

  % Question 80
  \begin{multi}[points=1]{Architectures de Sécurité Informatique}
    Quel protocole réseau est couramment utilisé pour sécuriser les communications entre un client et un serveur web ?
    \item SSH (Secure Shell)
    \item* HTTPS (Hypertext Transfer Protocol Secure)
    \item FTP (File Transfer Protocol)
    \item SMTP (Simple Mail Transfer Protocol)
  \end{multi}

  % Question 81
  \begin{multi}[points=1]{Sécurité des Systèmes d'Information}
    Quelle technologie permet de surveiller et de filtrer le trafic réseau pour détecter et bloquer les menaces en temps réel ?
    \item* IDS/IPS (Système de Détection et de Prévention des Intrusions)
    \item Pare-feu d'application
    \item VPN (Virtual Private Network)
    \item SIEM (Security Information and Event Management)
  \end{multi}

  % Question 82
  \begin{multi}[points=1]{Politiques de Sécurité}
    Quelle politique de sécurité définit les règles concernant l'utilisation des ressources informatiques pendant les heures de travail ?
    \item Politique de cryptographie
    \item Politique de sauvegarde
    \item Politique de gestion de la fin de vie des données
    \item* Politique de gestion du temps de travail sur les ordinateurs
  \end{multi}

  % Question 83
  \begin{multi}[points=1]{Cybersécurité}
    Quelle est la principale menace associée aux attaques par hameçonnage (phishing) ?
    \item* Vol d'informations sensibles
    \item Corruption des fichiers système
    \item Perte de connectivité réseau
    \item Attaque par déni de service (DDoS)
  \end{multi}

  % Question 84
  \begin{multi}[points=1]{Architectures de Sécurité Informatique}
    Quelle technologie est utilisée pour identifier les utilisateurs en se basant sur des caractéristiques physiques uniques, comme l'empreinte digitale ou l'iris  ?
    \item Authentification à deux facteurs
    \item Token d'authentification
    \item* Biométrie
    \item Chiffrement asymétrique
  \end{multi}

  % Question 85
  \begin{multi}[points=1]{Sécurité des Systèmes d'Information}
    Quelle mesure de sécurité garantit que les données sont accessibles uniquement par des personnes autorisées et que leur contenu n'est pas modifié en transit ?
    \item Chiffrement par substitution
    \item Pare-feu de zone
    \item* Intégrité des données
    \item Authentification à deux éléments
  \end{multi}

  % Question 86
  \begin{multi}[points=1]{Politiques de Sécurité}
    Quelle politique de sécurité définit les règles pour l'utilisation appropriée des équipements informatiques appartenant à l'entreprise ?
    \item Politique de gestion des clés de chiffrement
    \item Politique de cryptographie
    \item* Politique d'utilisation acceptable des équipements
    \item Politique de gestion de la fin de vie des données
  \end{multi}

  % Question 87
  \begin{multi}[points=1]{Cybersécurité}
    Quelle technique de sécurité consiste à tromper un attaquant en lui fournissant de fausses informations pour le détourner de la véritable cible ?
    \item* Leurre (Honeypot)
    \item Filtrage MAC
    \item VPN (Virtual Private Network)
    \item Chiffrement WPA3
  \end{multi}

  % Question 88
  \begin{multi}[points=1]{Architectures de Sécurité Informatique}
    Quel mécanisme de sécurité permet de garantir que l'expéditeur d'un message électronique est bien la personne qu'il prétend être ?
    \item Chiffrement des données
    \item Authentification à deux facteurs
    \item Contrôle d'accès
    \item* Signature électronique
  \end{multi}

  % Question 89
  \begin{multi}[points=1]{Sécurité des Systèmes d'Information}
    Quelle technologie permet de stocker les mots de passe de manière sécurisée en les transformant en une séquence aléatoire de caractères ?
    \item Biométrie
    \item Authentification à deux éléments
    \item* Hachage de mots de passe
    \item Chiffrement de fichiers
  \end{multi}

  % Question 90
  \begin{multi}[points=1]{Politiques de Sécurité}
    Quelle politique de sécurité définit les procédures à suivre en cas de violation de la sécurité des données personnelles des clients ?
    \item Politique de gestion des incidents de sécurité
    \item Politique de cryptographie
    \item* Politique de notification de violation de données
    \item Politique de contrôle d'accès
  \end{multi}


   % Question 91
  \begin{multi}[points=1]{Sécurité des Systèmes d'Information}
    Quel type d'attaque consiste à intercepter et à enregistrer le trafic réseau afin d'espionner les communications ?
    \item Attaque de phishing
    \item Attaque par déni de service (DDoS)
    \item* Attaque d'interception (sniffing)
    \item Attaque de ransomware
  \end{multi}

  % Question 92
  \begin{multi}[points=1]{Architectures de Sécurité Informatique}
    Quel composant réseau est conçu pour bloquer le trafic réseau non autorisé en fonction de règles prédéfinies ?
    \item SIEM (Security Information and Event Management)
    \item IDS (Système de Détection d'Intrusion)
    \item* Pare-feu
    \item VPN (Virtual Private Network)
  \end{multi}

  % Question 93
  \begin{multi}[points=1]{Politiques de Sécurité}
    Quelle politique de sécurité définit les règles pour l'accès physique aux locaux informatiques, aux serveurs et aux équipements ?
    \item Politique de gestion des mots de passe
    \item Politique des médias sociaux
    \item* Politique de sécurité physique
    \item Politique de cryptographie
  \end{multi}

  % Question 94
  \begin{multi}[points=1]{Cybersécurité}
    Quelle est la principale caractéristique d'une attaque de type "phishing" ?
    \item Elle vise à saturer un réseau de trafic malveillant
    \item Elle consiste à détruire physiquement les systèmes informatiques
    \item* Elle cherche à tromper les victimes en se faisant passer pour une source de confiance
    \item Elle exploite des vulnérabilités logicielles connues
  \end{multi}

  % Question 95
  \begin{multi}[points=1]{Sécurité des Systèmes d'Information}
    Quel mécanisme de sécurité permet de garantir l'intégrité des données en vérifiant si elles ont été altérées pendant le transport ?
    \item Chiffrement asymétrique
    \item Authentification à deux facteurs
    \item* Fonction de hachage (hash)
    \item Authentification biométrique
  \end{multi}

  % Question 96
  \begin{multi}[points=1]{Architectures de Sécurité Informatique}
    Quelle technologie de sécurité est utilisée pour empêcher les programmes malveillants d'accéder à certaines ressources du système ?
    \item IDS (Système de Détection d'Intrusion)
    \item SIEM (Security Information and Event Management)
    \item VPN (Virtual Private Network)
    \item* Contrôle d'accès aux applications
  \end{multi}

  % Question 97
  \begin{multi}[points=1]{Politiques de Sécurité}
    Quelle politique de sécurité traite des procédures pour la gestion des certificats numériques ?
    \item Politique de sauvegarde
    \item Politique de gestion de la fin de vie des données
    \item* Politique de gestion des certificats
    \item Politique de cryptographie
  \end{multi}

  % Question 98
  \begin{multi}[points=1]{Cybersécurité}
    Quelle mesure de sécurité implique la création de copies de données et leur stockage hors site pour la récupération en cas de sinistre ?
    \item Authentification forte
    \item* Sauvegarde et récupération des données
    \item Chiffrement de bout en bout
    \item Contrôle d'accès
  \end{multi}

  % Question 99
  \begin{multi}[points=1]{Sécurité des Systèmes d'Information}
    Quelle est la principale menace associée à l'utilisation de mots de passe faibles ou faciles à deviner ?
    \item Attaque de phishing
    \item Attaque par injection SQL
    \item Attaque par déni de service (DDoS)
    \item* Attaque par force brute
  \end{multi}

  % Question 100
  \begin{multi}[points=1]{Architectures de Sécurité Informatique}
    Quel type de pare-feu permet de prendre des décisions basées sur l'état actuel des connexions réseau, plutôt que sur des règles statiques ?
    \item Pare-feu d'application
    \item Pare-feu de zone
    \item VPN (Virtual Private Network)
    \item* Pare-feu d'état
  \end{multi}


  % Bloc SEC101 - SECOPS

  \begin{multi}[points=1]{Vulnérabilités informatiques}
     Qu'est-ce qu'une vulnérabilité informatique  ?
        \item Une attaque contre les données
        \item Un logiciel antivirus
        \item *Une faiblesse dans un système qui peut être exploitée
        \item Une méthode de chiffrement
      \end{multi} 
    
     \begin{multi}[points=1]{Vulnérabilités informatiques}
     Quel type de vulnérabilité est généralement corrigé par un correctif de sécurité  ?
        \item Vulnérabilité physique
        \item Vulnérabilité humaine
        \item *Vulnérabilité logicielle
        \item Vulnérabilité matérielle
      \end{multi} 
    
    \begin{multi}[points=1]{Vulnérabilités informatiques}
     Qu'est-ce qu'une analyse de vulnérabilité  ?
        \item Une attaque ciblée
        \item *L'évaluation des faiblesses potentielles dans un système
        \item Le chiffrement des données sensibles
        \item La gestion des incidents de sécurité
      \end{multi} 
    
     \begin{multi}[points=1]{Vulnérabilités informatiques}
     Quel est le rôle d'un scanner de vulnérabilités ?
        \item *Identifier et rapporter les vulnérabilités dans un système
        \item Défendre contre les attaques DDoS
        \item Chiffrer les communications réseau
        \item Gérer les identités utilisateur
      \end{multi} 
    
     \begin{multi}[points=1]{Vulnérabilités informatiques}
     Quelle est la principale raison de la présence de vulnérabilités dans les systèmes informatiques ?
        \item Manque de pare-feu
        \item *Complexité des logiciels et des systèmes
        \item Faible utilisation de l'authentification à deux facteurs
        \item Trop de correctifs de sécurité
      \end{multi} 
    
     \begin{multi}[points=1]{Vulnérabilités informatiques}
    % Menaces informatiques
     Qu'est-ce qu'une menace informatique ?
        \item *Un événement ou une situation potentiellement nuisible pour les systèmes informatiques
        \item Une mise à jour logicielle
        \item Un certificat SSL
        \item Un programme antivirus
      \end{multi} 
    
     \begin{multi}[points=1]{Vulnérabilités informatiques}
     Quel est le principal objectif d'une menace informatique de type "ransomware" ?
        \item *Chiffrer les données de la victime et demander une rançon
        \item Dérober des informations confidentielles
        \item Détecter les vulnérabilités dans un système
        \item Corrompre les fichiers système
      \end{multi} 
    
     \begin{multi}[points=1]{Vulnérabilités informatiques}
     Quelle est la caractéristique commune des attaques de type "phishing" ?
        \item *L'utilisation de l'ingénierie sociale pour tromper les utilisateurs
        \item L'exploitation de failles logicielles
        \item La diffusion de logiciels malveillants par courriel
        \item L'effacement de données sensibles
      \end{multi} 
    
     \begin{multi}[points=1]{Vulnérabilités informatiques}
     Qu'est-ce qu'un attaquant "hacker" ?
        \item Un employé de sécurité informatique
        \item *Une personne qui exploite des failles de sécurité pour accéder illégalement à un système
        \item Un utilisateur légitime du réseau
        \item Un concepteur de logiciels
      \end{multi} 
    
     \begin{multi}[points=1]{Vulnérabilités informatiques}
     Quelle est la principale motivation derrière les attaques informatiques ?
        \item Curiosité
        \item *Génération de profit financier
        \item Amélioration des compétences techniques
        \item Amélioration de la réputation en ligne
      \end{multi} 
    
     \begin{multi}[points=1]{Computer Emergency Response Team}
    % Codes malveillants
    Qu'est-ce qu'un logiciel malveillant (malware) ?
        \item *Un programme informatique conçu pour causer des dommages ou compromettre la sécurité
        \item Un programme antivirus
        \item Un outil de sauvegarde
        \item Un dispositif de chiffrement
      \end{multi} 
    
     \begin{multi}[points=1]{Codes malveillants}
     Quelle est la principale caractéristique d'un virus informatique ?
        \item *Capacité de se reproduire en infectant d'autres fichiers
        \item Chiffrer des données sensibles
        \item Exploitation de failles de sécurité
        \item Détection des vulnérabilités réseau
      \end{multi} 
    
     \begin{multi}[points=1]{Codes malveillants}
     Qu'est-ce qu'un cheval de Troie (trojan) dans le contexte des codes malveillants ?
        \item *Un programme apparemment légitime qui contient un logiciel malveillant
        \item Un virus informatique
        \item Un dispositif de chiffrement
        \item Une attaque par déni de service (DDoS)
      \end{multi} 
    
     \begin{multi}[points=1]{Codes malveillants}
     Quel est l'objectif d'un ransomware ?
        \item Collecter des informations de navigation
        \item *Chiffrer les fichiers de la victime et demander une rançon
        \item Détruire physiquement le matériel informatique
        \item Infecter les utilisateurs avec des publicités indésirables
      \end{multi} 
    
     \begin{multi}[points=1]{Codes malveillants}
     Quelle est la principale méthode de propagation d'un ver informatique ?
        \item *Exploitation des vulnérabilités réseau
        \item Ingénierie sociale
        \item Chiffrement des fichiers système
        \item Utilisation de courriels frauduleux
      \end{multi} 
    
     \begin{multi}[points=1]{Incidents informatiques}
    % Incidents informatiques
     Qu'est-ce qu'un incident informatique ?
        \item Une mise à jour logicielle
        \item Un certificat SSL
        \item *Un événement qui compromet la sécurité des systèmes informatiques
        \item Une attaque DDoS
      \end{multi} 
    
     \begin{multi}[points=1]{Incidents informatiques}
     Quelle est la première étape dans la gestion d'un incident informatique ?
        \item *Détection et identification de l'incident
        \item Isolation du système affecté
        \item Notification des autorités
        \item Analyse post-incident
      \end{multi} 
    
     \begin{multi}[points=1]{Incidents informatiques}
     Quelle est la principale responsabilité d'une équipe de réponse aux incidents (IRT) ?
       
     \item *Investigation et gestion des incidents de sécurité
        \item Mise en œuvre des politiques de sécurité
        \item Maintenance des serveurs
        \item Développement de logiciels sécurisés
      \end{multi} 
    
     \begin{multi}[points=1]{Incidents informatiques}
     Qu'est-ce qu'une analyse post-incident ?
        \item *Examen et évaluation des événements survenus après un incident de sécurité
        \item Prévention des incidents futurs
        \item Détection des vulnérabilités
        \item Réponse immédiate à un incident
      \end{multi} 
    
     \begin{multi}[points=1]{Incidents informatiques}
     Quel est l'objectif d'une politique de gestion des incidents ?
        \item Chiffrer les données sensibles
        \item *Fournir des directives pour répondre efficacement aux incidents de sécurité
        \item Gérer les vulnérabilités du réseau
        \item Analyser les journaux d'activité
      \end{multi} 
    
     \begin{multi}[points=1]{Security Operation Center}
    % Security Operation Center (SOC)
     Qu'est-ce qu'un Security Operation Center (SOC) ?
        \item Un logiciel antivirus
        \item *Un centre qui surveille et répond aux incidents de sécurité
        \item Un dispositif de chiffrement
        \item Un pare-feu
      \end{multi} 
    
     \begin{multi}[points=1]{Security Operation Center}
     Quel est le rôle principal d'un SOC dans une organisation ?
        \item Mise en œuvre des politiques de sécurité
        \item Gestion des identités utilisateurs
        \item *Surveillance continue de la sécurité et réponse aux incidents
        \item Maintenance des serveurs
      \end{multi} 
    
     \begin{multi}[points=1]{Security Operation Center}
     Qu'est-ce qu'une alerte de sécurité dans le contexte d'un SOC ?
        \item *Une notification signalant une activité suspecte ou un incident de sécurité
        \item Un test de pénétration automatisé
        \item Un rapport sur les performances du réseau
        \item Une mise à jour logicielle
      \end{multi} 
    
     \begin{multi}[points=1]{Security Operation Center}
     Quelle est la principale différence entre un SOC et un CERT ?
        \item *Un SOC se concentre sur la surveillance et la réponse aux incidents, tandis qu'un CERT se concentre sur la coordination des réponses aux incidents au niveau national ou sectoriel
        \item Un SOC est uniquement responsable de la gestion des vulnérabilités
        \item Un CERT est une extension d'un SOC
        \item Un SOC et un CERT sont des termes interchangeables
      \end{multi} 
    
     \begin{multi}[points=1]{Security Operation Center}
     Quel est l'avantage d'utiliser des outils d'automatisation dans un SOC ?
        \item *Réduction du temps de réponse aux incidents
        \item Augmentation de la complexité des opérations
        \item Diminution de la surveillance
        \item Amélioration de la gestion des identités utilisateurs
      \end{multi} 
    
     \begin{multi}[points=1]{Computer Emergency Response Team}
    % Computer Emergency Response Team (CERT)
    Qu'est-ce qu'un Computer Emergency Response Team (CERT) ?
        \item *Une équipe spécialisée dans la gestion des incidents de sécurité au niveau national ou sectoriel
        \item Un groupe de pirates informatiques
        \item Un logiciel antivirus
        \item Un centre de formation en cybersécurité
      \end{multi} 
    
     \begin{multi}[points=1]{Computer Emergency Response Team}
     Quel est le rôle principal d'un CERT ?
        \item Maintenance des serveurs
        \item Gestion des identités utilisateurs
        \item *Coordination des réponses aux incidents au niveau national ou sectoriel
        \item Mise en œuvre des politiques de sécurité
      \end{multi} 
    
     \begin{multi}[points=1]{Computer Emergency Response Team}
     Quelle est la principale responsabilité d'un CERT pendant une cyberattaque majeure ?
        \item Maintenance des serveurs
        \item *Coordination des efforts de réponse et d'information
        \item Développement de logiciels sécurisés
        \item Mise en œuvre des politiques de sécurité
      \end{multi} 
    
     \begin{multi}[points=1]{Computer Emergency Response Team}
     Qu'est-ce qu'une déclaration d'incidents électroniques ?
        \item *Une communication formelle informant le CERT d'un incident de sécurité
        \item Un rapport sur les performances du réseau
        \item Un test de pénétration automatisé
        \item Un certificat SSL
      \end{multi} 
    
     \begin{multi}[points=1]{Computer Emergency Response Team}
     Quelle est la principale différence entre un SOC et un CERT ?
        \item *Un SOC se concentre sur la surveillance et la réponse aux incidents, tandis qu'un CERT se concentre sur la coordination des réponses aux incidents au niveau national ou sectoriel
        \item Un SOC est uniquement responsable de la gestion des vulnérabilités
        \item Un CERT est une extension d'un SOC
        \item Un SOC et un CERT sont des termes interchangeables
      \end{multi} 
    

%BLOC SEC101 - 2024

\begin{multi}[points=1]{Vulnérabilités informatiques}
% Vulnérabilités informatiques
Qu'est-ce qu'un test d'intrusion ?
    \item Un logiciel antivirus
    \item Une analyse post-incident
    \item *Une simulation d'attaque pour identifier les vulnérabilités
    \item Un programme de gestion des incidents
\end{multi}

\begin{multi}[points=1]{Vulnérabilités informatiques}
Quel est le principal facteur contribuant aux vulnérabilités logicielles ?
    \item *Complexité du code
    \item Mises à jour fréquentes
    \item Utilisation d'authentification forte
    \item Isolation des systèmes
\end{multi}

\begin{multi}[points=1]{Menaces informatiques}
% Menaces informatiques
Qu'est-ce qu'un logiciel espion (spyware) ?
    \item *Un programme conçu pour collecter des informations sans le consentement de l'utilisateur
    \item Un dispositif de chiffrement
    \item Un outil de gestion des identités
    \item Une mise à jour logicielle
\end{multi}

\begin{multi}[points=1]{Menaces informatiques}
Quelle est la principale caractéristique d'une attaque par déni de service distribué (DDoS) ?
    \item *Inondation d'un service en ligne par un trafic excessif
    \item Chiffrer des fichiers système
    \item Utilisation de l'ingénierie sociale
    \item Exploitation de vulnérabilités réseau
\end{multi}

\begin{multi}[points=1]{Codes malveillants}
% Codes malveillants
Qu'est-ce qu'un virus polymorphique ?
    \item *Un virus capable de changer son apparence pour éviter la détection
    \item Un cheval de Troie
    \item Une alerte de sécurité
    \item Un programme antivirus
\end{multi}

\begin{multi}[points=1]{Codes malveillants}
Quel est l'objectif principal d'un ver informatique ?
    \item *Se propager rapidement à travers les réseaux
    \item Chiffrer les fichiers de la victime
    \item Collecter des informations sans autorisation
    \item Corrompre les fichiers système
\end{multi}

\begin{multi}[points=1]{Incidents informatiques}
% Incidents informatiques
Qu'est-ce qu'une analyse forensique ?
    \item *L'analyse des preuves numériques après un incident pour en déterminer la cause
    \item Un test d'intrusion
    \item La gestion des identités utilisateurs
    \item La coordination des réponses aux incidents
\end{multi}

\begin{multi}[points=1]{Incidents informatiques}
Quel est le rôle d'un responsable de la réponse aux incidents ?
     \item *Coordination des efforts pour résoudre les incidents de sécurité
    \item Développement de politiques de sécurité
    \item Maintenance des serveurs
    \item Analyse post-incident
\end{multi}

\begin{multi}[points=1]{Security Operation Center}
% Security Operation Center (SOC)
Qu'est-ce qu'un SIEM (Security Information and Event Management) ?
    \item Un programme antivirus
    \item Un dispositif de chiffrement
    \item *Un système qui analyse et corrèle les données liées à la sécurité
    \item Une équipe de réponse aux incidents
\end{multi}

\begin{multi}[points=1]{Security Operation Center}
Quel est l'objectif principal de la surveillance continue dans un SOC ?
    \item *Identifier et répondre rapidement aux incidents de sécurité
    \item Gérer les politiques de sauvegarde
    \item Développer des logiciels sécurisés
    \item Fournir une formation en cybersécurité
\end{multi}

\begin{multi}[points=1]{Vulnérabilités informatiques}
% Vulnérabilités informatiques
Qu'est-ce qu'un test de sécurité "black-box" ?
    \item *Une évaluation externe sans connaissance préalable du système
    \item Une analyse post-incident
    \item Une simulation d'attaque interne
    \item Un test de pénétration automatisé
\end{multi}

\begin{multi}[points=1]{Vulnérabilités informatiques}
Quel est le principal objectif d'une évaluation de la sécurité physique ?
    \item La détection d'attaques DDoS
    \item La collecte d'informations sensibles
    \item *L'identification des vulnérabilités physiques d'un site
    \item Le suivi des journaux d'activité réseau
\end{multi}

\begin{multi}[points=1]{Menaces informatiques}
% Menaces informatiques
Qu'est-ce qu'un attaquant "script kiddie" ?
    \item *Une personne qui utilise des outils préexistants sans comprendre leur fonctionnement
    \item Un professionnel de la cybersécurité
    \item Un pirate informatique expérimenté
    \item Un membre d'une équipe de réponse aux incidents
\end{multi}

\begin{multi}[points=1]{Menaces informatiques}
Quelle est la principale caractéristique d'une attaque par "zero-day" ?
    \item *Exploitation d'une vulnérabilité non corrigée
    \item Utilisation d'attaques sophistiquées
    \item Attaque pendant une journée spécifique
    \item Détection préalable par les systèmes de sécurité
\end{multi}

\begin{multi}[points=1]{Codes malveillants}
% Codes malveillants
Qu'est-ce qu'un cheval de Troie (trojan) dans le contexte des codes malveillants ?
    \item *Un programme apparemment légitime qui contient un logiciel malveillant
    \item Un virus polymorphique
    \item Une alerte de sécurité
    \item Un certificat SSL
\end{multi}

\begin{multi}[points=1]{Codes malveillants}
Quelle est la principale caractéristique d'un ransomware ?
    \item *Chiffrer les fichiers de la victime et demander une rançon
    \item Collecter des informations sans autorisation
    \item Se propager rapidement à travers les réseaux
    \item Corrompre les fichiers système
\end{multi}

\begin{multi}[points=1]{Incidents informatiques}
% Incidents informatiques
Quelle est la principale responsabilité d'une équipe d'analystes forensiques ?
    \item *Collecter et analyser des preuves numériques après un incident
    \item Coordination des réponses aux incidents
    \item Mise en œuvre des politiques de sécurité
    \item Maintenance des serveurs
\end{multi}

\begin{multi}[points=1]{Incidents informatiques}
Qu'est-ce qu'une déclaration d'incidents électroniques ?
    \item Un test d'intrusion automatisé
    \item *Une communication formelle informant d'un incident de sécurité
    \item Un programme antivirus
    \item Un rapport sur les performances du réseau
\end{multi}

\begin{multi}[points=1]{Security Operation Center}
% Security Operation Center (SOC)
Qu'est-ce qu'un indicateur de compromission (IoC) dans le contexte d'un SOC ?
    \item Un outil de chiffrement
    \item *Une donnée observable qui indique une activité suspecte ou malveillante
    \item Une équipe de réponse aux incidents
    \item Un certificat SSL
\end{multi}

\begin{multi}[points=1]{Security Operation Center}
Quelle est la principale fonction d'un tableau de bord de sécurité (Security Dashboard) dans un SOC ?
    \item *Afficher en temps réel les indicateurs de sécurité et les alertes
    \item Gérer les vulnérabilités du réseau
    \item Coordonner les réponses aux incidents
    \item Chiffrer les données sensibles
\end{multi}

\begin{multi}[points=1]{Vulnérabilités informatiques}
% Vulnérabilités informatiques
Qu'est-ce qu'une vulnérabilité "day-zero" ?
    \item *Une vulnérabilité qui est exploitée avant qu'un correctif ne soit disponible
    \item Une vulnérabilité identifiée zéro jour après son apparition
    \item Une vulnérabilité qui affecte seulement les systèmes de production
    \item Une vulnérabilité découvert zéro jours après sa correction
\end{multi}

\begin{multi}[points=1]{Vulnérabilités informatiques}
Quel est le rôle principal d'un scanner de vulnérabilités dans le contexte de la sécurité informatique ?
    \item Une protection contre les attaques DDoS
    \item *Identifier et signaler les vulnérabilités dans un système
    \item Un dispositif de chiffrement des données
    \item Un outil de détection de malware
\end{multi}

\begin{multi}[points=1]{Menaces informatiques}
% Menaces informatiques
Qu'est-ce qu'une attaque de type "man-in-the-middle" ?
    \item Une attaque par force brute
    \item *Une attaque où un attaquant intercepte et altère la communication entre deux parties
    \item Une attaque de déni de service distribué (DDoS)
    \item Une attaque par script kiddie
\end{multi}

\begin{multi}[points=1]{Menaces informatiques}
Quel est l'objectif principal d'une attaque de type "phishing" ?
    \item Corrompre les fichiers système
    \item *Utiliser des techniques d'ingénierie sociale pour tromper les utilisateurs et obtenir des informations sensibles
    \item Collecter des informations de navigation
    \item Lancer des attaques DDoS
\end{multi}

\begin{multi}[points=1]{Codes malveillants}
% Codes malveillants
Qu'est-ce qu'un ver informatique ?
    \item *Un programme autonome capable de se propager à travers les réseaux
    \item Un virus polymorphique
    \item Une équipe de réponse aux incidents
    \item Un outil de chiffrement
\end{multi}

\begin{multi}[points=1]{Codes malveillants}
Quelle est la principale caractéristique d'un cheval de Troie (trojan) ?
    \item *Un programme apparemment légitime qui cache un logiciel malveillant
    \item Une alerte de sécurité
    \item Un outil de détection de malware
    \item Un virus capable de se reproduire en infectant d'autres fichiers
\end{multi}

\begin{multi}[points=1]{Incidents informatiques}
% Incidents informatiques
Qu'est-ce qu'une analyse post-incident ?
    \item *L'examen et l'évaluation des événements survenus après un incident de sécurité
    \item Une simulation d'attaque pour identifier les vulnérabilités
    \item Un test de pénétration automatisé
    \item La collecte des preuves numériques pendant un incident
\end{multi}

\begin{multi}[points=1]{Incidents informatiques}
Quelle est la principale responsabilité d'une équipe de réponse aux incidents (IRT) ?
    \item La gestion des identités utilisateurs
    \item *Investigation et gestion des incidents de sécurité
    \item Maintenance des serveurs
    \item Mise en œuvre des politiques de sécurité
\end{multi}

\begin{multi}[points=1]{Security Operation Center}
% Security Operation Center (SOC)
Qu'est-ce qu'un indicateur de compromission (IoC) fournit par un CERT ?
    \item Une donnée de vulnérabilité
    \item *Une donnée observable qui indique une activité suspecte ou malveillante
    \item Un outil de chiffrement
    \item* Une signature de programme malveillant
\end{multi}

\begin{multi}[points=1]{Security Operation Center}
Quel est le rôle principal d'un analyste de sécurité dans un SOC ?
    \item *Surveiller et analyser les activités de sécurité, détecter les incidents et fournir une réponse initiale
    \item Développer des politiques de sécurité
    \item Gérer les vulnérabilités du réseau
    \item Coordonner les efforts de réponse aux incidents
\end{multi}

\begin{multi}[points=1]{Vulnérabilités informatiques}
% Vulnérabilités informatiques
Qu'est-ce qu'une vulnérabilité "zero-day" ?
    \item Une vulnérabilité qui n'a jamais été exploitée
    \item Une vulnérabilité qui ne nécessite aucune correction
    \item *Une vulnérabilité qui est exploitée avant qu'un correctif ne soit disponible
    \item Une vulnérabilité qui n'affecte que les anciennes versions de logiciels
\end{multi}

\begin{multi}[points=1]{Vulnérabilités informatiques}
Quel est le rôle d'un test d'intrusion "gray-box" ?
    \item Une évaluation externe sans connaissance préalable du système
    \item *Une évaluation avec une connaissance partielle du système
    \item Une simulation d'attaque interne
    \item Un test de pénétration automatisé
\end{multi}

\begin{multi}[points=1]{Menaces informatiques}
% Menaces informatiques
Qu'est-ce qu'une attaque par force brute ?
    \item Une attaque utilisant des tactiques furtives
    \item *Une attaque essayant toutes les combinaisons possibles de mots de passe
    \item Une attaque exploitant les failles matérielles
    \item Une attaque par déni de service distribué (DDoS)
\end{multi}

\begin{multi}[points=1]{Menaces informatiques}
Quelle est la principale caractéristique d'une menace persistante avancée (APT) ?
    \item Une menace de courte durée
    \item *Une menace discrète et soutenue visant à accéder à un système de manière non détectée
    \item Une menace uniquement basée sur l'ingénierie sociale
    \item Une menace affectant uniquement les utilisateurs inexpérimentés
\end{multi}

\begin{multi}[points=1]{Codes malveillants}
% Codes malveillants
Qu'est-ce qu'un logiciel espion (spyware) ?
    \item *Un programme conçu pour collecter des informations sans le consentement de l'utilisateur
    \item Un virus capable de se reproduire en infectant d'autres fichiers
    \item Un cheval de Troie
    \item Un outil de détection de malware
\end{multi}

\begin{multi}[points=1]{Codes malveillants}
Quelle est la principale caractéristique d'un ransomware ?
    \item *Chiffrer les fichiers de la victime et demander une rançon
    \item Collecter des informations sans autorisation
    \item Se propager rapidement à travers les réseaux
    \item Corrompre les fichiers système
\end{multi}

\begin{multi}[points=1]{Incidents informatique}
% Incidents informatiques
Qu'est-ce qu'une déclaration d'incidents électroniques ?
    \item Un test d'intrusion automatisé
    \item *Une communication formelle informant d'un incident de sécurité
    \item Un programme antivirus
    \item Un rapport sur les performances du réseau
\end{multi}

\begin{multi}[points=1]{Incidents informatique}
Quelle est la première étape dans la gestion d'un incident informatique ?
    \item L'analyse post-incident
    \item *La détection et l'identification de l'incident
    \item La coordination des réponses aux incidents
    \item L'analyse forensique
\end{multi}

\begin{multi}[points=1]{Security Operation Center}
% Security Operation Center (SOC)
Qu'est-ce qu'un tableau de bord de sécurité (Security Dashboard) dans un SOC ?
    \item Un outil de chiffrement
    \item *Un outil visuel affichant en temps réel les indicateurs de sécurité et les alertes
    \item Une équipe de réponse aux incidents
    \item Un certificat SSL
\end{multi}

\begin{multi}[points=1]{Security Operation Center}
Quel est l'objectif principal de la surveillance continue dans un SOC ?
    \item Gérer les vulnérabilités du réseau
    \item *Identifier et répondre rapidement aux incidents de sécurité
    \item Fournir une formation en cybersécurité
    \item Coordonner les réponses aux incidents
\end{multi}

\begin{multi}[points=1]{Gestion de crise cyber}
% Gestion de crise cyber

Quelle est la principale responsabilité d'une équipe de gestion de crise cyber pendant un incident majeur ?
    \item Maintenance des serveurs
    \item *Coordination des actions pour minimiser les impacts et restaurer les opérations normales
    \item Surveillance des journaux d'activité réseau
    \item Gestion des vulnérabilités du réseau
\end{multi}

\begin{multi}[points=1]{Gestion de crise cyber}
% Gestion de crise cyber
Qu'est-ce qu'un plan de gestion de crise cyber ?
    \item Un programme antivirus
    \item Une équipe de réponse aux incidents
    \item *Un ensemble de procédures définissant les actions à prendre pendant une cybercrise
    \item Un logiciel de détection de malware
\end{multi}

\begin{multi}[points=1]{Gestion de crise cyber}
Quel est le rôle principal d'un coordinateur de crise cyber ?
    \item La maintenance des serveurs
    \item *Coordonner les actions et les communications pendant une crise
    \item La gestion des vulnérabilités du réseau
    \item Analyser les journaux d'activité réseau
\end{multi}

\begin{multi}[points=1]{Gestion de crise cyber}
Quelle est la première étape d'une gestion efficace de crise cyber ?
    \item La coordination des réponses aux incidents
    \item *La détection et la compréhension de la cyberattaque
    \item La maintenance des serveurs
    \item Le déploiement d'un logiciel antivirus
\end{multi}

\begin{multi}[points=1]{Gestion de crise cyber}
Qu'est-ce qu'une communication proactive pendant une crise cyber ?
    \item *Fournir des mises à jour régulières aux parties prenantes
    \item Une analyse post-incident
    \item La gestion des identités utilisateurs
    \item La collecte des preuves numériques pendant un incident
\end{multi}

\begin{multi}[points=1]{Gestion de crise cyber}
Quel est le rôle d'une équipe de communication de crise cyber ?
    \item La maintenance des serveurs
    \item *Gérer la communication interne et externe pendant une crise
    \item La coordination des réponses aux incidents
    \item L'analyse forensique
\end{multi}

\begin{multi}[points=1]{Gestion de crise cyber}
Qu'est-ce qu'un exercice de simulation de crise cyber ?
    \item *Une simulation d'une situation de crise pour tester la préparation de l'équipe
    \item Un test d'intrusion automatisé
    \item La gestion des vulnérabilités du réseau
    \item Une alerte de sécurité
\end{multi}

\begin{multi}[points=1, tags={M3Sec101, Secops}]{Gestion de crise cyber}
Quelle est la principale responsabilité d'une équipe de résilience cyber ?
    \item La coordination des réponses aux incidents
    \item *Assurer la continuité des opérations et la reprise après une cyberattaque
    \item La maintenance des serveurs
    \item La gestion des identités utilisateurs
\end{multi}

\begin{multi}[points=1]{Gestion de crise cyber}
Quel est l'objectif principal d'une communication externe pendant une crise cyber ?
    \item *Maintenir la confiance du public et des partenaires
    \item La collecte des preuves numériques pendant un incident
    \item La gestion des vulnérabilités du réseau
    \item Une analyse post-incident
\end{multi}

\begin{multi}[points=1]{Gestion de crise cyber}
Quelle est la différence entre une gestion de crise cyber et une réponse aux incidents ?
    \item* La gestion de crise est l'étape suivante d'une réponse à incident qui dépasse les capacités de décisions et de remédiation de la direction
    \item Il n'y a pas de différence
    \item* Une réponse à incident englobe toutes les actions dont la gestion de crise
    \item La gestion de crise cyber est réservée aux équipes techniques
\end{multi}

\begin{multi}[points=1]{Gestion de crise cyber}
Qu'est-ce qu'une évaluation post-crise cyber ?
    \item *L'évaluation des actions prises et des leçons apprises après la gestion d'une crise cyber
    \item La coordination des réponses aux incidents
    \item La maintenance des serveurs
    \item Une simulation d'attaque pour identifier les vulnérabilités
\end{multi}




% GENERATION POLITIQUE DE SECURITE


% Question 1
\begin{multi}[points=1]{ISO27001}
Quelle est la principale caractéristique de la norme ISO 27001?
\item* Elle fournit un cadre pour établir, mettre en œuvre, maintenir et améliorer un système de management de la sécurité de l'information.
\item Elle spécifie des exigences précises pour la conception d'un pare-feu.
\item Elle définit les protocoles de cryptage pour les réseaux sans fil.
\item Elle établit les exigences pour les logiciels antivirus.
\end{multi}

% Question 2
\begin{multi}[points=1]{Système de management de la sécurité}
Quel est le principal objectif d'un SMSI (Système de Management de la Sécurité de l'Information)?
\item* Assurer la confidentialité, l'intégrité et la disponibilité de l'information.
\item Gérer les ressources humaines de l'entreprise.
\item Optimiser les processus de marketing.
\item Accroître les bénéfices financiers de l'entreprise.
\end{multi}

% Question 3
\begin{multi}[points=1]{Politique de sécurité}
Quel est l'élément essentiel d'une politique de sécurité?
\item* La définition des règles et des responsabilités en matière de sécurité.
\item L'installation de logiciels antivirus sur tous les ordinateurs.
\item L'utilisation de mots de passe simples pour faciliter l'accès.
\item La réduction des coûts liés à la sécurité informatique.
\end{multi}

% Question 4
\begin{multi}[points=1]{Architectures fonctionnelles de sécurité}
Quel est le principal avantage d'une architecture de sécurité en couches?
\item* Elle permet une défense en profondeur contre les menaces.
\item Elle réduit le nombre de dispositifs de sécurité nécessaires.
\item Elle rend la maintenance du réseau plus complexe.
\item Elle simplifie la configuration des pare-feux.
\end{multi}

% Question 5
\begin{multi}[points=1]{ISO27001}
Qu'est-ce que la certification ISO 27001 garantit?
\item* La conformité aux normes internationales en matière de sécurité de l'information.
\item La protection absolue contre toutes les menaces informatiques.
\item La réduction des coûts de maintenance des serveurs.
\item La compatibilité avec tous les systèmes d'exploitation.
\end{multi}

% Question 6
\begin{multi}[points=1]{Système de management de la sécurité}
Quelle est la première étape dans la mise en œuvre d'un SMSI?
\item* L'engagement de la direction.
\item L'achat de matériel de sécurité de haute technologie.
\item La formation du personnel informatique.
\item La rédaction de politiques de sécurité détaillées.
\end{multi}

% Question 7
\begin{multi}[points=1]{Politique de sécurité}
Quelle est l'une des principales raisons pour lesquelles les politiques de sécurité échouent souvent?
\item* Le manque de sensibilisation et de formation du personnel.
\item L'existence de politiques trop strictes.
\item L'absence de surveillance des activités en ligne.
\item La dépendance excessive aux logiciels de sécurité.
\end{multi}

% Question 8
\begin{multi}[points=1]{Architectures fonctionnelles de sécurité}
Quel composant d'une architecture de sécurité est responsable de la limitation du trafic réseau entrant et sortant?
\item* Le pare-feu.
\item Le commutateur réseau.
\item Le serveur DNS.
\item Le routeur.
\end{multi}

% Question 9
\begin{multi}[points=1]{ISO27001}
Qu'est-ce que l'évaluation des risques dans le cadre de la norme ISO 27001?
\item* L'identification des menaces potentielles et de leurs impacts sur l'organisation.
\item La mise en place de pare-feux sur tous les points d'accès au réseau.
\item La vérification de la conformité des politiques de sécurité.
\item La classification des employés en fonction de leur niveau d'accès à l'information.
\end{multi}

% Question 10
\begin{multi}[points=1]{Système de management de la sécurité}
Qu'est-ce qu'un processus clé dans un SMSI?
\item* L'amélioration continue.
\item La désactivation des mises à jour automatiques.
\item La réduction des effectifs informatiques.
\item La suppression périodique des journaux d'activité.
\end{multi}


% Question 11
\begin{multi}[points=1]{ISO27001}
Quel est l'objectif principal de l'ISO 27001?
\item* Établir, mettre en œuvre, maintenir et améliorer un système de management de la sécurité de l'information.
\item Définir les normes pour la gestion des ressources humaines.
\item Élaborer des politiques de sécurité spécifiques pour les entreprises.
\item Créer des pare-feux personnalisés pour chaque organisation.
\end{multi}

% Question 12
\begin{multi}[points=1]{Système de management de la sécurité}
Qu'est-ce qu'un audit de sécurité?
\item* Une évaluation systématique de la sécurité d'un système ou d'une organisation.
\item Une mise à jour régulière des logiciels de sécurité.
\item Un processus pour réduire les coûts de sécurité.
\item Une analyse des tendances de sécurité sur les réseaux sociaux.
\end{multi}

% Question 13
\begin{multi}[points=1]{Politique de sécurité}
Qu'est-ce qu'une politique de classification de l'information?
\item* Une directive indiquant comment l'information doit être étiquetée et gérée en fonction de sa sensibilité.
\item Une liste de logiciels approuvés pour une utilisation sur le réseau de l'entreprise.
\item Une procédure pour l'installation de mises à jour logicielles.
\item Une politique interdisant l'utilisation de médias sociaux sur le lieu de travail.
\end{multi}

% Question 14
\begin{multi}[points=1]{Architectures fonctionnelles de sécurité}
Quel est l'avantage d'une architecture de sécurité en périphérie?
\item* Elle permet de protéger le réseau interne en contrôlant le trafic entrant et sortant.
\item Elle réduit la nécessité d'avoir des politiques de sécurité claires.
\item Elle rend la configuration des pare-feux plus complexe.
\item Elle augmente le risque de compromission des données sensibles.
\end{multi}

% Question 15
\begin{multi}[points=1]{ISO27001}
Quel est le rôle d'un responsable de la sécurité de l'information selon l'ISO 27001?
\item* Superviser la mise en œuvre et le maintien du SMSI (Système de Management de la Sécurité de l'Information).
\item Gérer les ressources financières de l'entreprise.
\item Assurer la sécurité physique des locaux de l'entreprise.
\item Développer des logiciels de sécurité personnalisés.
\end{multi}

% Question 16
\begin{multi}[points=1]{Système de management de la sécurité}
Qu'est-ce que la roue de Deming dans le contexte de la sécurité de l'information?
\item* Un cycle d'amélioration continue composé des étapes Plan-Do-Check-Act.
\item Un modèle de menace largement utilisé dans l'industrie de la cybersécurité.
\item Un algorithme de cryptage asymétrique.
\item Un protocole pour la détection des intrusions.
\end{multi}

% Question 17
\begin{multi}[points=1]{Politique de sécurité}
Pourquoi est-il important de sensibiliser régulièrement les employés à la sécurité de l'information?
\item* Pour réduire les risques d'attaques basées sur l'ingénierie sociale.
\item Pour éviter les pannes matérielles.
\item Pour augmenter la vitesse de connexion Internet.
\item Pour permettre l'accès à des sites Web non sécurisés.
\end{multi}

% Question 18
\begin{multi}[points=1]{Architectures fonctionnelles de sécurité}
Qu'est-ce qu'un proxy dans une architecture de sécurité?
\item* Un serveur intermédiaire qui agit comme un intermédiaire entre les utilisateurs et Internet.
\item Un type de pare-feu spécialement conçu pour les réseaux sans fil.
\item Un protocole de sécurité utilisé pour crypter les communications.
\item Un dispositif qui contrôle l'accès physique aux locaux de l'entreprise.
\end{multi}

% Question 19
\begin{multi}[points=1]{ISO27001}
Qu'est-ce qu'une analyse des risques selon l'ISO 27001?
\item* Une évaluation des menaces potentielles et de leurs impacts sur l'organisation.
\item Un test de pénétration des systèmes informatiques.
\item Un examen de la conformité des politiques de sécurité.
\item Une vérification de la disponibilité des mises à jour logicielles.
\end{multi}

% Question 20
\begin{multi}[points=1]{Système de management de la sécurité}
Qu'est-ce qu'une revue de direction dans le contexte de la sécurité de l'information?
\item* Une évaluation régulière de la performance du SMSI par la direction de l'entreprise.
\item Un processus pour réviser les politiques de sécurité chaque année.
\item Un audit externe mené par des experts en sécurité.
\item Un examen des journaux d'activité informatique.
\end{multi}

% Question 21
\begin{multi}[points=1]{Politique de sécurité}
Quelle est la principale raison pour laquelle les politiques de sécurité doivent être adaptées à chaque organisation?
\item* Chaque organisation a des besoins, des risques et des environnements informatiques uniques.
\item Les politiques de sécurité standard sont suffisantes pour toutes les entreprises.
\item Il est moins coûteux de copier les politiques d'autres entreprises.
\item Les politiques de sécurité ne sont pas importantes pour la plupart des organisations.
\end{multi}

% Question 22
\begin{multi}[points=1]{Architectures fonctionnelles de sécurité}
Quel est l'objectif principal d'un pare-feu dans une architecture réseau?
\item* Contrôler le trafic entrant et sortant pour protéger le réseau contre les menaces.
\item Réduire la vitesse de connexion Internet.
\item Surveiller les activités des utilisateurs en ligne.
\item Fournir un stockage sécurisé des données sensibles.
\end{multi}

% Question 23
\begin{multi}[points=1]{ISO27001}
Quelle est la différence entre une politique de sécurité et une procédure de sécurité?
\item* Une politique de sécurité énonce les règles générales, tandis qu'une procédure de sécurité décrit les étapes spécifiques à suivre.
\item Une politique de sécurité est utilisée pour contrôler l'accès au réseau, tandis qu'une procédure de sécurité gère les mises à jour logicielles.
\item Une politique de sécurité est mise en œuvre par les utilisateurs finaux, tandis qu'une procédure de sécurité est gérée par le département informatique.
\item Une politique de sécurité concerne uniquement la sécurité physique, tandis qu'une procédure de sécurité concerne la sécurité informatique.
\end{multi}

% Question 24
\begin{multi}[points=1]{Système de management de la sécurité}
Quel est l'objectif principal d'une évaluation des risques dans le cadre d'un SMSI?
\item* Identifier les menaces potentielles et évaluer leur impact sur l'organisation.
\item Établir des politiques de sécurité strictes.
\item Optimiser les processus de marketing.
\item Améliorer l'efficacité des communications internes.
\end{multi}

% Question 25
\begin{multi}[points=1]{Politique de sécurité}
Quelle est l'une des principales raisons pour lesquelles les politiques de sécurité sont souvent contournées par les employés?
\item* Les politiques de sécurité peuvent être perçues comme étant trop contraignantes ou inefficaces.
\item Les employés sont généralement bien formés sur les politiques de sécurité.
\item Les politiques de sécurité sont souvent trop laxistes.
\item Les employés ne sont pas conscients des politiques de sécurité de leur entreprise.
\end{multi}

% Question 26
\begin{multi}[points=1]{Architectures fonctionnelles de sécurité}
Qu'est-ce qu'un DMZ (zone démilitarisée) dans une architecture réseau?
\item* Un sous-réseau situé entre le réseau interne et Internet, utilisé pour héberger des services accessibles au public.
\item Une zone réservée aux employés de l'entreprise pour leurs activités de loisirs en ligne.
\item Un espace de stockage sécurisé pour les données sensibles.
\item Un dispositif qui bloque les attaques provenant de l'extérieur du réseau.
\end{multi}

% Question 27
\begin{multi}[points=1]{ISO27001}
Qu'est-ce qu'un actif de l'information selon l'ISO 27001?
\item* Tout élément qui a de la valeur pour une organisation et qui nécessite une protection appropriée.
\item Un logiciel antivirus installé sur les ordinateurs.
\item Un dispositif de sécurité physique comme une serrure de porte.
\item Un fichier de sauvegarde stocké sur un disque dur externe.
\end{multi}

% Question 28
\begin{multi}[points=1]{Système de management de la sécurité}
Quel est le rôle d'un responsable de la sécurité de l'information dans un SMSI?
\item* Planifier, mettre en œuvre et maintenir le système de management de la sécurité de l'information.
\item Gérer les opérations quotidiennes du service informatique.
\item Rédiger des rapports financiers pour la direction de l'entreprise.
\item Développer des applications logicielles personnalisées pour l'entreprise.
\end{multi}

% Question 29
\begin{multi}[points=1]{Politique de sécurité}
Qu'est-ce qu'une attaque par hameçonnage (phishing)?
\item* Une tentative d'obtenir des informations sensibles en se faisant passer pour une entité de confiance.
\item Une technique pour intercepter les communications sans fil.
\item Un type d'attaque qui vise à submerger un système de requêtes malveillantes.
\item Un processus pour créer des sauvegardes de données régulières.
\end{multi}

% Question 30
\begin{multi}[points=1]{Architectures fonctionnelles de sécurité}
Quel est l'objectif principal d'un IPS (Système de Prévention des Intrusions)?
\item* Détecter et empêcher les intrusions non autorisées sur le réseau.
\item Contrôler l'accès physique aux locaux de l'entreprise.
\item Crypter les communications sur le réseau.
\item Surveiller l'activité des utilisateurs sur les ordinateurs de l'entreprise.
\end{multi}

% Question 31
\begin{multi}[points=1]{ISO27001}
Quelle est la principale différence entre la certification ISO 27001 et la conformité à la norme?
\item* La certification est délivrée par un organisme de certification tiers, tandis que la conformité est auto-déclarée.
\item La conformité est obligatoire, tandis que la certification est facultative.
\item La certification est moins contraignante que la conformité.
\item La certification est valable pour une année, tandis que la conformité est permanente.
\end{multi}

% Question 32
\begin{multi}[points=1]{Système de management de la sécurité}
Quelle est la première étape dans le processus de gestion des incidents de sécurité?
\item* Identifier et classer l'incident.
\item Isoler le système affecté du réseau.
\item Notifier les autorités locales.
\item Réparer les dommages causés par l'incident.
\end{multi}

% Question 33
\begin{multi}[points=1]{Politique de sécurité}
Qu'est-ce qu'une politique de gestion des mots de passe?
\item* Une directive qui établit des règles pour la création et l'utilisation de mots de passe sécurisés.
\item Un document qui répertorie tous les mots de passe utilisés dans l'entreprise.
\item Un processus pour réinitialiser les mots de passe oubliés.
\item Une politique qui interdit l'utilisation de mots de passe pour accéder au système.
\end{multi}

% Question 34
\begin{multi}[points=1]{Architectures fonctionnelles de sécurité}
Qu'est-ce qu'un VPN (réseau privé virtuel)?
\item* Un réseau sécurisé qui permet aux utilisateurs d'accéder à des ressources informatiques à distance.
\item Un logiciel antivirus pour les ordinateurs personnels.
\item Un dispositif de surveillance de l'activité Internet des employés.
\item Un protocole de cryptage pour les réseaux sans fil.
\end{multi}

% Question 35
\begin{multi}[points=1]{ISO27001}
Qu'est-ce qu'une analyse d'impact sur la sécurité?
\item* Une évaluation des consé

quences potentielles des menaces sur les actifs de l'information.
\item Un test de vulnérabilité des systèmes informatiques.
\item Un processus pour vérifier la conformité aux politiques de sécurité.
\item Une enquête sur les incidents de sécurité passés.
\end{multi}

% Question 36
\begin{multi}[points=1]{Système de management de la sécurité}
Qu'est-ce qu'un tableau de bord de sécurité?
\item* Un outil pour surveiller et mesurer les performances du SMSI.
\item Un dispositif de sécurité physique installé dans les locaux de l'entreprise.
\item Un document décrivant les procédures de réponse aux incidents de sécurité.
\item Un logiciel de gestion des mots de passe.
\end{multi}

% Question 37
\begin{multi}[points=1]{Politique de sécurité}
Quelle est l'une des meilleures pratiques pour assurer la sécurité des courriels?
\item* Utiliser le chiffrement pour protéger les données sensibles.
\item Ouvrir tous les courriels sans les examiner.
\item Partager les mots de passe par courriel.
\item Cliquer sur tous les liens dans les courriels, même s'ils semblent suspects.
\end{multi}

% Question 38
\begin{multi}[points=1]{Architectures fonctionnelles de sécurité}
Qu'est-ce qu'un NIDS (Système de Détection d'Intrusion sur Réseau)?
\item* Un système conçu pour détecter les activités suspectes sur un réseau.
\item Un dispositif de sécurité physique installé sur les ordinateurs.
\item Un logiciel antivirus pour les serveurs de messagerie.
\item Un protocole de cryptage pour les communications en ligne.
\end{multi}

% Question 39
\begin{multi}[points=1]{ISO27001}
Qu'est-ce qu'une politique de sauvegarde des données?
\item* Une directive qui établit des règles pour la sauvegarde régulière et sécurisée des données.
\item Un processus pour supprimer définitivement les données obsolètes.
\item Un protocole pour crypter les données lors de leur transmission.
\item Un dispositif de stockage sécurisé des données sensibles.
\end{multi}

% Question 40
\begin{multi}[points=1]{Système de management de la sécurité}
Qu'est-ce que la sensibilisation à la sécurité?
\item* L'éducation des employés sur les pratiques sécuritaires et les risques potentiels.
\item L'installation de logiciels de sécurité sur tous les ordinateurs.
\item La surveillance constante des activités en ligne des employés.
\item La restriction de l'accès à Internet sur le lieu de travail.
\end{multi}




\end{quiz}

 \end{document}