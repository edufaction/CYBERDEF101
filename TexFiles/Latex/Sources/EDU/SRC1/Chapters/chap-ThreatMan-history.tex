%-------------------------------------
% Chapitre
% Threat Management
% Intro
% File : chap-ThreatMan-history.tex
%--------------------------------------
% sources 
% https://channellife.co.nz/story/a-brief-history-of-cyber-threats-from-2000-to-2020
\uchap{Menaces et histoires}
%_-_-_-_-_-_-_-_-_-_-_-_-_-_-_-_-_-_-_-_-_-_-_-_-_-_-_


\subsection{Quelques éléments d'histoire des menaces d'aujourd'hui}

« Pour prévoir l'avenir, il faut connaître le passé, car les événements de ce monde ont en tout temps des liens aux temps qui les ont précédés. Créés par les hommes animés des mêmes passions, ces événements doivent nécessairement avoir les mêmes résultats. » disait Nicolas Machiavel.

 Les événements passés peuvent éclairer les tendances futures, selon la sagesse courante - et la cybersécurité ne fait pas exception.
Les rapports annuels sur les menaces offrent aux équipes de sécurité l'occasion de réfléchir aux cyber-événements importants des 12 derniers mois, dans le but d'identifier les tendances pour le développement futur, idéalement se traduisant par une meilleure protection. Mais il peut être interessant de replonger un peu dans l'histoire.
De nombreux événements de cybersécurité importants se sont produits depuis le fameux passage de l’an 2000, il y a donc plus de 20 ans.

\subsubsection{Les années 2000 - L'ère du développement des vers}
Cette époque a vu certains des vers les plus prolifiques que l'industrie de la sécurité de l'information ait jamais connus, coûtant plus de 100 milliards de dollars en dommages et en coûts de réparation. Cela marque également le début du malware en tant que sensation médiatique grand public.

On peut noter  en particulier le ver ILOVEYOU, lancé en 2000, qui ciblait les utilisateurs de Microsoft Outlook et infectait au moins 10\% des hôtes connectés à Internet en quelques heures et causait alors jusqu'à 15 milliards de dollars de dommages.

En réponse, Microsoft a publié une mise à jour d'Outlook avec des modifications visant à lutter contre les pires symptômes de ILOVEYOU, notamment en empêchant les utilisateurs d'accéder à des pièces jointes dangereuses et en avertissant les utilisateurs si un programme tentait d'envoyer du courrier en leur nom.

Puis ce fut une véritable vague de vers dans les cinq années suivantes, exploitant les vulnérabilités du système d'exploitation Windows en particulier et  de l'infrastructure réseau. On peut noter en particulier

\begin{itemize}
  \item CodeRed (juillet 2001)
  \item Code Red II (août 2001)
  \item Nimda (septembre 2001)
  \item SQL Slammer (janvier 2003)
  \item Blaster (août 2003)
  \item Welchia (août 2003)
  \item Sobig.F (août 2003)
  \item Sober (octobre 2003)
  \item Bagle (janvier 2004)
  \item MyDoom (janvier 2004)
  \item Netsky (février 2004)
  \item Sasser (avril 2004)
\end{itemize}


Beaucoup de ces vers ont fortement utilisé des vulnérabilités de type "débordement de tampon/pile" dans diverses versions de Windows ou dans des applications telles que Internet Information Services (IIS) ou SQL Server les applications phares de Microsoft dans les années 2000.

La prolifération de ces codes malveillants de type vers ont poussé les éditeurs à mettre en place des mécanismes de corrections organisés. Microsoft a par exemple lancé le "Patch Tuesday", une approche structurée et cohérente de la distribution des correctifs corrigeant la distribution des correctifs ad hoc ou dans le cadre de service packs.

Bien structuré autour des firewall, le filtrage périmètrique évolue aussi vers la vérification des e-mails qui sont sous 

Le filtrage des e-mails s'est également amélioré à cette époque, les fournisseurs utilisant davantage de logiciels de détection pour leurs produits. Mais alors que les menaces par e-mail continuaient leurs attaques malgré tout, de nombreux cybercriminels passaient à leur prochain programme: gagner de l'argent.

\subsubsection{2005-2012 - L'ère du business malveillant}
Les cybercriminels, qui jusque-là provoquaient des perturbations principalement pour la notoriété ou la curiosité, ont commencé à penser à leur passe-temps comme une source de revenu. La cybercriminalité, soutenue par un internet portant la digitalisation de échange marchant, a trouvé aussi dans ces nouveaux "chimistes" une capacité à faire fructifier dans le monde numérique leurs techniques et méthodes criminelles classiques. C'est le début de la digitalisation du crime, en même temps que l'ouverture de monde numérique comme  nouvel espace de conflits des états, nouvel espace de bataille des militaires.

De nombreuses voies sont apparues pour transférer les bonnes idées de la filouterie et de la criminalité du monde réel au monde numérique, exploitant l'éventail des talents de l'écosystème de la cybercriminalité : Le malvertising, le spam, les botnets et les chevaux de Troie n'étaient que le début que la production de codes et de services malveillants Les cybercriminels, comme leurs homologues du monde réel, se sont organisés et ont créé une véritable industrie du cybercrime et du crime numérisé.

\textbf{Spam}

L'usage malveillant de la messagerie, s'est fortement basé sur les techniques liées au spam. Le SPAM est un bon moyen efficace de propager des vers; une lettre en chaîne d'un ami ou d'un parent involontaire était typique. Mais il est rapidement devenu monétisé par des escroqueries en ligne.

L'une des campagnes de spam coordonnées les plus connues a été la prolifération du spam  pour les  pharmacies en-ligne. Les «webmasters» généreraient du trafic vers les magasins en ligne gérés via des services de SPAM, pour présenter de nombreux médicaments divers et varié sur ordonnance à des prix largement réduits. Ce fut une époque ou le monde fut inondé par la plus grande campagne pour une nouvelle molécule miracle, le VIAGRA essentiellement vendu par ces pharmacie en-ligne.

Avec des hôtes à l'épreuve des balles en place, les cybercriminels ont probablement gagné des milliards grâce au spam dans les pharmacies, et la cybercriminalité à motivation financière était là pour rester.

\textbf{Botnets}

L'un des botnets les plus prolifiques de l'époque était le botnet Storm, autrefois surnommé le « supercalculateur le plus puissant» du monde, certains experts estimaient en 2007 que les millions d'hôtes infectés le rendait plus puissant que le supercalculateur BlueGene d'IBM de l'époque.

Storm a été conçu pour la furtivité et le profit. À son apogée, les estimations de la taille du botnet allaient d’un million à 10 millions d’ordinateurs infectés. Storm s'est écarté des standarts de ses prédécesseurs sur le bruit et l'agressivité pour privilégier une approche plus patiente et silencieuse.

Dans le cadre de sa tactique furtive, le botnet a utilisé un modèle peer-to-peer distribué. Il a utilisé un DNS à flux rapide et un polymorphisme pour échapper aux défenseurs et infecter un nombre croissant d'ordinateurs.

Cette notoriété et cette efficacité a rapidement fait de Storm, la norme pour tous les futurs botnets tant sur le plan de l'architecture que des des techniques utilisées.

\textbf{Chevaux de Troie}

L'un des pères du business de la malveillance par  cheval de Troie est l'auteur du cheval de Troie bancaire Zeus / Zbot qui ciblait les utilisateurs principalement par le biais du spam, du phishing, de la publicité ou de l'ingénierie sociale. Ce code est rapidement passé d'un simple cheval de Troie bancaire à un kit de de type crimeware à part entière et a marqué le début du cybercrime-as-a-service.

Le prix des licences Zeus ont commencé aux alentour de 1000 euros, mais l'auteur a rapidement proposé des versions personnalisées à un prix plus élevé, qui comprenait une assistance 24h / 24 et 7j / 7 et étaient distribuées par le biais de sociétés affiliées.

Le code source de Zeus a été divulgué en ligne en 2011, ce qui a permis à d'autres  cybercriminels moins techniques d'apprendre de l'un des kits de logiciels malveillants efficace les techniques efficaces et les plus répandues à cette époque. Cette fuite de code Zeus serait responsable de plusieurs variantes, notamment Citadel, Gameover Zeus, ICE-9, CIDEX, Ramnit, Dridex, Kronos, Tinba et Panda.

C'est la grande époque qui a fait exploser les besoins en services de filtrage des e-mails en raison de la flambée du spam et du phishing, tandis que les kits d'exploitation et le malvertising ont conduit à un filtrage supplémentaire du contenu Web.  De nouveaux produits et services de sécurité ont vu le jour. 

La coopération entre l’industrie de la sécurité de l’information, l’application de la loi et les processeurs de paiement a profondément affecté les opérations lucratives des cybercriminels, comme celle mise en œuvre par le cheval de Troie Reveton.

% https://www.cert-ist.com/public/fr/SO_detail?code=201301_article


\subsection{2013 à aujourd'hui - le Rançon-logiciel}

Ce n'est peut-être pas la seule caractéristique déterminante des codes malveillants de cette époque, mais les rançon-logiciels ont sans aucun doute eu l'impact le plus significatif sur les stratégies de cybersécurité des systèmes d'information.

À partir de 2020, les estimations des dommages causés par les attaques de rançon-logiciels se chiffrent en millions d'euros sur l''ensemble de l'économie mondiale. Ces rançon-logiciels exploitent les vulnérabilités des systèmes informatiques qui déploient très rapidement des 
 nouvelles technologies pas ou peu matures.

\subsubsection{CryptoLocker}
Lancée en 2013, l'attaque rançon-logiciels CryptoLocker a fourni aux criminels une formule gagnante en combinant deux technologies : le chiffrement et la crypto-monnaie. Cryptolocker et beaucoup de ses descendants ont également ressuscité un ancien vecteur de menace qui était en sommeil depuis plus d'une décennie : les macros malveillants dans les logiciels de documents.

Depuis lors, de nombreux groupes à l'origine de certaines des familles de rançon-logiciels les plus prolifiques ont perfectionné leurs compétences et se sont adaptés à un environnement en mutation.

Aucun individu ni aucune industrie n'est à l'abri d'une attaque de rançon-logiciel. Aucune technologie seule ne suffit pour arrêter ces typologies de menaces car elles utilisent des vulnérabilités multiples (HOT : Humaines (ingénierie sociale), techniques, Organisationnelles). 

\subsubsection{Double-extorsion}

Avec le rançon-logiciel Maze, les criminels ont popularisé un méthode devenue une norme commune aujourd'hui : la double extorsion, par laquelle les cybercriminels non seulement chiffrent et volent des données, mais menacent de les publier si les cibles ne paient pas.

Même si une entreprise pouvait se remettre complètement d'une attaque de rançon-logiciel sans payer la rançon, elle pourrait toujours se retrouver avec de grandes difficultés tant réglementaire que contractuelle quand des données volées contenant des informations de clients ou de partenaires se retrouvent  publiées

Sous la menace d'une violation de données , certaines entreprises peuvent aussi choisir de payer la rançon plutôt que les amendes officielles qui leur sont imposées. Payer la rançon pourrait être l'option la moins chère des deux, compte tenu des coûts de récupération et des dommages à la marque associés à cette violation. 

\subsubsection{La bataille continue}

Les rançon-logiciels ont bouleversé le monde de la cybersécurité peut-être plus que tout autre vecteur d'attaque.

En 2020, les opérateurs de Maze ayant pris un retraite surement bien méritée, c'est le système EGREGOR qui a pris le relais avec des temps de latence réduite pour déployer ses mécanismes de chiffrement. Les opérateurs ont par ailleurs bien industrialisées les demandes de rançons.

De nombreuses organisations ont revu leur politique de sécurité pour faire face à ces menaces. La notion de plan de cyberdéfense  vu le jour. 

Mais cela nous  conduit aussi à réévaluer les contrôles de sécurité, à créer ou à renforcer les cultures de sécurité au sein des organisations, mais aussi innover de nouveaux produits et services de cybersécurité.

La bataille contre la criminalité numérique continue, elle est un autre d'une bataille plus large incluant la bataille des états dans le Cyberespace. Tous ces champs s'interpénètrent Cyberdefense d'Entreprises, d'Etat, cybercriminalité,  cyberespionnage. La bataille continue. 



