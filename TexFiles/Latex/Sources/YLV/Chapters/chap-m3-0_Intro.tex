


\section{Introduction Architecture}

CONSTRUIRE vs Politique Architecture de sécurité vs Sécurité des architectures (Crypto, Sécurité Perimétrique). - PERIMETRE, CLOISENEMENT, ZERO TRUST, FILTRAGE, DETECTION, BASTION  … CHÂTEAU FORT (Mur) / AEROPORT (Portique) - Contrôle d'accès à tout objets

Appréhender la complexité de l'environnement technique du client (Protection, Défense, Résilience) et identifier les composants services, terminaux et réseaux du SI afin de les intégrer dans la politique de sécurité protective.
Découvrir les modèles d'architecture de sécurité périmétrique afin d'assurer le filtrage nécessaire et la détection des menaces sur le SI.


%\picframe{../Latex/Sources/YLV/Pictures/img-crypto1}{Label sous titre de l'image}{0.8}{labelRef}

%Begin FRAME----------------------------
\mode<presentation>{\texframe
% contenu affiché sur Article et Beamer
%- - - - - - - - - - - - - - - - - - - - - - - - 
{Architectures, Composants et Sécurité} % titre de la diapo
{Introduction} % sous titre de la diapo
{
%begin slide- - - - - - - - - - - - - - - - - 
	
 \begin{enumerate}
     \item De l’analyse de risques aux fonctions de sécurité
     \item Architecture et composants de système d’information
     \item Modèles de sécurité et technologies de sécurité protectrices
     \item Sécurité Endpoints
 \end{enumerate}
    
%end slide- - - - - - - - - - - - - - - - - - - 
}}
%End FRAME------------------------------
