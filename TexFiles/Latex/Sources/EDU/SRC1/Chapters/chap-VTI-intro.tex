%-------------------------------------------------------------
%               FR CYBERDEF SECOPS COURSE
%                                        SECOPS
%.                    Vulnerability/Threat/Incident
%
%                                    Introduction
%                                chap-VTI-intro.tex
%
%                              2020 eduf@ction
%-------------------------------------------------------------
\uchap{Chapitre introductif de la Partie 3 du cours SEC 101 (VTI-INTRO)}


%=======================================
\section{Sécurité opérationnelle}
%--------------------------------------------------------------

En terme de gouvernance, après avoir construit une structure de sécurité cohérente sur les aspects de gestion des flux, de gestion des accès et des identités, et construit une gouvernance efficace sur la base de l'\uindex{ISO27001}, il est  nécessaire de maintenir le niveau de sécurité de l'entreprise ou du système. La dynamique de sécurité de l'entreprise en exploitation nécessite une organisation et des politiques orientées vers cette efficience de l'anticipation, de la détection et de la réaction que nous avons présentées comme le volet cyberdéfense de la cybersécurité de l'entreprise.


% begin frame 
\mode<all>{\texframe{sécurité opérationnelle}{Dans le cycle de vie}
{
Dans certains ouvrages ce processus est dénommé  \g{Maintien en Condition de sécurité}. En utilisant  les termes anglo-saxons définissant le cycle des projets, nous pourrions positionner ses activités dans la phase dite de  \textbf{RUN}. Les autres phases en amont pouvant être définies comme : 

\begin{itemize}
 \item \textbf{THINK/DESIGN} : Des risques évalués à la politique sécurité établie en fonction des risques ;
 \item \textbf{BUILD}: De la politique de sécurité déployée à la construction d’une sécurité implémentée;
\end{itemize}
}} % end frame




% begin PRZ==========================
\begin{frame}
\frametitle<presentation>{sécurité opérationnelle}
\framesubtitle<presentation>{l'activité RUN}
% end header PRZ======================
\upicture{../Latex/Sources/EDU/SRC1/Pictures/img-thinkrun}{les phases du cycle de vie}{0.7}{lblthinkrun}

Et nous classons donc dans la dernière phase du cycle de vie : les activités d'exploitation de la sécurité, \textbf{RUN} : Des évènements de sécurité gérés à une \textbf{Cybercrise} maitrisée, ce que certains appellent \textbf{\uindex{SECOPS}} : \g{\uindex{sécurité Opérationnelle}}.
\end{frame}
% end PRZ ===========================


% begin PRZ==========================
\begin{frame}
\frametitle<presentation>{Objectifs adaptés aux finalités}
\framesubtitle<presentation>{de l'activité de l'entreprise}
% end header PRZ======================
Ce modèle se développe bien entendu en fonction des finalités de l'entreprise.

\begin{itemize}
    \item Soit nous sommes dans \textbf{une dynamique entreprise} et ces processus sont ceux mis en place pour s'assurer que  l'ensemble de actions sont prises en compte pour maîtriser les fragilités dont les vulnérabilités informatiques, détecter les menaces tant en anticipation que pendant des attaques (bruyantes, ou discrètes), et réagir pour maintenir l'activité et limiter l'impact.
    \item  Soit nous sommes \textbf{fabricant d'un produit ou d'un service}, et au delà des engagements sécuritaires de toute entreprise (cf. ci-dessus) des processus de \g{maintien en condition de sécurité} des produits et services sont à ajouter pour maîtriser les vulnérabilités, les correctifs et leur cycle de vie  (audit, communication, gestion des découvertes de fragilités par des tiers, rémunération de BugHunters ...).
\end{itemize}
\end{frame}
% end PRZ ===========================

% begin PRZ==========================
\begin{frame}
\frametitle<presentation>{sécurité opérationnelle}
\framesubtitle<presentation>{Une définition}
% end header PRZ======================
 Le terme de \g{sécurité opérationnelle}, est relativement jeune dans l’histoire de la sécurité des technologies de l’information. Le terme de \uindex{SSI}(sécurité des Systèmes d’Information) était né pour distinguer des disciplines qui s’attachaient à protéger l’information qui circulent dans les systèmes d'information de l’entreprise (cf. protection et classification de l’information) vis à vis de la sécurité des biens et des personnes. La sécurité des réseaux et la sécurité informatique ont été les précurseurs de la cybersécurité, le cyber recouvrant en un seul terme, les enjeux de sécurité liés au réseau et à l’informatique, mais plus largement à la sécurité de l'économie numérique.
\end{frame}
% end PRZ ===========================

Comme nous l’avons abordé dans l'introduction et dans les chapitres précédents, la cybersécurité est un domaine vaste qui regroupe de nombreuses disciplines. Elle peuvent intervenir dans des cycles projets pour construire des systèmes sûrs ou pour assurer la continuité d’activité et la protection des patrimoines dans l'entreprise. Il faut aussi, penser à y ajouter aussi tout un espace de gestion des conformités (législatives, réglementaires, normatives, contractuelles).


C’est plutôt dans un contexte d'opération sécurité au quotidien que l’on parle de sécurité opérationnelle. Ces activités opérationnelles supportent donc le maintien en condition de sécurité au quotidien de l’entreprise. En France, au sein des armées, on parle de lutte informatique défensive permettant de différencier les activités des Cyber-défense des activités de Cyber-protection. Ces activités sont à opposer à la lutte informatique\g {offensive} qui ne sera pas abordé dans c\ecours car elle relève de prérogative des états et non des entreprises. Nous aurons toutefois l'occasion d'aborder le \textbf{Hackback}, dans un prochain.
La sécurité opérationnelle ajoute par ailleurs à son périmètre de surveillance de l'intérieur de la zone de responsabilité de l'entreprise SI, réseaux sociaux, services cloud ...), celle extérieure à l'entreprise via des mécanismes de veille sur la menace et de surveillance des compromissions potentielles. Nous pourrions évoqué l'image d'une cité où \g{Les murs sont épais et solides, les douaniers sont aux portes de la cité, la police doit toutefois veiller à la sécurité des biens et des citoyens dans la ville, car certains sont néanmoins des brigands. Quand à l'armée, elle veille aux frontières du pays informée par des agents à l'étranger}.

On traitera donc cette partie avec une équivalence dans les terminologies suivantes :

% begin PRZ==========================
\begin{frame}
\frametitle<presentation>{Plusieurs terminologies, une dynamique}
% end header PRZ======================
	\begin{itemize}
		\item Maintien en condition de sécurité (\uindex{MCS}); 
		\item Sécurité opérationnelle (\uindex{SECOPS});
		\item Lutte informatique défensive (\uindex{LID});
		\item Cyberdéfense au sens de la cyberdéfense d'entreprise (CYBERDEFENSE).
	\end{itemize}
\end{frame}
% end PRZ ===========================

Le but de cette sécurité opérationnelle est d’être au coeur de l’action de la sécurité de l’entreprise. En effet, la sécurité de l’entreprise est une propriété multiforme. Elle est d’abord statique dans la mesure où elle correspond à un niveau de confiance dans l’environnement pour conserver la disponibilité, la confidentialité et l’intégrité de l’entreprise. Cette forme statique est souvent liée à la conformité de l’entreprise, aux différents référentiels sécuritaires (ISO27000, GDPR, LPM, NIS ...), mais surtout aux objectifs sécurité de l'entreprise face à ses risques et aux exigences de sécurité des clients souvent inscrites dans des plans d’assurance sécurité (Cf. PAS). 
Elle est aussi dynamique car c’est aussi une propriété systémique qui mesure la capacité à anticiper les menaces, identifier les fragilités , détecter en temps réel les attaques et réagir à temps ou au pire disposer des capacités de revenir dans un état de fonctionnement compatible avec la survie de l’entreprise (Modes dégradés par exemple).
Le système évolue, faisant apparaitre ici et là de nouvelles fragilités, l’entreprise se transforme, vit, suscitant de nouveaux potentiels d’attaques. L’entreprise doit s’organiser pour disposer de fonctions opérationnelles adaptées et dédiées à cette activité. Ces fonctions nécessitent des savoirs, des savoirs-faire et de l’outillage. C’est l’ensemble de ces techniques que nous allons tenter d’aborder dans c\edoc.

Globalement, on peut remarquer que le cycle de vie est à prendre dans le sens inverse de notre présentation. Dans les entreprises moins matures en gouvernance de la sécurité, la dynamique de cette sécurité opérationnelle est la première visible et opérée. L'entreprise va réagir le plus souvent dans une dynamique de réponse immédiates aux problèmes de sécurité sans pour autant investiguer plus avant dans les fragilités globales. Les mécanismes de cybersécurité sont donc construits dans une entreprise peu mature dans le sens suivant :

% begin PRZ==========================
\begin{frame}
\frametitle<presentation>{Enjeux SECOPS}
% end header PRZ======================
	\begin{itemize}
		\item Répondre aux incidents de sécurité, tenter de répondre à la question : \g{qui nous attaque et pourquoi}; 
		\item Améliorer les filtrages;
		\item Couvrir les vulnérabilités découvertes;
		\item Rechercher les vulnérabilités existantes dans le périmètre de responsabilité ;
		\item Anticiper les attaques;
		\item Anticiper les risques informatiques;
		\item Anticiper les risques sur l'information;
		\item Anticiper la menace.
	\end{itemize}
\end{frame}
% end PRZ ===========================

\section{Lutte contre la menace}

La finalité de cette défense d’entreprise est de lutter contre ces attaques qui ne sont pas qu’informatiques. L’attaquant peut utiliser des scenarii utilisant de nombreux vecteurs qui peuvent utiliser des fragilités organisationnelles ou humaines. On peut dire qu’une attaque est une fonction complexe, qui peut viser ou utiliser de nombreux facteurs internes et externes à l’entreprise. Ces facteurs constituent ce que certains nomment l’environnement numérique ou digital de l’entreprise. Cet environnement est globalement constitué de l’ensemble des outils, services, moyens informatiques ou réseaux utilisés par l’entreprise.
Mai 2017 a été un tournant dans la prise de conscience de la menace de la part des entreprises. Le Rançon-logiciel WannaCry a plus fait trembler les médias que les entreprises, mais a permis de faire comprendre au grand public les enjeux des menaces informatiques.


\begin{nota}[Paramètres d’une attaque]
\begin{equation}
Attaque = Fonction \left[ Fragilit\acute{e}s\,HOT\, Entreprise\otimes Gains\,Escompt\acute{e}s\,PF \right]
\end{equation}
\end{nota}

\begin{itemize}
	\item Fragilités HOT : Humaines, Organisationnelles, Techniques 
	\item Gains pour l'attaquant : Idéologiques, Politiques, Financiers, ...
\end{itemize}

On peut classer la majorité des attaques informatiques dans quatre grandes classes :

% begin PRZ==========================
\begin{frame}
\frametitle<presentation>{Grandes typologies des attaques numériques}
% end header PRZ======================
\begin{itemize}
\item Attaques \textbf{d’interception} d’information, vols par écoutes passives ou actives dans les flux transitant entre un émetteur et un récepteur;
\item Attaques par \textbf{déni de services}, généralement sur le réseau : Ce type d’attaque est une atteinte à la DISPONIBILITE du système, basé souvent sur la saturation d’une capacité de traitement. Le système saturé dans l’exécution de certaines de ses fonctions, ne peut plus répondre aux demandes légitimes, car il est occupé à traiter d’autres sollicitations;
\item Attaques par \textbf{exploitation de failles }logiciels : Ce type d’attaque va utiliser une vulnérabilité, d’un système d’exploitation ou d’un logiciel pour exécuter du code malveillant. Ce code réalisera alors sa mission;
\item Attaques par \textbf{exploitation de défauts} de configuration : Ce type d’attaque utilise simplement un ou des défauts de configuration pour que légitimement l’agresseur puisse dérouler un scénario, qui pourra lui donner par exemple des droits particuliers pour conduire des attaques.
\end{itemize}
\end{frame}
% end PRZ==========================

Nous pourrions remarquer que ce nombre est relativement faible. Toutefois, la vrai difficulté réside dans la multiplicité des vulnérabilités, et des défauts de configuration. Les développeurs réalisent des logiciels possédant des failles (vulnérabilités), les utilisateurs ou les administrateurs déploient des systèmes en faisant des erreurs de configuration, ou ne les configurent que très rarement en pensant à la malveillance.

Les motivations des attaquants sont nombreuses, et leurs objectifs variés :

% begin PRZ==========================
\begin{frame}
\frametitle<presentation>{Motivations de l'attaquant}
% end header PRZ======================

\begin{itemize}
\item obtenir un accès au système pour s’y maintenir en attendant une opportunité ;
\item récupérer de l’information, secrets, données personnelles exploitables (en gros toute information ayant de la valeur)
\item récupérer des données bancaires ;
\item s'informer sur l'organisation (entreprise de l'utilisateur, etc.) ;
\item troubler, couper, bloquer le fonctionnement d'un service (les \uindex{rançongiciels} entre dans cette catégories) ;
\item utiliser le système d’un utilisateur, pour rebondir vers un autre système ;
\item détourner les ressources du système d’un utilisateur (utiliser de la bande passante, utiliser de la capacité de calcul) ;
\end{itemize}
\end{frame}
% end PRZ==========================

Bien entendu, il n’y a que très rarement un seul objectif, c’est la combinaison des méthodes d’attaques, des objectifs unitaires qui définissent globalement une mission ou un objectif final. L’exploitation de vulnérabilités au sein de l’entreprise va permettre le déploiement par l’attaquant d’un scénario.

\subsection{Politiques et Stratégies}

% begin PRZ==========================
\begin{frame}
\frametitle<presentation>{Politique vs stratégie}
% end header PRZ======================

\upicture{../Latex/Sources/EDU/SRC1/Pictures/img-cyclevie-pol-strat}{Positionnement de la sécurité opérationnelle}{0.8}{lblpol-strat}

\end{frame}
% end PRZ==========================

A partir des risques identifiés, l’entreprise a posé des politiques de sécurité qui ont permis de mettre en place des mesures de sécurité. Ces mesures sont d’ordre techniques avec des systèmes de sécurité, ou des SI avec des architectures particulières, mais aussi d’ordre organisationnel avec des procédures et des mécanismes à respecter.
L’ensemble de cette dynamique construit un niveau de sécurité qu’il va être nécessaire de maintenir dans le temps. Toutefois ce niveau de sécurité n’est pas suffisant pour une simple et bonne raison : la menace évolue, les vulnérabilités apparaissent (découvertes, ou créées), la valeur\g{marchande} des actifs d’une entreprise change aussi. 
Les occurrences de ces éléments de vie sont considérés comme des évènements qu’il convient de détecter avec suffisamment d’avance sur l’attaquant pour pouvoir le plus rapidement les prendre en compte.

La gestion des événements qui peuvent être une source de mesure de l’évolution du niveau de sécurité de l’entreprise est au coeur des stratégies de cyberdéfense. Ces évènements sont corrélés avec des sources provenant de deux processus particuliers qui seront décrits dans ce c\edoc.

% begin PRZ==========================
\begin{frame}
\frametitle<presentation>{Politique vs stratégie}
% end header PRZ======================
Il est à noter qu'un attaquant ne raisonne pas en politiques d'attaque face à une politique de sécurité, mais par des stratégies auxquelles il faut opposer aussi par des stratégies de défense, dont 

\begin{itemize}
\item Recherche des vulnérabilités : Processus qui permet de rechercher, découvrir, couvrir les vulnérabilités ou fragilités de l’entreprise ou ayant un impact sur l’entreprise que celles-ci soient techniques, humaines ou organisationnelles ;
\item Prévention de la menace : Processus qui permet de connaître les menaces directes sur l’entreprise ou potentielles afin d’anticiper et/ou se préparer à un type d’attaque.
\end{itemize}
\end{frame}
% end PRZ==========================

C’est la confrontation entre les vulnérabilités, les menaces et la détection de l’activité de l’entreprise qui va permettre d’être efficace dans le processus de réponse. Il y a de nombreuses manière d’aborder la cyberdéfense d’entreprise.

C\edoc présente donc une dynamique de cyberdéfense en trois\g{volets }

% begin PRZ==========================
\begin{frame}
\frametitle<presentation>{3 volets d'une cyberdéfense}
% end header PRZ======================


\begin{itemize}
\item Gestion des vulnérabilités (\textit{Vulnerability Management and CERT}) : maîtriser ses vulnérabilités mais aussi surveiller l’environnement technologique. 
\item Surveillance, Détection de la menace (\textit{Event and Threat Management}) : Analyser en temps réel l’environnement protégé mais aussi surveiller l’écosystème lié à la menace pour anticiper 
\item Gestion des incidents et réponse aux incidents (\textit{Incident Response – CSIRT}) : Réagir en cas d’incident et assurer la remédiation
\end{itemize}

\end{frame}
% end PRZ==========================


% begin PRZ==========================
\begin{frame}
\frametitle<presentation>{3 volets d'une cyberdéfense}
% end header PRZ======================
\upicture{../Latex/Sources/EDU/SRC1/Pictures/img-triangle}{Des 3 des volets de la sécurité opérationnelle}{0.6}{lbltriptyque}

\end{frame}
% end PRZ==========================

Ces trois volets ne sont pas les seuls qui concourent à la cyberdéfense d’entreprise, mais il en reste les trois faces principales. Il est à noter que ces trois volets correspondent aussi en France à trois référentiels de qualification de l’ANSSI des prestataires de services de cybersécurité au profit des entreprises. Ces labels sont obtenus par les entreprises qui respectent un cahier des charges rigoureux sur le plan de l'éthique, du professionnalisme, et de la compétence des experts intervenants. Il y trois cadres principaux de certifications sont :

% begin PRZ==========================
\begin{frame}
\frametitle<presentation>{Référentiels ANSSI}
% end header PRZ======================
\begin{itemize}
\item PASSI : Prestataire d’Audit de la sécurité des systèmes d’information ;
\item PDIS : Prestataire de détection d’incident de sécurité ;
\item PRIS : Prestataire de réponse à incident.
\end{itemize}

Ces trois référentiels définissent l’ensemble des exigences d’assurance pour\g {qualifier} des prestataires de services en cybersécurité sur ces trois thématiques. En effet, il serait en effet important de confier la recherche de ses vulnérabilités, leurs remédiations à des sociétés de confiance. 
\end{frame}
% end PRZ==========================
A ces trois volets il ne faut pas oublier, le volet administration des briques informatiques et de télécommunications de l’environnement de l’entreprise. C’est un volet que nous traiterons pas directement dans c\edoc pour se concentrer sur les mécanismes de maintien en continu du niveau de sécurité de l’entreprise avec des mécanismes de veille, d’alerte et de réaction.

%TODO Leure et Hony Posts
%Leurres et Pots de Miel 
%Renseignement

\subsection{Stratégies d'action}

\upicture{../Latex/Sources/EDU/SRC1/Pictures/img-vti}{Les différentes actions de veille}{0.8}{lblvti}

% Begin PRZ ===========================
\begin{frame}
\frametitle<presentation>{Stratégies de l'action}

% end header PRZ =======================
La cyberdéfense est un ensemble de mécanismes liés à une stratégie de l'action. Les outils de cyberdéfense sont construits pour aider à surveiller l'environnement, détecter des menaces et/ou des attaques mais surtout agir et réagir pour limiter les impacts. Si les outils de protection sont configurés à partir d'éléments de politique de sécurité (droits, accès, filtrage ...), les outils de défense sont basés sur les stratégies des attaquants.
On distinguera donc ici trois grands mécanismes de Cyberdefense que les anglo-saxons appellent : 

\begin{itemize}
	\item Predictive Cyberdefense;
	\item Active and Proactive Cyberdefense;
	\item Reactive Cyberdefense.
\end{itemize}
\end{frame}

Nous aborderons, en particulier ces concepts quand nous évoquerons la notion de SOC (Security Operational Center) activité qui opère ce volet de cyberdéfense sachant que la veille sur l'environnement numérique reste un axe important.



% End PRZ ===========================

Il ne faut pas, par ailleurs, oublier le renseignement (\textit{Intelligence}), qui reste une des grandes étapes de la cyberdéfense domaine que nous explorerons sous son volet cyber avec les sources de \g{threat intelligence}, mais aussi avec le Renseignement d'Origine Cyber que les anglo-saxons nomme \g{intelligence cyber}

Dans les grandes organisations, une autre stratégie globale de la cyberdéfense est de penser l'anticipation et la détection de manière globale à l'environnement digital de l'entreprise tout en structurant la réaction de manière locale. 

%TODO https://en.wikipedia.org/wiki/Proactive_cyber_defence

Nous avons positionné l'audit technique comme une des activités fondamentales de la gestion des vulnérabilités.
En effet les techniques d'audit font partie des méthodes de référence pour disposer d'un état des fragilités de l'entreprise. On y trouvera donc les grands basiques des audits techniques que sont les tests d'intrusion, la sécurité applicative, l'audit de configuration, et le fuzzing.

Par ailleurs nous explorerons rapidement, les techniques de déception et de leurre qui font partie cette défense proactive avec les honeypots qui peuvent être couplés avec le \UKword{cyber-hunting}, technique de chasse aux codes malveillants dans l'entreprise.
 

\subsection{Les modèles de cybersécurité}

Les modèles ou framework de sécurité sont intéressants à plusieurs titres. Ils permettent en particulier, de disposer d'un ensemble d'exigences ou de bonnes pratiques organisées dans un référentiel connu et reconnu d'une communauté et  offre un cadre pour : 

\begin{itemize}
  \item Passer en revue les pratiques des organisations en matière de cybersécurité;
  \item Etablir ou améliorer son propre programme de cybersécurité;
  \item Effectuer une auto-évaluation des risques en matière de cybersécurité;
  \item Sensibiliser collaborateurs, partenaires, sous-traitants;
  \item Améliorer la communication entre organisations grâce à un échange d’exigences cybersécurité entre les partenaires commerciaux, les fournisseurs et les régulateurs (ANSSI par exemple)
\end{itemize}


% begin PRZ==========================
\begin{frame}
\frametitle<presentation>{Modèle NIST}
% end header PRZ======================

Il existe de nombreux modèles de description de l'activité de Cyberdefense dans un contexte de cybersécurité.
Certains sont totalement intégrés au modèle de cybersécurité comme l'ISO 27K, ou le Cybersecurity FrameWork du NIST \uref{FrameWork du NIST}{lblnist} avec les activités \textbf{DETECT, RESPOND et RECOVER};
\upicture{../Latex/Sources/EDU/SRC1/Pictures/img-nist}{modèle NIST}{0.7}{lblnist}
\end{frame}
% end PRZ==========================


% begin PRZ==========================
\begin{frame}
\frametitle<presentation>{NIST vs ISO 27k}
% end header PRZ======================
Ce que l'on peut reprocher au modèle du NIST, c'est qu'il ne possède pas explicitement la gestion des fragilités / vulnérabilités, mais il apporte toutefois un modèle très détaillé, que nous utiliserons pour partie.
\upicture{../Latex/Sources/EDU/SRC1/Pictures/img-risk27}{modèle ISO27 et risques}{0.8}{lblrisk27}
\end{frame}
% end PRZ==========================
Dans l'environnement ISO 27000, le modèle est piloté par les risques.\uref{Gestion des risques}{lblrisk27}.

Nous avons fait le choix de positionner la présentation du volet sécurité opérationnelle en nous éloignant un peu des modèles pour présenter les trois grands moteurs de la sécurité opérationnelle. En effet les modèles cités sont orientés sur un axe de cycle de vie.
En sécurité opérationnelle ou cyberdéfense, l'objectif est de conduire en continu et de front des processus de maîtrise des risques cyber opérationnels.
\begin{itemize}
  \item Les systèmes d'information évoluent en continu et des vulnérabilités peuvent s'insérer et/ou être découvertes chaque jour au grès des modifications et évolutions,
  \item Des menaces se concrétisent quotidiennement par des attaques ciblées ou non, nécessitant de réagir vite et en cohérence avec des enjeux de l'entreprise
  \item Avoir la capacité de réagir, et d'assurer la continuité d'activité face à des attaques d'ampleur, ou à fort impact techniques ou médiatique.
\end{itemize}

\newpage
\section{Processus SECOPS}


% Begin PRZ ===========================
\begin{frame}
\frametitle<presentation>{SECOPS en 3 thématiques}
% end header PRZ =======================
Notre propos sera donc centré sur ces trois axes  qui nous déclinerons dans trois chapitres. Le travail de fond d'une équipe de sécurité opérationnelle, ou simplement de l'activité SECOPS est de pouvoir gérer de front trois grandes tâches : 
\begin{itemize}
 \item maîtriser les fragilités numériques de l'entreprise (\UKword{Vulnerability Management)} quelles soient au sein du SI ou dans l'environnement dit digital de cette entreprise (réseaux sociaux, partenaires, ...);
 \item anticiper les menaces et les scénarios associés (\UKword{Threat Management)}, détecter les attaques et gérer au quotidien les événements de sécurité ;
 \item réagir vite et en cohérence avec l'activité de l'entreprise en cas d'incident (\UKword{Incident Management)}.
\end{itemize}
\end{frame}
% end PRZ ===========================

Nous aborderons aussi quelques compléments à ces processus SECOPS, comme la détection des fuites de données (Leak Detection), qui peut s'entendre comme un incident de sécurité externe, ou une détection d'évènements hors de périmètre du système informatique, mais dans le périmètre de surveillance.

% Begin PRZ ===========================
\begin{frame}
\frametitle<presentation>{Les processus SECOPS}
% end header PRZ =======================
\upicture{../Latex/Sources/EDU/SRC1/Pictures/img-chapvar}{Synthèse des méta-processus SECOPS}{0.8}{lblprosecops}

\end{frame}
% end PRZ ===========================

Ces activités nécessitent, pour être efficace, une symbiose parfaite entre les équipes qui gèrent l'activité digitale (Systèmes d'informations, réseaux sociaux,  communication...) et les équipes de sécurité opérationnelle.
Il ne faut pas oublier bien entendu les mécanismes de gouvernance sécurité globale (ISO 27001 par exemple) dans lesquels s'inscrit la sécurité opérationnelle. On trouve souvent dans les entreprises un RSSI dédié cette activité relevant soit du RSSI de la DSI soit d'un DSSI (Directeur de la SSI).

\section{Les métiers de la SECOPS}

Au delà des métiers de l'audit de sécurité qui existent depuis de nombreuses années, la sécurité opérationnelle est le champ de développement de nombreux métiers nouveaux ou en devenir. Nous en explorerons quelques-uns dans chacune des parties qui présentent les opérations de cette SECOPS.

% Begin PRZ ===========================
\begin{frame}
\frametitle<presentation>{Les métiers SECOPS}
% end header PRZ =======================
\upicture{../Latex/Sources/EDU/SRC1/Pictures/img-metiersVTI}{Des métiers SECOPS}{1}{lblmetiersVTI}
\end{frame}
% end PRZ ===========================

\section{Eléments Communs}

Nous trouverez dans chacun des chapitres un petit condensé du processus avec :

\begin{itemize}
  \item Les objectifs;
  \item Les outils;
  \item Les méthodologies et standards;
  \item Les métiers et compétences.
\end{itemize}
