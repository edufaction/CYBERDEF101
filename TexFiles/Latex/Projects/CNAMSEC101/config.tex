%-----------------------------------------------
%               FR CYBERDEF SECOPS COURSE
%                         GLOBAL CONFIG  File
%                             2020 eduf@ction
%-----------------------------------------------

% just used for quick access to samples on project editing tool
%%-------------------------------------------------------------
%               FR CYBERDEF SECOPS COURSE
%              SAMPLES for edx & u  commands
%                             2020 eduf@ction
%-------------------------------------------------------------


% 												I M A G E S   
%-------------------------------------------------------------------------
%\upicture{../Latex/Sources/EDU/Pictures/img-liovar-incidents}{Incidents}{01}{lb:mtbf}
%-------------------------------------------------------------------------


%\mode<all>{\picframe{../Latex/Sources/EDU/Pictures/img-cycle}{Cycle de vie de gouvernance Cyberdef}{0.3}{lbl:cycle}}


% Planche PRZ 
% begin frame \mode<all>{\texframe{Déroulement}{présentation Gestion de la menace}
%{

%}} % end frame



%\begin{table}[!hbtp]
%\resizebox{\textwidth}{!}{%
%\begin{tabular}{ l|l l}
%%\begin{tabular}{l*{3}{S[table-format=2.1]}}
%\toprule
%\textbf{Package\_name }                 & \textbf{C\&C }                                        & \textbf{sha256}                                                            \\ \hline
%caracal.raceinspace.astronaut & http://api.lulquid.xyz                       & f1d32c17a169574369088...\\ \hline
%
%\bottomrule
%\end{tabular}
%}
%\caption{Exemple de  données de TI (Threat Intelligence)}
%\label{tab:ExempleTI}
%\end{table}

%----------------------------
% Path
\newcommand{\upath}{Latex}

\newcommand{\utemplatepath}{Latex/Templates/Magic}
\newcommand{\utemplatetheme}{cnam} % MAGIC TEMPLATE Var
\newcommand{\uproject}{CNAMSEC101}


\newcommand{\upictureext}{pdf}

 %----------------------------
 % Basic Var
 
 %\setcounter{cntx}{\year+1}
\newcommand{\INFODistrib}{Notes de cours SECOPS 2022-2023}
\newcommand{\uJournalInfo}{Orange Cyber et Cloud avec CNAM Bretagne, Cybersécurité SEC101, edufaction}

\newcommand{\udescription}{La sécurité opérationnelle au coeur de la cyberdéfense d'entreprise}

\newcommand{\uCnam}{Conservatoire National des Arts et Métiers}
\newcommand{\uOCD}{Orange Campus Cyber}
\newcommand{\uauthor}{Eric DUPUIS}
\newcommand{\uauthorwriter}{edufaction}
\newcommand{\uproa}{Enseignement sous la direction du Professeur Véronique Legrand, \uCnam, Paris, France}
\newcommand{\uprob}{Directeur \uOCD}
\newcommand{\umaila}{eric.dupuis@lecnam.net}
\newcommand{\usitea}{http://www.cnam.fr}
\newcommand{\usiteb}{http://www.orange.com}
\newcommand{\umailb}{eric.dupuis@orange.com}
 %\newcommand{\uJournalInfo}{CNAM Bretagne, Cybersécurité SEC101, eduf@ction}
\newcommand{\uinstituteshort}{CNAM}
\newcommand{\uinstitute}{\uCnam}
\newcommand{\uchaire}{Chaire de Cybersécurité}
\newcommand{\uversion}{Cnam Bretagne}





%----------------------------
% All other variables are defined in
 % file : *.lesson.tex
\newcommand{\ushorttitle}{CYBERDEF SEC101}
\newcommand{\uCoursetittle}{Eléments de Sécurité Opérationnelle}
\newcommand{\uMinilogo}{Cyberdéfense d'entreprise}
\newcommand{\uCourseLongName} {Eléments de sécurité opérationnelle en cyberdéfense d'entreprise}
%\newcommand{\utitle}{Cyberdéfense d'entreprise}

%----------------------------
% Book versus Course Notes
%----------------------------
% Use "e document' "e cours"
\newcommand{\edoc}{e document\xspace}
\newcommand{\ecours}{e cours\xspace} 
\newcommand{\etitle}{éléments de sécurité opérationnelle en cyberdéfense d'entreprise} 
\newcommand{\fichetech}{une fiche TECHNO}


%%----------------------------
%% Spécifique BOOK
% %----------------------------
\newcommand{\printer}{Publication limitée pour le \uCnam \xspace et Orange Campus Cyber}

%%----------------------------
% Attention : Use hard coded Pictures Path
% (PicframeS)
\newcommand{\picpath}{Latex/Sources/EDU/Pictures}
%-------------------------------------------------------------

%%----------------------------
%% Spécifique COURSE
% %----------------------------

\newcommand {\parttitlecourse}{TITRE PARTIE (config)}
\newcommand{\coursenumber}{1}

\newcommand{\uGlossaryAcronyms}{
\loadglsentries{Latex/Sources/commons/acronymes.tex}
\loadglsentries{Latex/Sources/commons/glossaire.tex}
}
