%-----------------------------------------------
%               FR CYBERDEF SECOPS COURSE
%                         GLOBAL CONFIG  File
%                             2020 eduf@ction
%-----------------------------------------------

% just used for quick access to samples on project editing tool
%\input{../Latex/Projects/CNAMSEC101/Lessons/Commons/edxsamples.tex}

%----------------------------
% Path
\def\upath{../Latex}

%-------------------------------------------------------------
%			 FR CYBERDEF SECOPS COURSE
% 					INIT Commands File
% 		Select BEAMER or Article/book Mode
% 			with sub extension .doc or .prz
%                           2020 eduf@ction
%-------------------------------------------------------------
\RequirePackage{xstring}
\RequirePackage[realmainfile]{currfile}

\newcommand{\udocname}{\jobname}

\IfSubStr*{\currfilebase}{.prz}{\newcommand{\PRZMODE}{YES}}{\newcommand{\DOCMODE}{YES}}
   
\IfSubStr*{\currfilebase}{.edt}{\newcommand{\EDTMODE}{YES}}{\newcommand{\NOMODE}{YES}}

%===========================
% EDX LIBRARY
%===========================


\def\utemplatepath{*}
\def\utemplatetheme{*} % orange ou cnam
\def\uproject{*} % nom du projet


% Chargement des éléments de l'auteur
\newcommand{\setauthor}[1]{\newcommand{\uauthor}{Yann Arzel LE VILLIO}
\newcommand{\uauthorwriter}{\uauthor} %Beamer
\newcommand{\uproa}{Enseignant Sécurité ESIR}
\newcommand{\uprob}{Directeur Technique et Scientifique Orange CyberSchool}
\newcommand{\umaila}{yann-arzel.levillio@orange.com}
\newcommand{\usitea}{http://campus.orange.com}
\newcommand{\usiteb}{http://www.esir.fr}
\newcommand{\umailb}{umaila}}

\newcommand{\lessonID}[1]{\renewcommand{\lessonID}{#1}}



\def\utemplatepath{../Latex/Templates/Magic}

\def\uauthorphoto{\utemplatepath/commons.inc/CommonsPictures/missionauthorphoto.png}


\def\utemplatetheme{cnam}
\def\ubeamermodel{edx}
\def\uproject{CNAMSEC101}


\newcommand{\upictureext}{pdf}

 %----------------------------
 % Basic Var
 
 %\setcounter{cntx}{\year+1}
\newcommand{\INFODistrib}{Notes de cours SECOPS 2025-2026}
\newcommand{\uJournalInfo}{Orange Cyber et Cloud avec CNAM Bretagne et ESNA, Cybersécurité SEC101, edufaction}

\newcommand{\udescription}{La sécurité opérationnelle au coeur de la cyberdéfense d'entreprise}

\newcommand{\uUniversity}{Conservatoire National des Arts et Métiers}
\newcommand{\uCompany}{CNAM-ORANGE-ESNA}
 %\newcommand{\uJournalInfo}{CNAM Bretagne, Cybersécurité SEC101, eduf@ction}
\newcommand{\uinstituteshort}{CNAM}
\newcommand{\uinstitute}{\uUniversity}
\newcommand{\uchaire}{Chaire de Cybersécurité}
\newcommand{\uversion}{Cnam Bretagne}





%----------------------------
% All other variables are defined in
 % file : *.lesson.tex
\newcommand{\ushorttitle}{CYBERDEF SEC101}
\newcommand{\uCoursetittle}{SECOPS SEC101}
\newcommand{\uMinilogo}{Cyberdéfense d'entreprise}
\newcommand{\uCourseLongName} {Eléments de sécurité opérationnelle en cyberdéfense d'entreprise}
%\newcommand{\utitle}{Cyberdéfense d'entreprise}

%----------------------------
% Book versus Course Notes
%----------------------------
% Use "e document' "e cours"
\newcommand{\edoc}{e document\xspace}
\newcommand{\ecours}{e cours\xspace} 
\newcommand{\etitle}{éléments de sécurité opérationnelle en cyberdéfense d'entreprise} 
\newcommand{\fichetech}{une fiche TECHNO}


%%----------------------------
%% Spécifique BOOK
% %----------------------------
\newcommand{\printer}{Publication limitée pour le \uUniversity \xspace pour Orange et ESNA}

%%----------------------------
% Attention : Use hard coded Pictures Path
% (PicframeS)
\newcommand{\picpath}{../Latex/Sources/EDU/SRC1/Pictures}
%-------------------------------------------------------------

%%----------------------------
%% Spécifique COURSE
% %----------------------------

\newcommand {\parttitlecourse}{TITRE PARTIE (config)}
\newcommand{\coursenumber}{1}

\newcommand{\uGlossaryAcronyms}{
\loadglsentries{../Latex/Sources/commons/acronymes.tex}
\loadglsentries{../Latex/Sources/commons/glossaire.tex}
}

\newcommand{\ucourseresume}{\normalsize Aborder la sécurité des systèmes d’information sous l’angle d'une sécurité dynamique est un axe qui depuis quelques années apporte de nouvelle manière d’aborder la protection, la défense, et la résilience des systèmes d’information. La transformation digitale de l'entreprise modifie et rend plus flous les périmètres des systèmes d’informations. Cela nécessite une approche élargie du risque numérique et des nouvelles architectures de systèmes d'information intégrant des technologies de cloud et de cyberdéfense. Malgré la mise en place de mesures et de technologies de protection de plus en plus élaborées, l'impact d'une attaque ayant franchi ces frontières poreuses à considérablement augmenté.  Cette compilation des notes de cours élaborée dans le cadre d'un cours d'introduction à la cybersécurité et la cyberdéfense d'entreprise aborde les grands éléments fondamentaux permettant d'appréhender le domaine. Il permet d'en comprendre les enjeux, les codes, et acteurs. Protéger l'ensemble de l'entreprise alors qu'il est complexe de définir ses frontières est illusoire. Identifier les actifs essentiels ou vitaux et mettre en place les moyens adaptés à leur protection et leur défense est une démarche tactique qui permet de graduellement réduire ses cyber-risques tout en assurant un pilotage globale de la gouvernance de la sécurité sur ses volets de conformité.}
