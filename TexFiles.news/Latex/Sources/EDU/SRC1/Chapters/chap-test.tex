%-------------------------------------------------------------
%               FR CYBERDEF SECOPS COURSE
%                                        SECOPS
%                                            Intro
%
%                           Introduction Cyberdefense
%                            % Chap-Intro-Ref.tex
%
%                              2020 eduf@ction
%-------------------------------------------------------------



\section{Chapitre de test}

Pour l'entreprise la cybersécurité est un domaine  de nombreux cadres normatifs et réglementaires soutenus bien souvent par contraintes legislatives propres à chaque pays.

Cette normalisation et ses réglementations sont riches mais certaines fois complexes.
Le plus simple pour s'enrichir de ces savoirs et surtout pour disposer des meilleurs informations à la sources autant \g{fréquenter} les sites internet institutionnels des organismes qui sont et continuent à être les points de  référence dans le domaine de cybersécurité.

De nombreux services étatiques et de normalisation possèdent des activités dites Cyber dans leur structures :

\mode<all>{\texframe{Fonctions RSSI}{différents métiers} 
{
\begin{itemize}
\item \textbf{RSSI d'entreprise} : Responsable de la sécurité de sa structure.
\item  \textbf{RSSI d'un département, d'une organisation intermédiaire} : A l'image d'un RSSI d'entreprise, il assure toute les tâches de gouvernance, il applique et fait appliquer les directives et politique de sécurité aux équipes du département / division / structure intermédiaire, il déploie les actions décidées dans la chaîne fonctionnelle sécurité
\item   \textbf{RSSI d'un contrat, d'un projet contractualisé (Security Manager)} : Responsable de la sécurité du  déroulement d'un contrat. Souvent lié à un plan d'assurance sécurité, le RSSI contrat se doit d'assurer pour le client ou pour le fournisseur le suivi des exigences de sécurité du contrat.
\end{itemize}
}} % end 


