%-------------------------------------------------------------
%               FR CYBERDEF SECOPS COURSE
%              $Chapitre : Threat Management
%                 $theme : Architectures Sécurité
%     			 $File :  chap-secman-crypto.tex
%                             2020 eduf@ction
%-------------------------------------------------------------



\section{les 4 critères de la sécurité}

\begin{itemize}
  \item la confidentialité des données informatiques
  \item l’intégrité des données
  \item la disponibilité des données informatiques
  \item l’authentification et la non-répudiation
\end{itemize}


\mode<all>{\picframeS{acid1}{4 critères fondamentaux}}



\section{Politiques de sécurité}

\utodo

\section{Gouvernance de la sécurité}

\utodo


\subsection{Système de management de la sécurité}

\utodo


\section{l'assurance de la sécurité}

Grace aux outils d’analyse de risque, nous avons pu définir les biens essentiels de l’entreprise vu sous différents aspects :
\begin{itemize}
\item Continuité d’activité;
\item Protection du patrimoine informationnel et des données personnels;
\item Protection de l’image et de la réputation;
\end{itemize}

Globalement, ces critères définissent le niveau de confiance que peuvent avoir les clients, et les partenaires dans l’entreprise.

La gouvernance de la sécurité, est un ensemble de responsabilités et de processus permettant de maintenir au quotidien le niveau de sécurité d’une organisation, de réagir au plus vite, mais surtout de construire l’ensemble des mécanismes qui vont permettre à l’entreprise de construire la confiance.

Comme on le dit souvent la confiance ne se décrète pas elle se construit, et il faut très peu pour la rompre. Un premier élément pour la confiance, s’appelle la certification ou la labélisation des services ou de l'entreprise. l'ISO 27001 est un des cadre de cette assurance sécurité.

\utocomplete