
%===========================================
% QUIZZ CYBEREDU ANSSI
% https://www.ssi.gouv.fr/administration/formations/cyberedu/contenu-pedagogique-cyberedu/
%===========================================

% Q.........................................
\begin{multi}[multiple=true]{CyberEdu-X1.1}
	Entourer les exemples d'enjeu de la cybersécurité ? (Voir \href{https://www.ssi.gouv.fr/administration/formations/cyberedu/contenu-pedagogique-cyberedu/}{ANSSI CyberEdu} Slide n° 7 - Les enjeux de la sécurité des S.I.)
\item 	Augmenter les risques pesant sur le Système d'information
\item 	Révéler les secrets
\item	Rendre difficile la vie des utilisateurs en ajoutant plusieurs contraintes comme les mots de passe longs et complexes
\item* 	Protéger le système d'information
\end{multi}

% Q.........................................
\begin{multi}[multiple=true]{CyberEdu-X1.2}
	Impacts sur la vie privée  ?
\item* 	Impact sur l'image / le caractère / la vie privée : Diffamation de caractère , Divulgation d'informations personnelles (photos dénudées) Harcèlement
\item*	Impact sur l'identité: Usurpation d'identité
 \end{multi}

% Q.........................................
\begin{multi}[multiple=true]{CyberEdu-X1.3}
	Quels sont les trois principaux besoins de sécurité (Voir slide 23 - 24)
\item* 	D: Disponibilité
\item* 	I : Intégrité
\item*	C : Confidentialité
\item  P : Prouvabilité
\item  A : InputAbilité
\end{multi}


% Q.........................................
\begin{multi}[multiple=true]{CyberEdu-X1.4}
	Entourer la (ou les) phrase(s) correcte(s)
\item 	Le chiffrement permet de garantir que la donnée sera toujours disponible/accessible
\item* 	La sécurité physique permet d'assurer la disponibilité des équipements et des données
\item 	La signature électronique permet de garantir la confidentialité de la donnée
\item*	Les dénis de service distribués (DDoS) portent atteinte à la disponibilité des données
\end{multi}



% Q.........................................
\begin{multi}[multiple=true]{CyberEdu-X1.5}
	Vous développez un site web www.asso-etudiants-touristes.org  pour une association qui regroupe les étudiants souhaitant effectuer des voyages ensemble à l'étranger. Sur ce site on retrouve les informations concernant les voyages proposés telles que : le pays, les villes à visiter, le prix du transport, les conditions d'hébergement, les dates potentielles du voyage. Ces informations ont un besoin en confidentialité :
\item* 	Faible
\item 	Fort
\end{multi}

% Q.........................................
\begin{multi}[multiple=true]{CyberEdu-X1.6}
	Je viens de développer un site web pour une association qui regroupe les étudiants souhaitant effectuer des voyages en groupe à l'étranger. Les informations relatives aux étudiants inscrits sur le site (login et mot de passe, nom, prénom, numéro de téléphone, adresse), ont un besoin en confidentialité :
\item 	Faible
\item* 	Fort
\end{multi}

% Q.........................................
\begin{multi}[multiple=true]{CyberEdu-X1.6}
	Je peux réussir une attaque sur un bien qui n'a aucune vulnérabilité (voir Slide 34 - Notions de vulnérabilité, menace, attaque - attaque):
\item 	Vrai
\item* 	Faux
\end{multi}

% Q.........................................
\begin{multi}[multiple=true]{CyberEdu-X1.7}
	Toutes les organisations et tous les individus font face aux mêmes menaces (voir slide 40 - Exemples de sources de menaces):
\item 	Vrai
\item* 	Faux
\end{multi}

% Q.........................................
\begin{multi}[multiple=true]{CyberEdu-X1.8}
	Entourer les attaques généralement de type  ciblée (Voir \href{https://www.ssi.gouv.fr/administration/formations/cyberedu/contenu-pedagogique-cyberedu/}{ANSSI CyberEdu} Slide n° 42 - 52: Panorama de quelques menaces):
\item 	Phishing ou hameçonnage
\item 	Ransomware ou rançongiciel
\item* 	Social engineering ou ingénierie sociale
\item* 	Spear phishing ou l'arnaque au président
\end{multi}

% Q.........................................
\begin{multi}[multiple=true]{CyberEdu-X1.9}
	Entourer les attaques généralement de type non  ciblée  (Voir \href{https://www.ssi.gouv.fr/administration/formations/cyberedu/contenu-pedagogique-cyberedu/}{ANSSI CyberEdu} Slide n° 42 - 52: Panorama de quelques menaces):
\item 	Intrusion informatique
\item* 	Virus informatique
\item 	Déni de service distribué
\item* 	Phishing ou hameçonnage
\end{multi}

% Q.........................................
\begin{multi}[multiple=true]{CyberEdu-X1.10}
	Quels sont les éléments facilitateurs de fraudes internes (Voir \href{https://www.ssi.gouv.fr/administration/formations/cyberedu/contenu-pedagogique-cyberedu/}{ANSSI CyberEdu} Slide n° 47 - Panorama de quelques menaces : Fraude interne)
\item* 	Des comptes utilisateurs partagés entre plusieurs personnes
\item 	L'existence de procédures de contrôle interne
\item* 	Peu ou pas de supervision des actions internes
\item 	Une gestion stricte et revue des habilitations
\end{multi}

% Q.........................................
\begin{multi}[multiple=true]{CyberEdu-X1.11}
	Entourer les éléments qui peuvent réduire ou empêcher des fraudes internes
\item* 	Une gestion stricte et une revue des habilitations
\item* 	Une séparation des rôles des utilisateurs
\item 	Peu ou pas de surveillance interne
\item* 	Des comptes utilisateurs individuels pour chacun
\end{multi}

% Q.........................................
\begin{multi}[multiple=true]{CyberEdu-X1.12}
	Citer les vecteurs d'infection de virus 
\item* 	Une pièce jointe attachéà un message électronique
\item* 	Un support amovible infecté par exemple une clé USB
\item* 	Un site web malveillant ou ayant des pages web corrompues
\item* 	Un partage réseau ouvert
\item* 	Un système vulnérable
\end{multi}

% Q.........................................
\begin{multi}[multiple=true]{CyberEdu-X1.13}
	Qu'est-ce qu'un botnet? (Voir \href{https://www.ssi.gouv.fr/administration/formations/cyberedu/contenu-pedagogique-cyberedu/}{ANSSI CyberEdu} Slide n° -  Panorama de quelques menaces : Déni de service distribué)
\item*  un réseau d'ordinateurs infectés et contrôlés par une personne malveillante.
\item un logiciel maveillant s'autorepliquant sur internet
\item un système controlé à distance par un logiciel malveillant
\end{multi}

% Q.........................................
\begin{multi}[multiple=true]{CyberEdu-X1.14}
	Vous devez systématiquement donner votre accord avant de faire partir d'un réseau de botnets? (Voir \href{https://www.ssi.gouv.fr/administration/formations/cyberedu/contenu-pedagogique-cyberedu/}{ANSSI CyberEdu} Slide n° 52 - Panorama de quelques menaces : Déni de service distribué - illustration d'un botnet)
\item 	Vrai
\item* 	Faux
\end{multi}

% Q.........................................
\begin{multi}[multiple=true]{CyberEdu-X1.15}
	En France, la cybersécurité ne concerne que les entreprises du secteur privé et les individus (Voir \href{https://www.ssi.gouv.fr/administration/formations/cyberedu/contenu-pedagogique-cyberedu/}{ANSSI CyberEdu} Slide n° 54 : L'organisation de la sécurité en France)
\item 	Vrai
\item* 	Faux
\end{multi}

% Q.........................................
\begin{multi}[multiple=true]{CyberEdu-X1.16}
	L'usage d'outils pour obtenir les clés Wifi et accéder au réseau  Wifi du voisin tombe sous le coup de la loi (Voir \href{https://www.ssi.gouv.fr/administration/formations/cyberedu/contenu-pedagogique-cyberedu/}{ANSSI CyberEdu} Slide n° 58 - Dispositif juridique français de lutte contre la cybercriminalité):
\item 	Vigipirate
\item* 	Godfrain
\item 	Hadopi
\item 	Patriot act
\end{multi}

% Q.........................................
\begin{multi}[multiple=true]{CyberEdu-X1.17}
	Mon réseau wifi personnel est mal sécurisé, par exemple par l'usage d'une clé Wifi faible (exemple: 12345678). Une personne (intrus) se connecte à mon réseau  pour effectuer des actions malveillantes comme attaquer un site gouvernemental :
\item 	J'encours des sanctions
\item 	Seul l'intrus encourt des sanctions
\item* 	L'intrus et moi encourons des sanctions.
\item 	Aucune sanction n'est encourue
\end{multi}

% Q.........................................
\begin{multi}[multiple=true]{CyberEdu-X1.18}
	 Données à caractère personnel lesquels ?
\item*	Nom, prénom
\item* 	Nom, téléphone
\item* 	Date de naissance et commune
\item* 	Lieu de naissance
\item* 	Nationalité ou pays de naissance des parents ou des grands parents
\item* 	Adresse
\item* 	No carte d'identité / No de passeport / No de permis de conduire, …
\item* 	Empreinte digitale
\end{multi}

% Q.........................................
\begin{multi}[multiple=true]{CyberEdu-X1.20}
	Lors de la création du site Web de notre association étudiante, si vous stockez les informations suivantes pour chaque membre : nom, prénom, adresse, adresse email. Auprès de quel organisme devez-vous faire une déclaration (Voir \href{https://www.ssi.gouv.fr/administration/formations/cyberedu/contenu-pedagogique-cyberedu/}{ANSSI CyberEdu} Slide n° 60 - 64 : Droit de protection des données à caractère personnel)?
\item 	Gendarmerie
\item 	Université
\item* 	CNIL
\item 	Hadopi
\end{multi}

