%-------------------------------------------------------------
%               FR CYBERDEF SECOPS COURSE
%                            Contributions Rules
%                              2020 eduf@ction
%-------------------------------------------------------------

%\newcommand{\unotespdf}{}

\if unotespdf exists


\else

\fi

\section{Creative Common}

% ----------- Planche PRZ ------------------- 
\mode<all>{\texframe{Contributions}{quelques éléments}
{
Ce document a pour objectif de définir les règles de contribution et de participation au projet de notes de cours SEC101. Vous suivez ce cours, et un certains nombre de notes de cours et d'éléments pédagogiques peuvent être améliorer et être enrichis par vos expériences.

L'objectif des contributions des auditeurs est donc l'enrichissement des notes avec vos expériences et vos éléments fournis chaque semestre contenant l'actualité et les évolutions du domaine. Il bénéficieront aux auditeurs des années suivantes et à la communauté. Vos contributions seront tracées et indiquées dans le chapitre contribution de chaque note. 

}} % end frame

% ----------- Planche PRZ ------------------- 
\mode<all>{\texframe{Contributions}{quelques éléments}
{
Les notes de cours publiées sur GITHUB sont sous licences CC BY-NC-SA (Attribution - Pas d’utilisation commerciale - Partage dans les mêmes conditions)(Cf. \url{https://creativecommons.org/licenses/?lang=fr-FR}).

Cette licence permet aux autres de remixer, arranger, et adapter les documents de ce cours à des fins non commerciales tant que les auteurs sont portés au crédit de cette oeuvre et que les nouvelles œuvres sont diffusées selon les mêmes conditions.
}} % end frame

Toutes les éléments de cours SEC101, ne sont pas disponibles sous cette licence. Seules les éléments accessibles sur GITHUB sont publiées sous cette licence. Tout les cours SEC101 ne sont donc pas publiés sous cette forme. Seules des notes de cours sont fournies.  Elles ne représentent que partiellement le cours SEC 101 du CNAM.

%-------------------------------------------------------------
%               FR CYBERDEF SECOPS COURSE
%                            Contributions Rules
%                              2020 eduf@ction
%-------------------------------------------------------------

% \section{Contributions}

% \subsection{Comment contribuer}

Les notes et les présentations sont réalisées sous \hologo{LaTeX}. 

Vous pouvez contribuer au projet du cours CYBERDEF101. Les contributions peuvent se faire sous deux formes :

\begin{itemize}
  \item Corriger, amender, améliorer les notes publiées. A chaque session des modifications et évolutions sont apportées pour tenir compte des corrections de fond et de formes. 
  \item Ajouter, compléter, modifier des parties de notes sur la base de votre lecture du cours et de vos expertises dans chacun des domaines évoqués.
\end{itemize}

 Les fichiers sources sont publiés sur GITHUB dans l'espace : \ulink{https://github.com/edufaction/CYBERDEF101}{(edufaction/CYBERDEF101)}.
 

\subsection{Organisation de l'architecture de contribution}

L'ensemble des fichiers sources est publié sour GITHUB, et vous pouvez contribuer en utilisant GITHUB pour participer à titre individuel. Vous devez disposer d'une compétence LATEX et d'un environnement LATEX et GIT sur votre poste de travail. L'édition peut se faire sur le GIT mais cela est déconseillé (uniquement pour faciliter l'édition des documents pour des contributions, modifications limitées et rapides).  

\section{Architecture des projets LATEX}

% ----------- Planche PRZ ------------------- 
\mode<all>{\texframe{Contributions}{quelques éléments}
{
Les fichiers racines des projets de notes de cours sont dans le répertoire : \textbf{/Builder} . 
La syntaxe de ces fichiers est généralement \g{SEC101-Cx-title}. 
Cx étant le numéro du chapitre du cours, et \g{title} le titre de la note.
Afin d'uniformiser les notes de cours, une architecture documentaire standardisée est proposée. 
Les images en PDF sont issues d'un fichier source qui doit être fourni (OpenDocument)

Chaque compilation de notes de cours pour un thème  technique donné est un article au sens LATEX. Chaque article est configuré, dans un fichier\textbf{*.lesson.tex} qui définit le contenu de l'article. Le corps des documents est dans le répertoire \textbf{/Chapters}. 
}} % end frame

\subsection{Utilisation de GITHUB}

Veuillez vous référer au manuel d'usage de GITHUB. 

Le projet est sur  \ulink{https://github.com/edufaction/CYBERDEF}{\textbf{edufaction/CYBERDEF}}


%\subsection{Utilisation d'Overleaf }
%
%L'utilisation d'Overleaf sur le projet nécessite un peu d'habitude et de connaissance de l'outil. Pour accéder à la compilation du fichier LATEX du projet de notes correspondant il faut sélectionner le DOCUMENT PRINCIPAL dans le Menu d'overleaf. Le document principal se trouve dans le répertoire \textbf{Builder}. Les documents principaux sont nommés. SEC101-Cx-zzzzz.doc.tex et SEC101-Cx-zzzzz.prz.tex, où x est  le numéro de chapitre, zzzzz le titre du document. Les extensions sont PRZ pour les présentations et DOC pour les notes de cours à imprimer.
%
%Pour participer via Overleaf, vous devez créer un compte sur \ulink{https://www.overleaf.com}{Overleaf}.
%
%En cliquant sur le lien \url {https://www.overleaf.com/read/qjttpndkypsh} vous pouvez accéder en lecture seul au projet.
%
%Pour contribuer vous pouvez accéder via le lien  \url{https://www.overleaf.com/7966162813pykmwbhggmfg}. je vous conseille d'accéder en lecture seule dans un premier temps.


\subsection{../Latex}

Les sources des documents sont en LATEX. Si vous souhaitez éditer ces fichiers et recompiler les fichiers PDF, vous devez disposer d'une distribution et d'un éditeur LATEX.

% Contributions Individuelles (éditer le fichier ci dessous)

%-------------------------------------------------------------
%               FR CYBERDEF SECOPS COURSE
%                Contributions DATAS & NAMES
%                              2020 eduf@ction
%-------------------------------------------------------------

\subsection{Les contributeurs/auteurs du cours}

\subsubsection{co-auteurs}

%-------------------------------------------------------------
%               FR CYBERDEF SECOPS COURSE
%                Contributions DATAS & NAMES
%                              2020 eduf@ction
%-------------------------------------------------------------
%           CO AUTEURS - Rédacteur de chapitre
%-------------------------------------------------------------

\coauthor{David  BATANY}{2019-2020}{Architecture et fonctionnement des Botnets}{Cnam SEC101}

\subsubsection{contributeurs}

%-------------------------------------------------------------
%               FR CYBERDEF SECOPS COURSE
%                Contributions DATAS & NAMES
%                              2020 eduf@ction
%-------------------------------------------------------------
%                 CONTRIBUTIONS AU COURS
%-------------------------------------------------------------


\coauthor{Céline JUBY}{2020}{Contributions d'amélioration et relectures}{Orange Cyberdefense}



\subsection{Organisation du modèle contribution}

Toute personne contribuant à l'évolution du cours CYBERDEF101 sera indiquée avec sa contribution dans le fichier \bf{Contribs.tex}.

