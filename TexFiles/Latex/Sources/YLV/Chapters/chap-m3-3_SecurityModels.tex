

\section{Modèles de sécurité et technologies de sécurité protectrices}
\subsection{Château fort (Firewall, Proxy, anti DDoS), cloisonnement, accès admin dédié}
\subsubsection{Château fort}
•	En s’appuyant sur un schéma global, notion de DMZ externe, DMZ interne, illustration des flux entrant et sortant, positionnement de composants clés : proxy, serveurs, middleware, sondes, etc.
•	Cloisonnement
•	Accès admin – réseau dédié
•	Accès partenaires (VPN), flux vers Cloud, etc.
\subsubsection{Firewall}
Filtrage en entrée, flux autorisés sont ceux connus et validé dans la politique de sécurité implémentée sur le firewall
o	Historiquement : stateless (ACL CISCO), puis statefull, puis next Gen (jusqu’à la couche applicative)
o	Stateless : exemple d’une ACL CISCO, associée à du NAT, i.e. définition du NAT avec le besoin (visibilité extérieure, gestion des adresses IP, pénurie, etc.) \& le comment (Schéma)
o	Statefull UDP/TCP mainly
o	Next Gen (L7, déchiffrement, proxy, voire IDP, DLP, etc.)
\subsubsection{Proxy et Reverse Proxy}
Schéma Proxy (firewall applicatif), lié à de l’authentification, antivirus, URL Filtering
Schéma ReverseProxy, rupture protocolaire, protection serveur, réécriture, LB , WAF
\subsection{sondes de détection (IDS/IDP)}
Schéma sondes, positionnement dans l’architecture
Rôle : détection ou coupure, quid des flux chiffrés ?
\subsection{IAM , ZeroTrust, Bastion, VPN SSL, NAC}
\subsubsection{IAM}
Déf IAM, process ET procédures
\subsubsection{Zerotrust}
: pkoi ? et comment ?
\subsubsection{Bastion}
: schema, fonctions
\subsubsection{VPN SSL}
: schema, functions 
\subsubsection{NAC}
: schema, fonctions
\subsection{Multi Cloud - CASB}
Schémas , mix cloud pub et cloud privé, positionnement du CASB (fonction/pkoi)

