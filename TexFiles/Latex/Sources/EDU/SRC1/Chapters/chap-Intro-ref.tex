%-------------------------------------------------------------
%               FR CYBERDEF SECOPS COURSE
%                                        SECOPS
%                                            Intro
%
%                           Introduction Cyberdefense
%                            % Chap-Intro-Ref.tex
%
%                              2020 eduf@ction
%-------------------------------------------------------------



\section{Quelques organismes de référence}

Pour l'entreprise la cybersécurité est un domaine  de nombreux cadres normatifs et réglementaires soutenus bien souvent par contraintes legislatives propres à chaque pays.

Cette normalisation et ses réglementations sont riches mais certaines fois complexes.
Le plus simple pour s'enrichir de ces savoirs et surtout pour disposer des meilleurs informations à la sources autant \g{fréquenter} les sites internet institutionnels des organismes qui sont et continuent à être les points de  référence dans le domaine de cybersécurité.

De nombreux services étatiques et de normalisation possèdent des activités dites Cyber dans leur structures :

\begin{itemize}
    \item  Organismes français : AFNOR, Cert FR, CNIL, HADOPI, ANSSI, DGSE, DGSI, DGA/MI, Commandement de la cyberdéfense, C3N, OCLCTIC, BEFTI ...
  \item  Organismes internationaux : ISO, ETSI, CERT, Europo, lnterpol , ENISA , FIRST ...
  \item  Organismes étrangers : FBI, CIA, NSA, GCHQ, Unité 8200, Fapsi, The SANS institute, CISA ...
\end{itemize}

Je vous propose de donner quelques pointeurs par portée sur des organismes de référence du point de vue occidental.

\subsection{International et Etats-Unis}

Au niveau international, on ne peut éviter les Etats-Unis, un pays qui oeuvre fortement dans le domaine des standards.

\subsubsection{Le NIST}
Le National Institute of Standards and Technology, ou NIST est une agence du département du Commerce des États-Unis. Son but est de promouvoir l'économie en développant des technologies, la métrologie et des standards avec l'industrie. 

\begin{itemize}
  \item \ulink{https://csrc.nist.gov/}{NIST COMPUTER SECURITY RESOURCE CENTER}
  \item \ulink{https://www.nist.gov/itl/fips-general-information}{NIST INFORMATION TECHNOLOGY LABORATORY}
\end{itemize}

On notera en particulier les référentiels cryptographiques du NIST et ceux liées à la cyberdéfense en particulier avec le \UKword{CyberSecurity FrameWork}

\subsubsection{SEI - Université de Carnegie Mellon}

Le Software Engineering Institute (SEI) est un centre de recherche-développement financé par des fonds fédéraux et placé sous le parrainage du département de la Défense des États-Unis ; son fonctionnement incombe à Carnegie Mellon University. Le SEI travaille avec des organisations pour apporter des améliorations significatives à leurs capacités d’ingénierie logicielle en leur fournissant le leadership technique afin de faire progresser la pratique de l’ingénierie logicielle. Le CERT Division du SEI est l’entité qui fait autorité et cherche à améliorer la sécurité et la résilience des systèmes et réseaux en particulier dans le domaine du logiciel(\href{https://www.sei.cmu.edu/research-capabilities/cybersecurity/}{Carnegie Mellon University - Cybersecurity research}).

\subsubsection{l'ISO : International Organization for Standardization }


L’ISO est une Organisation Internationale participant à l’élaboration de Standards. En ce sens la conformité à une norme a l’avantage d’être reconnue internationalement.
 
Les normes de la famille ISO 27000 permettent d’organiser et structurer la démarche de la gestion de la sécurité des systèmes d’information, une grande famille de normes avec des  positionnement sur l'ensemble du spectre  de la sécurité des systèmes d'information : 

\begin{itemize}
  \item ISO 27001 décrit les processus permettant le management de la sécurité de l’information (SMSI);
  \item ISO 27002 présente un catalogue de bonnes pratiques de sécurité;
  \item ISO 27003 décrit les différentes phases initiales à accomplir afin d’aboutir à un système de Management tel que décrit dans la norme ISO 27001;
  \item ISO 27004 permet de définir les contrôles de fonctionnement du SMSI;
  \item ISO 27005 décrit les processus de la gestion des risques;
  \item ISO 27006 décrit les exigences relatives aux organismes qui auditent et certifient les SMSI des sociétés.
\end{itemize}

Nous aborderons dans le chapitre sur les politiques de sécurité, l'usage de ce cadre normatif dans la gouvernance globale de la cybersécurité au sein de l'entreprise.

%TODO : Pourquoi se faire certifier ISO27K, les points durs.

\subsection{Europe}

Au niveau européen, le règlement (CE) 460/2004 du Parlement européen et du Conseil du 10 mars 2004 à institué l'Agence européenne chargée de la sécurité des réseaux et de l'information  Agence européenne chargée de la sécurité des réseaux et de l'information \ulink{https://www.enisa.europa.eu}{ENISA}. Sont rôle est de : 

%BEGIN wikpedia
% TODO
\begin{itemize}
  \item Conseiller et assister la Commission et les États membres en matière de sécurité de l'information et les aider, en concertation avec le secteur, à faire face aux problèmes de sécurité matérielle et logicielle.
  \item Recueillir et analyser les données relatives aux incidents liés à la sécurité en Europe et aux risques émergents.
  \item Promouvoir des méthodes d'évaluation et de gestion des risques afin d'améliorer notre capacité de faire face aux menaces pesant sur la sécurité de l'information.
  \item Favoriser l'échange de bonnes pratiques en matière de sensibilisation et de coopération avec les différents acteurs du domaine de la sécurité de l'information, notamment en créant des partenariats entre le secteur public et le secteur privé avec des entreprises spécialisées.
  \item Suivre l'élaboration des normes pour les produits et services en matière de sécurité des réseaux et de l'information.
\end{itemize}
%END Wikipeida

\subsection{France}

En France, la Cybersécurité est pilotée par un organisme dépendant des services du 1er Ministre,  l'Agence National des Systèmes d'information (\uac{aANSSI}).
L'\uac{aANSSI} possède plusieurs rôles de fait. C'est un \g{régulateur} c'est à dire qu'elle définit des cadres réglementaires pour les entreprises mais c'est aussi une agence qui édicte des préconisations et des guides.

 \ulink{https://www.ssi.gouv.fr/agence/cybersécurité/ssi-en-france/}{Le site de l'agence} est riche en informations et guides sur la cybersécurité.

Dépendant aussi de l'état, la \ulink{https://www.cnil.fr/}{CNIL} (Commission National Informatique et Liberté) est une autorité dont la mission est de protéger le citoyen. Avec l'avènement du règlement de protection des données personnelles, la \uac{aCNIL} a vu son pouvoir étendu.  

Il faut aussi citer l'AFNOR \uac{aAFNOR}, qui relaie en France la normalisation internationale dont l'ISO au delà de ses actions de normalisation purement françaises.

\section{Quelques associations et groupements professionnels} 

A titre d'information, vous trouverez en France aussi quelques clubs et associations historiques de la sécurité de systèmes d'information qui offrent à leurs adhérents des lieux d'échanges très intéressants et publient régulièrement  :



\begin{itemize}

  \item\head{Observatoire de la Sécurité des Systèmes d'Information et des Réseaux} (Technique)
\ulink{https://www.ossir.org/}{OSSIR}, association plutôt technique, qui propose de nombreux échanges sur la cybersécurité et existant depuis les années 90.
  \item\head{Club de la sécurité de l'information Français} (Gouvernance)
\ulink{https://clusif.fr}{CLUSIF}, association qui propose de nombreux échanges sur la cybersécurité.

  \item\head{Club CyberEdu} (Education)
\ulink{https://www.cyberedu.fr}{CyberEdu}, issu des travaux sur la formation des enseignants en cybersécurité de l'ANSSI, l'association regroupe les écoles et les utilisateurs des travaux de CyberEdu.

  \item\head{Club HexaTrust} (Editeurs de produits et services de cybersécurité Français)
\ulink{https://www.hexatrust.com/le-club/}{HexaTrust}, regroupe les éditeurs et fournisseurs de services français en cybersécurité.

  \item \head{Club des Experts de la sécurité de l’Information et du Numérique}. (Club de RSSI)
le \ulink{https://www.cesin.fr}{CESIN} est une association regroupant les RSSI d'entreprises, l'adhésion à cette association nécessite un parrainage et vous devez être RSSI.
\end{itemize}

