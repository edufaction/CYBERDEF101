\subsection{La détection}
Recherche d'anomalie, comparaison de signatures, pots-de-miel, toutes ces techniques basées sur l'activité du réseau reposent sur l'inspection des flux et des paquets.

\subsubsection{La détection passive}
L'identification et l'analyse passive des flux (adresses IP,port source et destination, étiquette MPLS\footnote{MultiProtocol Label Switching}, etc.) permettent de classifier les protocoles suspects et les serveurs de C\&C.
\newline BotFinder, par exemple, permet de décomposer un flux (durée moyenne des connexions, nombre d'octets transférés, etc.) et de le comparer à l'activité normale du réseau.
\newline L'observation des DNS\footnote{Domain Name Server} permet aussi d'identifier les domaines malveillants afin de caractériser le botnet suspecté.
BotGad, un système d'exploitation permettant d'analyser le trafic DNS sur un réseau local, utilise un algorithme basé sur l'apprentissage afin de définir la stratégie de groupe du botnet.
\newline EXPOSURE, un autre système d'exploitation déployé au sein de l'ISP\footnote{Internet Service Provider}, permet d'analyser à large échelle mais sur une durée de plusieurs mois le trafic DNS. Produisant une liste de domaines malveillants, il permet, par exemple, d'identifier un volume conséquent de requêtes pour un même domaine.
\newline Enfin le recours aux pots-de-miel et l'analyse des journaux d'activité sont les éléments de base d'une recherche d'activité liée aux botnets.
Suivant cette idée, le SIEM\footnote{Security Information and Event Management}, une solution de gestion de la sécurité, représente un outil précieux et novateur afin d'optimiser la veille du trafic et d'automatiser les processus de sécurité en cas de comportement anormal.

\subsubsection{La détection active}
Différentes techniques existent comme le sinkholing\footnote{également appelé serveur gouffre, gouffre Internet ou Blackhole DNS } redirigeant le trafic vers des serveurs afin de simuler le comportement du C\&C et de diminuer la puissance du réseau du botnet.
\newline L'infiltration, fonction de l'architecture du botnet, consiste à simuler le comportement d'un botnet contrôlé à l'aide de drones IRC ou de script afin de capturer le trafic et de remonter jusqu'au botmaster.
Le projet Pebbletrace reprend cette idée d’identification du botmaster en piégeant les équipements infectés avec une charge défensive afin de retourner le trafic contre le botnet.
