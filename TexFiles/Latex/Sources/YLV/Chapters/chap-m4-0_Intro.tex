\section{Introduction}
%PROCESS SMS Organisation des entreprises SMSI (80\% du temps a définir ce qui doit être fait, et prouver qu'il a fait ce qu'il devait faire), 27001, SMSI, 27005 (Cadre méthdologie EBIOS …MEHARI …) - POLITIQUE et PILOTAGE \& AUDIT (V\&V) SMSI, 27001 comment raisonne un RSSI pour structurer ses douleurs et les solutions, plan de traitement des risques ... 80\% du temps STATIQUE

Nous aborderons dans ce chapitre la mise en place d'un Système de Management de la Sécurité de l'Information (SMSI) au sein des entreprises. Cette activité nécessite une approche méthodique et structurée. En effet, environ 80\% du temps consacré à la gestion du SMSI est dédié à la définition des actions à entreprendre et à la démonstration de leur mise en œuvre effective. Cela implique une attention particulière à la conformité avec la norme ISO/IEC 27001, qui fournit un cadre pour établir, mettre en œuvre, maintenir et améliorer le SMSI. Nous présenterons cette norme dans la première partie.

Nous étudierons le cadre Méthodologique qui pourra être utilisé pour soutenir cette démarche. En utilisant par exemple des méthodologies telles que l'ISO/IEC 27005 qui offre un cadre pour la gestion des risques liés à la sécurité de l'information, intégrant des outils comme EBIOS et MEHARI. Ces approches permettent d'identifier, d'évaluer et de traiter les risques de manière systématique, facilitant ainsi la prise de décision éclairée. Nous verrons dans la deuxième partie son utilisation en soutien de la politique et du pilotage pour in fine définir une politique de sécurité claire essentielle pour orienter les actions du SMSI. Le pilotage et l'audit du système, notamment à travers des processus de vérification et de validation (V&V), garantissent que les mesures mises en place sont efficaces et conformes aux exigences de la norme ISO/IEC 27001. Cela permet également d'identifier les domaines nécessitant des améliorations.

%Rôle du Responsable de la Sécurité des Systèmes d'Information (RSSI)
Le Responsable de la Sécurité des Systèmes d'Information (RSSI) joue un rôle clé dans cette organisation. Il doit structurer ses préoccupations et identifier les solutions appropriées pour traiter les risques. Cela inclut l'élaboration d'un plan de traitement des risques, qui doit être régulièrement mis à jour pour refléter l'évolution des menaces et des vulnérabilités.

Il est important de noter que 80\% du temps consacré à la gestion du SMSI est souvent statique, axé sur la documentation, la mise en conformité et la gestion des processus. Cela souligne la nécessité d'une approche proactive et rigoureuse pour garantir la sécurité de l'information au sein de l'organisation.

En résumé, l'organisation d'un SMSI efficace repose sur une méthodologie bien définie, une politique claire, un pilotage rigoureux et l'implication active du RSSI. Cela permet de garantir une gestion optimale des risques et de renforcer la sécurité des informations au sein de l'entreprise.






%Begin FRAME----------------------------
\mode<presentation>{\texframe
% contenu affiché sur Article et Beamer
%- - - - - - - - - - - - - - - - - - - - - - - - 
{Gouvernance et management de la sécurité} % titre de la diapo
{Introduction} % sous titre de la diapo
{
%begin slide- - - - - - - - - - - - - - - - - 
	
 \begin{itemize}
    \item Organisation des entreprises
    \item Gouvernance : cadre, méthodologie, objectifs
    \item Comment raisonne un Responsable de la sécurité de l'information (RSSI) ?
 \end{itemize}
%end slide- - - - - - - - - - - - - - - - - - - 
}}
%End FRAME------------------------------


