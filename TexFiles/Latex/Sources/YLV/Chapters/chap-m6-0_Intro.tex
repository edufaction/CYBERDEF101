
\section{Introduction}
% PROCESS type SECU In TTM : Secu by design, secu les projets vs securité d'entreprise (Ingénierie de la sécurité, OPERATION et PROJET) - Sécurité des produits (ISO 15504, CSPN …)

% différentier la sécurité dans les projets et la sécurité de l'entreprise afin de découvrir les règles techniques de sécurisation des composants du SI, l'organisation des équipes sécurité dans les projets et les enjeux de conformité technique des produits

Ce chapitre se propose d'explorer la distinction entre la sécurité dans les projets et la sécurité de l'entreprise, deux concepts souvent confondus mais aux implications différentes. 

Nous aborderons les règles techniques essentielles pour sécuriser les composants des systèmes d'information (SI), ainsi que l'organisation des équipes de sécurité au sein des projets. Enfin, nous mettrons en lumière les enjeux de conformité technique des produits, afin de garantir une approche intégrée et efficace de la sécurité dans le développement de projets. Cette compréhension approfondie est cruciale pour anticiper les risques et assurer la pérennité des initiatives numériques.

%Begin FRAME----------------------------
\mode<presentation>{\texframe
% contenu affiché sur Article et Beamer
%- - - - - - - - - - - - - - - - - - - - - - - - 
{Sécurité des projets et sécurité d'entreprise} % titre de la diapo
{Introduction} % sous titre de la diapo
{
%begin slide- - - - - - - - - - - - - - - - - 
Différentier la sécurité dans les projets et la sécurité de l'entreprise
 \begin{enumerate}
    \item security by design 
    \item les règles techniques de sécurisation des composants du SI
    \item organisation des équipes sécurité
    \item enjeux de conformité technique des produits
 \end{enumerate}
%end slide- - - - - - - - - - - - - - - - - - - 
}}
%End FRAME------------------------------