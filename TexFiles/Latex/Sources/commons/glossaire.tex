%-------------------------------------------------------------
%          FR CYBERDEF SECOPS COURSE
%
%                        GLOSSAIRE
%
%                 $File :  Sources/EDU/glossaire.tex
%                    2020 eduf@ction
%-------------------------------------------------------------
% Require : glossaries package
%-------------------------------------------------------------

%\newglossaryentry{xxx}
%{
%name={xxx},
%description={xxx}
%}

\newglossaryentry{gLPM}
{
name={Loi de programmation militaire},
description={Une loi de programmation militaire (LPM), est une loi visant à établir une programmation pluriannuelle des dépenses et donc du budget que l'État français consacre à ses forces armées. Les lois de programmation militaire ont des durées d'application de quatre, cinq ou six ans. La cyberdéfense fait partie des axes importants des lois de programmation}
}

\newglossaryentry{gCTF}
{
name={Capture The Flag},
description={Challenge consistant à exploiter ses connaissance et les vulnérabilités affectant des logiciels  et systèmes numériques de manière à s’introduire sur des réseaux , des ordinateurs, pour récupérer (capturer) des preuves de l’intrusion réussie dénommées G\g{des drapeaux (\textit{Flags} en anglais)}.}
}

\newglossaryentry{gDTF}
{
name={Defend The Flag},
description={Challenge consistant à détecter, résister à des attaquants qui tentent de s'introduire ou sont introduits dans des systèmes numériques de manière à défendre le bien considéré comme le plus sensible convoité par l'attaquant dénommé G\g{drapeau (\textit{Flags} en anglais)}.}
}


\newglossaryentry{gROOTKIT}
{
name={Rootkit},
description={ou simplement « kit » (est un « outil de dissimulation d'activité » ou  « code furtif » qui est une sorte de trousse à outils pour un pirate informatique). Il est constitué  de techniques mises en œuvre par un ou plusieurs logiciels, dont le but est d'obtenir et de pérenniser un accès ceci avec en déployant des techniques de furtivité et de dissimulation.}
}

%-------------------------------------------------------
% GLS DATABASE XLS Generation (cf. RESSOURCES DIR)
%-------------------------------------------------------
% DO NOT MODIFY DIRECTLY, USE XLS File (BEGIN)

\newglossaryentry{gAffiliate}{name={\textit{Affiliates}},description={(Affiliés): Cybercriminels qui mènent des cyberattaques, parfois sophistiquées, par le biais de rançongiciels qui sont mis à disposition par des développeurs ou concepteurs. Les gains générés par les affiliés sont partagés avec les développeurs via un système de commissions.}}

% DO NOT MODIFY (END)
%-------------------------------------------------------

