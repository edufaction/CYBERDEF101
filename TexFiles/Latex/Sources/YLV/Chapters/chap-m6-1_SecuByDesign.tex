\section{Security By Design}

Le concept de \textbf{Security By Design} s'impose comme une approche fondamentale dans le développement de systèmes et d'applications sécurisés. Nous aborderons dans ce chapitre les principes clés de cette "philosophie", qui intègre la sécurité dès les premières étapes de conception, plutôt que de l'ajouter en fin de processus (et donc trop tard!). Nous examinerons les éléments essentiels qui sous-tendent cette démarche proactive, tels que \textbf{l'évaluation des risques}, \textbf{la minimisation des surfaces d'attaque} et l'implémentation de \textbf{contrôles de sécurité robustes}.

Dans un second temps, nous aborderons les méthodes de \textbf{maintien en condition de sécurité (MCS)}, qui garantissent que les systèmes restent protégés tout au long de leur cycle de vie. Cela inclut des pratiques telles que la \textbf{surveillance continue}, \textbf{les mises à jour régulières} et \textbf{les audits de sécurité}, permettant ainsi d'adapter les mesures de protection face à l'évolution des menaces. 

En intégrant ces deux volets, ce chapitre vise à fournir une compréhension complète du Security By Design et de son rôle crucial dans la sécurité des projets.

\subsection{Concepts et principes}

Ce chapitre se penchera d'abord sur les principes clés du security by design. Parmi ceux-ci, l'évaluation des risques joue un rôle central. 
Cette activité implique d'identifier les vulnérabilités potentielles et d'analyser les impacts possibles sur les systèmes et les données. Une évaluation rigoureuse permet de prioriser les efforts de sécurité en fonction des menaces les plus critiques (cf. chapitre 4.1 pour plus de détails sur les différentes méthodes d'analyse de risques).

Ensuite, la minimisation des surfaces d'attaque consiste à réduire le nombre de points d'entrée potentiels pour les attaquants, en limitant les fonctionnalités non essentielles et en appliquant le principe du moindre privilège. En restreignant l'accès aux ressources et en désactivant les services inutilisés, les organisations peuvent considérablement diminuer leur exposition aux cyberattaques. l'exemple ci-dessous représente schématiquement un système ou seul les ports liés au fonctionnement du serveur de base de données sont ouverts. Nous reverrons ce sujet et son application technique dans le chapitre suivant sur le durcissement.

%Begin PICFRAME------------------------
\mode<all>{\picframe
%- - - - - - - - - - - - - - - - - - - - - - - - 
{../Latex/Sources/YLV/Pictures/Minimisation}% PDF image sans extension
{Minimisation : exemple d'un serveur de Base de données} % texte sous l'image en article
{0.7} % echelle
{Minimisation} % label de référence
}
%End PICFRAME--------------------------

Enfin, l'implémentation de contrôles de sécurité robustes est essentielle pour garantir la protection des systèmes. Cela inclut l'intégration de mécanismes de sécurité tels que l'authentification forte, le chiffrement des données et la surveillance des activités suspectes. Ces contrôles doivent être conçus pour fonctionner de manière cohérente tout au long du cycle de vie du produit, assurant ainsi une défense en profondeur.

\subsection{Méthodes de MCS (Maintien en Condition de Sécurité)}

On appelle méthodes de maintien en condition de sécurité ou \textbf{MCS} les méthodes qui garantissent que les systèmes restent protégés tout au long de leur cycle de vie. Ces pratiques sont essentielles pour s'assurer que les mesures de sécurité initialement mises en place continuent d'être efficaces face à l'évolution constante des menaces ou l'apparition de vulnérabilités.
Nous verrons par la suite les méthodes de surveillance continue, d'application des mises à jour régulières, l'utilisation des audits de sécurité et enfin la sensibilisation des utilisateurs.

\subsubsection{Surveillance Continue}

La surveillance continue est une méthode et une activité clé pour détecter et répondre rapidement aux incidents de sécurité. Cela implique l'utilisation par exemple de systèmes de détection d'intrusion (IDS) (cf. chapitre 3.3) et de solutions de gestion des informations et de corrélation des événements de sécurité (SIEM). Par exemple, une entreprise peut déployer un SIEM pour collecter, analyser et corréler les journaux d'activité de ses serveurs et applications. En surveillant ces données en temps réel, l'organisation peut identifier des comportements anormaux, tels que des tentatives de connexion suspectes ou d'exfiltration de données, et réagir rapidement pour atténuer les risques et dommages en isolant des serveurs ou en coupant des flux au niveau des pare feux par exemple.

\subsubsection{Mises à Jour Régulières}

Les mises à jour régulières des logiciels et des systèmes sont cruciales pour maintenir leur sécurité. Les vulnérabilités découvertes dans les logiciels doivent être corrigées par des mises à jour fournies par les éditeurs. Par exemple, un système d'exploitation peut recevoir des correctifs mensuels pour combler des failles de sécurité. Ignorer ces mises à jour expose les systèmes à des attaques exploitant ces vulnérabilités. Ainsi, établir un processus de gestion des correctifs, qui inclut l'évaluation et l'application rapide et régulière des mises à jour, est essentiel pour protéger les environnements numériques.

\subsubsection{Audits de Sécurité}

Les audits de sécurité, tels que les tests d'intrusion et les évaluations de vulnérabilité, sont également des pratiques importantes pour maintenir la sécurité. Par exemple, une entreprise peut engager des experts en sécurité pour réaliser des tests d'intrusion sur son système d'information. Ces tests simulent des attaques réelles afin d'identifier les failles de sécurité avant qu'elles ne soient exploitées par des attaquants. De plus, des évaluations régulières des vulnérabilités permettent de détecter les points faibles et d'appliquer des mesures correctives. Les rapports de ces tests permettent aux propriétaires et acteurs du système de consolider et maintenir la sécurité de leur produit. Ces tests peuvent être faits au plus tôt dans la phase de conception mais aussi (et surtout!) tout au long de la vie du système.

\subsubsection{Formation et Sensibilisation}

Enfin, la formation et la sensibilisation des employés jouent un rôle crucial dans le maintien de la sécurité. Les utilisateurs finaux sont souvent la première ligne de défense contre les cybermenaces. Par exemple, des sessions de formation régulières sur la reconnaissance des tentatives de phishing et les bonnes pratiques de sécurité peuvent réduire considérablement le risque d'incidents liés à l'erreur humaine.

%Begin FRAME----------------------------
\mode<presentation>{\texframe
% contenu affiché sur Article et Beamer
%- - - - - - - - - - - - - - - - - - - - - - - - 
{Security By Design} % titre de la diapo
{} % sous titre de la diapo
{
%begin slide- - - - - - - - - - - - - - - - - 

\begin{itemize}
    \item Concepts et principes
        \begin{itemize}
            \item évaluation des risques
            \item minimisation des surfaces d'attaque
            \item contrôles de sécurité
        \end{itemize}
    \item MCS (Maintien en Condition de Sécurité)
        \begin{itemize}
            \item surveillance continue
            \item mises à jour 
            \item audits de sécurité
            \item formation et sensibilisation
        \end{itemize}
\end{itemize}
%end slide- - - - - - - - - - - - - - - - - - - 
}}
%End FRAME------------------------------

%Begin FRAME----------------------------
\mode<all>{\texframe
% contenu affiché sur Article et Beamer
%- - - - - - - - - - - - - - - - - - - - - - - - 
{Points à retenir} % titre de la diapo
{} % sous titre de la diapo
{
%begin slide- - - - - - - - - - - - - - - - - 
\begin{itemize}
    \item
    \item
\end{itemize}
%end slide- - - - - - - - - - - - - - - - - - - 
}}
%End FRAME------------------------------