\documentclass[a4paper,10pt,twocolumn]{article}

\usepackage[french]{babel}
\usepackage[utf8]{inputenc}
\usepackage[T1]{fontenc}
\usepackage{lmodern}
	
	
\usepackage{listings}      

\lstset{literate=%
{é}{{\'e}}{1}%
{è}{{\`e}}{1}%
{à}{{\`a}}{1}%
{ç}{{\c{c}}}{1}%
{œ}{{\oe}}{1}%
{ù}{{\`u}}{1}%
{É}{{\'E}}{1}%
{È}{{\`E}}{1}%
{À}{{\`A}}{1}%
{Ç}{{\c{C}}}{1}%
{Œ}{{\OE}}{1}%
{Ê}{{\^E}}{1}%
{ê}{{\^e}}{1}%
{î}{{\^i}}{1}%
{ô}{{\^o}}{1}%
{û}{{\^u}}{1}%
{ë}{{\¨{e}}}1
{û}{{\^{u}}}1
{â}{{\^{a}}}1
{Â}{{\^{A}}}1
{Î}{{\^{I}}}1
} 

\lstdefinestyle{LATEX}
{
  language=[LaTeX]{TeX},
  texcsstyle=*\color{red},
  basicstyle=\ttfamily,
  moretexcs={lquote,rquote,xtodo,xreview}, % user command highlight
  frame=single,
}

\lstset{breaklines=true}

\def\EDX{\textit{\textbf{EDX}}}
\def\lib{librairie }

%---------------------------------------------------------------
% Placeholder Module - eduf@ction - 2023
%---------------------------------------------------------------

%\usepackage[usenames,svgnames,table]{xcolor}

\usepackage{xcolor}

\newcommand{\placeholder}[1]{\noindent%
   \fcolorbox{LightBlue}{LightCyan}{%
      \begin{minipage}{0.9\columnwidth}%
         \color{DodgerBlue}\textbf{\tiny{TODO~~}}%
         \color{MidnightBlue}#1%
      \end{minipage}%
   }%
   \par%
}
\newcommand{\placeholderinl}[1]{\noindent%
   \fcolorbox{LightBlue}{LightCyan}{\color{DodgerBlue}!}%
   ~{\color{MidnightBlue}#1}~%
   \fcolorbox{LightBlue}{LightCyan}{\color{DodgerBlue}<}%
}

%---------------------------------------------------------------
% Epigraphe Module - eduf@ction - 2023
%---------------------------------------------------------------


\usepackage{times}
\usepackage{xcolor}
\usepackage{epigraph}
\usepackage{tabto}
\usepackage{ifthen}
\setlength{\epigraphwidth}{0.75\linewidth}
\setlength{\afterepigraphskip}{\baselineskip}
\setlength{\beforeepigraphskip}{0.4\baselineskip}


\definecolor{epigraphecolor}{RGB}{125,125,125}


\makeatletter
\newenvironment{@leftepigraph}{
   \setlength\topsep{0pt}
   \setlength\parskip{0pt}
   \renewcommand{\epigraphflush}{flushleft}
}{}
\newenvironment{@rightepigraph}{
   \setlength\topsep{0pt}
   \setlength\parskip{0pt}
   \renewcommand{\epigraphflush}{flushright}
   \renewcommand{\sourceflush}{flushright}
   \renewcommand{\textflush}{flushright}
}{}
\newcommand{\@bgquote}[2]{\tabto*{#1}{\fontsize{100pt}{0pt}\selectfont{}\color{epigraphecolor!50}\smash{\raisebox{-60pt}{#2}}}\tabto*{0pt}}
\newcommand{\@generalquote}[6]{\begin{#6}\epigraph{\@bgquote{#5}{#4}#3}{\textsc{#2}\ifthenelse{\equal{#1}{}}{}{\\\textit{#1}}}\end{#6}}
\newcommand{\lquote}[3][]{\@generalquote{#1}{#2}{#3}{''}{\dimexpr(\epigraphwidth-50pt)}{@leftepigraph}}
\newcommand{\rquote}[3][]{\@generalquote{#1}{#2}{#3}{``}{0pt}{@rightepigraph}}
\makeatother

\title{Magic Documentation}
\author{Eric Dupuis}

\begin{document}
\maketitle

\section{Avant propos}

Ce document définit les différentes macros, du template EDX MAGIC utilisé pour le cours CYBERDEF101.

\section{Librairies EDX}

\subsection{Librairies des macros de mise en forme}

\subsubsection{Epigraphe (edxlib-epigraphe)}

Utilisation :
\begin{lstlisting}[style=LATEX]
% inclusion de librairies EDX
%---------------------------------------------------------------
% Epigraphe Module - eduf@ction - 2023
%---------------------------------------------------------------


\usepackage{times}
\usepackage{xcolor}
\usepackage{epigraph}
\usepackage{tabto}
\usepackage{ifthen}
\setlength{\epigraphwidth}{0.75\linewidth}
\setlength{\afterepigraphskip}{\baselineskip}
\setlength{\beforeepigraphskip}{0.4\baselineskip}


\definecolor{epigraphecolor}{RGB}{125,125,125}


\makeatletter
\newenvironment{@leftepigraph}{
   \setlength\topsep{0pt}
   \setlength\parskip{0pt}
   \renewcommand{\epigraphflush}{flushleft}
}{}
\newenvironment{@rightepigraph}{
   \setlength\topsep{0pt}
   \setlength\parskip{0pt}
   \renewcommand{\epigraphflush}{flushright}
   \renewcommand{\sourceflush}{flushright}
   \renewcommand{\textflush}{flushright}
}{}
\newcommand{\@bgquote}[2]{\tabto*{#1}{\fontsize{100pt}{0pt}\selectfont{}\color{epigraphecolor!50}\smash{\raisebox{-60pt}{#2}}}\tabto*{0pt}}
\newcommand{\@generalquote}[6]{\begin{#6}\epigraph{\@bgquote{#5}{#4}#3}{\textsc{#2}\ifthenelse{\equal{#1}{}}{}{\\\textit{#1}}}\end{#6}}
\newcommand{\lquote}[3][]{\@generalquote{#1}{#2}{#3}{''}{\dimexpr(\epigraphwidth-50pt)}{@leftepigraph}}
\newcommand{\rquote}[3][]{\@generalquote{#1}{#2}{#3}{``}{0pt}{@rightepigraph}}
\makeatother

\end{lstlisting}	

Où \$libpath est le chemin des libs \EDX. Cette \lib utilisant le package \textbf{xcolor} avec des options, elle doit être chargé avant d'autres packages utilisant \textbf{xcolor}.

il y deux commandes :

\textbf{\textbackslash rquote} pour un alignement à droite (right)

\begin{lstlisting}[style=LATEX]
\rquote{Napoleon}{Quand on veut on peut, quand on peut on doit.}
\end{lstlisting}	
% Left aligned epigraph
\rquote{Napoleon}{Quand on veut on peut, quand on peut on doit.}

% Right aligned epigraph with a listed occupation

\textbf{\textbackslash rquote} pour un alignement à gauche (left) avec l'option d'ajout soit du titre de l'auteur, soit de la source de la citation.

\begin{lstlisting}[style=LATEX]
\lquote[Le calendrier de Pudd'nhead Wilson]{Mark Twain}{Quand vous êtes en colère, comptez jusqu'à quatre. Quand vous êtes très en colère, jurez!}
\end{lstlisting}

\lquote[Le calendrier de Pudd'nhead Wilson]{Mark Twain}{Quand vous êtes en colère, comptez jusqu'à quatre. Quand vous êtes très en colère, jurez!}

\subsubsection{Placeholder (edxlib-placeholder)}	
Cette librairie inclue des marques pour les TODO ou les reviews

\textbf{\textbackslash xtodo} pour ajouter une marque de TODO.

\xtodo{ne pas oublier d'ajouter des compléments}

\textbf{\textbackslash xtodo} pour ajouter une marque de REVIEW.
\xreview{ne pas oublier de modifier ce texte}
\begin{lstlisting}[style=LATEX]
\xtodo{ne pas oublier d'ajouter des compléments}
\xreview{ne pas oublier de modifier ce texte}
\end{lstlisting}

\section{Davinci Resolve - Video}
\subsection{Redim PDF for Beamer}
Les PDF beamer font 16,7 cm sur 9 cm

\begin{verbatim}	
gs -sDEVICE=pdfwrite -sPAPERSIZE=a4 -dFIXEDMEDIA -dPDFFitPage -dCompatibilityLevel=1.4 -o out.pdf CourseNotes-FR-SEC101-00-intro.prz.pdf
\end{verbatim}
	
\end{document}