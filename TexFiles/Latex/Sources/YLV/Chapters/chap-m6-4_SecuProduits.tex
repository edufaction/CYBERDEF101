\section{Sécurité des produits}

%La conformité technique aux référentiels/normes est un élément fondamental pour s'assurer que les produits logiciels et hardware sont conformes aux normes et réglementations du secteur.

La conformité technique aux référentiels et normes constitue un pilier essentiel pour garantir que les produits logiciels et matériels répondent aux exigences réglementaires et aux standards de sécurité de l'industrie. Cette démarche de conformité permet non seulement de valider la robustesse des solutions déployées mais aussi d'assurer leur interopérabilité et leur niveau de sécurité.

Ce chapitre présentera la conformité aux implementations normatives les plus connues ainsi que les dispositifs de certifications existants.

\subsection{Conformité aux implémentations normatives}
Pour aider à protéger les informations sensibles et les infrastructures critiques, divers référentiels ont été développés pour définir l'état de l'art en matière de sécurité informatique. Ces référentiels fournissent des cadres, des normes et des meilleures pratiques pour guider les organisations dans la mise en place de mesures de sécurité efficaces. Ce chapitre présente une liste ci-après de ces référentiels. Cette liste n'est évidemment pas exhaustive et est évolutive.

Atout concurrentiel ?

\begin{itemize}
	\item ISO/IEC 27001
	Norme internationale pour les systèmes de management de la sécurité de l'information (SMSI). Elle spécifie les exigences pour établir, mettre en œuvre, maintenir et améliorer continuellement un SMSI. Norme qu sert de livre de chevet pour tous les dirigeants de la sécurité, RSSI ou directeur de la sécurité. Elle donne les bonnes pratiques de gouvernance ainsi que les mesures à appliquer pour contrôler la mise en oeuvre. Cette norme est certifiante pour les SMSI mais aussi pour les utilisateurs implémenteurs ou auditeurs.
	
	\item NIST Cybersecurity Framework (CSF)
	Cadre développé par le National Institute of Standards and Technology (NIST) pour améliorer la gestion des risques de cybersécurité. Il se compose de cinq fonctions principales : Identifier, Protéger, Détecter, Répondre et Récupérer. De nombreux documents ont été édité par le NIST et servent aux organisations et responsables.
	
	\item CIS Controls
	Ensemble de 20 contrôles de sécurité critiques publiés par le Center for Internet Security (CIS). Ces contrôles sont conçus pour aider les organisations à se défendre contre les cybermenaces les plus courantes. Cf. chapitre 6.2 pour plus d'information sur l'application et l'utilisation de ces controles.
	
	\item COBIT
	Framework de gouvernance et de management des technologies de l'information (TI) développé par ISACA. COBIT fournit des outils pour aligner les objectifs de l'entreprise avec les objectifs de la TI.
	
	\item PCI DSS
	Norme de sécurité des données pour l'industrie des cartes de paiement. Elle vise à protéger les informations des titulaires de carte et à réduire la fraude liée aux cartes de paiement.
	
	\item GDPR
	Règlement général sur la protection des données de l'Union européenne. Il impose des obligations strictes aux organisations qui collectent ou traitent des données personnelles de résidents de l'UE.
	
	\item  HIPAA
	Loi américaine sur la portabilité et la responsabilité en matière d'assurance maladie. Elle établit des normes pour la protection des informations de santé des patients.
	
	\item FISMA
	Loi fédérale américaine sur la gestion de la sécurité de l'information. Elle impose des exigences de sécurité aux agences fédérales pour protéger les informations et les systèmes fédéraux.
	
	\item  SOC 2
	Norme de conformité pour les organisations de services. SOC 2 se concentre sur les contrôles relatifs à la sécurité, la disponibilité, l'intégrité du traitement, la confidentialité et la vie privée.
	
	\item OWASP Top Ten
	Liste des dix principales vulnérabilités de sécurité des applications web, publiée par l'Open Web Application Security Project (OWASP). Elle sert de guide pour les développeurs et les professionnels de la sécurité.
	
	\item ITIL
	Bibliothèque pour l'infrastructure des technologies de l'information. ITIL fournit des pratiques pour la gestion des services informatiques, y compris la gestion de la sécurité de l'information.
	
	\item SANS Top 20
	Liste des 20 contrôles de sécurité critiques publiée par le SANS Institute. Ces contrôles sont conçus pour aider les organisations à se protéger contre les cybermenaces les plus courantes.
	
	\item CSA CCM
	Matrice de contrôles de sécurité pour le cloud, publiée par la Cloud Security Alliance. Elle fournit des recommandations pour sécuriser les environnements de cloud computing.
	
	\item ISO/IEC 27017
	Code de pratique pour les contrôles de sécurité de l'information basés sur ISO/IEC 27002 pour les services cloud. Il fournit des lignes directrices spécifiques pour les fournisseurs de services cloud.
	
	\item ISO/IEC 27018
	Code de pratique pour la protection des informations personnelles (PII) dans les services cloud publics. Il se concentre sur la protection des données personnelles dans les environnements de cloud computing.
	
	\item ENISA
	Description : L'Agence de l'Union européenne pour la cybersécurité publie des lignes directrices et des recommandations pour améliorer la cybersécurité en Europe. Elle fournit des ressources pour aider les organisations à se protéger contre les cybermenaces. Par exemple, on peut citer le texte sur la sécurité de la 5G.
	
\end{itemize}

\subsection{La confiance certifiée}

Dans le domaine de la cybersécurité on peut faire certifier des produits logiciels ou matériels avec la Certification de Sécurité de Premier Niveau (CSPN) ou les critères communs. En France cela passe par l'ANSSI.

\subsubsection{Certification de Sécurité de Premier Niveau (CSPN)}
La CSPN mise en place par l’ANSSI en 2008 consiste en des tests en « boîte noire » effectués en temps et délais contraints. La CSPN est une alternative aux évaluations Critères Communs, dont le coût et la durée peuvent être un obstacle, et lorsque le niveau de confiance visé est moins élevé. Cette certification s’appuie sur des critères, une méthodologie et un processus élaborés par l’ANSSI publiés sur le présent site.
(source ANSSI)

\subsubsection{critères communs}
La certification dite tierce partie est la certification de plus haut niveau, qui permet à un client de s’assurer par l’intervention d’un professionnel indépendant, compétent et contrôlé, appelé organisme certificateur, de la conformité d’un produit à un cahier des charges ou à une spécification technique. 

La certification tierce partie apporte au client la confirmation indépendante et impartiale qu’un produit répond à un cahier des charges ou à des spécifications techniques publiées. Ces spécifications techniques peuvent être élaborées dans un cadre normatif ou non.
(source ANSSI)
AJOUTER EXEMPLES
%Begin FRAME----------------------------
\mode<presentation>{\texframe
% contenu affiché sur Article et Beamer
%- - - - - - - - - - - - - - - - - - - - - - - - 
{Sécurité des produits} % titre de la diapo
{Conformité aux implémentations normatives} % sous titre de la diapo
{
%begin slide- - - - - - - - - - - - - - - - - 

\begin{itemize}
	\item Protocoles réseaux
	\item Normes environnementales
\end{itemize}
%end slide- - - - - - - - - - - - - - - - - - - 
}}
%End FRAME------------------------------
%Begin FRAME----------------------------
\mode<presentation>{\texframe
% contenu affiché sur Article et Beamer
%- - - - - - - - - - - - - - - - - - - - - - - - 
{Sécurité des produits} % titre de la diapo
{La confiance certifiée} % sous titre de la diapo
{
%begin slide- - - - - - - - - - - - - - - - - 

\begin{itemize}
	\item Certification de Sécurité de Premier Niveau (CSPN) : tests en « boîte noire »
	\item critères communs : certification qui permet à un client de s’assurer par l’intervention organisme certificateur, de la conformité d’un produit à un cahier des charges ou à une spécification technique
\end{itemize}
%end slide- - - - - - - - - - - - - - - - - - - 
}}

%End FRAME------------------------------

%Begin FRAME----------------------------
\mode<all>{\texframe
% contenu affiché sur Article et Beamer
%- - - - - - - - - - - - - - - - - - - - - - - - 
{Points à retenir} % titre de la diapo
{} % sous titre de la diapo
{
%begin slide- - - - - - - - - - - - - - - - - 
\begin{itemize}
    \item
    \item
\end{itemize}
%end slide- - - - - - - - - - - - - - - - - - - 
}}
%End FRAME------------------------------