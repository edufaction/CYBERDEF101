%-------------------------------------
% Chapitre
% Vulnerability Management
% CERT
% File : chap-Vulman-CERT.tex
%--------------------------------------
\uchap{\jobname}

\subsection{Les services de veille en vulnérabilités}

Vous trouverez quelques éléments sur les CERTs (Computer Emergency Response Team) dans le chapitre sur les vulnérabilités, toutefois le périmètre de fonctions et services des CERTs s'est rapidement élargie ces dernières années. Au delà de la diffusion et alertes sur des vulnérabilités, ils couvrent maintenant avec précision les menaces (analyse et alerte sur codes malveillants, ...) et les incidents. Les CERTs restent les acteurs principaux de cette veille et capacité d'alerte.  
Nous trouvons toutefois de nombreux services de veille en vulnérabilités qui ne sont pas des CERTs, mais qui offrent des services dédiés à des typologies de produits, ou des secteurs ....

\subsubsection{Les CERT de l'ANSSI}


le \uac{aCERT} gouvernemental de l'\uac{aANSSI} (\ulink{https://www.cert.ssi.gouv.fr}{CERT GOUV FR}) publie régulièrement plusieurs types d'information :

\begin{itemize}
  \item Alertes de sécurité;
  \item Rapports sur des menaces et incidents;
  \item Avis de sécurité Indicateur de compromission;
  \item Bulletins et notes d'information.
\end{itemize}

% TODO XXXX


\subsubsection{Les CERTs commerciaux}

\begin{techworkbox}{CERT}
	Les CERTs commerciaux est un bon sujet d'exploration des sociétés qui délivrent des services de veille. Vous trouverez 
	\ulink{https://www.ssi.gouv.fr/agence/cybersécurité/ssi-en-france/les-cert-francais/}{les Certs en France} sur le site de l'ANSSI.  Excellent sujet pour \fichetech.
\end{techworkbox}

\subsubsection{La relation avec un CSIRT Interne }
% Analyse de malware. (Un malware utilise des vulnérabilités, et l’analyse d’une attaque donne aussi les éléments pour comber les fragilités

	Une méthodologie efficace de gestion des vulnérabilités comprend une équipe d’intervention en cas d’incident de sécurité informatique (CSIRT). Le CSIRT est responsable de la publication des avis de sécurité, de la tenue d'informations régulières pour échanger sur les activités malveillantes et des dernières attaques du jour zéro, de la simplification et de la diffusion des alertes de sécurité et de l’élaboration de directives compréhensibles et efficaces en matière de réaction aux incidents pour tous les salariés. De cette manière, chacun sera en mesure de réagir aux indicateurs de compromis potentiels conformément aux pratiques recommandées par l'équipe CSIRT.
	
	
\subsection{les agences de notation/cotation Cyber}

\begin{techworkbox}{Cotation Cyber}
	La cotation des entreprises sur le volet CYBER est un sujet en soi. Des entreprises comme (BiSight, Risk IQ, CyberVadis,..) se sont spécialisées dans la cotation du niveau de sécurité d'une entreprise. Sur la base de méthodes, de référentiels et d'algorithmes souvent propriétaires, elles fournissent une cotation CYBER de l'entreprise permettant à ses clients et partenaires de disposer d'une note ou d'un profil de solidité CYBER de leur infrastructure et organisation. Un excellent sujet pour \fichetech.
\end{techworkbox}

%bitsight, risk IQ, Cybervadis

%TODO (01) COTATION CYBER CYBERVADIS - mettre un exemple