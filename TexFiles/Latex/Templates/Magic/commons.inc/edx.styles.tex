%-------------------------------------------------------------
%               FR CYBERDEF SECOPS COURSE
%
%                                    EDX STYLE
%
%                              2020 eduf@ction
%-------------------------------------------------------------
\usepackage{parskip}
\setlength{\parindent}{0em}
\setlength{\parskip}{5pt} % 1ex plus 0.5ex minus 0.2ex}

%---------------------------------------------------------------
%	MARGINS
%---------------------------------------------------------------

\usepackage{geometry}

\geometry{
	paper=a4paper, 
%	top=1cm, 
%	bottom=1cm, 
%	left=1cm, 
%	right=1cm, 
%	headheight=20pt, 
%	footskip=1.4cm, 
%	headsep=10pt, 
%	showframe, 
	showcrop 
}

%---------------------------------------------------------------
%	FOOTNOTE (format)
%---------------------------------------------------------------

\makeatletter
\long\def\@makefntextFB#1{%
    \ifx\thefootnote\ftnISsymbol
        \@makefntextORI{#1}%
    \else
        \rule\z@\footnotesep
        \setbox\@tempboxa\hbox{\@thefnmark}%
            \ifdim\wd\@tempboxa>\z@
                \kern2em\llap{\@thefnmark.\kern0.5em}%
            \fi
        \hangindent2em\hangafter\@ne#1
    \fi}
\makeatother

%-----------------------------------------------------
%	COLOR
%-----------------------------------------------------
%\usepackage{xcolor} % Required for specifying colors by name
%\definecolor{uMainColor}{RGB}{160,0,0} 
\colorlet{edxcolorcover}{uMainColor}
%\definecolor{edxcolorcover}{RGB}{160,0,0} 
\definecolor{grey}{RGB}{50,80,80}
%\definecolor{cnam}{RGB}{128, 0, 32} 
                 

  \definecolor{comments}{rgb}{0.7,0,0}    % rouge foncé
  \definecolor{link}{rgb}{0,0.4,0.6}      % ~RoyalBlue de dvips
  \definecolor{url}{rgb}{0.6,0,0}         % rouge-brun
  \definecolor{citation}{rgb}{0,0.5,0}    % vert foncé
  \definecolor{ULlinkcolor}{rgb}{0,0,0.3} % de ulthese.cls
  \definecolor{rouge}{rgb}{0.85,0,0.07}   % rouge bandeau identitaire
  \definecolor{or}{rgb}{1,0.8,0}          % or bandeau identitaire

%-----------------------------------------------------
%	FONTS
%-----------------------------------------------------

%%\usepackage{avant} % Use the Avantgarde font for headings
%\usepackage{times} % Use the Times font for headings
%\usepackage{mathptmx} % Use the Adobe Times Rgman as the default text font together with math symbols from the Sym­bol, Chancery and Com­puter Modern fonts

%%\usepackage{microtype} % Slightly tweak font spacing for aesthetics
%%\usepackage[utf8]{inputenc} % Required for including letters with accents
%\usepackage[T1]{fontenc} % Use 8-bit encoding that has 256 glyphs

% Using AVANT GARDE Family

%\renewcommand{\familydefault}{\sfdefault}

%\fontfamily{pag}\selectfont
%\renewcommand{\familydefault}{pag} % sfdefault pag lmss

%bch         Charter
%lmr         Latin Modern Roman
%lmss        Latin Modern Sans Serif
%lmssq       Latin Modern Sans Serif extended
%lmtt        Latin Modern Typewriter
%lmvtt       Latin Modern Typewriter proportional
%pag         Avant Garde
%pbk         Bookman
%pcr         Courier
%phv         Helvetica
%pnc         New Century Schoolbook
%ppl         Palatino
%ptm         Times
%put         Utopia
 
%\renewcommand{\rmdefault}{pag} % text 
%\renewcommand{\sfdefault}{pag} % titre


\renewcommand{\rmdefault}{pag} % text 
\renewcommand{\sfdefault}{phv} % titre


%-----------------------------------------------------
%	BIBLIOGRAPHY AND INDEX
%-----------------------------------------------------


\usepackage{calc} % For simpler calculation - used for spacing the index letter headings correctly

\usepackage{imakeidx}
%\makeindex % Tells LaTeX to create the files required for indexing

\makeindex[columns=2,intoc=true,options={-s \upath/indexstyle.ist}]

%-----------------------------------------------------
%	HEADERS AND FOOTERS
%-----------------------------------------------------


\usepackage{fancyhdr} % Required for header and footer configuration
\pagestyle{fancy} % Enable the custom headers and footers

\ifcsname chapter \endcsname   % verify if command name exists in the CLASS (Book, report, article ...)
% BEGIN IF
\renewcommand{\chaptermark}[1]{\markboth{\sffamily\normalsize\bfseries\chaptername\ \thechapter.\ #1}{}} % Styling for the current chapter in the header
  \else
%nothing
\fi
% END IF

\renewcommand{\sectionmark}[1]{\markright{\sffamily\normalsize\thesection\hspace{5pt}#1}{}} % Styling for the current section in the header

\fancyhf{} % Clear default headers and footers
\fancyhead[LE,RO]{\sffamily\normalsize\thepage} % Styling for the page number in the header
\fancyhead[LO]{\sffamily\normalsize\rightmark} % Print the nearest section name on the left side of odd pages
\fancyhead[RE]{\sffamily\normalsize\leftmark} % Print the current chapter name on the right side of even pages
\fancyfoot[RE,LO]{\sffamily\normalsize\uCoursetittle} % Uncomment to include a footer
\fancyfoot[LE,RO]{\hyperlink{toc}{\sffamily\normalsize\uMinilogo}} % Uncomment to include a footer
% \hyperlink{toc}
\renewcommand{\headrulewidth}{1.0pt} % Thickness of the rule under the header

\fancypagestyle{plain}{% Style for when a plain pagestyle is specified
	\fancyhead{}\renewcommand{\headrulewidth}{0pt}} 

% Removes the header from odd empty pages at the end of chapters
\makeatletter
\renewcommand{\cleardoublepage}{
\clearpage\ifodd\c@page\else
\hbox{}
\vspace*{\fill}
\thispagestyle{empty}
\newpage
\fi}


%----------------------------------------------------------------------------------------
%	MAIN TABLE OF CONTENTS
%----------------------------------------------------------------------------------------
%\usepackage{titletoc} % Required for manipulating the table of contents

\contentsmargin{0cm} % Removes the default margin

% Part text styling (this is mostly taken care of in the PART HEADINGS section of this file)
\titlecontents{part}
	[0cm] % Left indentation
	{\addvspace{20pt}\bfseries} % Spacing and font options for parts
	{}
	{}
	{}

% Chapter text styling
\titlecontents{chapter}
	[1.25cm] % Left indentation
	{\addvspace{12pt}\large\sffamily\bfseries} % Spacing and font options for chapters
	{\color{uMainColor!60}\contentslabel[\Large\thecontentslabel]{1.25cm}\color{uMainColor}} % Formatting of numbered sections of this type
	{\color{uMainColor}} % Formatting of numberless sections of this type
	{\color{uMainColor!60}\normalsize\;\titlerule*[.5pc]{.}\;\thecontentspage} % Formatting of the filler to the right of the heading and the page number

% Section text styling
\titlecontents{section}
	[1.25cm] % Left indentation
	{\addvspace{3pt}\sffamily\bfseries} % Spacing and font options for sections
	{\contentslabel[\thecontentslabel]{1.25cm}} % Formatting of numbered sections of this type
	{} % Formatting of numberless sections of this type
	{\hfill\color{black}\thecontentspage} % Formatting of the filler to the right of the heading and the page number

% Subsection text styling
\titlecontents{subsection}
	[1.25cm] % Left indentation
	{\addvspace{1pt}\sffamily\small} % Spacing and font options for subsections
	{\contentslabel[\thecontentslabel]{1.25cm}} % Formatting of numbered sections of this type
	{} % Formatting of numberless sections of this type
	{\ \titlerule*[.5pc]{.}\;\thecontentspage} % Formatting of the filler to the right of the heading and the page number

%\titlecontents{subsubsection}
%	[1.25cm] % Left indentation
%	{\addvspace{1pt}\sffamily\tiny} % Spacing and font options for subsections
%	{\contentslabel[\thecontentslabel]{1.25cm}} % Formatting of numbered sections of this type
%	{} % Formatting of numberless sections of this type
%	{\ \titlerule*[.5pc]{.}\;\thecontentspage} % Formatting of the filler to the right of the heading and the page number

\titlecontents{subsubsection}
[1.25cm]
  {\addvspace{1pt}\sffamily\tiny}
 {\contentslabel[\thecontentslabel]{1.25cm}}
  {}
  {}

% Figure text styling
\titlecontents{figure}
	[1.25cm] % Left indentation
	{\addvspace{1pt}\sffamily\small} % Spacing and font options for figures
	{\thecontentslabel\hspace*{1em}} % Formatting of numbered sections of this type
	{} % Formatting of numberless sections of this type
	{\ \titlerule*[.5pc]{.}\;\thecontentspage} % Formatting of the filler to the right of the heading and the page number

% Table text styling
\titlecontents{table}
	[1.25cm] % Left indentation
	{\addvspace{1pt}\sffamily\small} % Spacing and font options for tables
	{\thecontentslabel\hspace*{1em}} % Formatting of numbered sections of this type
	{} % Formatting of numberless sections of this type
	{\ \titlerule*[.5pc]{.}\;\thecontentspage} % Formatting of the filler to the right of the heading and the page number

%----------------------------------------------------------------------------------------
%	MINI TABLE OF CONTENTS IN PART HEADS
%----------------------------------------------------------------------------------------

% Chapter text styling
\titlecontents{lchapter}
	[0em] % Left indentation
	{\addvspace{15pt}\large\sffamily\bfseries} % Spacing and font options for chapters
	{\color{uMainColor}\contentslabel[\Large\thecontentslabel]{1.25cm}\color{uMainColor}} % Chapter number
	{}  
	{\color{uMainColor}\normalsize\sffamily\bfseries\;\titlerule*[.5pc]{.}\;\thecontentspage} % Page number

% Section text styling
\titlecontents{lsection}
	[0em] % Left indentation
	{\sffamily\small} % Spacing and font options for sections
	{\contentslabel[\thecontentslabel]{1.25cm}} % Section number
	{}
	{}

% Subsection text styling (note these aren't shown by default, display them by searchings this file for tgcdepth and reading the commented text)
%\titlecontents{lsubsection}
%	[.5em] % Left indentation
%	{\sffamily\footnotesize} % Spacing and font options for subsections
%	{\contentslabel[\thecontentslabel]{1.25cm}}
%	{}
%	{}


\usepackage[]{ccicons}


%----------------------------------------------------------------------------------------
%	THEOREM STYLES
%----------------------------------------------------------------------------------------

\usepackage{amsmath,amsfonts,amssymb,amsthm} % For math equations, theorems, symbols, etc

\newcommand{\intoo}[2]{\mathopen{]}#1\,;#2\mathclose{[}}
\newcommand{\ud}{\mathop{\mathrm{{}d}}\mathopen{}}
\newcommand{\intff}[2]{\mathopen{[}#1\,;#2\mathclose{]}}
\renewcommand{\qedsymbol}{$\blacksquare$}

\ifcsname chapter \endcsname   % verify if command name exists in the CLASS (Book, report, article ...)
	\newtheorem{notation}{Notation}[chapter]
 \else
	\newtheorem{notation}{Notation}[section]
\fi


% Boxed/framed environments
\newtheoremstyle{ocrenumbox}% Theorem style name
{0pt}% Space above
{0pt}% Space below
{\normalfont}% Body font
{}% Indent amount
{\small\bf\sffamily\color{uMainColor}}% Theorem head font
{\;}% Punctuation after theorem head
{0.25em}% Space after theorem head
{\small\sffamily\color{uMainColor}\thmname{#1}\nobreakspace\thmnumber{\@ifnotempty{#1}{}\@upn{#2}}% Theorem text (e.g. Theorem 2.1)
\thmnote{\nobreakspace\the\thm@notefont\sffamily\bfseries\color{black}---\nobreakspace#3.}} % Optional theorem note

\newtheoremstyle{ocrenumbox}% Theorem style name
{0pt}% Space above
{0pt}% Space below
{\normalfont}% Body font
{}% Indent amount
{\small\bf\sffamily\color{uMainColor}}% Theorem head font
{\;}% Punctuation after theorem head
{0.25em}% Space after theorem head
{\small\sffamily\color{uMainColor}\thmname{#1}\nobreakspace\thmnumber{\@ifnotempty{#1}{}\@upn{#2}}% Theorem text (e.g. Theorem 2.1)
\thmnote{\nobreakspace\the\thm@notefont\sffamily\bfseries\color{black}---\nobreakspace#3.}} % Optional theorem note

\newtheoremstyle{blacknumex}% Theorem style name
{5pt}% Space above
{5pt}% Space below
{\normalfont}% Body font
{} % Indent amount
{\small\bf\sffamily}% Theorem head font
{\;}% Punctuation after theorem head
{0.25em}% Space after theorem head
{\small\sffamily{\tiny\ensuremath{\blacksquare}}\nobreakspace\thmname{#1}\nobreakspace\thmnumber{\@ifnotempty{#1}{}\@upn{#2}}% Theorem text (e.g. Theorem 2.1)
\thmnote{\nobreakspace\the\thm@notefont\sffamily\bfseries---\nobreakspace#3.}}% Optional theorem note


\newtheoremstyle{blacknumboxS}% name of the style to be used
{5pt}% measure of space to leave above the theorem. E.g.: 3pt
{5pt}% measure of space to leave below the theorem. E.g.: 3pt
{\normalfont}% name of font to use in the body of the theorem
{}% measure of space to indent
{\small\bf\sffamily}% name of head font
{}% punctuation between head and body
{ }% space after theorem head; " " = normal interword space
{\color{black}\faEye~  \color{uMainColor}\small\sffamily\thmnote{#3 : }}

% Non-boxed/non-framed environments
\newtheoremstyle{ocrenum}% Theorem style name
{5pt}% Space above
{5pt}% Space below
{\normalfont}% Body font
{}% Indent amount
{\small\bf\sffamily\color{uMainColor}}% Theorem head font
{\;}% Punctuation after theorem head
{0.25em}% Space after theorem head
{\small\sffamily\color{uMainColor}\thmname{#1}\nobreakspace\thmnumber{\@ifnotempty{#1}{}\@upn{#2}}% Theorem text (e.g. Theorem 2.1)
\thmnote{\nobreakspace\the\thm@notefont\sffamily\bfseries\color{black}---\nobreakspace#3.}} % Optional theorem note
\makeatother

% Defines the theorem text style for each type of theorem to one of the three styles above



\ifcsname chapter \endcsname

\newcounter{dummy} 

\numberwithin{dummy}{section}
	\theoremstyle{ocrenumbox}
		\newtheorem{theoremeT}[dummy]{Proposition}
		\newtheorem{problem}{Problem}[chapter]
		\newtheorem{exerciseT}{Outillage}[chapter]
	\theoremstyle{blacknumex}
		\newtheorem{exampleT}{Example}[chapter]
	\theoremstyle{blacknumbox}
		\newtheorem{vocabulary}{Vocabulary}[chapter]
		\newtheorem{definitionT}{Definition}[section]
		\newtheorem{corollaryT}[dummy]{Concept}
	\theoremstyle{blacknumboxS}
		\newtheorem{remarqueST}[]{}
		\newtheorem{remarqueT}[dummy]{Remarque}
	\theoremstyle{ocrenum}
	\newtheorem{proposition}[dummy]{Proposition}
  
 \else
 
\newcounter{dummy} 
\numberwithin{dummy}{section}

	\theoremstyle{ocrenumbox}
		\newtheorem{theoremeT}[dummy]{Proposition}
		%\newtheorem{problem}{Problem}[section]
		\newtheorem{exerciseT}{Outillage}[section]
	\theoremstyle{blacknumex}
		\newtheorem{exampleT}{Example}[section]
	\theoremstyle{blacknumbox}
		\newtheorem{vocabulary}{Vocabulary}[section]
		\newtheorem{definitionT}{Definition}[section]
		\newtheorem{corollaryT}[dummy]{concept}
	\theoremstyle{blacknumboxS}
		\newtheorem{remarqueST}[]{}
		\newtheorem{remarqueT}[dummy]{Remarque}
	\theoremstyle{ocrenum}
		\newtheorem{proposition}[dummy]{Proposition}
\fi



%----------------------------------------------------------------------------------------
%	DEFINITION OF COLORED BOXES
%----------------------------------------------------------------------------------------

\RequirePackage[framemethod=default]{mdframed} % Required for creating the theorem, definition, exercise and corollary boxes

% Theorem box
\newmdenv[skipabove=7pt,
skipbelow=7pt,
backgroundcolor=black!5,
linecolor=uMainColor,
innerleftmargin=5pt,
innerrightmargin=5pt,
innertopmargin=5pt,
leftmargin=0cm,
rightmargin=0cm,
innerbottommargin=5pt]{tBox}

% Exercise box	  
\newmdenv[skipabove=7pt,
skipbelow=7pt,
rightline=false,
leftline=true,
topline=false,
bottomline=false,
backgroundcolor=uMainColor!10,
linecolor=uMainColor,
innerleftmargin=5pt,
innerrightmargin=5pt,
innertopmargin=5pt,
innerbottommargin=5pt,
leftmargin=0cm,
rightmargin=0cm,
linewidth=4pt]{eBox}	

% Definition box
\newmdenv[skipabove=7pt,
skipbelow=7pt,
rightline=false,
leftline=true,
topline=false,
bottomline=false,
linecolor=uMainColor,
innerleftmargin=5pt,
innerrightmargin=5pt,
innertopmargin=0pt,
leftmargin=0cm,
rightmargin=0cm,
linewidth=4pt,
innerbottommargin=0pt]{dBox}	

% Corollary box
\newmdenv[skipabove=7pt,
skipbelow=7pt,
rightline=false,
leftline=true,
topline=false,
bottomline=false,
linecolor=gray,
backgroundcolor=black!5,
innerleftmargin=5pt,
innerrightmargin=5pt,
innertopmargin=5pt,
leftmargin=0cm,
rightmargin=0cm,
linewidth=4pt,
innerbottommargin=5pt]{cBox}

% Creates an environment for each type of theorem and assigns it a theorem text style from the "Theorem Styles" section above and a colored box from above
%\newenvironment{theorem}{\begin{tBox}\begin{theoremeT}}{\end{theoremeT}\end{tBox}}
\newenvironment{exercise}{\begin{eBox}\begin{exerciseT}}{\hfill{\color{uMainColor}\tiny\ensuremath{\blacksquare}}\end{exerciseT}\end{eBox}}				  
%\newenvironment{definition}{\begin{dBox}\begin{definitionT}}{\end{definitionT}\end{dBox}}	
%\newenvironment{example}{\begin{exampleT}}{\hfill{\tiny\ensuremath{\blacksquare}}\end{exampleT}}		
%\newenvironment{corollary}{\begin{cBox}\begin{corollaryT}}{\end{corollaryT}\end{cBox}}	
\newenvironment{nota}{\begin{cBox}\begin{remarqueT}}{\end{remarqueT}\end{cBox}}	

%----------------------------------------------------------------------------------------
%	REMARK ENVIRONMENT
%----------------------------------------------------------------------------------------

\newenvironment{remark}{\par\vspace{10pt}\small % Vertical white space above the remark and smaller font size
\begin{list}{}{
\leftmargin=35pt % Indentation on the left
\rightmargin=25pt}\item\ignorespaces % Indentation on the right
\makebox[-2.5pt]{\begin{tikzpicture}[overlay]
\node[draw=uMainColor!60,line width=1pt,circle,fill=uMainColor!25,font=\sffamily\bfseries,inner sep=2pt,outer sep=0pt] at (-15pt,0pt){\textcolor{uMainColor}{i}};\end{tikzpicture}} % Orange R in a circle
\advance\baselineskip -1pt}{\end{list}\vskip5pt} % Tighter line spacing and white space after remark

%----------------------------------------------------------------------------------------
%	SECTION NUMBERING IN THE MARGIN
%----------------------------------------------------------------------------------------

\makeatletter
% \renewcommand{\@seccntformat}[1]{\textcolor{uMainColor}{\csname the#1\endcsname}\quad}   

%\makeatother

%----------------------------------------------------------------------------------------
%	PART HEADINGS
%----------------------------------------------------------------------------------------

% Numbered part in the table of contents
\newcommand{\@mypartnumtocformat}[2]{%
	\setlength\fboxsep{0pt}%
	\noindent\colorbox{uMainColor!20}{\strut\parbox[c][.7cm]{\ecart}{\color{uMainColor!70}\Large\sffamily\bfseries\centering#1}}\hskip\esp\colorbox{uMainColor!40}{\strut\parbox[c][.7cm]{\linewidth-\ecart-\esp}{\Large\sffamily\centering#2}}%
}

% Unnumbered part in the table of contents
\newcommand{\@myparttocformat}[1]{%
	\setlength\fboxsep{0pt}%
	\noindent\colorbox{uMainColor!40}{\strut\parbox[c][.7cm]{\linewidth}{\Large\sffamily\centering#1}}%
}

\newlength\esp
\setlength\esp{4pt}
\newlength\ecart
\setlength\ecart{1.2cm-\esp}
\newcommand{\thepartimage}{}%
\newcommand{\partimage}[1]{\renewcommand{\thepartimage}{#1}}%
\def\@part[#1]#2{%
\ifnum \c@secnumdepth >-2\relax%
\refstepcounter{part}%
\addcontentsline{toc}{part}{\texorpdfstring{\protect\@mypartnumtocformat{\thepart}{#1}}{\partname~\thepart\ ---\ #1}}
\else%
\addcontentsline{toc}{part}{\texorpdfstring{\protect\@myparttocformat{#1}}{#1}}%
\fi%
\startcontents%
\markboth{}{}%
{\thispagestyle{empty}%
\begin{tikzpicture}[remember picture,overlay]%
\node at (current page.north west){\begin{tikzpicture}[remember picture,overlay]%	
\fill[uMainColor!20](0cm,0cm) rectangle (\paperwidth,-\paperheight);
\node[anchor=north] at (4cm,-3.25cm){\color{uMainColor!40}\fontsize{220}{100}\sffamily\bfseries\thepart}; 
\node[anchor=south east] at (\paperwidth-1cm,-\paperheight+1cm){\parbox[t][][t]{8.5cm}{
\printcontents{l}{0}{\setcounter{tocdepth}{1}}% The depth to which the Part mini table of contents displays headings; 0 for chapters only, 1 for chapters and sections and 2 for chapters, sections and subsections
}};
\node[anchor=north east] at (\paperwidth-1.5cm,-3.25cm){\parbox[t][][t]{15cm}{\strut\raggedleft\color{white}\fontsize{30}{30}\sffamily\bfseries#2}};
\end{tikzpicture}};
\end{tikzpicture}}%
\@endpart}
\def\@spart#1{%
\startcontents%
\phantomsection
{\thispagestyle{empty}%
\begin{tikzpicture}[remember picture,overlay]%
\node at (current page.north west){\begin{tikzpicture}[remember picture,overlay]%	
\fill[uMainColor!20](0cm,0cm) rectangle (\paperwidth,-\paperheight);
\node[anchor=north east] at (\paperwidth-1.5cm,-3.25cm){\parbox[t][][t]{15cm}{\strut\raggedleft\color{white}\fontsize{30}{30}\sffamily\bfseries#1}};
\end{tikzpicture}};
\end{tikzpicture}}
\addcontentsline{toc}{part}{\texorpdfstring{%
\setlength\fboxsep{0pt}%
\noindent\protect\colorbox{uMainColor!40}{\strut\protect\parbox[c][.7cm]{\linewidth}{\Large\sffamily\protect\centering #1\quad\mbox{}}}}{#1}}%
\@endpart}
\def\@endpart{\vfil\newpage
\if@twoside
\if@openright
\null
\thispagestyle{empty}%
\newpage
\fi
\fi
\if@tempswa
\twocolumn
\fi}

%----------------------------------------------------------------------------------------
%	CHAPTER HEADINGS
%-------------------------------------------------%--------------------------------------

%
%
%% A switch to conditionally include a picture, implemented by Christian Hupfer
\newif\ifusechapterimage
\usechapterimagetrue
\newcommand{\thechapterimage}{}%
\newcommand{\chapterimage}[1]{\ifusechapterimage\renewcommand{\thechapterimage}{#1}\fi}%
\newcommand{\autodot}{.}

\def\@makechapterhead#1{%
{\parindent \z@ \raggedright \normalfont
\ifnum \c@secnumdepth >\m@ne
\if@mainmatter
\begin{tikzpicture}[remember picture,overlay]
	
\node at (current page.north west)
			{\begin{tikzpicture}[remember picture,overlay]
			\node[anchor=north west,inner sep=0pt] at (0,0) {\ifusechapterimage\includegraphics[width=\paperwidth]{\thechapterimage}\fi};
			\draw[anchor=west] (\Gm@lmargin,-9cm) node [line width=2pt,rounded corners=15pt,draw=uMainColor,fill=white,fill opacity=1,inner sep=15pt]{\strut\makebox[22cm]{}};
			
			\draw[anchor=west] (\Gm@lmargin+.3cm,-9cm) node {\fontsize{20}{30}\selectfont\sffamily\bfseries\color{uMainColor}~#1\strut};
			%{\Huge\sffamily\bfseries\color{uMainColor}\thechapter\autodot~#1\strut};
			\node at (current page.north east)  [xshift=-3.95cm, yshift=-4.05cm, text opacity=1]  {{\color{white}\centering\fontsize{200}{30}\selectfont \bfseries\sffamily\thechapter\strut} };
			\node at (current page.north east)  [xshift=-4cm, yshift=-4cm, text opacity=1]  {{\color{uMainColor}\centering\fontsize{200}{30}\selectfont \bfseries\sffamily\thechapter\strut} };
			
			\end{tikzpicture}};
	\end{tikzpicture}

\else

\begin{tikzpicture}[remember picture,overlay]
\node at (current page.north west)
{\begin{tikzpicture}[remember picture,overlay]
\node[anchor=north west,inner sep=0pt] at (0,0) {\ifusechapterimage\includegraphics[width=\paperwidth]{\thechapterimage}\fi};
\draw[anchor=west] (\Gm@lmargin,-9cm) node [line width=2pt,rounded corners=15pt,draw=uMainColor,fill=white,fill opacity=1,inner sep=15pt]{\strut\makebox[22cm]{}};

\draw[anchor=east] (\Gm@lmargin,-9cm) node [line width=2pt,rounded corners=15pt,draw=uMainColor,fill=white,fill opacity=1,inner sep=15pt]{\strut\makebox[22cm]{}};
\draw[anchor=west] (\Gm@lmargin+.3cm,-9cm) node {\huge\sffamily\bfseries\color{uMainColor}#1\strut};

\end{tikzpicture}};
 
\end{tikzpicture}
\fi\fi\par\vspace*{270\p@}}}

\makeatother
%-------------------------------------------

%\def\@makeschapterhead#1{%
%\begin{tikzpicture}[remember picture,overlay]
%\node at (current page.north west)
%{\begin{tikzpicture}[remember picture,overlay]
%\node[anchor=north west,inner sep=0pt] at (0,0) {\ifusechapterimage\includegraphics[width=\paperwidth]{\thechapterimage}\fi};
%\draw[anchor=west] (\Gm@lmargin,-9cm) node [line width=2pt,rounded corners=15pt,draw=uMainColor,fill=white, fill opacity=0.8,inner sep=15pt]{\strut\makebox[22cm]{}};
%\draw[anchor=west] (\Gm@lmargin+.3cm,-9cm) node {\huge\sffamily\bfseries\color{uMainColor}#1\strut};
%\end{tikzpicture}};
%\end{tikzpicture}
%\par\vspace*{270\p@}}




%----------------------------------------------------------------------------------------
%	LINKS
%----------------------------------------------------------------------------------------


\usepackage{bookmark}
\bookmarksetup{
open,
numbered,
addtohook={%
\ifnum\bookmarkget{level}=0 % chapter
\bookmarksetup{bold}%
\fi
\ifnum\bookmarkget{level}=-1 % part
\bookmarksetup{color=uMainColor,bold}%
\fi
}
}

%----------------------------------------------------------------------------------------
%	FANCY CHAPTER Header
%----------------------------------------------------------------------------------------



